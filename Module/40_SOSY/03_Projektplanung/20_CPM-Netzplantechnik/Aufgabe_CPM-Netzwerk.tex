\documentclass{bschlangaul-aufgabe}
\bLadePakete{cpm,mathe,checkbox}
\begin{document}
\bAufgabenMetadaten{
  Titel = {CPM-Netzwerk},
  Thematik = {CPM-Netzwerk},
  Referenz = SOSY.Projektplanung.CPM-Netzplantechnik.CPM-Netzwerk,
  RelativerPfad = Module/40_SOSY/03_Projektplanung/20_CPM-Netzplantechnik/Aufgabe_CPM-Netzwerk.tex,
  ZitatSchluessel = sosy:e-klausur,
  BearbeitungsStand = mit Lösung,
  Korrektheit = unbekannt,
  Ueberprueft = {unbekannt},
  Stichwoerter = {CPM-Netzplantechnik},
}

\let\f=\footnotesize
\let\FZ=\bCpmFruehI
\let\SZ=\bCpmSpaetI
\let\v=\bCpmVon
\let\vz=\bCpmVonZu
\let\z=\bCpmZu

\begin{center}
\begin{tikzpicture}
\bCpmEreignis{1}{0}{0}
\bCpmEreignis{2}{1}{1}
\bCpmEreignis{3}{1}{-1}
\bCpmEreignis{4}{2}{0}
\bCpmEreignis{5}{3}{1}
\bCpmEreignis{6}{4}{0}
\bCpmEreignis{7}{5}{1}
\bCpmEreignis{8}{5}{-1}
\bCpmEreignis{9}{6}{0}

\bCpmVorgang{1}{2}{3}
\bCpmVorgang{1}{3}{4}
\bCpmVorgang{2}{5}{1}
\bCpmVorgang{3}{4}{3}
\bCpmVorgang{3}{8}{5}
\bCpmVorgang{5}{6}{2}
\bCpmVorgang{6}{7}{1}
\bCpmVorgang{6}{8}{3}
\bCpmVorgang{7}{9}{1}
\bCpmVorgang{8}{9}{2}

\bCpmVorgang[schein]{2}{4}{}
\bCpmVorgang[schein]{4}{6}{}
\bCpmVorgang[schein]{6}{9}{}
\bCpmVorgang[schein]{5}{7}{}
\end{tikzpicture}
\end{center}

\begin{enumerate}

%%
%
%%

\item Welche Scheinvorgänge könnten aus dem Netzwerk entfernt werden,
ohne dass Informationen verloren gehen?
\index{CPM-Netzplantechnik}
\footcite{sosy:e-klausur}

\begin{itemize}
\bCheckboxAngekreuzt 2 $\rightarrow$ 4
\bCheckboxLeer 4 $\rightarrow$ 6
\bCheckboxAngekreuzt 5 $\rightarrow$ 7
\bCheckboxAngekreuzt 6 $\rightarrow$ 9
\end{itemize}

%%
%
%%

\item Berechnen Sie für jedes Ereignis den frühesten Termin, wobei
angenommen wird, dass das Projekt zum Zeitpunkt 0 startet.

\begin{bAntwort}
\begin{center}
\begin{tabular}{|l|r|r|}
\hline
$i$ & Nebenrechnung & \FZ\\\hline\hline
1 &                                   & 0 \\\hline
2 & \f$0 + \vz{3}(1-2) = 3$           & 3 \\\hline
3 & \f$0 + \vz{4}(1-3) = 4$           & 4 \\\hline
4 & \f$\vz{3}(1-2) + \vz{0}(2-4) = 3$ & \\
  & \f$\vz{4}(1-3) + \vz{3}(3-4) = 7$ & \\
  & \f$\max(3,7)$                     & 7 \\\hline
5 & \f$\vz{3}(1-2) + \vz{1}(2-5) = 4$ & 4 \\\hline
6 & \f$\max(7 + 0, 4 + 2)$            & 7 \\\hline
7 & \f$\max(4 + 0, 7 + 1)$            & 8 \\\hline
8 & \f$\max(4 + 5, 7 + 3)$            & 10 \\\hline
9 & \f$\max(8 + 1, 7 + 0, 10 + 2)$    & 12 \\\hline
\end{tabular}
\end{center}
\end{bAntwort}

%%
%
%%

\item Berechnen Sie für jedes Ereignis auch die spätesten Zeiten, indem
Sie für das letzte Ereignis den frühesten Termin als spätesten Termin
ansetzen.

\begin{bAntwort}
\begin{center}
\begin{tabular}{|l|r|r|}
\hline
$i$ & Nebenrechnung & \SZ \\\hline\hline
1 & \f$\min(4 - 3, 4 - 4)$           & 0 \\
2 & \f$\min(5 - 1, 7 - 0)$           & 4 \\
3 & \f$\min(10 - 5, 7 - 3)$          & 4 \\
4 & \f$7 - 0$                        & 7 \\
5 & \f$\min(11 - 0, 7 - 2)$          & 5 \\
6 & \f$\min(12 - 0, 11 - 1, 10 - 3)$ & 7 \\
7 & \f$12 - 1$                       & 11 \\
8 & \f$12 - 2$                       & 10 \\
9 & \f{}siehe $\text{FZ}_9$  & 12 \\\hline
\end{tabular}
\end{center}
\end{bAntwort}

%%
%
%%

\item Geben Sie nun die Pufferzeiten der Ereignisse an.

\begin{bAntwort}
\begin{center}
\begin{tabular}{|l||l|l|l|l|l|l|l|l|l|}
\hline
Ereignis         & 1 & 2 & 3 & 4 & 5 & 6 & 7  & 8  & 9 \\\hline\hline
frühester Termin & 0 & 3 & 4 & 7 & 4 & 7 & 8  & 10 & 12 \\\hline
spätester Termin & 0 & 4 & 4 & 7 & 5 & 7 & 11 & 10 & 12 \\\hline
Puffer           & 0 & 1 & 0 & 0 & 1 & 0 & 3  & 0  & 0 \\\hline
\end{tabular}
\end{center}
\end{bAntwort}

\item Wie verläuft der kritische Pfad durch das Netzwerk?

\begin{bAntwort}
1 3 4 6 8 9
\end{bAntwort}

\end{enumerate}

\end{document}
