\documentclass{bschlangaul-aufgabe}
\bLadePakete{cpm,mathe}
\begin{document}
\bAufgabenMetadaten{
  Titel = {Aufgabe 1: CPM-Netzplantechnik},
  Thematik = {CPM mit Scheinvorgang},
  Referenz = SOSY.Projektplanung.CPM-Netzplantechnik.CPM-Scheinvorgang,
  RelativerPfad = Module/40_SOSY/03_Projektplanung/20_CPM-Netzplantechnik/Aufgabe_CPM-Scheinvorgang.tex,
  ZitatSchluessel = sosy:ab:5,
  ZitatBeschreibung = {Seite 1},
  BearbeitungsStand = mit Lösung,
  Korrektheit = unbekannt,
  Ueberprueft = {unbekannt},
  Stichwoerter = {CPM-Netzplantechnik},
}

\let\f=\footnotesize
\let\FZ=\bCpmFruehI
\let\SZ=\bCpmSpaetI
\let\v=\bCpmVon
\let\vz=\bCpmVonZu
\let\z=\bCpmZu

Gegeben ist das nachfolgende CPM-Netz. Gestrichelte Linien zwischen
Ereignissen stellen Scheinvorgänge mit einer Dauer von $0$ dar.
\index{CPM-Netzplantechnik}
\footcite[Seite 1]{sosy:ab:5}

\begin{center}
\begin{tikzpicture}
\bCpmEreignis{1}{-1}{2}
\bCpmEreignis{2}{1}{4}
\bCpmEreignis{3}{1}{0}
\bCpmEreignis{4}{3}{2}
\bCpmEreignis{5}{5}{4}
\bCpmEreignis{6}{5}{0}
\bCpmEreignis{7}{7}{2}
\bCpmEreignis{8}{9}{4}
\bCpmEreignis{9}{9}{0}
\bCpmEreignis{10}{11}{2}

\bCpmVorgang{1}{2}{2}
\bCpmVorgang{1}{3}{3}
\bCpmVorgang{2}{5}{4}
\bCpmVorgang{3}{4}{4}
\bCpmVorgang{3}{6}{3}
\bCpmVorgang{4}{5}{1}
\bCpmVorgang{5}{7}{3}
\bCpmVorgang{5}{8}{2}
\bCpmVorgang{6}{7}{3}
\bCpmVorgang{7}{10}{1}
\bCpmVorgang{7}{9}{2}
\bCpmVorgang{9}{10}{3}

\bCpmVorgang[schein]{1}{4}{}
\bCpmVorgang[schein]{4}{6}{}
\bCpmVorgang[schein]{6}{9}{}
\bCpmVorgang[schein]{8}{10}{}
\end{tikzpicture}
\end{center}

\begin{enumerate}

%%
% (a)
%%

\item Begründen Sie, welche Scheinvorgänge aus dem Netzplan ohne
Informationsverlust gestrichen werden könnten.

\begin{bAntwort}
Die Scheinvorgänge zwischen den Ereignissen $1$ und $4$ bzw. zwischen
$6$ und $9$ können jeweils gestrichen werden, da Ereignis $4$ schon auf
$1$ wartet (über $3$) und $9$ wartet auf $6$ (über $7$).
\end{bAntwort}

%%
% (b)
%%

\item Berechnen Sie für jedes Ereignis den \emph{frühesten Termin}, den
\emph{spätesten Termin} sowie die \emph{Gesamtpufferzeiten}.

\begin{bAntwort}
\begin{tabular}{|l|r|r|}
\hline
$i$ & Nebenrechnung & \FZ \\\hline\hline
1 &                                                       & $0$ \\\hline
2 &                                                       & $2$ \\\hline
3 &                                                       & $3$ \\\hline
4 &                                                       & $7$ \\\hline
5 & \f$\max(\v{3}(3) + 3,\v{7}(4) + 1)$                   & $8$ \\\hline
6 & \f$\max(\v{3}(3) + 3,\v{7}(4) + 0)$                   & $7$ \\\hline
7 & \f$\max(\v{8}(5) + 3,\v{7}(6) + 3)$                   & $11$ \\\hline
8 & \f$\v{8}(5) + 2$                                      & $10$ \\\hline
9 & \f$\max(\v{7}(6) + 0,\v{11}(7) + 2)$                  & $13$ \\\hline
10 & \f$\max(\v{10}(7) + 1, \v{8}(8) + 0, \v{13}(9) + 3)$ & $16$ \\\hline
\end{tabular}

\begin{tabular}{|l|r|r|}
\hline
$i$ & Nebenrechnung & \SZ \\\hline\hline
1 &                                         & $0$ \\\hline
2 & \f$\min(\v{8}(5) - 4$                   & $4$ \\\hline
3 & \f$\min(\v{8}(6) - 3,\v{7}(4) - 4)$     & $3$ \\\hline
4 & \f$\min(\v{8}(5) - 1,\v{8}(6) - 0)$     & $7$ \\\hline
5 & \f$\min(\v{16}(8) - 2,\v{11}(7) - 3)$   & $8$ \\\hline
6 & \f$\min(\v{11}(7) - 3,\v{13}(9) - 0)$   & $8$ \\\hline
7 & \f$\min(\v{16}(10) - 1, \v{13}(9) - 2)$ & $11$ \\\hline
8 & \f$\v{16}(10) - 0$                      & $16$ \\\hline
9 & \f$\v{16}(10) - 3$                      & $13$ \\\hline
10 & \f{}siehe $\text{FZ}_10$               & $16$ \\\hline
\end{tabular}

\begin{tabular}{|l||l|l|l|l|l|l|l|l|l|l|}
\hline
i   & 1 & 2  & 3   & 4  & 5  & 6  & 7  & 8  & 9  & 10 \\\hline\hline
\FZ & 0 & 2  & 3   & 7  & 8  & 7  & 11 & 10 & 13 & 16 \\\hline
\SZ & 0 & 4  & 3   & 7  & 8  & 8  & 11 & 16 & 13 & 16 \\\hline
GP  & 0 & 2  & 0   & 0  & 0  & 1  & 0  & 6  & 0  & 0 \\\hline
\end{tabular}
\end{bAntwort}

%%
% (c)
%%

\item Bestimmen Sie den kritischen Pfad.

\begin{bAntwort}
$1 \rightarrow 3 \rightarrow 4 \rightarrow 5 \rightarrow 7 \rightarrow 9 \rightarrow 10$

\begin{center}
\begin{tikzpicture}[scale=0.8,transform shape]
\bCpmEreignis{1}{-1}{2}
\bCpmEreignis{2}{1}{4}
\bCpmEreignis{3}{1}{0}
\bCpmEreignis{4}{3}{2}
\bCpmEreignis{5}{5}{4}
\bCpmEreignis{6}{5}{0}
\bCpmEreignis{7}{7}{2}
\bCpmEreignis{8}{9}{4}
\bCpmEreignis{9}{9}{0}
\bCpmEreignis{10}{11}{2}

\bCpmVorgang{1}{2}{2}
\bCpmVorgang{1}{3}{3}
\bCpmVorgang{2}{5}{4}
\bCpmVorgang{3}{4}{4}
\bCpmVorgang{3}{6}{3}
\bCpmVorgang{4}{5}{1}
\bCpmVorgang{5}{7}{3}
\bCpmVorgang{5}{8}{2}
\bCpmVorgang{6}{7}{3}
\bCpmVorgang{7}{10}{1}
\bCpmVorgang{7}{9}{2}
\bCpmVorgang{9}{10}{3}

\bCpmVorgang[schein]{1}{4}{}
\bCpmVorgang[schein]{4}{6}{}
\bCpmVorgang[schein]{6}{9}{}
\bCpmVorgang[schein]{8}{10}{}
\end{tikzpicture}
\end{center}
\end{bAntwort}
\end{enumerate}
\end{document}
