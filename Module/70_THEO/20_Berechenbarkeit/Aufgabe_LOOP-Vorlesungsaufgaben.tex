\documentclass{bschlangaul-aufgabe}
\bLadePakete{syntax}
\begin{document}
\bAufgabenMetadaten{
  Titel = {Vorlesungsaufgaben},
  Thematik = {Vorlesungsaufgaben},
  Referenz = THEO.Berechenbarkeit.LOOP-Vorlesungsaufgaben,
  RelativerPfad = Module/70_THEO/20_Berechenbarkeit/Aufgabe_LOOP-Vorlesungsaufgaben.tex,
  ZitatSchluessel = theo:fs:4,
  ZitatBeschreibung = {Seite 11},
  BearbeitungsStand = mit Lösung,
  Korrektheit = unbekannt,
  Ueberprueft = {unbekannt},
  Stichwoerter = {Berechenbarkeit},
}

LOOP-Implementierung\footcite[Seite 11]{theo:fs:4}\index{Berechenbarkeit}

\begin{enumerate}
\item Geben Sie eine LOOP-Implementierung für

\begin{enumerate}

%%
%
%%

\item $add(x_i, x_j)$

\begin{bAntwort}
\begin{minted}{md}
x_0 := x_i;
LOOP x_j DO
  x0 := succ(x_0);
END
\end{minted}
\end{bAntwort}

%%
%
%%

\item $mult(x_i, x_j)$

\begin{bAntwort}
\begin{minted}{md}
x_0 := x_i;
LOOP x_j DO
  x0 := add(x_0, x_i);
END
\end{minted}
\end{bAntwort}

%%
%
%%

\item $power(x_i, x_j)$

\begin{bAntwort}
\begin{minted}{md}
x_0 := succ(0);
LOOP x_j DO
  x0 := mult(x_0, x_i);
END
\end{minted}
\end{bAntwort}

%%
%
%%

\item $hyper(x_i, x_j)$

\begin{bAntwort}
\begin{minted}{md}
x_0 := succ(0);
LOOP x_j DO
  x_0 := power(x_i, x_0);
END
\end{minted}
\end{bAntwort}

\item $2^{x_i}$

\begin{bAntwort}
% Info_2021-05-07-2021-05-07_09.32.08.mp4 1h42min
Mit \emph{power}

\begin{minted}{md}
x_0 := power(2, x_i);
\end{minted}

Mit \emph{mult}

\begin{minted}{md}
x_0 := 1;
x_2 := 2;
LOOP x_i DO
  x0 := mult(x_0, x_2);
END
\end{minted}
\end{bAntwort}
\end{enumerate}

an.

\item Beweisen Sie, dass der größte gemeinsame Teiler zweier natürlicher
Zahlen LOOP-berechenbar ist.

\begin{bAntwort}
% Info_2021-05-07-2021-05-07_09.32.08.mp4 1h47min
\begin{minted}{md}
ggT(x_1, x_2)

x_3 := MAX(x_1, x_2);
x_4 := MIN(x_1, x_2);

LOOP x_4 DO
  x_5 := x_3 - x_4;
  x_3 := MAX(x_4, x_5);
  x_4 := MIN(x_4, x_5);
END
x_0 := x_3;
\end{minted}
\end{bAntwort}
\end{enumerate}

\end{document}
