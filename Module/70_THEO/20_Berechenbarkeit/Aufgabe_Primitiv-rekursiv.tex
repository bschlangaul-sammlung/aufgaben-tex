\documentclass{bschlangaul-aufgabe}

\begin{document}
\bAufgabenMetadaten{
  Titel = {Primitiv-rekursiv},
  Thematik = {Primitiv-rekursiv},
  Referenz = THEO.Berechenbarkeit.Primitiv-rekursiv,
  RelativerPfad = Module/70_THEO/20_Berechenbarkeit/Aufgabe_Primitiv-rekursiv.tex,
  ZitatSchluessel = theo:ab:4,
  ZitatBeschreibung = {Seite 7, Aufgabe 5},
  BearbeitungsStand = mit Lösung,
  Korrektheit = unbekannt,
  Ueberprueft = {unbekannt},
  Stichwoerter = {Berechenbarkeit},
}

\begin{enumerate}
%%
% a)
%%

\item Begründe, dass folgende Funktionen primitiv-rekursiv sind:
1. f (x) = x!
(
1 wennx > 0
2. sig(x) =
0 sonst
\index{Berechenbarkeit}
\footcite[Seite 7, Aufgabe 5]{theo:ab:4}

\begin{bAntwort}
Begründung durch eine Angabe einer Funktion:
1. f (0) = 1, f (n + 1) = mult(n + 1, f (n))
2. sig(0) = 0, sig(n + 1) = 1
\end{bAntwort}

%%
% (b)
%%

\item Gebe eine konkrete primitiv-rekursive Implementierung für if x1 than x2 else
x3 an.
Wobei wie bei Programmiersprachen x1 als wahr gilt, wenn der Wert nicht Null
ist.

\begin{bAntwort}
 Zusätzlich werden die folgenden Funktionen festgelegt:

isZero(0) = 1
isZero(n) =
isZero(n + 1) = 0
not(n) = 1 − n
ite(x1, x2, x3) = isZero(x1) * x3 + not(isZero(x1)) * x2
\end{bAntwort}

\end{enumerate}
\end{document}
