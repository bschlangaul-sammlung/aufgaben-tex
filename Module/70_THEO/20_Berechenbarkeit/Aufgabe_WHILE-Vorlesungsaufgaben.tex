\documentclass{bschlangaul-aufgabe}
\bLadePakete{syntax,mathe}
\begin{document}
\bAufgabenMetadaten{
  Titel = {Vorlesungsaufgaben WHILE-Programm},
  Thematik = {Vorlesungsaufgaben},
  Referenz = THEO.Berechenbarkeit.WHILE-Vorlesungsaufgaben,
  RelativerPfad = Module/70_THEO/20_Berechenbarkeit/Aufgabe_WHILE-Vorlesungsaufgaben.tex,
  ZitatSchluessel = theo:fs:4,
  ZitatBeschreibung = {Seite 16},
  BearbeitungsStand = mit Lösung,
  Korrektheit = unbekannt,
  Ueberprueft = {unbekannt},
  Stichwoerter = {Berechenbarkeit},
}

Geben Sie ein WHILE-Programm an, dass
\index{Berechenbarkeit}
\footcite[Seite 16]{theo:fs:4}

% Info_2021-05-07-2021-05-07_09.32.08.mp4 2h29

\begin{itemize}
\item $2^{x_i}$

\begin{bAntwort}
Ausnutzen der 2er-Potenzeigenschaft:

$2^1 = 1 + 1 = 2$

$2^2 = 2 + 2 = 4$

$2^3 = 4 + 4 = 8$

Hier werden nur die elementaren Bestandteile der WHILE-Sprache ausgenutzt.

\begin{minted}{md}
x_2 := 1;
WHILE x_1 DO
  x_3 := x_2;
  WHILE x_3 DO
    x_2 := x_2 + 1;
    x_3 := x_3 - 1;
  END
  x_1 := x_1 - 1;
END
x_0 := x_2;
\end{minted}
\end{bAntwort}

\item $\text{ggt}(x_i, x_j)$

\begin{bAntwort}
Zusätzliche Voraussetzungen:

\begin{minted}{md}
x_1 > x_2;
MOD(x_1, x_2);
\end{minted}

\begin{minted}{md}
x_0 := MOD(x_1, x_2);
WHILE x_0 DO
  x_1 := x_2;
  x_2 := x_0;
  x_0 := MOD(x_1, x_2);
END
\end{minted}
\end{bAntwort}
\end{itemize}

berechnet.

\end{document}
