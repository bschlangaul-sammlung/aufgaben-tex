\documentclass{bschlangaul-aufgabe}
\bLadePakete{formale-sprachen,syntaxbaum,automaten}
\begin{document}
\bAufgabenMetadaten{
  Titel = {Kontextfreie Grammatik},
  Thematik = {Vorlesungsaufgabe},
  Referenz = THEO.Formale-Sprachen.Typ-2_Kontextfrei.Grammatik.Vorlesungsaufgabe,
  RelativerPfad = Module/70_THEO/10_Formale-Sprachen/20_Typ-2_Kontextfrei/Grammatik/Aufgabe_Vorlesungsaufgabe.tex,
  ZitatSchluessel = theo:fs:2,
  BearbeitungsStand = mit Lösung,
  Korrektheit = unbekannt,
  Ueberprueft = {unbekannt},
  Stichwoerter = {Kontextfreie Sprache, Ableitung (Kontextfreie Sprache), Kontextfreie Grammatik},
}

\begin{enumerate}
\item Erstellen Sie eine Ableitung für die Wörter der Sprache zur
vorgegeben Grammatik\index{Kontextfreie Sprache}
\footcite{theo:fs:2}
\footcite[Seite 10]{theo:fs:2}
\index{Ableitung (Kontextfreie Sprache)}

\bGrammatik{alphabet={0, 1}, variablen={S, A, B }}

\begin{bProduktionsRegeln}
S -> A 1 B,
A -> 0 A | EPSILON,
B -> 0 B | 1 B | EPSILON,
\end{bProduktionsRegeln}

\bFlaci{Gi1rgpemg}

\begin{itemize}
\item 00101

\begin{bAntwort}
\bAbleitung{
S ->
A1B ->
0A1B ->
00A1B ->
001B ->
0010B ->
00101B ->
00101
}
\end{bAntwort}

\item 1001

\begin{bAntwort}
\bAbleitung{
S ->
A1B ->
1B ->
10B ->
100B ->
1001B ->
1001
}
\end{bAntwort}
\end{itemize}

\item Erstellen Sie eine kontextfreie Grammatik, die alle Wörter mit
gleich vielen $1$‘s, gefolgt von gleich vielen $0$‘s enthält.
\index{Kontextfreie Grammatik}

\begin{bAntwort}
\begin{bProduktionsRegeln}
S -> 1S0 | epsilon
\end{bProduktionsRegeln}
\bFlaci{Grxmyw2ia}
\end{bAntwort}

\item Erstellen Sie eine kontextfreie Grammatik, die alle regulären
Ausdrücke über den Zeichen $0,1$ darstellt. (Beispiel:
\texttt{01*(1+0)0} für einen möglichen regulären Ausdruck (Das
\texttt{+}-Zeichen ist hier anstelle des Oder-Zeichens (|)))
\footcite[Aufgabe 2a)]{theo:ab:5}

\begin{bAntwort}
\bGrammatik{alphabet={1; 0; (; ); +; *}, variablen={S}}

\begin{bProduktionsRegeln}
S -> EPSILON | 0 | 1 | S * | ( S ) | S S | S + S
\end{bProduktionsRegeln}
\bFlaci{Ghfgrv027}
\end{bAntwort}

\end{enumerate}

\end{document}
