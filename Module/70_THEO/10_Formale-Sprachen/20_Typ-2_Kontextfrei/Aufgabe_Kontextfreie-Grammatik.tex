\documentclass{bschlangaul-aufgabe}
\bLadePakete{formale-sprachen}
\begin{document}
\bAufgabenMetadaten{
  Titel = {Kontextfreie Grammatiken},
  Thematik = {Kontextfreie-Grammatik},
  Referenz = THEO.Formale-Sprachen.Typ-2_Kontextfrei.Kontextfreie-Grammatik,
  RelativerPfad = Module/70_THEO/10_Formale-Sprachen/20_Typ-2_Kontextfrei/Aufgabe_Kontextfreie-Grammatik.tex,
  ZitatSchluessel = theo:ab:2,
  BearbeitungsStand = mit Lösung,
  Korrektheit = unbekannt,
  Ueberprueft = {unbekannt},
  Stichwoerter = {Kontextfreie Grammatik},
}

Geben Sie für die folgenden Sprachen eine kontextfreie Grammatik an:
\index{Kontextfreie Grammatik}
\footcite{theo:ab:2}

\begin{itemize}

%%
%
%%

\item Alle Wörter der Sprachen bestehen aus $a$’s, gefolgt von gleich
vielen $b$’s.

\begin{bAntwort}
\begin{bProduktionsRegeln}
S -> a S b | EPSILON
\end{bProduktionsRegeln}
\end{bAntwort}

%%
%
%%

\item Die Wörter der Sprache bestehen aus gleich vielen $x$ wie $y$.

\begin{bAntwort}
\begin{bProduktionsRegeln}
S -> x S Y | y S X | EPSILON,
X -> x S,
Y -> y S
\end{bProduktionsRegeln}
\end{bAntwort}
\end{itemize}

\end{document}
