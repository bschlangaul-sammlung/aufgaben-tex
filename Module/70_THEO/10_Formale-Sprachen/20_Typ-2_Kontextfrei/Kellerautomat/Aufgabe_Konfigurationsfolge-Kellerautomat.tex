\documentclass{bschlangaul-aufgabe}
\bLadePakete{mathe,automaten,formale-sprachen,mathe}
\begin{document}
\bAufgabenMetadaten{
  Titel = {Konfigurationsfolge von Kellerautomaten},
  Thematik = {Konfigurationsfolge doppelt so viele b’s wie a’s},
  Referenz = THEO.Formale-Sprachen.Typ-2_Kontextfrei.Kellerautomat.Konfigurationsfolge-Kellerautomat,
  RelativerPfad = Module/70_THEO/10_Formale-Sprachen/20_Typ-2_Kontextfrei/Kellerautomat/Aufgabe_Konfigurationsfolge-Kellerautomat.tex,
  ZitatSchluessel = theo:ab:2,
  BearbeitungsStand = mit Lösung,
  Korrektheit = korrekt,
  Ueberprueft = {unbekannt},
  Stichwoerter = {Kellerautomat},
}

\let\m=\bMengeOhneMathe
\let\z=\bZustandsnameTiefgestellt

Gegeben ist der folgende nichtdeterministische Kellerautomat mit
\index{Kellerautomat}
\footcite{theo:ab:2}

\begin{center}
\bKellerAutomat{
  zustaende={z_0, z_1, z_2, z_3, z_4},
  alphabet={a, b},
  kelleralphabet={\#, A, B},
  ende={z_4},
}
\end{center}

\begin{center}
\begin{tikzpicture}[li kellerautomat]
  \node[state,initial] (z0) at (2.71cm,-6cm) {$z_0$};
  \node[state] (z1) at (4.71cm,-2.43cm) {$z_1$};
  \node[state] (z2) at (8.71cm,-2.43cm) {$z_2$};
  \node[state] (z3) at (9.71cm,-5.71cm) {$z_3$};
  \node[state,accepting] (z4) at (4.29cm,-8.14cm) {$z_4$};

  \bKellerKante[left,bend left]{z0}{z1}{
    a, KELLERBODEN, A A KELLERBODEN;
  }

  \bKellerKante[left]{z0}{z4}{
    EPSILON, KELLERBODEN, EPSILON;
  }

  \bKellerKante[below,bend left]{z0}{z3}{
    b, KELLERBODEN, B KELLERBODEN;
  }

  \bKellerKante[above right,bend left]{z1}{z0}{
    EPSILON, KELLERBODEN, KELLERBODEN;
  }

  \bKellerKante[above,loop]{z1}{z1}{
    a, A, A A A;
    b, A, EPSILON;
  }

  \bKellerKante[right,bend left]{z2}{z3}{
    EPSILON, B, EPSILON;
  }

  \bKellerKante[above]{z2}{z1}{
    EPSILON, KELLERBODEN, A KELLERBODEN;
  }

  \bKellerKante[below,bend left]{z3}{z0}{
    EPSILON, KELLERBODEN, KELLERBODEN;
  }

  \bKellerKante[above,loop below]{z3}{z3}{
    b, B, B B;
  }

  \bKellerKante[left,bend left]{z3}{z2}{
    a, B, EPSILON;
  }
\end{tikzpicture}
\end{center}
\bFlaci{Apk0ic3s9}

\begin{enumerate}

\item Geben Sie für die folgenden Wörter, die in der Sprache enthalten sind,
eine Berechnung (Folge von Konfigurationen) des Kellerautomaten an:

\begin{enumerate}

%%
% (a)
%%

\item $w_1 = \texttt{bab}$

\begin{bAntwort}
$(\z0, bab, \#) \vdash
(\z3, ab, B\#) \vdash
(\z2, b, \#) \vdash
(\z1, b, A\#) \vdash
(\z1, \varepsilon, \#) \vdash
(\z0, \varepsilon, \#) \vdash
(\z4, \varepsilon, \varepsilon)$
\end{bAntwort}

%%
% (b)
%%

\item $w_2 = \texttt{abb}$

\begin{bAntwort}
$(\z0, abb, \#) \vdash
(\z1, bb, AA\# )\vdash
(\z1, b, A\#) \vdash
(\z1, \varepsilon, \#) \vdash
(\z0, \varepsilon, \#) \vdash
(\z4, \varepsilon, \varepsilon)$
\end{bAntwort}

%%
% (c)
%%

\item $w_3 = \texttt{abababbbb}$

\begin{bAntwort}
$(\z0, abababbbb, \#) \vdash
(\z1, bababbbb, AA\#) \vdash
(\z1, ababbbb, A\#) \vdash
(\z1, babbbb, AAA\#) \vdash
(\z1, abbbb, AA\#) \vdash
(\z1, bbbb, AAAA\#) \vdash
(\z1, bbb, AAA\#) \vdash
(\z1, bb, AA\#) \vdash
(\z1, b, A\#) \vdash
(\z1, \varepsilon, \#) \vdash
(\z0, \varepsilon, \#) \vdash
(\z4, \varepsilon, \varepsilon)$
\end{bAntwort}

\end{enumerate}

\item Charakterisiere die Wörter der Sprache in eigenen Worten.

\begin{bAntwort}
\bAusdruck{w}
{
  w
  \text{ enthält genau doppelt so viele }
  b \text{’s wie }
  a \text{’s}
}
\end{bAntwort}

\end{enumerate}
\end{document}
