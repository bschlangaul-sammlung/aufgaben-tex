\documentclass{bschlangaul-aufgabe}
\bLadePakete{formale-sprachen,cyk-algorithmus}
\begin{document}
\bAufgabenMetadaten{
  Titel = {Aufgabe},
  Thematik = {AB5},
  Referenz = THEO.Formale-Sprachen.Typ-2_Kontextfrei.CYK-Algorithmus.AB5,
  RelativerPfad = Module/70_THEO/10_Formale-Sprachen/20_Typ-2_Kontextfrei/CYK-Algorithmus/Aufgabe_AB5.tex,
  ZitatSchluessel = theo:ab:5,
  ZitatBeschreibung = {Aufgabe 2b},
  BearbeitungsStand = mit Lösung,
  Korrektheit = unbekannt,
  Ueberprueft = {unbekannt},
  Stichwoerter = {CYK-Algorithmus},
}

\let\l=\bKurzeTabellenLinie

Teste jeweils mit dem CYK-Algorithmus, ob das angegebene Wort zur Sprache
der Grammatik gehört.\index{CYK-Algorithmus}
\footcite[Aufgabe 2b]{theo:ab:5}

\begin{bProduktionsRegeln}
S -> S S | R_a A | R_b B | R_c C,
A -> R_b R_c | R_c R_b,
B -> R_a R_c | R_c R_a,
C -> R_a R_b | R_b R_a,
R_a -> a,
R_b -> b,
R_c -> c,
\end{bProduktionsRegeln}

\begin{enumerate}

%%
% 1.
%%

\item $\omega_1 = acbcab$

\begin{bAntwort}
\begin{tabular}{|c|c|c|c|c|c|}
a     & c     & b     & c     & a     & b \\\hline\hline

$R_a$ & $R_c$ & $R_b$ & $R_c$ & $R_a$ & $R_b$ \l6
B     & A     & A     & B     & C \l5
S     & -     & S     & S \l4
-     & -     & - \l3
-     & - \l2
S \l1
\end{tabular}

\bWortInSprache{\omega_1}
\end{bAntwort}

%%
% 2.
%%

\item $\omega_2 = cabb$

\begin{bAntwort}
\begin{tabular}{|c|c|c|c|}
c     & a     & b     & b    \\\hline\hline

$R_c$ & $R_a$ & $R_b$ & $R_b$  \l4
B     & C     & - \l3
S     & - \l2
- \l1
\end{tabular}

\bWortNichtInSprache{\omega_2}
\end{bAntwort}

\end{enumerate}

\end{document}
