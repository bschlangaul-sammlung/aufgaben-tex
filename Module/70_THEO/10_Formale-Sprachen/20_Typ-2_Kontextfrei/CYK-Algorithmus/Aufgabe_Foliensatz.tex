\documentclass{bschlangaul-aufgabe}
\bLadePakete{formale-sprachen,cyk-algorithmus}
\begin{document}
\bAufgabenMetadaten{
  Titel = {CYK-Algorithmus},
  Thematik = {Foliensatz},
  Referenz = THEO.Formale-Sprachen.Typ-2_Kontextfrei.CYK-Algorithmus.Foliensatz,
  RelativerPfad = Module/70_THEO/10_Formale-Sprachen/20_Typ-2_Kontextfrei/CYK-Algorithmus/Aufgabe_Foliensatz.tex,
  ZitatSchluessel = theo:fs:2,
  ZitatBeschreibung = {Seite 46},
  BearbeitungsStand = mit Lösung,
  Korrektheit = unbekannt,
  Ueberprueft = {unbekannt},
  Stichwoerter = {CYK-Algorithmus},
}

\let\l=\bKurzeTabellenLinie

\begin{bProduktionsRegeln}
S -> A B | B T,
A -> B A | a,
B -> T T | b,
T -> A B | a
\end{bProduktionsRegeln}
\index{CYK-Algorithmus}
\footcite[Seite 46]{theo:fs:2}

\begin{bAntwort}
\begin{tabular}{|c|c|c|c|c|}
b     & a     & a   & a   & b \\\hline\hline

B     & A,T   & A,T & A,T & B \l5
A,S   & B     & B   & S,T \l4
-     & S,T,A & B \l3
S,A,T & S,T \l2
S,T   \l1
\end{tabular}

\bWortInSprache{baaab}
\end{bAntwort}

\end{document}
