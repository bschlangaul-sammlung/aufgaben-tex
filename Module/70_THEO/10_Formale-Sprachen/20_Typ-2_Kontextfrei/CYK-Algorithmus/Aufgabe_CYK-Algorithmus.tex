\documentclass{bschlangaul-aufgabe}
\bLadePakete{formale-sprachen,cyk-algorithmus}
\begin{document}
\bAufgabenMetadaten{
  Titel = {Wortproblem - CYK-Algorithmus},
  Thematik = {CYK-Algorithmus},
  Referenz = THEO.Formale-Sprachen.Typ-2_Kontextfrei.CYK-Algorithmus.CYK-Algorithmus,
  RelativerPfad = Module/70_THEO/10_Formale-Sprachen/20_Typ-2_Kontextfrei/CYK-Algorithmus/Aufgabe_CYK-Algorithmus.tex,
  ZitatSchluessel = theo:ab:2,
  BearbeitungsStand = mit Lösung,
  Korrektheit = unbekannt,
  Ueberprueft = {unbekannt},
  Stichwoerter = {CYK-Algorithmus},
}

\let\l=\bKurzeTabellenLinie

Teste jeweils mit dem CYK-Algorithmus, ob das angegebene Wort zur
Sprache der Grammatik gehört.
\index{CYK-Algorithmus}
\footcite{theo:ab:2}

\begin{enumerate}

%%
% (a)
%%

\item Grammatik $G_1$:

\begin{bProduktionsRegeln}
S -> S S | R_a A | R_b B | R_c C,
A -> R_b R_c | R_c R_b,
B -> R_a R_c | R_c R_a,
C -> R_a R_b | R_b R_a,
R_a -> a,
R_b -> b,
R_c -> c
\end{bProduktionsRegeln}

\begin{enumerate}
\item $w_1 = abccab$
\item $w_2 = abcba$
\end{enumerate}

%%
% (b)
%%

\item Grammatik $G_2$:

\begin{bProduktionsRegeln}
Z -> X D | X A | EPSILON,
S -> X D | X A,
A -> V_a D | V_a A | A B | c,
B -> B B | C C | c | a,
C -> C C | c,
D -> A B,
X -> V_a D | V_a A | A B | b | c,
V_a -> a
\end{bProduktionsRegeln}

\begin{enumerate}
\item $w_3 = bacac$

\begin{bAntwort}
\begin{tabular}{|c|c|c|c|c|}
b   & a   & c   & a   & c \\\hline\hline

X   & B,$V_a$ & A,B,C,X     & B,$V_a$ & A,B,C,X \l5
-   & B,A,X   & A,D,X,B     & B,A,X \l4
Z,S & A,B,D,X & B,A,X,D,S,Z \l3
Z,S & A,B,D,S,X,Z \l2
Z,S \l1
\end{tabular}
\end{bAntwort}

\item $w_4 = baca$
\end{enumerate}
\end{enumerate}
\end{document}
