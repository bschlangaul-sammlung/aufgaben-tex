\documentclass{bschlangaul-aufgabe}

\begin{document}
\bAufgabenMetadaten{
  Titel = {Reguläre Sprache in kontextfreier Sprache},
  Thematik = {Reguläre Sprache in kontextfreier Sprache},
  Referenz = THEO.Formale-Sprachen.Typ-2_Kontextfrei.Regulaere-Sprache-in-kontextfreier-Sprache,
  RelativerPfad = Module/70_THEO/10_Formale-Sprachen/20_Typ-2_Kontextfrei/Aufgabe_Regulaere-Sprache-in-kontextfreier-Sprache.tex,
  ZitatSchluessel = theo:ab:2,
  BearbeitungsStand = mit Lösung,
  Korrektheit = unbekannt,
  Ueberprueft = {unbekannt},
  Stichwoerter = {Reguläre Sprache, Kontextfreie Sprache},
}

Zeigen Sie, dass sich eine reguläre Sprache ebenfalls als kontextfreie
Sprache auffassen lässt.\index{Reguläre Sprache}
\index{Kontextfreie Sprache}
\footcite{theo:ab:2}

\begin{bAntwort}
Im Grunde genommen kann ein DPDA einen deterministischen endlichen
Automaten simulieren. Weil ein PDA einen Stack besitzen muss, erhält der
PDA ein Symbol $Z_0$ auf seinem Stack. Der PDA ignoriert den Stack aber
und arbeitet lediglich mit seinen Zuständen. Formal ausgedrückt, sei

\begin{displaymath}
A = (Q, \Sigma, \delta_A, q_0, F)
\end{displaymath}

\noindent
ein DFA. Wir konstruieren einen DPDA

\begin{displaymath}
P = (Q, \Sigma, \{ Z_0 \}, \delta_P, q_0, Z_0, F),
\end{displaymath}

\noindent
indem $\delta_P(q, a, Z_0) = \{(p, Z_0)\}$ für alle Zustände $p$ und $q$
aus $Q$ definiert wird, derart dass $\delta_A(q, a) = p$. Wir behaupten,
dass $(q_0, w, Z_0) \rightarrow (p, \varepsilon, Z_0)$ genau dann, wenn
$\delta_A (q_0, w) = p$. Das heißt, $P$ simuliert $A$ über seinen
Zustand. Beide Richtungen lassen sich durch einfache Induktionsbeweise
über $|w|$ zeigen. Da sowohl $A$ als auch $P$ akzeptieren, indem sie
einen der Zustände aus $F$ annehmen, schließen wir darauf, dass ihre
Sprachen identisch sind.
\end{bAntwort}

\end{document}
