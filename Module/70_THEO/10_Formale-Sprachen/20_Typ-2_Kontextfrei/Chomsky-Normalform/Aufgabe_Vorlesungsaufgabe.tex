\documentclass{bschlangaul-aufgabe}
\bLadePakete{formale-sprachen,chomsky-normalform}
\begin{document}
\bAufgabenMetadaten{
  Titel = {Kontextfreien Grammatiken in CNF},
  Thematik = {Vorlesungsaufgabe (S, SAB, SABCD)},
  Referenz = THEO.Formale-Sprachen.Typ-2_Kontextfrei.Chomsky-Normalform.Vorlesungsaufgabe,
  RelativerPfad = Module/70_THEO/10_Formale-Sprachen/20_Typ-2_Kontextfrei/Chomsky-Normalform/Aufgabe_Vorlesungsaufgabe.tex,
  ZitatSchluessel = theo:fs:2,
  ZitatBeschreibung = {Seite 37},
  BearbeitungsStand = mit Lösung,
  Korrektheit = unbekannt,
  Ueberprueft = {unbekannt},
  Stichwoerter = {Chomsky-Normalform},
}

\let\schrittE=\bChomskyUeberErklaerung

Überführen Sie die folgenden kontextfreien Grammatiken in CNF
(Chomsky-Normalform).
\index{Chomsky-Normalform}
\footcite[Seite 37]{theo:fs:2}

\begin{enumerate}

%-----------------------------------------------------------------------
% a
%-----------------------------------------------------------------------

\item

\begin{bProduktionsRegeln}
S -> 0S1 | epsilon
\end{bProduktionsRegeln}

\begin{bAntwort}
\begin{enumerate}
\item \schrittE{1}

\begin{bProduktionsRegeln}
S -> 0S1 | 01
\end{bProduktionsRegeln}
\bFlaci{Ghje1ygz9}
\item \schrittE{2}

\bNichtsZuTun

\item \schrittE{3}

N = Null, E = Eins

\begin{bProduktionsRegeln}
S -> NSE | NE,
N -> 0,
E -> 1
\end{bProduktionsRegeln}

\item \schrittE{4}

\begin{bProduktionsRegeln}
S -> NR | NE,
R -> SE,
N -> 0,
E -> 1
\end{bProduktionsRegeln}
\end{enumerate}
\end{bAntwort}

%-----------------------------------------------------------------------
% b
%-----------------------------------------------------------------------

\item

\begin{bProduktionsRegeln}
S -> a | aA | B,
A -> aBB | epsilon,
B -> Aa | b
\end{bProduktionsRegeln}
\bFlaci{G54gubr9i}

\begin{bAntwort}
\begin{enumerate}
\item \schrittE{1}

\begin{bProduktionsRegeln}
S -> a | aA | B,
A -> aBB,
B -> Aa | b | a
\end{bProduktionsRegeln}

Das leere Wort ist nicht in der Sprache ($\varepsilon \notin L(G )$). In
der Sprache sind immer Wörter mit mindestens einem Buchstaben. In der
ersten Produktionsregel wird aus $aA  \rightarrow a \varepsilon$ nur das
$a$. Das ist aber bereits in der ersten Regel enthalten. In der zweiten
Regel wird das leere Wort weg gelassen. In der dritten Regel wird noch
ein $a$ hinzugefügt, das aus $Aa \rightarrow \varepsilon a \rightarrow a$
entstanden ist.

\item \schrittE{2}

\begin{bProduktionsRegeln}
S -> a | aA | Aa | b,
A -> aBB,
B -> Aa | b | a
\end{bProduktionsRegeln}

Wir schreiben die Regel, die keine einzelnes Nonterminal auf der rechten
Seite enthalten, ab. In der ersten Regel wird $B$ mit $Aa | b | a$
ersetzte, wobei das letzte $a$, dann weggelassen werden kann, da es
bereits am Anfang der rechten Seite vorkommt. Die $B$-Regel kann nicht
weggelassen werden, weil sie in der $A$-Regel vorkommt.

\item \schrittE{3}

\begin{bProduktionsRegeln}
S -> a | VA | AV | b,
A -> VBB,
B -> AV | b | a,
V -> a
\end{bProduktionsRegeln}

\item \schrittE{4}

\begin{bProduktionsRegeln}
S -> a | VA | AV | b,
A -> VC,
B -> AV | b | a,
V -> a,
C -> BB
\end{bProduktionsRegeln}
\end{enumerate}
\end{bAntwort}

%-----------------------------------------------------------------------
% c
%-----------------------------------------------------------------------

\item % Video ab 1h26

\begin{bProduktionsRegeln}
S -> ABC,
A -> aCD,
B -> bCD,
C -> D | epsilon,
D -> C
\end{bProduktionsRegeln}
\bFlaci{Grxwcync2}

\begin{bAntwort}
\begin{enumerate}
\item \schrittE{1}

\begin{bProduktionsRegeln}
S -> ABC | AB,
A -> aCD | aD,
B -> bCD | bD,
C -> D,
D -> C | epsilon
\end{bProduktionsRegeln}

In der letzten Regel entsteht ein neues $\varepsilon$. Es muss in der
nächsten Iteration entfernt werden.

\begin{bProduktionsRegeln}
S -> ABC | AB,
A -> aCD | aD | aC | a,
B -> bCD | bD | bC | b,
C -> D,
D -> C
\end{bProduktionsRegeln}

\item \schrittE{2}

\begin{bProduktionsRegeln}
S -> AB,
A -> a,
B -> b,
\end{bProduktionsRegeln}

$C$ und $D$ sind nicht produktiv. $C \rightarrow D$ und $D \rightarrow
C$ können gestrichen werden.

\item \schrittE{3}

\bNichtsZuTun

\item \schrittE{4}

\bNichtsZuTun
\end{enumerate}
\end{bAntwort}
\end{enumerate}

\end{document}
