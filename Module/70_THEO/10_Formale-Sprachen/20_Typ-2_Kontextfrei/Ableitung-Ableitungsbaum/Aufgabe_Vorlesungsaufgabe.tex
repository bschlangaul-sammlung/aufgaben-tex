\documentclass{bschlangaul-aufgabe}
\bLadePakete{formale-sprachen,syntaxbaum}
\begin{document}
\bAufgabenMetadaten{
  Titel = {Ableitung und Parsebaum},
  Thematik = {Vorlesungsaufgabe},
  Referenz = THEO.Formale-Sprachen.Typ-2_Kontextfrei.Ableitung-Ableitungsbaum.Vorlesungsaufgabe,
  RelativerPfad = Module/70_THEO/10_Formale-Sprachen/20_Typ-2_Kontextfrei/Ableitung-Ableitungsbaum/Aufgabe_Vorlesungsaufgabe.tex,
  ZitatSchluessel = theo:fs:2,
  ZitatBeschreibung = {Seite 18},
  BearbeitungsStand = mit Lösung,
  Korrektheit = unbekannt,
  Ueberprueft = {unbekannt},
  Stichwoerter = {Kontextfreie Sprache, Ableitung (Kontextfreie Sprache), Ableitungsbaum},
}

% Info_2021-03-12-2021-03-12_13.03.18.mp4 1h43
\let\m=\bMengeOhneMathe

\begin{enumerate}

%-----------------------------------------------------------------------
%
%-----------------------------------------------------------------------

\item Erstelle eine Ableitung und einen Parsebaum für die folgende
Grammatik für das Wort
\index{Kontextfreie Sprache}
\footcite[Seite 18]{theo:fs:2}
\index{Ableitung (Kontextfreie Sprache)}
\index{Ableitungsbaum}
\bFlaci{Gf6s60uxg}

\bGrammatik{variablen={P}, alphabet={0, 1}}

\begin{bProduktionsRegeln}
S -> EPSILON | 0 | 1 | 0 S 0 | 1 S 1
\end{bProduktionsRegeln}

\begin{itemize}

%%
%
%%

\item 0000

\begin{bAntwort}
\bAbleitung{S -> 0S0 -> 00S00 -> 0000}

\begin{center}
\begin{tikzpicture}[b syntaxbaum,level distance=1cm]
\Tree [.S 0 [ 0 [.S $\varepsilon$ ] 0 ] 0 ]
\end{tikzpicture}
\end{center}
\end{bAntwort}

%%
%
%%

\item 01010

\begin{bAntwort}
\bAbleitung{S -> 0S0 -> 01S10 -> 01010}

\begin{center}
\begin{tikzpicture}[b syntaxbaum,level distance=1cm]
\Tree [.S 0 [ 1 [.S 0 ] 1 ] 0 ]
\end{tikzpicture}
\end{center}
\end{bAntwort}
\end{itemize}

%-----------------------------------------------------------------------
%
%-----------------------------------------------------------------------

\item Erstelle eine Ableitung und einen Parsebaum für die nebenstehende
Grammatik für das Wort
\bFlaci{Gpmohr81a}

$V = \m{ S, A, B }$

\bAlphabet{0, 1}

\begin{bProduktionsRegeln}
S -> A1B,
A -> 0A | epsilon,
B -> 0B | 1B | epsilon
\end{bProduktionsRegeln}

$S = S$

\begin{itemize}
\item 10101

\begin{bAntwort}
\bAbleitung{S -> A1B -> 1B -> 10B -> 101B -> 1010B -> 10101B -> 10101}

\begin{center}
\begin{tikzpicture}[b syntaxbaum,level distance=1cm]
\Tree [.S [.A $\varepsilon$ ] 1 [.B 0 [.B 1 [.B 0 [.B 1 [.B $\varepsilon$ ] ] ] ] ] ]
\end{tikzpicture}
\end{center}
\end{bAntwort}

\item 00100

\begin{bAntwort}
\bAbleitung{S -> A1B -> 0A1B -> 00A1B -> 001B -> 0010B -> 00100B -> 00100}

\begin{center}
\begin{tikzpicture}[b syntaxbaum,level distance=1cm]
\Tree [.S [.A 0 [.A 0 [.A ] $\varepsilon$ ] ] 1 [.B 0 [.B 0 [.B $\varepsilon$  ] ] ] ]
\end{tikzpicture}
\end{center}
\end{bAntwort}
\end{itemize}

\item Sind die Parsebäume eindeutig?

\begin{bAntwort}
Ja, die Parsebäume sind eindeutig.
\end{bAntwort}
\end{enumerate}

\end{document}
