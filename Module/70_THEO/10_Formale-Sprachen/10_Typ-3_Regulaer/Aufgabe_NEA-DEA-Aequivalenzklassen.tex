\documentclass{bschlangaul-aufgabe}
\bLadePakete{mathe,automaten,minimierung}
\begin{document}
\bAufgabenMetadaten{
  Titel = {NEA-DEA-Äquivalenzklassen},
  Thematik = {NEA-DEA-Aequivalenzklassen},
  Referenz = THEO.Formale-Sprachen.Typ-3_Regulaer.NEA-DEA-Aequivalenzklassen,
  RelativerPfad = Module/70_THEO/10_Formale-Sprachen/10_Typ-3_Regulaer/Aufgabe_NEA-DEA-Aequivalenzklassen.tex,
  ZitatSchluessel = theo:ab:1,
  ZitatBeschreibung = {Seite 11, Aufgabe 7},
  BearbeitungsStand = mit Lösung,
  Korrektheit = unbekannt,
  Ueberprueft = {unbekannt},
  Stichwoerter = {Reguläre Sprache, Deterministisch endlicher Automat (DEA), Minimierungsalgorithmus, Reguläre Ausdrücke, Äquivalenzklassen},
}

\let\f=\bFussnote
\let\l=\bLeereZelle
\def\Z#1#2{(#1, #2)}

Gegeben ist der deterministische endliche Automat
\bAutomat{
  zustaende={A, B, C, D, E},
  alphabet={0, 1},
  start=A,
  ende=E,
}.
\index{Reguläre Sprache}
\footcite[Seite 11, Aufgabe 7]{theo:ab:1}
\index{Deterministisch endlicher Automat (DEA)}

\begin{center}
\begin{tabular}{l||l|l}
$\delta$ & 0 & 1 \\\hline\hline
A & B & C \\\hline
B & E & C \\\hline
C & D & C \\\hline
D & E & A \\\hline
E & E & E \\\hline
\end{tabular}
\end{center}

\begin{enumerate}
\item Minimieren Sie den Automaten mit dem bekannten
Minimierungsalgorithmus. Dokumentieren Sie die Schritte geeignet.
\index{Minimierungsalgorithmus}

\begin{bAntwort}
\begin{center}
\begin{tikzpicture}[li automat]
\node[state,initial] (A) {A};
\node[state,right=of A] (C) {C};
\node[state,above=of C] (B) {B};
\node[state,below=of C] (D) {D};
\node[state,right=5cm of A,accepting] (E) {E};

\path (A) edge[above] node{0} (B);
\path (A) edge[above] node{1} (C);
\path (B) edge[above] node{0} (E);
\path (B) edge[right] node{1} (C);
\path (C) edge[right] node{0} (D);
\path (C) edge[above,loop right] node{1} (C);
\path (D) edge[above] node{1} (A);
\path (D) edge[above] node{0} (E);
\path (E) edge[above,loop right] node{0,1} (E);
\end{tikzpicture}
\bFlaci{A5amu40wc}
\end{center}

\begin{center}
\begin{tabular}{|c||c|c|c|c|c|}
\hline
A & \l  & \l  & \l  & \l  & \l \\ \hline
B & \f2 & \l  & \l  & \l  & \l \\ \hline
C &     & \f2 & \l  & \l  & \l \\ \hline
D & \f2 &     & \f2 & \l  & \l \\ \hline
E & \f1 & \f1 & \f1 & \f1 & \l \\ \hline\hline
  & A   & B   & C   & D   & E  \\ \hline
\end{tabular}
\end{center}

\bFussnoten

\begin{liUebergangsTabelle}{0}{1}
\Z A B & \Z B E \f2 & \Z C C \\
\Z A C & \Z B D     & \Z C C \\
\Z A D & \Z B E \f2 & \Z C A \\
\Z B C & \Z E D \f2 & \Z C C \\
\Z B D & \Z E E     & \Z C A \\
\Z C D & \Z D E \f2 & \Z C A \\
\end{liUebergangsTabelle}

\bPseudoUeberschrift{Minimiert}

\begin{center}
\begin{tikzpicture}[li automat]
  \node[state,initial] (A+C) at (2.14cm,-2.14cm) {A+C};
  \node[state] (B+D) at (5.29cm,-2.14cm) {B+D};
  \node[state,accepting] (E) at (8.43cm,-2.14cm) {E};

  \path (A+C) edge[auto,bend left] node{$0$} (B+D);
  \path (A+C) edge[auto,loop above] node{$1$} (A+C);
  \path (B+D) edge[auto,bend left] node{$1$} (A+C);
  \path (B+D) edge[auto] node{$0$} (E);
  \path (E) edge[auto,loop above] node{$0,1$} (E);
\end{tikzpicture}
\end{center}
\bFlaci{Ara57j4oa}
\end{bAntwort}

\item Geben Sie einen regulären Ausdruck für die erkannte Sprache an.
\index{Reguläre Ausdrücke}

\begin{bAntwort}
$r = (0|1)^*00(0|1)^*$
\end{bAntwort}

\item Geben Sie die Äquivalenzklassen der Myhill-Nerode-Äquivalenz der
Sprache durch reguläre Ausdrücke an.
\index{Äquivalenzklassen}

\begin{bAntwort}
Die Äquivalenzklassen lauten: $[A, C], [B, D], [E]$

\begin{align*}
r_A &= (1^*(01)^*)^*\\
r_B &= (1^*(01)^*)^*0\\
r_C &= r\\
\end{align*}
\end{bAntwort}
\end{enumerate}

\end{document}
