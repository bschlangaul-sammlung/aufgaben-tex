\documentclass{bschlangaul-aufgabe}
\bLadePakete{formale-sprachen,pumping-lemma}
\begin{document}
\bAufgabenMetadaten{
  Titel = {Pumping-Lemma},
  Thematik = {„w w“},
  Referenz = THEO.Formale-Sprachen.Typ-3_Regulaer.Pumping-Lemma.Koblenz,
  RelativerPfad = Module/70_THEO/10_Formale-Sprachen/10_Typ-3_Regulaer/Pumping-Lemma/Aufgabe_Koblenz.tex,
  BearbeitungsStand = mit Lösung,
  Korrektheit = unbekannt,
  Ueberprueft = {unbekannt},
  Stichwoerter = {Pumping-Lemma (Reguläre Sprache)},
}

\let\m=\bMenge

Zeigen oder widerlegen Sie: Die folgenden Sprachen über dem Alphabet
\bAlphabet{a, b, c} sind regulär.\index{Pumping-Lemma (Reguläre Sprache)}
\footnote{\url{https://userpages.uni-koblenz.de/~dpeuter/teaching/17ss_gti/blatt04_loesung.pdf}}

\begin{bExkurs}[Pumping-Lemma für Reguläre Sprachen]
\bPumpingRegulaer
\end{bExkurs}

\begin{displaymath}
L_1 = \m{w w | w \in \m{a, b}^* }
\end{displaymath}

\begin{bAntwort}
\noindent
Angenommen $L_1$ sei regulär, dann müsste $L_1$ die Bedingungen der
stärkeren Variante des Pumping-Lemmas erfüllen.

\bPseudoUeberschrift{Beweis durch Widerspruch:}

\noindent
Sei $j \in \mathbb{N}$ die Konstante aus dem Pumping-Lemma und $\omega =
a^j b a^j b$ ein Wort aus $L_1$ ($|\omega| > j$ gilt offensichtlich).

Dann müsste $\omega$ nach dem Pumping-Lemma zerlegbar sein in $\omega =
uvw$ mit $|v| \geq 1$ und $|uv| < j$.
%
$uv$ kann wegen $|uv| < j$ kein $b$ enthalten und liegt komplett im
ersten $a^j$.

\noindent
Also:

\begin{displaymath}
a^j b a^j b = uvw \text{ mit }
u = a^x \text{, }
v = a^y \text{, }
w = a^{n - x - y} b a^j b (n \geq x + y, x > 0)
\end{displaymath}

\noindent
Dann gilt

\begin{displaymath}
u v^0 w = a^x a^{j - x - y} b a^j b = a^{j - y} b a^j b \notin L_1
\end{displaymath}

\noindent
Wir haben gezeigt, dass es keine gültige Zerlegung für $\omega$ gibt.
Also gilt für $L_1$ die stärkere Variante des Pumping-Lemmas nicht.
Somit kann $L_1$ nicht regulär sein.
\end{bAntwort}
\end{document}
