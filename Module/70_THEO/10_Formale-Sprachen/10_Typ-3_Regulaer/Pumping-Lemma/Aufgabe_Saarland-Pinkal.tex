\documentclass{bschlangaul-aufgabe}
\bLadePakete{formale-sprachen,pumping-lemma}
\begin{document}
\bAufgabenMetadaten{
  Titel = {Pumping-Lemma},
  Thematik = {w c wR},
  Referenz = THEO.Formale-Sprachen.Typ-3_Regulaer.Pumping-Lemma.Saarland-Pinkal,
  RelativerPfad = Module/70_THEO/10_Formale-Sprachen/10_Typ-3_Regulaer/Pumping-Lemma/Aufgabe_Saarland-Pinkal.tex,
  BearbeitungsStand = mit Lösung,
  Korrektheit = unbekannt,
  Ueberprueft = {unbekannt},
  Stichwoerter = {Pumping-Lemma (Reguläre Sprache)},
}

\index{Pumping-Lemma (Reguläre Sprache)}
\bFussnoteUrl{http://www.coli.uni-saarland.de/courses/I2CL-10/material/Uebungsblaetter/Musterloesung4.4.pdf}

\begin{bExkurs}[Pumping-Lemma für Reguläre Sprachen]
\bPumpingRegulaer
\end{bExkurs}

\begin{displaymath}
L_1 = \bMenge{ w c w^R | w \in \bMenge{a, b}^* }
\end{displaymath}

\noindent
Erläuterung: $w^R$ ist die Spiegelung von $w$, \dh es enthält die
Zeichen von $w$ in umgekehrter Reihenfolge. Worte von $L_1$ sind also
\zB $c$, $abcba$, $bbbaabacabaabbb$

\begin{bAntwort}
$L_1$ ist kontexfrei.

\bPseudoUeberschrift{Beweis, dass $L_1$ nicht regulär ist, durch das
Pumping Lemma: }

\noindent
Wir nehmen an $L_1$ wäre regulär. Dann gibt es einen endlichen
Automaten, der $L_1$ erkennt. Die Anzahl der Zustände dieses Automaten
sei $j$. Wir wählen jetzt das Wort $\omega = a^j c a^j$. $\omega$ liegt
in $L_1$, und ist offensichtlich länger als $j$. Dieses Wort muss
irgendwo eine Schleife, also einen aufpumpbaren Teil enthalten, \dh
man kann es so in $uvw$ zerlegen, dass für jede natürliche Zahl $i$ auch
$uv^iw$ zu $L_1$ gehört. Wo könnte dieser aufpumpbare Teil liegen?

\begin{description}
\item[Fall 1:]

Der aufpumbare Teil $v$ liegt komplett im Bereich des ersten
$a^j$-Blocks. Dann würde aber $uv^2w = a^{j + |v|} c a^j$ mehr $a$’s im
ersten Teil als im zweiten Teil enthalten und läge nicht mehr in $L_1$.

\item[Fall 2:]

$v$ enthält das $c$. Dann würde aber $u v^2 w$ zwei $c$’s enthalten und
läge damit nicht mehr in $L_1$.

\item[Fall 3:]

Der aufpumpbare Teil liegt komplett im Bereich des zweiten $a^j$-Blocks.
Dann liegt analog zu Fall 1 $u v^2 w$ nicht mehr in $L_1$. Unser Wort
lässt sich also nicht so zerlegen, dass man den Mittelteil aufpumpen
kann, also ist die Annahme, dass $L_1$ regulär ist, falsch.
\end{description}
\noindent
Beweis, dass $L_1$ kontextfrei ist, durch Angabe einer kontexfreien
Grammatik:

\noindent
\begin{bProduktionsRegeln}
S -> a S a,
S -> b S b,
S -> c
\end{bProduktionsRegeln}
\end{bAntwort}

\end{document}
