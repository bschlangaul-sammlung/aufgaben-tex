\documentclass{bschlangaul-aufgabe}
\bLadePakete{automaten,formale-sprachen}
\begin{document}
\bAufgabenMetadaten{
  Titel = {Grammatik aus Automat},
  Thematik = {Grammatik aus Automat},
  Referenz = THEO.Formale-Sprachen.Typ-3_Regulaer.Grammatik-aus-Automat,
  RelativerPfad = Module/70_THEO/10_Formale-Sprachen/10_Typ-3_Regulaer/Aufgabe_Grammatik-aus-Automat.tex,
  ZitatSchluessel = theo:ab:1,
  ZitatBeschreibung = {Seite 4, Aufgabe 3},
  BearbeitungsStand = mit Lösung,
  Korrektheit = unbekannt,
  Ueberprueft = {unbekannt},
  Stichwoerter = {Reguläre Sprache, Deterministisch endlicher Automat (DEA), Reguläre Grammatik},
}

Sei \bAutomat{zustaende={z_0, z_1, z_2}, alphabet={a, b}, ende={z_2}} ein
endlicher Automat. Die Übergangsfunktion sei wie in dem unten
abgebildeten Diagramm definiert.
\index{Reguläre Sprache}
\footcite[Seite 4, Aufgabe 3]{theo:ab:1}
\index{Deterministisch endlicher Automat (DEA)}

\begin{center}
\begin{tikzpicture}[li automat]
  \node[state,initial] (z0) at (2.43cm,-5.14cm) {$z_0$};
  \node[state] (z1) at (6.29cm,-1.71cm) {$z_1$};
  \node[state,accepting] (z2) at (8.43cm,-5.57cm) {$z_2$};

  \path (z0) edge[auto] node{b} (z1);
  \path (z0) edge[auto,bend left] node{a} (z2);
  \path (z1) edge[auto,bend left] node{b} (z2);
  \path (z1) edge[auto,loop] node{a} (z1);
  \path (z2) edge[auto,bend left] node{b} (z0);
  \path (z2) edge[auto,bend left] node{a} (z1);
\end{tikzpicture}
\end{center}
\bFlaci{Apk0iyqyg}

\begin{enumerate}

%%
%
%%

\item Gebe eine reguläre Grammatik $G$ an, sodass $L(G) = L(M)$ gilt.
\index{Reguläre Grammatik}

\begin{bAntwort}
\bGrammatik{variablen={Z_0, Z_1, Z_2}, alphabet={a, b}, start=Z_0} mit
folgender Produktionsmenge

\begin{bProduktionsRegeln}
Z_0 -> b Z_1 | a Z_2,
Z_1 -> a Z_1 | b Z_2,
Z_2 -> b Z_0 | a Z_1 | EPSILON
\end{bProduktionsRegeln}
\end{bAntwort}

%%
%
%%

\item Überlegen Sie sich ein systematisches Verfahren, um einen
deterministischen endlichen Automaten in eine reguläre Grammatik
umzuwandeln.

\begin{bAntwort}
Analog zu obigem Beispiel folgender Algorithmus benutzt werden:

\begin{itemize}
\item Setze $V = \bMenge{Z_0, Z_1, \dots Z_n}$ und $S$ auf den
Startzustand des Automaten.

\item Für jeden Übergang $\delta(Z_i, a) = Z_j$ füge die Produktion
\bProduktionen{Z_i -> a Z_j} zu $P$ hinzu.

\item Für jeden Zustand $Z_i \in Z$ füge die Produktion
\bProduktionen{Z_i -> EPSILON} zu $P$ dazu.
\end{itemize}
\end{bAntwort}
\end{enumerate}

\end{document}
