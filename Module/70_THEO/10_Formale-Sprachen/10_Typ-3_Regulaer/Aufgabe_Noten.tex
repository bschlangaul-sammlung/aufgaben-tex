\documentclass{bschlangaul-aufgabe}
\bLadePakete{musik-symbole}

\begin{document}
\bAufgabenMetadaten{
  Titel = {Noten},
  Thematik = {Noten},
  Referenz = THEO.Formale-Sprachen.Typ-3_Regulaer.Noten,
  RelativerPfad = Module/70_THEO/10_Formale-Sprachen/10_Typ-3_Regulaer/Aufgabe_Noten.tex,
  ZitatSchluessel = theo:ab:1,
  BearbeitungsStand = mit Lösung,
  Korrektheit = unbekannt,
  Ueberprueft = {unbekannt},
  Stichwoerter = {Reguläre Sprache},
}

Der Viervierteltakt ist vor allem in der Unterhaltungsmusik die weitaus
häufigste Taktart. Viervierteltakt bedeutet, dass ein Takt eine Länge
von vier Vierteln hat. Das Symbol \quarterNote{} bedeutet eine
Viertelnote. Eine halbe Note \halfNote{} hat den Wert von zwei Vierteln,
eine punktierte halbe Note \halfNoteDotted{} den Wert von drei Vierteln
und eine ganze Note \wholeNote{} den Wert von vier Vierteln. Andere
Notenwerte sollen hier nicht vorkommen.\index{Reguläre Sprache}
\footcite{theo:ab:1}

Die Prüfsoftware eines Notenverlages stellt eine Methode bereit, die testet, ob die
Notenwerte eines Taktes tatsächlich vier Viertel ergeben.

Durch das Alphabet

\begin{center}
$\Sigma =$ \{ \wholeNote, \halfNoteDotted, \halfNote, \quarterNote\ \},
\end{center}

die Menge der Nichtterminale

\begin{center}
V = \{ <Takt>; <3/4>; <1/2> \} ,
\end{center}

das Startsymbol <Takt> sowie die Produktionsregeln

\begin{center}
$R_1$: <Takt> $\rightarrow$ \wholeNote{} | <3/4> \quarterNote{}

$R_2$: <3/4> $\rightarrow$ \halfNoteDotted{} | <1/2> \quarterNote{}

$R_3$: <1/2> $\rightarrow$ \halfNote{} | \quarterNote{} \quarterNote{}
\end{center}

ist eine Grammatik für eine formale Sprache $S$ gegeben.

\begin{enumerate}

%%
% (a)
%%

\item Entscheide begründet, ob die Zeichenketten
\quarterNote\quarterNote\quarterNote\quarterNote\ und
\quarterNote\halfNote\quarterNote{} zu $S$ gehören. Gebe eine weitere
Zeichenkette an, die zwar einen Viervierteltakt darstellt, aber nicht zu
$S$ gehört.

%%
% (b)
%%

\item Ergänze die oben angegebenen Produktionsregeln so, dass jeder mit
dem Alphabet $\Sigma$ mögliche Viervierteltakt dargestellt werden kann.

%%
% (c)
%%

\item Mit $T_4$ wird die formale Sprache bezeichnet, die genau alle
Viervierteltakte enthält, die mit den Zeichen aus $\Sigma$ gebildet werden
können. Verwende für die weiteren Teilaufgaben anstelle der Notensymbole
Buchstaben gemäß folgender Tabelle:

Zeichne das Zustandsübergangsdiagramm eines erkennenden endlichen
Automaten, der genau $T_4$ akzeptiert.

%%
% (d)
%%

\item Entwerfe eine Implementierung des Automaten aus Teilaufgabe c in
Java. Dabei soll es u. a. eine Methode istViervierteltakt(eingabe)
geben, die überprüft, ob die übergebene Zeichenkette eingabe den
Vorgaben für einen Viervierteltakt entspricht, und einen entsprechenden
Wahrheitswert zurückgibt. Dazu ruft sie für jedes Zeichen der Eingabe
jeweils die Methode zustandWechseln() auf.
\end{enumerate}
\end{document}
