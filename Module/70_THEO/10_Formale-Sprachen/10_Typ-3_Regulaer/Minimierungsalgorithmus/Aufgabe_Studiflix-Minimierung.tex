\documentclass{bschlangaul-aufgabe}
\bLadePakete{automaten,minimierung,formale-sprachen}

\begin{document}
\bAufgabenMetadaten{
  Titel = {Minimierungsalgorithmus},
  Thematik = {Studiflix-Minimierung},
  Referenz = THEO.Formale-Sprachen.Typ-3_Regulaer.Minimierungsalgorithmus.Studiflix-Minimierung,
  RelativerPfad = Module/70_THEO/10_Formale-Sprachen/10_Typ-3_Regulaer/Minimierungsalgorithmus/Aufgabe_Studiflix-Minimierung.tex,
  BearbeitungsStand = mit Lösung,
  Korrektheit = unbekannt,
  Ueberprueft = {unbekannt},
  Stichwoerter = {Minimierungsalgorithmus},
}

\let\z=\bZustandsnameTiefgestellt
\let\f=\bFussnote
\let\l=\bLeereZelle
\let\Z=\bZustandsPaar

\begin{center}
\begin{tikzpicture}[->,node distance=1.5cm]
\node[state,initial] (0) {$z_0$};
\node[state,above=of 0] (1) {$z_1$};
\node[state,right=of 0,accepting] (2) {$z_2$};
\node[state,right=of 1,accepting] (3) {$z_3$};
\node[state,right=of 2,accepting] (4) {$z_4$};
\node[state,right=of 3] (5) {$z_5$};

\path (0) edge[above] node{1} (2);
\path (0) edge[left,bend left] node{0} (1);
\path (1) edge[above] node{1} (3);
\path (1) edge[right,bend left] node{0} (0);
\path (2) edge[above] node{0} (4);
\path (2) edge[above,pos=0.2] node{1} (5);
\path (3) edge[above,pos=0.2] node{0} (4);
\path (3) edge[above] node{1} (5);
\path (4) edge[right] node{1} (5);
\path (5) edge[right,loop right] node{0,1} (5);
\path (4) edge[right,loop right] node{0} (4);
\end{tikzpicture}
\end{center}

\begin{bAntwort}
\begin{center}
\begin{tabular}{|c||c|c|c|c|c|c|}
\hline
\z0 & \l  & \l  & \l  & \l  & \l  & \l  \\ \hline
\z1 &     & \l  & \l  & \l  & \l  & \l  \\ \hline
\z2 & \f1 & \f1 & \l  & \l  & \l  & \l  \\ \hline
\z3 & \f1 & \f1 &     & \l  & \l  & \l  \\ \hline
\z4 & \f1 & \f1 &     &     & \l  & \l  \\ \hline
\z5 & \f2 & \f2 & \f1 & \f1 & \f1 & \l  \\ \hline\hline
    & \z0 & \z1 & \z2 & \z3 & \z4 & \z5 \\ \hline
\end{tabular}
\end{center}
\index{Minimierungsalgorithmus}

\bFussnoten

\begin{liUebergangsTabelle}{0}{1}
\Z01 & \Z10 & \Z23     \\
\Z05 & \Z15 & \Z25 \f2 \\
\Z15 & \Z05 & \Z35 \f2 \\
\Z23 & \Z44 & \Z55     \\
\Z24 & \Z44 & \Z55     \\
\Z34 & \Z44 & \Z55     \\
\end{liUebergangsTabelle}

\noindent
\Z23, \Z24 und \Z34 können zu einem Zustand verschmolzen werden, weil
sie alle drei bei der Eingabe von 0 zu \Z44 und bei 1 zu \Z55 werden.
%
\z5 kann nicht verschmolzen werden, weil er in der Tabelle markiert ist.

\begin{center}
\begin{tikzpicture}[->,node distance=1.5cm]
\node[state,initial] (01) {$z_{01}$};
\node[state,right=of 01] (234) {$z_{234}$};
\node[state,right=of 234,accepting] (5) {$z_5$};

\path (01) edge[above] node{1} (234);
\path (234) edge[above] node{1} (5);
\path (01) edge[above,loop] node{0} (01);
\path (234) edge[above,loop] node{0} (234);
\path (5) edge[above,loop] node{0,1} (5);
\end{tikzpicture}
\end{center}
\end{bAntwort}

\bFussnoteUrl{https://studyflix.de/informatik/dea-minimieren-1212}

\end{document}
