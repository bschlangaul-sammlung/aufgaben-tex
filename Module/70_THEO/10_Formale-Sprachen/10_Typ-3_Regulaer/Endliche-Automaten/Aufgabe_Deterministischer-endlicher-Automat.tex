\documentclass{bschlangaul-aufgabe}
\bLadePakete{automaten,formale-sprachen}
\begin{document}
\bAufgabenMetadaten{
  Titel = {Deterministischer endlicher Automat},
  Thematik = {Deterministischer endlicher Automat},
  Referenz = THEO.Formale-Sprachen.Typ-3_Regulaer.Endliche-Automaten.Deterministischer-endlicher-Automat,
  RelativerPfad = Module/70_THEO/10_Formale-Sprachen/10_Typ-3_Regulaer/Endliche-Automaten/Aufgabe_Deterministischer-endlicher-Automat.tex,
  ZitatSchluessel = theo:ab:1,
  BearbeitungsStand = mit Lösung,
  Korrektheit = unbekannt,
  Ueberprueft = {unbekannt},
  Stichwoerter = {Reguläre Sprache, Deterministisch endlicher Automat (DEA)},
}

Geben Sie einen DFA über dem Alphabet \bAlphabet{0, 1} an, der die
folgende Sprache erkennt: Die Menge aller Zeichenketten, die eine durch
4 teilbare Anzahl von Einsen besitzt.
\index{Reguläre Sprache}
\footcite{theo:ab:1}
\index{Deterministisch endlicher Automat (DEA)}

\begin{bAntwort}
\begin{center}
\begin{tikzpicture}[li automat]
\node[state,initial,accepting] (0) {$z_0$};
\node[state,right of=0] (1) {$z_1$};
\node[state,right of=1] (2) {$z_2$};
\node[state,right of=2] (3) {$z_3$};

\path (0) edge[above] node{1} (1);
\path (1) edge[above] node{1} (2);
\path (2) edge[above] node{1} (3);
\path (3) edge[bend left,above] node{1} (0);

\path (0) edge[loop,above] node{0} (0);
\path (1) edge[loop,above] node{0} (1);
\path (2) edge[loop,above] node{0} (2);
\path (3) edge[loop,above] node{0} (3);
\end{tikzpicture}
\end{center}
\end{bAntwort}

\end{document}
