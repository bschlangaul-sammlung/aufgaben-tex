\documentclass{bschlangaul-aufgabe}
\bLadePakete{mathe,automaten,formale-sprachen}

\begin{document}
\bAufgabenMetadaten{
  Titel = {Potenzmengenalgorithmus},
  Thematik = {Vorlesungsaufgaben},
  Referenz = THEO.Formale-Sprachen.Typ-3_Regulaer.Potenzmengenalgorithmus.Vorlesungsaufgaben-Potenzmengenalgorithmus-erstes-Beispiel,
  RelativerPfad = Module/70_THEO/10_Formale-Sprachen/10_Typ-3_Regulaer/Potenzmengenalgorithmus/Aufgabe_Vorlesungsaufgaben-Potenzmengenalgorithmus-erstes-Beispiel.tex,
  ZitatSchluessel = theo:fs:1,
  ZitatBeschreibung = {Seite 35-45},
  BearbeitungsStand = mit Lösung,
  Korrektheit = unbekannt,
  Ueberprueft = {unbekannt},
  Stichwoerter = {Potenzmengenalgorithmus},
}

% Info_2021-02-26-2021-02-26_13.01.25.mp4 2h23- 2h38

Gegeben ist der folgende NEA:
\footcite[Seite 35-45]{theo:fs:1}
\index{Potenzmengenalgorithmus}

\begin{center}
\begin{tikzpicture}[->,node distance=2cm]
\node[state,initial] (0) {$z_0$};
\node[state,right of=0] (1) {$z_1$};
\node[state,right of=1,accepting] (2) {$z_2$};

\path (0) edge[above] node{b} (1);
\path (1) edge[above] node{a,b} (2);
\path (0) edge[loop,above] node{a,b} (0);
\path (2) edge[loop,above] node{a,b} (2);
\end{tikzpicture}
\end{center}

\begin{enumerate}

%%
%
%%

\item Welche Sprache akzeptiert dieser Automat? Beschreiben Sie in
Worten und stellen Sie einen regulären Ausdruck hierfür auf.
\footcite[Seite 35]{theo:fs:1}

\begin{bAntwort}
$(a|b)^*b(a|b)(a|b)^*$
\end{bAntwort}

%%
%
%%

\item Überführe den gegebenen NEA mit dem Potenzmengenalgorithmus in
einen DEA.

\begin{bAntwort}
\let\p=\bPotenzmenge

\begin{tabular}{l|l|l|l}
Name & Zustandsmenge & Eingabe a & Eingabe b \\\hline
$z'_0$ &
\p{z_0} &
\p{z_0} &
\p{z_0, z_1} \\

$z'_1$ &
\p{z_0, z_1} &
\p{z_0, z_2} &
\p{z_0, z_1, z_2} \\

$z'_2$ &
\p{z_0, z_2} &
\p{z_0, z_2} &
\p{z_0, z_1, z_2} \\

$z'_3$ &
\p{z_0, z_1, z_2} &
\p{z_0, z_2} &
\p{z_0, z_1, z_2} \\
\end{tabular}

\begin{center}
\begin{tikzpicture}[->,node distance=2cm]
\node[state,initial,label=270:\p{z_0}] (0) {$z'_0$};
\node[state,right of=0,label=270:\p{z_0, z_1}] (1) {$z'_1$};
\node[state,accepting,above right of=1,label=90:\p{z_0, z_2}] (2) {$z'_2$};
\node[state,accepting,below right of=2,label=270:\p{z_0, z_1, z_2}] (3) {$z'_3$};

\path (0) edge[above] node{b} (1);
\path (1) edge[above] node{a} (2);
\path (1) edge[above] node{b} (3);
\path (2) edge[right, bend left] node{b} (3);
\path (3) edge[above] node{a} (2);

\path (0) edge[above,loop] node{a} (0);
\path (2) edge[above,loop right] node{a} (2);
\path (3) edge[above,loop right] node{b} (3);
\end{tikzpicture}
\end{center}
\end{bAntwort}

\end{enumerate}

\end{document}
