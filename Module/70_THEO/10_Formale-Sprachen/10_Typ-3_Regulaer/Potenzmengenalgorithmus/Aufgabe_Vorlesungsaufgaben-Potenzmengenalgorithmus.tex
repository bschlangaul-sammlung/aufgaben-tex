\documentclass{bschlangaul-aufgabe}
\bLadePakete{mathe,automaten,formale-sprachen}

\begin{document}
\bAufgabenMetadaten{
  Titel = {Potenzmengenalgorithmus},
  Thematik = {Vorlesungsaufgaben},
  Referenz = THEO.Formale-Sprachen.Typ-3_Regulaer.Potenzmengenalgorithmus.Vorlesungsaufgaben-Potenzmengenalgorithmus,
  RelativerPfad = Module/70_THEO/10_Formale-Sprachen/10_Typ-3_Regulaer/Potenzmengenalgorithmus/Aufgabe_Vorlesungsaufgaben-Potenzmengenalgorithmus.tex,
  ZitatSchluessel = theo:fs:1,
  ZitatBeschreibung = {Seite 46},
  BearbeitungsStand = mit Lösung,
  Korrektheit = unbekannt,
  Ueberprueft = {unbekannt},
  Stichwoerter = {Potenzmengenalgorithmus},
}

% Info_2021-02-26-2021-02-26_13.01.25.mp4 2h:40 - 2h:

Gegeben ist der folgende nichtdeterministische endliche Automat:
\index{Potenzmengenalgorithmus}
\footcite[Seite 46]{theo:fs:1}

\begin{center}
\begin{tikzpicture}[->,node distance=2cm]
\node[state,initial] (0) {$z_0$};
\node[state,right of=0] (2) {$z_2$};
\node[state,above of=2] (1) {$z_1$};
\node[state,right of=2,accepting] (3) {$z_3$};

\path (0) edge[above] node{a} (1);
\path (0) edge[above] node{a} (2);
\path (2) edge[right] node{b} (1);
\path (2) edge[above] node{b} (3);
\path (3) edge[loop,above] node{a,b} (3);
\end{tikzpicture}
\end{center}

\noindent
Überführenn Sie den gegebenen nichtdeterministischen endlichen Automaten
mit dem Potenzmengenalgorithmus in einen deterministischen endlichen
Automaten.

\begin{bAntwort}
\let\p=\bPotenzmenge

\begin{tabular}{l|l|l|l}
neuer Name & Zustandsmenge & Eingabe a & Eingabe b \\\hline\hline
$z'_0$ &
\p{z_0} &
\p{z_1, z_2} &
\p{} \\

$z'_1$ &
\p{z_1, z_2} &
\p{}&
\p{z_1, z_3} \\

$z'_2$ &
\p{z_1, z_3} &
\p{z_3} &
\p{z_3} \\

$z'_3$ &
\p{z_3} &
\p{z_3} &
\p{z_3} \\

$z'_t$ &
\p{} &
\p{} &
\p{} \\
\end{tabular}

\begin{center}
\begin{tikzpicture}[->,node distance=2cm]
\node[state,initial,label=270:\p{z_0}] (0) {$z'_0$};
\node[state,right of=0,label=270:\p{z_1, z_2}] (1) {$z'_1$};
\node[state,right of=1,accepting,label=270:\p{z_1, z_3}] (2) {$z'_2$};
\node[state,right of=2,accepting,label=270:\p{z_3}] (3) {$z'_3$};
\node[state,above right of=0,label=90:\p{}] (t) {$z'_t$};

\path (0) edge[above] node{a} (1);
\path (0) edge[left] node{b} (t);
\path (1) edge[right] node{a} (t);
\path (1) edge[above] node{b} (2);
\path (2) edge[above] node{a,b} (3);
\path (3) edge[above,loop right] node{a,b} (3);
\path (t) edge[above,loop right] node{a,b} (t);

\end{tikzpicture}
\end{center}
\end{bAntwort}

\end{document}
