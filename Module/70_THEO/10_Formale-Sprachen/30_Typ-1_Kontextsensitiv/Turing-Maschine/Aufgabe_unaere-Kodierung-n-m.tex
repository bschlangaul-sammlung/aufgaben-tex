\documentclass{bschlangaul-aufgabe}
\bLadePakete{mathe,automaten,formale-sprachen}
\begin{document}
\bAufgabenMetadaten{
  Titel = {Prädikat unäre Kodierung von n und m},
  Thematik = {unäre Kodierung von n und m},
  Referenz = THEO.Formale-Sprachen.Typ-1_Kontextsensitiv.Turing-Maschine.unaere-Kodierung-n-m,
  RelativerPfad = Module/70_THEO/10_Formale-Sprachen/30_Typ-1_Kontextsensitiv/Turing-Maschine/Aufgabe_unaere-Kodierung-n-m.tex,
  ZitatSchluessel = theo:ab:3,
  ZitatBeschreibung = {Aufgabe 4: TM: Vergleicher},
  BearbeitungsStand = mit Lösung,
  Korrektheit = unbekannt,
  Ueberprueft = {unbekannt},
  Stichwoerter = {Turing-Maschine},
}

Konstruieren Sie eine Einband-Turingmaschine, die für eine Eingabe der
Form $|^n \circ |^m$ mit $n, m \in \mathbb{N}$ (also die unäre Kodierung
von $n$ und $m$ durch $\circ$ getrennt) das Prädikat $n \leq m$
berechnet.\index{Turing-Maschine}
\footcite[Aufgabe 4: TM: Vergleicher]{theo:ab:3}

\begin{equation*}
f(x) =
\begin{cases}
1 \text{ (akzeptiert)} & n \leq m \\
0 \text{ (nicht akzeptiert)} & \text{sonst}
\end{cases}
\end{equation*}

\begin{bAntwort}
\begin{center}
\begin{tikzpicture}[li turingmaschine]
  \node[state,initial] (z0) at (2.29cm,-3cm) {$z_0$};
  \node[state] (z1) at (5.57cm,-2cm) {$z_1$};
  \node[state] (z2) at (9.29cm,-1.86cm) {$z_2$};
  \node[state] (z3) at (7.57cm,-3.71cm) {$z_3$};
  \node[state] (z4) at (4cm,-5.57cm) {$z_4$};
  \node[state] (z5) at (5.29cm,-8.57cm) {$z_5$};
  \node[state] (z7) at (12.14cm,-1.57cm) {$z_7$};
  \node[state] (z6) at (8.86cm,-6.29cm) {$z_6$};
  \node[state,accepting] (z8) at (12cm,-6.43cm) {$z_8$};

  \bTuringKante[above]{z0}{z1}{
    |, LEER, R;
  }

  \bTuringKante[above]{z0}{z4}{
    °, LEER, R;
  }

  \bTuringKante[above]{z1}{z2}{
    LEER, LEER, L;
  }

  \bTuringKante[above,loop above]{z1}{z1}{
    |, |, R;
    °, °, R;
  }

  \bTuringKante[above]{z2}{z3}{
    |, LEER, L;
  }

  \bTuringKante[above]{z2}{z7}{
    °, LEER, L;
  }

  \bTuringKante[above]{z3}{z0}{
    LEER, LEER, R;
  }

  \bTuringKante[above,loop above]{z3}{z3}{
    |, |, L;
    °, °, L;
  }

  \bTuringKante[above]{z4}{z6}{
    LEER, LEER, N;
  }

  \bTuringKante[above]{z4}{z5}{
    |, LEER, R;
  }

  \bTuringKante[above,loop above]{z5}{z5}{
    |, LEER, R;
  }

  \bTuringKante[above]{z5}{z6}{
    LEER, LEER, N;
  }

  \bTuringKante[above,loop above]{z7}{z7}{
    |, LEER, N;
  }

  \bTuringKante[above]{z7}{z8}{
    LEER, 0, R;
  }

  \bTuringKante[above]{z6}{z8}{
    LEER, 1, R;
  }
\end{tikzpicture}
\end{center}
\bFlaci{Ajf6mi7e7}
\end{bAntwort}

\end{document}
