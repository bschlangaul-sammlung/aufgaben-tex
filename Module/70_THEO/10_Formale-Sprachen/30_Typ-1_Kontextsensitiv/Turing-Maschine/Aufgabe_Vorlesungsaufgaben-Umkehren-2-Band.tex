\documentclass{bschlangaul-aufgabe}
\bLadePakete{formale-sprachen,mathe,syntax}
\begin{document}
\bAufgabenMetadaten{
  Titel = {Wörter umkehren auf der 2-Band-Turingmaschine},
  Thematik = {Vorlesungsaufgaben ab-Wörter umkehren 2-Band-Turingmaschine},
  Referenz = THEO.Formale-Sprachen.Typ-1_Kontextsensitiv.Turing-Maschine.Vorlesungsaufgaben-Umkehren-2-Band,
  RelativerPfad = Module/70_THEO/10_Formale-Sprachen/30_Typ-1_Kontextsensitiv/Turing-Maschine/Aufgabe_Vorlesungsaufgaben-Umkehren-2-Band.tex,
  ZitatSchluessel = theo:fs:3,
  ZitatBeschreibung = {Seite 27},
  BearbeitungsStand = mit Lösung,
  Korrektheit = unbekannt,
  Ueberprueft = {unbekannt},
  Stichwoerter = {Turing-Maschine},
}

\def\z#1{\,z_#1\,}
\def\p{&\rightarrow}
\def\l{\,\bTuringLeerzeichen\,}

% Info_2021-04-23-2021-04-23_13.17.40.mp4 3h20min

\begin{enumerate}
\item Geben Sie eine 2-Band-Turingmaschine an, die die Eingabe über dem
Alphabet \bAlphabet{a, b} umkehrt.\index{Turing-Maschine}
\footcite[Seite 27]{theo:fs:3}

Beispiele:

\begin{itemize}
\item $abb \rightarrow bba$
\item $aaaaba \rightarrow abaaaa$
\item $aaa \rightarrow aaa$
\end{itemize}

Tipp:
Das Ergebniswort muss nicht auf dem 1. Band stehen.
\end{enumerate}

\begin{bAntwort}
\begin{minted}{md}
name: ab-Wörter umkehren 2-Band-Turingmaschine
init: z0
accept: z0

z0,a,_
z0,_,a,>,<

z0,b,_
z0,_,b,>,<
\end{minted}
\bFussnoteUrl{http://turingmachinesimulator.com/shared/iqwxphngwc}
\end{bAntwort}

\end{document}
