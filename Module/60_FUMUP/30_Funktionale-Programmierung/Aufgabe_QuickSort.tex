\documentclass{bschlangaul-aufgabe}
\bLadePakete{syntax}
\begin{document}
\bAufgabenMetadaten{
  Titel = {Aufgabe},
  Thematik = {QuickSort},
  Referenz = FUMUP.Funktionale-Programmierung.QuickSort,
  RelativerPfad = Module/60_FUMUP/30_Funktionale-Programmierung/Aufgabe_QuickSort.tex,
  ZitatSchluessel = fumup:ab:3,
  BearbeitungsStand = mit Lösung,
  Korrektheit = unbekannt,
  Ueberprueft = {unbekannt},
  Stichwoerter = {Funktionale Programmierung mit Haskell},
}

Implementiere ebenfalls in der Datei Sortierverfahren.hs die Funktion
quickSort, die den Quicksort-Algorithmus umsetzt. Sie erhält eine
unsortierte Liste mit Elementen (auf denen eine Ordnung definiert ist),
sortiert diese in aufsteigender Reihenfolge und gibt die sortierte Liste
zurück. Die Signatur der Sortierfunktion lautet:
\index{Funktionale Programmierung mit Haskell}
\footcite{fumup:ab:3}

\bHaskellCode{quickSort :: (Ord a) => [a] -> [a]}

Falls eine leere Liste übergeben wird, ist das Ergebnis die leere Liste.
Ansonsten wird das erste Element der Liste als Pivot-Element gewählt. Es
soll nun für zwei Teillisten (die mittels Listengenerator erzeugt werden
sollen und eine Liste alle Elemente kleiner gleich dem Pivot-Element und
die andere Liste alle Elemente größer dem Pivot-Element enthält) die
Funktion \bHaskellCode{quickSort} (Rekursion!) aufgerufen werden. Dabei
sollen die beiden Teillisten mit dem Pivot-Element in der Mitte
konkateniert werden. (Tipp: Diese Funktion kann (ausgenommen von der
Signatur und der Abbruchbedingung der Rekursion) als Einzeiler
programmiert werden.)

\end{document}
