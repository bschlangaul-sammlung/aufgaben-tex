\documentclass{bschlangaul-aufgabe}
\bLadePakete{syntax}
\begin{document}
\bAufgabenMetadaten{
  Titel = {Listen},
  Thematik = {Seven-of-Nine},
  Referenz = FUMUP.Funktionale-Programmierung.Seven-of-Nine,
  RelativerPfad = Module/60_FUMUP/30_Funktionale-Programmierung/Aufgabe_Seven-of-Nine.tex,
  ZitatSchluessel = fumup:ab:3,
  BearbeitungsStand = mit Lösung,
  Korrektheit = unbekannt,
  Ueberprueft = {unbekannt},
  Stichwoerter = {Funktionale Programmierung mit Haskell},
}

Implementiere in der Datei sevenOfNine.hs nachfolgende Funktion in
Haskell. Die parameterlose Funktion \bHaskellCode{sevenOfNine} liefert
eine (unendliche) Liste aller natürlicher Zahlen, die durch 7 oder 9
teilbar sind, zurück. \bHaskellCode{sevenOfNine :: [Int]}
\index{Funktionale Programmierung mit Haskell}
\footcite{fumup:ab:3}

\begin{bAntwort}
\bHaskellDatei{sevenOfNine.hs}
\end{bAntwort}

\end{document}
