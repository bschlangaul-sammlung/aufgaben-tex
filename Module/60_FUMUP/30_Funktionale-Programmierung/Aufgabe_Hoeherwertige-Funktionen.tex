\documentclass{bschlangaul-aufgabe}
\bLadePakete{syntax}
\begin{document}
\bAufgabenMetadaten{
  Titel = {Höherwertige Funktionen},
  Thematik = {Höherwertige Funktionen},
  Referenz = FUMUP.Funktionale-Programmierung.Hoeherwertige-Funktionen,
  RelativerPfad = Module/60_FUMUP/30_Funktionale-Programmierung/Aufgabe_Hoeherwertige-Funktionen.tex,
  ZitatSchluessel = fumup:ab:3,
  BearbeitungsStand = mit Lösung,
  Korrektheit = unbekannt,
  Ueberprueft = {unbekannt},
  Stichwoerter = {Funktionale Programmierung mit Haskell},
}

Implementiere in der Datei switch.hs. Die Funktion
\bHaskellCode{switch} bekommt eine einwertige Entscheidungsfunktion
\bHaskellCode{s}, sowie ein Feld aus einwertigen Funktionen übergeben.
Ihr Ergebnis ist eine einwertige Funktion (mit Parameter
\bHaskellCode{x}). Diese Funktion interpretiert den Rückgabewert der
Entscheidungsfunktion (\bHaskellCode{s x}) als Index, mit dem eine
Funktion aus dem Feld auswählt wird. Diese gewählte Funktion wird dann
für \bHaskellCode{x} auswertet. Wenn der ermittelte Funktionsindex
nicht im Feld liegt, soll \bHaskellCode{x} unverändert zurückgegeben
werden.
\index{Funktionale Programmierung mit Haskell}
\footcite{fumup:ab:3}

\begin{bAntwort}
\bHaskellDatei{switch.hs}
\end{bAntwort}

\end{document}
