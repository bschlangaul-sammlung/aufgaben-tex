\documentclass{bschlangaul-aufgabe}
\bLadePakete{java}
\begin{document}
\bAufgabenMetadaten{
  Titel = {Intervallschachtelung von Wurzeln},
  Thematik = {Intervallschachtelung von Wurzeln},
  Referenz = TECH.Mehr-Adress.09-Wurzel,
  RelativerPfad = Module/50_TECH/20_Mehr-Adress/Aufgabe_09-Wurzel.tex,
  BearbeitungsStand = mit Lösung,
  Korrektheit = unbekannt,
  Ueberprueft = {unbekannt},
  Stichwoerter = {Mehr-Adress-Befehl-Assembler},
}

Mit Hilfe einer Intervallschachtelung lässt sich die Wurzel einer
Quadratzahl bestimmen. Erstelle ein Assemblerprogramm, dass zu einer
Quadratzahl als Eingabe die Wurzel berechnet. Das Ergebnis soll in $R0$
liegen.\index{Mehr-Adress-Befehl-Assembler}

\begin{bAntwort}
\bAssemblerDatei{Aufgabe_09-Wurzel.mi}

\bJavaDatei[firstline=3]{aufgaben/tech_info/assembler/mehr_adress/QuadratWurzel}
\end{bAntwort}

\end{document}
