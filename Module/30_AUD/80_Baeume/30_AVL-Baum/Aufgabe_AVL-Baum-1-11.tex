\documentclass{bschlangaul-aufgabe}

\begin{document}
\bAufgabenMetadaten{
  Titel = {Einfügen von Knoten},
  Thematik = {AVL-Baum-1-11},
  Referenz = AUD.Baeume.AVL-Baum.AVL-Baum-1-11,
  RelativerPfad = Module/30_AUD/80_Baeume/30_AVL-Baum/Aufgabe_AVL-Baum-1-11.tex,
  ZitatSchluessel = aud:ab:7,
  ZitatBeschreibung = {Seite 3, Aufgabe 5: AVL-Baum},
  BearbeitungsStand = nur Angabe,
  Korrektheit = unbekannt,
  Ueberprueft = {unbekannt},
  Stichwoerter = {AVL-Baum},
}

Gegeben sei folgender AVL-Baum:
\index{AVL-Baum}
\footcite[Seite 3, Aufgabe 5: AVL-Baum]{aud:ab:7}

Dabei sind die Blätter der Übersichtlichkeit halber weggelassen worden
und über jedem Knoten v ist folgender Wert angegeben:

\begin{center}
Höhe linker Teilbaum von v – Höhe rechter Teilbaum von v
\end{center}

\begin{enumerate}

%%
% (a)
%%

\item Fügen Sie in diesen Baum den Schlüssel 1 ein.

%%
% (b)
%%

\item Fügen Sie in diesen Baum den Schlüssel 11 ein.

\end{enumerate}
\end{document}
