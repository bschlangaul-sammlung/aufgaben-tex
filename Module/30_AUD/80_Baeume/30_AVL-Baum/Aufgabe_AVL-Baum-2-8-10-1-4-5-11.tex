\documentclass{bschlangaul-aufgabe}
\bLadePakete{baum}
\begin{document}
\bAufgabenMetadaten{
  Titel = {AVL-Baum},
  Thematik = {AVL-Baum 2, 8, 10, 1, 4, 5, 11},
  Referenz = AUD.Baeume.AVL-Baum.AVL-Baum-2-8-10-1-4-5-11,
  RelativerPfad = Module/30_AUD/80_Baeume/30_AVL-Baum/Aufgabe_AVL-Baum-2-8-10-1-4-5-11.tex,
  ZitatSchluessel = aud:e-klausur,
  BearbeitungsStand = mit Lösung,
  Korrektheit = unbekannt,
  Ueberprueft = {unbekannt},
  Stichwoerter = {AVL-Baum},
}

Fügen Sie die Zahlen 2, 8, 10, 1, 4, 5, 11 in der vorgegebenen
Reihenfolge in einen AVL-Baum ein. Wie sieht der finale AVL-Baum aus?\index{AVL-Baum}
\footcite{aud:e-klausur}

\begin{bAntwort}
\begin{bBaum}{Nach dem Einfügen von „2“}
\begin{tikzpicture}[b binaer baum]
\Tree
[.\node[label=0]{2}; ]
\end{tikzpicture}
\end{bBaum}

\begin{bBaum}{Nach dem Einfügen von „8“}
\begin{tikzpicture}[b binaer baum]
\Tree
[.\node[label=+1]{2};
  \edge[blank]; \node[blank]{};
  [.\node[label=0]{8}; ]
]
\end{tikzpicture}
\end{bBaum}

\begin{bBaum}{Nach dem Einfügen von „10“}
\begin{tikzpicture}[b binaer baum]
\Tree
[.\node[label=+2]{2};
  \edge[blank]; \node[blank]{};
  [.\node[label=+1]{8};
    \edge[blank]; \node[blank]{};
    [.\node[label=0]{10}; ]
  ]
]
\end{tikzpicture}
\end{bBaum}

\begin{bBaum}{Nach der Linksrotation}
\begin{tikzpicture}[b binaer baum]
\Tree
[.\node[label=0]{8};
  [.\node[label=0]{2}; ]
  [.\node[label=0]{10}; ]
]
\end{tikzpicture}
\end{bBaum}

\begin{bBaum}{Nach dem Einfügen von „1“}
\begin{tikzpicture}[b binaer baum]
\Tree
[.\node[label=-1]{8};
  [.\node[label=-1]{2};
    [.\node[label=0]{1}; ]
    \edge[blank]; \node[blank]{};
  ]
  [.\node[label=0]{10}; ]
]
\end{tikzpicture}
\end{bBaum}

\begin{bBaum}{Nach dem Einfügen von „4“}
\begin{tikzpicture}[b binaer baum]
\Tree
[.\node[label=-1]{8};
  [.\node[label=0]{2};
    [.\node[label=0]{1}; ]
    [.\node[label=0]{4}; ]
  ]
  [.\node[label=0]{10}; ]
]
\end{tikzpicture}
\end{bBaum}

\begin{bBaum}{Nach dem Einfügen von „5“}
\begin{tikzpicture}[b binaer baum]
\Tree
[.\node[label=-2]{8};
  [.\node[label=+1]{2};
    [.\node[label=0]{1}; ]
    [.\node[label=+1]{4};
      \edge[blank]; \node[blank]{};
      [.\node[label=0]{5}; ]
    ]
  ]
  [.\node[label=0]{10}; ]
]
\end{tikzpicture}
\end{bBaum}

\begin{bBaum}{Nach der Linksrotation}
\begin{tikzpicture}[b binaer baum]
\Tree
[.\node[label=-2]{8};
  [.\node[label=-1]{4};
    [.\node[label=-1]{2};
      [.\node[label=0]{1}; ]
      \edge[blank]; \node[blank]{};
    ]
    [.\node[label=0]{5}; ]
  ]
  [.\node[label=0]{10}; ]
]
\end{tikzpicture}
\end{bBaum}

\begin{bBaum}{Nach der Rechtsrotation}
\begin{tikzpicture}[b binaer baum]
\Tree
[.\node[label=0]{4};
  [.\node[label=-1]{2};
    [.\node[label=0]{1}; ]
    \edge[blank]; \node[blank]{};
  ]
  [.\node[label=0]{8};
    [.\node[label=0]{5}; ]
    [.\node[label=0]{10}; ]
  ]
]
\end{tikzpicture}
\end{bBaum}

\begin{bBaum}{Nach dem Einfügen von „11“}
\begin{tikzpicture}[b binaer baum]
\Tree
[.\node[label=+1]{4};
  [.\node[label=-1]{2};
    [.\node[label=0]{1}; ]
    \edge[blank]; \node[blank]{};
  ]
  [.\node[label=+1]{8};
    [.\node[label=0]{5}; ]
    [.\node[label=+1]{10};
      \edge[blank]; \node[blank]{};
      [.\node[label=0]{11}; ]
    ]
  ]
]
\end{tikzpicture}
\end{bBaum}
\end{bAntwort}

\end{document}
