\documentclass{bschlangaul-aufgabe}
\bLadePakete{java}
\begin{document}
\bAufgabenMetadaten{
  Titel = {Feld-Invertierer},
  Thematik = {Feld-Invertierer},
  Referenz = AUD.Rekursion.Feld-Invertierer,
  RelativerPfad = Module/30_AUD/10_Rekursion/Aufgabe_Feld-Invertierer.tex,
  ZitatSchluessel = aud:ab:2,
  ZitatBeschreibung = {Aufgabenblatt 2: Rekursion, Sortieren, Komplexität, Arrays und
Rekursion, Diese Aufgabe stammt aus der Vorlesung Konzepte der
Programmierung von Prof. Bernhard Westfechtel der Universität Bayreuth,
WS 2017/18, Übungsblatt 8 und wurde dankenswerterweise zur Verwendung in
diesem Aufgabenblatt zur Verfügung gestellt, Aufgabe 1},
  BearbeitungsStand = mit Lösung,
  Korrektheit = unbekannt,
  Ueberprueft = {unbekannt},
  Stichwoerter = {Rekursion, Implementierung in Java},
}

\begin{enumerate}

%%
%
%%

\item Erstellen Sie eine neue Klasse \bJavaCode{ArrayInvertierer} mit
einer rekursiven Methode, die den Inhalt eines ihr übergebenen 1D-Arrays
gefüllt mit Strings invertiert. Auf diese Weise kann \zB ein deutscher
Satz im Array gespeichert und dann verkehrt herum ausgegeben werden.
\index{Rekursion}
\footcite[Aufgabenblatt 2: Rekursion, Sortieren, Komplexität, Arrays und
Rekursion, Diese Aufgabe stammt aus der Vorlesung Konzepte der
Programmierung von Prof. Bernhard Westfechtel der Universität Bayreuth,
WS 2017/18, Übungsblatt 8 und wurde dankenswerterweise zur Verwendung in
diesem Aufgabenblatt zur Verfügung gestellt, Aufgabe 1]{aud:ab:2}
\index{Implementierung in Java}

\emph{Wichtig:} Nicht das übergebene Array soll verändert werden,
sondern ein Neues erstellt und von der Methode zurückgegeben werden.

\emph{Tipp:} Sie dürfen dafür gerne auch rekursive Hilfsmethoden
benutzen.

%%
%
%%

\item Implementieren Sie dann eine \bJavaCode{main}-Methode, in der Sie
zwei verschieden lange \bJavaCode{String}-Arrays erzeugen und die
Wortreihenfolge umkehren lassen. Das Ergebnis soll auf der Konsole
ausgegeben werden und könnte \zB wie folgt aussehen.

\bigskip

{
\ttfamily
Den Satz\\
Ich find dich einfach klasse!\\
wuerde Meister Yoda so aussprechen:\\
klasse! einfach dich find Ich\\

Den Satz\\
Das war super einfach/schwer\\
wuerde Meister Yoda so aussprechen:\\
einfach/schwer super war Das
}

\bigskip

[optional] Wenn das ursprüngliche \bJavaCode{String}-Array selbst
verändert werden soll, braucht die rekursive Methode keine Rückgabe.
Versuchen Sie, diese Aufgabe ohne das Nutzen einer Hilfsmethode zu
lösen.

\begin{bAntwort}
\bJavaDatei[firstline=3]{aufgaben/aud/ab_2/ArrayInvertierer}
\end{bAntwort}
\end{enumerate}

\end{document}
