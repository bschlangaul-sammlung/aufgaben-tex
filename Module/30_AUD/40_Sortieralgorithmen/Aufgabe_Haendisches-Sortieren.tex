\documentclass{bschlangaul-aufgabe}
\bLadePakete{sortieren,java}
\begin{document}
\bAufgabenMetadaten{
  Titel = {Händisches Sortieren},
  Thematik = {Händisches Sortieren},
  Referenz = AUD.Sortieralgorithmen.Haendisches-Sortieren,
  RelativerPfad = Module/30_AUD/40_Sortieralgorithmen/Aufgabe_Haendisches-Sortieren.tex,
  ZitatSchluessel = aud:pu:1,
  ZitatBeschreibung = {Seite 1},
  BearbeitungsStand = mit Lösung,
  Korrektheit = unbekannt,
  Ueberprueft = {unbekannt},
  Stichwoerter = {Bubblesort, Mergesort, Quicksort},
}

Gegeben sei ein Array $a$, welches die Werte $5,7,9,3,6,1,2,8$ enthält.
Sortieren Sie das Array händisch mit:\footcite[Seite 1]{aud:pu:1}

\begin{enumerate}

%%
% (a)
%%

\item Bubblesort\index{Bubblesort}

\begin{bAntwort}
erster Durchgang

\bVertauschen{5 7 9 >3 <6 1 2 8}

\bVertauschen{5 7 9 6 >3 <1 2 8}

\bVertauschen{5 7 9 6 1 >3 <2 8}

Zweiter Durchgang

\bVertauschen{5 7 9 >6 <1 2 3 8}

\bVertauschen{5 7 9 1 >6 <2 3 8}

\bVertauschen{5 7 9 1 2 >6 <3 8}

Dritter Durchgang

\bVertauschen{5 7 >9 <1 2 3 6 8}

\bVertauschen{5 7 1 >9 <2 3 6 8}

\bVertauschen{5 7 1 2 >9 <3 6 8}

\bVertauschen{5 7 1 2 3 >9 <6 8}

\bVertauschen{5 7 1 2 3 6 >9 <8}

Vierter Durchgang

\bVertauschen{5 >7 <1 2 3 6 8 9}

\bVertauschen{5 1 >7 <2 3 6 8 9}

\bVertauschen{5 1 2 >7 <3 6 8 9}

\bVertauschen{5 1 2 3 >7 <6 8 9}

Fünfter Durchgang

\bVertauschen{>5 <1 2 3 6 7 8 9}

\bVertauschen{1 >5 <2 3 6 7 8 9}

\bVertauschen{1 2 >5 <3 6 7 8 9}

fertig

\bVertauschen{1 2 3 5 6 7 8 9}

% bschlangaul-werkzeug.java sortiere 5 7 9 3 6 1 2 8 -b
\begin{verbatim}
 5  7  9  3  6  1  2  8  Eingabe
 5  7  9  3  6  1  2  8  Durchlauf Nr. 1
 5  7 >9  3< 6  1  2  8  vertausche [2<>3]
 5  7  3 >9  6< 1  2  8  vertausche [3<>4]
 5  7  3  6 >9  1< 2  8  vertausche [4<>5]
 5  7  3  6  1 >9  2< 8  vertausche [5<>6]
 5  7  3  6  1  2 >9  8< vertausche [6<>7]
 5  7  3  6  1  2  8  9  Durchlauf Nr. 2
 5 >7  3< 6  1  2  8  9  vertausche [1<>2]
 5  3 >7  6< 1  2  8  9  vertausche [2<>3]
 5  3  6 >7  1< 2  8  9  vertausche [3<>4]
 5  3  6  1 >7  2< 8  9  vertausche [4<>5]
 5  3  6  1  2  7  8  9  Durchlauf Nr. 3
>5  3< 6  1  2  7  8  9  vertausche [0<>1]
 3  5 >6  1< 2  7  8  9  vertausche [2<>3]
 3  5  1 >6  2< 7  8  9  vertausche [3<>4]
 3  5  1  2  6  7  8  9  Durchlauf Nr. 4
 3 >5  1< 2  6  7  8  9  vertausche [1<>2]
 3  1 >5  2< 6  7  8  9  vertausche [2<>3]
 3  1  2  5  6  7  8  9  Durchlauf Nr. 5
>3  1< 2  5  6  7  8  9  vertausche [0<>1]
 1 >3  2< 5  6  7  8  9  vertausche [1<>2]
 1  2  3  5  6  7  8  9  Durchlauf Nr. 6
 1  2  3  5  6  7  8  9  Ausgabe
\end{verbatim}
\end{bAntwort}

%%
% (b)
%%

\item Mergesort\index{Mergesort}
\begin{bAntwort}

\begin{forest}
  /tikz/arrows=->, /tikz/>=latex, /tikz/nodes={draw},
  for tree={delay={sort}}, sort level=2
[5 7 9 3 6 1 2 8
  [5 7 9 3
    [5 7
      [5]
      [7]
    ]
    [9 3
      [9]
      [3]
    ]
  ]
  [6 1 2 8
    [6 1
      [6]
      [1]
    ]
    [2 8
      [2]
      [8]
    ]
  ]
]
%
\coordinate (m) at (!|-!\forestOnes);
\myNodes
\end{forest}

% bschlangaul-werkzeug.java sortiere 5 7 9 3 6 1 2 8 -m
\begin{verbatim}
 5  7  9  3  6  1  2  8  Eingabe
 5  7  9  3||6  1  2  8  teile [3]
 5  7||9  3              teile [1]
 5||7                    teile [0]
 5  7                    Nach dem Mischen [0-1]
       9||3              teile [2]
       3  9              Nach dem Mischen [2-3]
 3  5  7  9              Nach dem Mischen [0-3]
             6  1||2  8  teile [5]
             6||1        teile [4]
             1  6        Nach dem Mischen [4-5]
                   2||8  teile [6]
                   2  8  Nach dem Mischen [6-7]
             1  2  6  8  Nach dem Mischen [4-7]
 1  2  3  5  6  7  8  9  Nach dem Mischen
 1  2  3  5  6  7  8  9  Ausgabe
\end{verbatim}
\end{bAntwort}

%%
% (c)
%%

\item Quicksort\index{Quicksort}

\begin{bAntwort}

% bschlangaul-werkzeug.java sortiere 5 7 9 3 6 1 2 8 -q
\begin{verbatim}
 5  7  9  3  6  1  2  8  Eingabe
 5  7  9  3  6  1  2  8  zerlege
 5  7  9  3* 6  1  2  8  markiere [3]
 5  7  9 >3  6  1  2  8< vertausche [3<>7]
>5  7  9  8  6  1< 2  3  vertausche [0<>5]
 1 >7  9  8  6  5  2< 3  vertausche [1<>6]
 1  2 >9  8  6  5  7  3< vertausche [2<>7]
 1  2                    zerlege [0-1]
 1* 2                    markiere [0]
>1  2<                   vertausche [0<>1]
>2  1<                   vertausche [0<>1]
          8  6  5  7  9  zerlege [3-7]
          8  6  5* 7  9  markiere [5]
          8  6 >5  7  9< vertausche [5<>7]
         >8  6  9  7  5< vertausche [3<>7]
             6  9  7  8  zerlege [4-7]
             6  9* 7  8  markiere [5]
             6 >9  7  8< vertausche [5<>7]
            >6< 8  7  9  vertausche [4<>4]
             6 >8< 7  9  vertausche [5<>5]
             6  8 >7< 9  vertausche [6<>6]
             6  8  7 >9< vertausche [7<>7]
             6  8  7     zerlege [4-6]
             6  8* 7     markiere [5]
             6 >8  7<    vertausche [5<>6]
            >6< 7  8     vertausche [4<>4]
             6 >7< 8     vertausche [5<>5]
             6  7 >8<    vertausche [6<>6]
             6  7        zerlege [4-5]
             6* 7        markiere [4]
            >6  7<       vertausche [4<>5]
            >7  6<       vertausche [4<>5]
 1  2  3  5  6  7  8  9  Ausgabe
\end{verbatim}
\end{bAntwort}

\end{enumerate}

\end{document}
