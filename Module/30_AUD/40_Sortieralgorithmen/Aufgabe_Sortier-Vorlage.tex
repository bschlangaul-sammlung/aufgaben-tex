\documentclass{bschlangaul-aufgabe}
\bLadePakete{spalten,java}
\begin{document}
\bAufgabenMetadaten{
  Titel = {Aufgabe 2: Sortieren},
  Thematik = {Sortier-Vorlage},
  Referenz = AUD.Sortieralgorithmen.Sortier-Vorlage,
  RelativerPfad = Module/30_AUD/40_Sortieralgorithmen/Aufgabe_Sortier-Vorlage.tex,
  ZitatSchluessel = aud:ab:2,
  ZitatBeschreibung = {Aufgabenblatt 2: Rekursion,
Sortieren, Komplexität, Diese Aufgabe stammt aus dem Übungsblatt 1 zu
Algorithmen und Datenstrukturen von Prof. Dr. Martin Hennecke und Rainer
Gall an der Universität Würzburg und wurde dankenswerterweise zur
Verwendung in diesem Aufgabenblatt zur Verfügung gestellt, Aufgabe
2},
  BearbeitungsStand = mit Lösung,
  Korrektheit = unbekannt,
  Ueberprueft = {unbekannt},
  Stichwoerter = {Selectionsort, Bubblesort},
}

Für diese Aufgabe wird die Vorlage Sortieralgorithmen benötigt, die auf
dem Beiblatt genauer erklärt wird.\footcite[Aufgabenblatt 2: Rekursion,
Sortieren, Komplexität, Diese Aufgabe stammt aus dem Übungsblatt 1 zu
Algorithmen und Datenstrukturen von Prof. Dr. Martin Hennecke und Rainer
Gall an der Universität Würzburg und wurde dankenswerterweise zur
Verwendung in diesem Aufgabenblatt zur Verfügung gestellt, Aufgabe
2]{aud:ab:2}

Die fertigen Methoden sollen in der Lage sein, beliebige Arrays zu
sortieren. Im gelben Textfeld des Eingabefensters soll dabei wieder
ausführlich und nachvollziehbar angezeigt werden, wie die jeweilige
Methode vorgeht. Beispielsweise so:

{
\tiny
\begin{multicols}{2}
\begin{verbatim}
Führe die Methode selectionSort() aus:
Sortiere dieses Feld: 5, 3, 17, 7, 42, 23
Der Marker liegt bei: 5
Das Maximum liegt bei: 4
Diese beiden Elemente werden nun vertauscht.
Ergebnis dieser Runde: 5, 3, 17, 7, 23, 42
Der Marker liegt bei: 4
Das Maximum liegt bei: 4
Diese beiden Elemente werden nun vertauscht.
Ergebnis dieser Runde: 5, 3, 17, 7, 23, 42
Der Marker liegt bei: 3
Das Maximum liegt bei: 2
Diese beiden Elemente werden nun vertauscht.
Ergebnis dieser Runde: 5, 3, 7, 17, 23, 42
Der Marker liegt bei: 2
Das Maximum liegt bei: 2
Diese beiden Elemente werden nun vertauscht.
Ergebnis dieser Runde: 5, 3, 7, 17, 23, 42
Der Marker liegt bei: 1
Das Maximum liegt bei: 0
Diese beiden Elemente werden nun vertauscht.
Ergebnis dieser Runde: 3, 5, 7, 17, 23, 42
Der Marker liegt bei: 0
Das Maximum liegt bei: 0
Diese beiden Elemente werden nun vertauscht.
Ergebnis dieser Runde: 3, 5, 7, 17, 23, 42
\end{verbatim}

\begin{verbatim}
Führe die Methode bubbleSort() aus:
Sortiere dieses Feld: 5, 3, 17, 7, 42, 23
Das Element an der Stelle 0 ist größer als sein Nachfolger.
Diese beiden werden nun vertauscht.
Ergebnis dieser Runde: 3, 5, 17, 7, 42, 23
Das Element an der Stelle 2 ist größer als sein Nachfolger.
Diese beiden werden nun vertauscht.
Ergebnis dieser Runde: 3, 5, 7, 17, 42, 23
Das Element an der Stelle 4 ist größer als sein Nachfolger.
Diese beiden werden nun vertauscht.
Ergebnis dieser Runde: 3, 5, 7, 17, 23, 42
\end{verbatim}
\end{multicols}
}

\begin{enumerate}

%%
% (a)
%%

\item Vervollständige die Methode
\bJavaCode{selectionSort()}.\index{Selectionsort}

\begin{bAntwort}
\bJavaDatei[firstline=75,lastline=105]{aufgaben/aud/ab_2/Sortieralgorithmen}
\end{bAntwort}

%%
% (b)
%%

\item Vervollständige die Methode
\bJavaCode{bubbleSort()}.\index{Bubblesort}

\begin{bAntwort}
\bJavaDatei[firstline=107,lastline=136]{aufgaben/aud/ab_2/Sortieralgorithmen}
\end{bAntwort}
\end{enumerate}
\end{document}
