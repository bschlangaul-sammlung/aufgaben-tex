\documentclass{bschlangaul-aufgabe}
\bLadePakete{java}
\begin{document}
\bAufgabenMetadaten{
  Titel = {Sortieren},
  Thematik = {iterativ-rekursiv},
  Referenz = AUD.Sortieralgorithmen.Selectionsort.iterativ-rekursiv,
  RelativerPfad = Module/30_AUD/40_Sortieralgorithmen/20_Selectionsort/Aufgabe_iterativ-rekursiv.tex,
  ZitatSchluessel = aud:e-klausur,
  ZitatBeschreibung = {Aufgabe 2},
  BearbeitungsStand = mit Lösung,
  Korrektheit = unbekannt,
  Ueberprueft = {unbekannt},
  Stichwoerter = {Iterative Realisation, Rekursion, Selectionsort},
}

In dieser Aufgabe soll ein gegebenes Integer Array mit Hilfe von
\textbf{Selection Sort} sortiert werden. Es soll eine
iterative\index{Iterative Realisation} und eine
rekursive\index{Rekursion} Methode geschrieben werden.
\index{Selectionsort}
\footcite[Aufgabe 2]{aud:e-klausur}
%
Verwenden Sie zur Implementierung jeweils die Methodenköpfe
\bJavaCode{selectionSortIterativ()} und
\bJavaCode{selectionSortRekursiv()}. Eine \bJavaCode{swap}-Methode,
die für ein gegebenes Array und zwei Indizes die Einträge an den
jeweiligen Indizes des Arrays vertauscht, ist gegeben und muss nicht
implementiert werden.
%
Es müssen keine weiteren Methoden geschrieben werden!

\begin{bAntwort}
\bPseudoUeberschrift{iterativ}

\bJavaDatei[firstline=11,lastline=21]{aufgaben/aud/sortier/SelectionSort}

\bPseudoUeberschrift{rekursiv}

\bJavaDatei[firstline=23,lastline=35]{aufgaben/aud/sortier/SelectionSort}
\end{bAntwort}

\end{document}
