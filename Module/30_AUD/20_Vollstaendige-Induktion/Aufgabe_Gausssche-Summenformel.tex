\documentclass{bschlangaul-aufgabe}
\bLadePakete{java,vollstaendige-induktion}
\begin{document}
\bAufgabenMetadaten{
  Titel = {Gaußsche Summenformel},
  Thematik = {Gaußsche Summenformel},
  Referenz = AUD.Vollstaendige-Induktion.Gausssche-Summenformel,
  RelativerPfad = Module/30_AUD/20_Vollstaendige-Induktion/Aufgabe_Gausssche-Summenformel.tex,
  BearbeitungsStand = mit Lösung,
  Korrektheit = unbekannt,
  Ueberprueft = {unbekannt},
  Stichwoerter = {Vollständige Induktion},
}

\let\m=\bInduktionMarkierung
\let\e=\bInduktionErklaerung

Gegeben sei folgende rekursive Methodendeklaration in der Sprache Java.
Es wird als Vorbedingung vorausgesetzt, dass die Methode \bJavaCode{sum}
nur für Werte $n \geq 0$ aufgerufen wird.\index{Vollständige Induktion}

\bJavaDatei[firstline=8,lastline=14]{aufgaben/aud/induktion/Gauss}

\noindent
Beweisen Sie mittels vollständiger Induktion, dass der Methodenaufruf
\bJavaCode{sum(n)} die Summe der ersten $n$ aufeinanderfolgenden
natürlichen Zahlen für alle $n \geq 0$ berechnet, wobei gilt

\begin{displaymath}
\sum_{k=0}^{n}k = \frac{n(n+1)}{2}
\end{displaymath}

\begin{bAntwort}

%%
%
%%

\bInduktionAnfang

\begin{displaymath}
\sum_{k=0}^{0}k = \frac{0(0+1)}{2} = \frac{0}{2} = 0
\end{displaymath}

\begin{displaymath}
\texttt{sum(0)} = 0
\end{displaymath}

\bInduktionVoraussetzung

\begin{displaymath}
\sum_{k=0}^{n}k = \frac{n(n+1)}{2}
\end{displaymath}

\begin{displaymath}
\texttt{sum(n)} = n + \frac{(n-1)((n-1)+1)}{2} = n + \frac{(n-1)n}{2}
\end{displaymath}

\bInduktionSchritt

\begin{displaymath}
\sum_{k=0}^{n+1}k = \frac{(n+1)\bigl((n+1)+1\bigr)}{2}
\end{displaymath}

\begin{align*}
\texttt{sum(n+1)}
&= \m{(n+1)} + \frac{(\m{(n+1)}-1)\m{(n+1)})}{2}
& \e{$n+1-1=n$}\\
%
& = (n + 1) + \m{\frac{n(n + 1)}{2}}
& \e{$(n+1)$ eingesetzt}\\
%
& = \frac{\m{2}(n + 1)}{\m{2}} + \frac{n(n + 1)}{2}
& \e{$(n+1)$ als Bruch geschrieben}\\
%
&
= \frac{2(n+1) + n(n+1)}{\m{2}} &
\e{Hauptnenner $2$}\\
%
&
= \frac{(2 + n)\m{(n+1)}}{2} &
\e{$(n+1)$ ausgeklammert} \\
%
& = \frac{(\m{n + 2})(n+1)}{2} &
\e{Kommutativgesetz angewandt} \\
%
&
= \frac{\m{(n+1)(n+2)}}{2} &
\e{getauscht nach Kommutativgesetz} \\
%
&
= \frac{\m{(n+1)}\bigl(\m{(n+1)}+1\bigr)}{2} &
\e{mit $(n+1)$ an der Stelle von $n$}\\
\end{align*}
\end{bAntwort}

\bJavaTestDatei{aufgaben/aud/induktion/GaussTest}

\end{document}
