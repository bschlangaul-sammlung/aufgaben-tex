\documentclass{bschlangaul-aufgabe}

\begin{document}
\bAufgabenMetadaten{
  Titel = {Pseudocode (Sackbahnhof)},
  Thematik = {Sackbahnhof},
  Referenz = AUD.Listen.Stapel.Sackbahnhof,
  RelativerPfad = Module/30_AUD/70_Listen/30_Stapel/Aufgabe_Sackbahnhof.tex,
  ZitatSchluessel = aud:ab:7,
  ZitatBeschreibung = {Seite 3, Aufgabe 4: Stack},
  BearbeitungsStand = mit Lösung,
  Korrektheit = unbekannt,
  Ueberprueft = {unbekannt},
  Stichwoerter = {Stapel (Stack)},
}

In einem Sackbahnhof mit drei Gleisen befinden sich in den Gleisen
\texttt{S1} und \texttt{S2} zwei Züge jeweils mit Waggons für
Zielbahnhof \texttt{A} und \texttt{B}. Gleis \texttt{S3} ist leer.
Stellen Sie die Züge zusammen, die nur Waggons für einen Zielbahnhof
enthalten! Betrachten Sie \texttt{S1}, \texttt{S2} und \texttt{S3} als
Stapel und entwerfen Sie einen Algorithmus (Pseudocode genügt), der die
Züge so umordnet, dass anschließend alle Waggons für \texttt{A} in
\texttt{S1} und alle Waggons für \texttt{B} in \texttt{S2} stehen.\index{Stapel (Stack)}
\footcite[Seite 3, Aufgabe 4: Stack]{aud:ab:7}
\end{document}
