\documentclass{bschlangaul-aufgabe}

\begin{document}
\bAufgabenMetadaten{
  Titel = {Aufgabe aus dem Informatik-Biber zu Stapeln},
  Thematik = {Informatik-Biber},
  Referenz = AUD.Listen.Stapel.Informatik-Biber,
  RelativerPfad = Module/30_AUD/70_Listen/30_Stapel/Aufgabe_Informatik-Biber.tex,
  ZitatSchluessel = net:pdf:informatik-biber-2012,
  ZitatBeschreibung = {Seite 23},
  BearbeitungsStand = mit Lösung,
  Korrektheit = unbekannt,
  Ueberprueft = {unbekannt},
  Stichwoerter = {Stapel (Stack)},
}

Der Güterzug der Biberbahn wurde in der Wagenreihung \verb|D-E-B-C-A|
abgestellt: Die Lok kann vorwärts und rückwärts fahren und dabei
beliebig viele Waggons ziehen und schieben. Jedes Mal, wenn ein Waggon
angekoppelt oder ein Waggon abgekoppelt wird, zählt das als eine
Rangieroperation. Wie viele Rangieroperationen sind mindestens nötig, um
die Wagenreihung \verb|A-B-C-D-E| herzustellen?\index{Stapel (Stack)}

\begin{bAntwort}
Die Anzahl 8 ist richtig: Um einen Zug mit nur zwei Waggons umzuordnen,
muss jeder der beiden Waggons einmal an- und einmal abgekoppelt werden,
das sind vier Operationen. Bei dieser Aufgabe kann man die bereits
geordneten Zugteile \verb|D-E| und \verb|B-C| als einzelne Waggons
behandeln. Die ersten beiden umzuordnen, etwa \verb|D-E| und \verb|B-C|,
erfordert also vier Operationen. Den so gewonnenen Zugteil
\verb|B-C-D-E| und den verbleibenden Waggon \verb|A| umzuordnen
erfordert weitere vier Operationen. Die Reihenfolge der Schritte mag
varüeren, aber nur mit mehr Gleisen könnten Operationen eingespart
werden.

Die zwei Abstellgleise können als Stapelspeicher (stacks) angesehen
werden. Man kann Objekte hineintun und wieder herausholen – aber nicht
in beliebiger Reihenfolge. Was zuletzt hineinkam (push), muss zuerst
wieder heraus (pop). Stapelspeicher, manchmal auch Kellerspeicher
genannt, werden von der Informatik in Programmen und Hardwareschaltungen
für vielfältige Zwecke eingesetzt.
\footcite[Seite 23]{net:pdf:informatik-biber-2012}
\end{bAntwort}
\end{document}
