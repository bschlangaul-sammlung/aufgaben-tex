\documentclass{bschlangaul-aufgabe}
\bLadePakete{java,uml}
\begin{document}
\bAufgabenMetadaten{
  Titel = {Musikliste},
  Thematik = {Playlist},
  Referenz = AUD.Listen.Listen.Playlist,
  RelativerPfad = Module/30_AUD/70_Listen/10_Listen/Aufgabe_Playlist.tex,
  ZitatSchluessel = aud:ab:4,
  ZitatBeschreibung = {Seite 1-2, Aufgaben 1-3},
  BearbeitungsStand = mit Lösung,
  Korrektheit = unbekannt,
  Ueberprueft = {unbekannt},
  Stichwoerter = {Einfach-verkettete Liste, Implementierung in Java, Doppelt-verkettete Liste, Rekursion},
}

\begin{center}
\begin{tikzpicture}[scale=0.8, transform shape]
\tiny
\umlclass
{MusikListe}
{
  int anzahl
}
{
  void setzeErsten(Knoten e)\\
  int aktualisiereAnzahl()\\
  String gibMusikstückListe()\\
  Knoten entnimmOben()\\
  Knoten gibKnoten()
}

\umlclass[x=5.5]
{Knoten}
{
}
{
  Knoten gibNächsten()\\
  void setzeNächsten(Knoten k)\\
  MusikStück gibMusikstück()
}

\umlaggreg[%
  angle1=10,
  angle2=90,
  loopsize=3cm,
  mult1=1,
  mult2=0..1,
  pos1=0.05,
  pos2=0.95,
  name=nächster,
]{Knoten}{Knoten}
\bUmlLeserichtung[pos=above right,dir=right,distance=3cm]{nächster}

\umlclass[x=11]
{MusikStueck}
{
  String titel
}
{
  String gibTitel()
}

\umlassoc[pos1=0.4,name=erster]{MusikListe}{Knoten}
\bUmlLeserichtung[pos=below,dir=right,distance=0.2cm]{erster}

\umlaggreg[mult1=1,mult2=1,name=verwaltet]{Knoten}{MusikStueck}
\bUmlLeserichtung[pos=below right,dir=right,distance=0.2cm and -0.3cm]{verwaltet}

\end{tikzpicture}
\end{center}

Das Klassendiagramm zeigt den Aufbau einer Playlist.
\footcite[Seite 1-2, Aufgaben 1-3]{aud:ab:4}

\begin{enumerate}
\item Implementieren Sie das Klassendiagramm.

\begin{bAntwort}
\bPseudoUeberschrift{Klasse „MusikListe“}
\bJavaDatei[firstline=3,lastline=66]{aufgaben/aud/listen/musikliste/MusikListe}

\bPseudoUeberschrift{Klasse „Knoten“}
\bJavaDatei[firstline=3,lastline=24]{aufgaben/aud/listen/musikliste/Knoten}

\bPseudoUeberschrift{Klasse „MusikStueck“}
\bJavaDatei{aufgaben/aud/listen/musikliste/MusikStueck}
\end{bAntwort}

\item Schreiben Sie eine Testklasse, in der Sie eine Playlist mit
mindestens vier Liedern erstellen.

\begin{bAntwort}
\bJavaTestDatei[firstline=8,lastline=30]{aufgaben/aud/listen/musikliste/MusikListeTest}
\end{bAntwort}
\end{enumerate}

\subsection{Aufgabe 2}

Die Playlist aus Aufgabe 1 soll nun erweitert werden. Aktualisieren Sie
Ihren Code entsprechend!
\index{Einfach-verkettete Liste}

\begin{enumerate}

%%
%
%%

\item Bisher wurde das erste Element der Musikliste in einer öffentlich
sichtbaren Variable gespeichert, dies ist jedoch nicht sinnvoll.
Erstellen Sie eine Methode \bJavaCode{setzeErsten()}, mit der anstatt dessen
die Liste der erstellten Musikstücke angesprochen werden kann.
\index{Implementierung in Java}

\begin{bAntwort}
\bJavaDatei[firstline=12,lastline=15]{aufgaben/aud/listen/musikliste/MusikListe}
\end{bAntwort}

%%
%
%%

\item Außerdem wird ein Attribut \bJavaCode{anzahl} benötigt, dass mit Hilfe
der Methode \bJavaCode{aktualisiereAnzahl()} auf dem aktuellen Stand
gehalten werden kann.

\begin{bAntwort}
\bJavaDatei[firstline=5,lastline=6]{aufgaben/aud/listen/musikliste/MusikListe}
\end{bAntwort}

%%
%
%%

\item Eine weitere Methode \bJavaCode{gibMusikstückListe()} soll die Titel
aller Lieder in der Liste als \bJavaCode{String} zurückgeben.

\begin{bAntwort}
\bJavaDatei[firstline=32,lastline=43]{aufgaben/aud/listen/musikliste/MusikListe}
\end{bAntwort}

%%
%
%%

\item Mit \bJavaCode{entnimmOben()} soll der erste Titel aus der Liste
entnommen werden können.

\begin{bAntwort}
\bJavaDatei[firstline=45,lastline=53]{aufgaben/aud/listen/musikliste/MusikListe}
\end{bAntwort}

%%
%
%%

\item Es soll der Titel des Musikstücks ermittelt werden, das an einer
bestimmten Position in der Musikliste abgespeichert ist. Implementieren
Sie dazu die Methode \bJavaCode{gibKnoten()}.

\begin{bAntwort}
\bJavaDatei[firstline=55,lastline=66]{aufgaben/aud/listen/musikliste/MusikListe}
\end{bAntwort}

%%
%
%%

\item Die Musikliste soll nun in eine doppelt verkettete Liste umgebaut
werden. Fügen Sie entsprechende Attribute, getter- und setter-Methoden
hinzu.
\index{Doppelt-verkettete Liste}

\begin{bAntwort}
\bJavaDatei[firstline=5,lastline=5]{aufgaben/aud/listen/musikliste/Knoten}
\bJavaDatei[firstline=22,lastline=28]{aufgaben/aud/listen/musikliste/Knoten}
\end{bAntwort}

%%
%
%%

\item Testen Sie die Funktionalität der neuen Methoden in Ihrer
Testklasse.

\begin{bAntwort}
\bJavaTestDatei{aufgaben/aud/listen/musikliste/MusikListeTest}
\end{bAntwort}

\end{enumerate}

\subsection{Rekursion}

Die Anzahl der Titel in der Musikliste aus Aufgabe 1 kann auch unter
Verwendung einer rekursiven Methode ermittelt werden. Implementieren Sie
eine Methode \bJavaCode{zaehleEintraege()}, die analog zu
\bJavaCode{aktualisiereAnzahl()} angibt, wie viele Titel in der Musikliste
gespeichert sind, dies aber rekursiv ermittelt! Testen Sie diese Methode
in der Testklasse. Hinweis: Um für die gesamte Musikliste aufgerufen
werden zu können, muss diese Methode in der Musikliste selbst und auch
in der Klasse Knoten existieren!
\index{Rekursion}

\begin{bAntwort}
\bPseudoUeberschrift{Klasse „MusikListe“}
\bJavaDatei[firstline=67,lastline=74]{aufgaben/aud/listen/musikliste/MusikListe}

\bPseudoUeberschrift{Klasse „Knoten“}
\bJavaDatei[firstline=34,lastline=40]{aufgaben/aud/listen/musikliste/Knoten}

\bPseudoUeberschrift{Klasse „TestKlasse“}
\bJavaTestDatei[firstline=45,lastline=50]{aufgaben/aud/listen/musikliste/MusikListeTest}
\end{bAntwort}
\end{document}
