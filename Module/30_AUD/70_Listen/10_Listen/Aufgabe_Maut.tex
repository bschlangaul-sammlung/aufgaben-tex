\documentclass{bschlangaul-aufgabe}
\bLadePakete{uml}
\begin{document}
\bAufgabenMetadaten{
  Titel = {Aufgabe zu Heterogenen Listen},
  Thematik = {Maut},
  Referenz = AUD.Listen.Listen.Maut,
  RelativerPfad = Module/30_AUD/70_Listen/10_Listen/Aufgabe_Maut.tex,
  ZitatSchluessel = aud:pu:4,
  ZitatBeschreibung = {Seite 1, Aufgabe 1: Heterogene Liste},
  BearbeitungsStand = mit Lösung,
  Korrektheit = unbekannt,
  Ueberprueft = {unbekannt},
  Stichwoerter = {Einfach-verkettete Liste, Klassendiagramm, Kompositum (Composite)},
}

Für die Umsetzung der Maut auf deutschen Autobahnen soll eine
Java-basierte Lösung entworfen werden. Dazu sollen alle Fahrzeuge, die
von einer Mautbrücken erfasst werden, in einer \emph{einfach verketteten
Liste} abgelegt werden. Um einen besseren Überblick über die Einnahmen
zu erhalten, soll zwischen \emph{LKWs}, \emph{PKWs} und
\emph{Motorrädern} unterschieden werden. Als Informatiker schlagen Sie
eine \emph{heterogene Liste} zur Realisierung vor. Notieren Sie unter
Verwendung des \emph{Entwurfsmusters Kompositum} ein entsprechendes
\emph{Klassendiagramm} zur Realisierung der Lösung für eine Mautbrücke.
Auf die Angabe von Attributen und Methoden kann verzichtet werden.
Kennzeichen Sie in Ihrem Klassendiagramm die \emph{abstrakten Klassen}
und benennen Sie die bestehenden \emph{Beziehungen}.\index{Einfach-verkettete Liste}
\footcite[Seite 1, Aufgabe 1: Heterogene Liste]{aud:pu:4}
\index{Klassendiagramm}
\index{Kompositum (Composite)}

\begin{bAntwort}
\begin{center}
\begin{tikzpicture}
\umlsimpleclass[type=abstract,x=4.5,y=5]{Listenelement}
\umlsimpleclass[x=3,y=3]{Abschluss}
\umlsimpleclass[x=6,y=3]{Datenknoten}
\umlsimpleclass[x=0,y=5]{Mautbrücke}
\umlassoc[mult2=1,name=erster,pos2=0.9]{Mautbrücke}{Listenelement}
\bUmlLeserichtung[pos=below,dir=right,distance=0cm]{erster}
\umlVHVinherit{Abschluss}{Listenelement}
\umlVHVinherit{Datenknoten}{Listenelement}

\umlsimpleclass[type=abstract,x=5,y=2]{Fahrzeug}
\umlsimpleclass[x=3,y=0]{Motorrad}
\umlsimpleclass[x=5,y=0]{PKW}
\umlsimpleclass[x=7,y=0]{LKW}
\umlVHVinherit{PKW}{Fahrzeug}
\umlVHVinherit{Motorrad}{Fahrzeug}
\umlVHVinherit{LKW}{Fahrzeug}

\umlHVHassoc[%
  anchor1=east,
  anchor2=east,
  arm1=3cm,name=enthält,pos1=0.3,mult1=1,
]{Fahrzeug}{Datenknoten}
\bUmlLeserichtung[pos=right,dir=left,distance=1cm]{enthält}

\umlHVHassoc[%
  anchor1=east,
  anchor2=east,
  arm1=3cm,name=nächster,pos1=0.3,mult1=1
]{Listenelement}{Datenknoten}
\bUmlLeserichtung[pos=above right,dir=left,distance=1cm and 1cm]{nächster}

\end{tikzpicture}
\end{center}
\end{bAntwort}
\end{document}
