\documentclass{bschlangaul-aufgabe}
\bLadePakete{java}
\begin{document}
\bAufgabenMetadaten{
  Titel = {Aufgabe 5: Listen},
  Thematik = {Wörterbuch},
  Referenz = AUD.Listen.Listen.Woerterbuch,
  RelativerPfad = Module/30_AUD/70_Listen/10_Listen/Aufgabe_Woerterbuch.tex,
  ZitatSchluessel = aud:ab:5,
  ZitatBeschreibung = {Seite 4, Aufgabe 5},
  BearbeitungsStand = mit Lösung,
  Korrektheit = unbekannt,
  Ueberprueft = {unbekannt},
  Stichwoerter = {Einfach-verkettete Liste, Kompositum (Composite)},
}

Erstellen Sie ein Deutsch-Englisch Wörterbuch. Verwenden Sie dazu eine
einfach verkettete Liste mit Kompositum. Identifizieren Sie die
benötigten Klassen, legen Sie das Wörterbuch an und implementieren Sie
anschließend die geforderten Methoden.\index{Einfach-verkettete Liste}
\footcite[Seite 4, Aufgabe 5]{aud:ab:5}
\index{Kompositum (Composite)}

\begin{itemize}
\item Ein Listenelement, welches immer jeweils auf seinen Nachfolger
verweisen kann, enthält jeweils einen Eintrag des Wörterbuchs. Ein
Eintrag besteht aus dem deutschen und dem zugehörigen englischen Wort.
Diese können natürlich jeweils zurückgegeben werden.

\item Mit der Methode \bJavaCode{einfuegen (String deutsch, String
englisch)} soll ein neuer Eintrag in das Wörterbuch eingefügt werden
können. Wie in jedem Wörterbuch müssen die (deutschen) Einträge jedoch
alphabetisch sortiert sein, sodass nicht an einer beliebigen Stelle
eingefügt werden kann. Um die korrekte Einfügeposition zu finden, ist
das Vergleichen von Strings notwendig. Recherchieren Sie dazu, wie die
Methode \bJavaCode{compareTo()} in Java funktioniert!

\item Der Aufruf der Methode \bJavaCode{uebersetze(String deutsch)} auf
der Liste soll nun für ein übergebenes deutsches Wort die englische
Übersetzung ausgeben.

\end{itemize}

\begin{bAntwort}

\bPseudoUeberschrift{Klasse \texttt{WörterbuchEintrag}}

\noindent
Die abstrakte Klasse im Kompositumentwurfsmuster von der sowohl die
primitive Klasse als auch die Behälterklasse erben.

\bJavaDatei[firstline=3]{aufgaben/aud/listen/woerterbuch/WoerterbuchEintrag}

\bPseudoUeberschrift{Klasse \texttt{WortPaar}}

\noindent
Das Listenelement (die primitive Klasse im Kompositumentwurfsmuster).

\bJavaDatei[firstline=3]{aufgaben/aud/listen/woerterbuch/WortPaar}

\bPseudoUeberschrift{Klasse \texttt{Wörterbuch}}

\noindent
Die Behälterklasse im Kompositumentwurfsmuster.

\bJavaDatei[firstline=6]{aufgaben/aud/listen/woerterbuch/Woerterbuch}

\bPseudoUeberschrift{Test}

\bJavaTestDatei{aufgaben/aud/listen/woerterbuch/WoerterbuchTest}

\end{bAntwort}
\end{document}
