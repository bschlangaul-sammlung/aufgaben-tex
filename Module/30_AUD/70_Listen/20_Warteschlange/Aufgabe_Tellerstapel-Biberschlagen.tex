\documentclass{bschlangaul-aufgabe}

\begin{document}
\bAufgabenMetadaten{
  Titel = {Aufgabe aus dem Informatik-Biber: Tellerstapel - Biberschlagen},
  Thematik = {Tellerstapel-Biberschlagen},
  Referenz = AUD.Listen.Warteschlange.Tellerstapel-Biberschlagen,
  RelativerPfad = Module/30_AUD/70_Listen/20_Warteschlange/Aufgabe_Tellerstapel-Biberschlagen.tex,
  ZitatSchluessel = net:pdf:informatik-biber-2010,
  ZitatBeschreibung = {Seite 35},
  BearbeitungsStand = mit Lösung,
  Korrektheit = unbekannt,
  Ueberprueft = {unbekannt},
  Stichwoerter = {Warteschlange (Queue), Stapel (Stack)},
}

Im Restaurant der Biberschule gibt es normalerweise zwei Warteschlangen:
In der einen holen sich die kleinen Biber ihre hohen grünen Teller, in
der anderen holen sich die großen Biber ihre flachen braunen Teller.
Wegen Bauarbeiten kann es heute nur eine Warteschlange für alle Biber
geben. Die Küchenbiber müssen deshalb einen Tellerstapel vorbereiten,
der zur Schlange passt: Sie müssen die grünen und braunen Teller so
stapeln, dass jeder Biber in der Schlange den passenden Teller bekommt.
Schau dir zum Beispiel diese Warteschlange an. Für diese Warteschlange
müssen die Teller so gestapelt sein.\index{Warteschlange (Queue)}
\index{Stapel (Stack)}
\footcite[Seite 35]{net:pdf:informatik-biber-2010}

\begin{bAntwort}
Daten, die mit Computerprogrammen verarbeitet werden sollen, müssen
passend organisiert sein. Informatiker beschäftigen sich deshalb
intensiv mit Datenstrukturen. Zwei einfache Datenstrukturen sind
„Schlange” (Queue) und „Stapel” (Stack). Bei einer „Schlange” kann man
nur auf die zuerst eingereihten Daten zugreifen (nach dem Prinzip FIFO:
„first in, first out”). Bei einem „Stapel” kann man nur auf die zuletzt
eingereihten Daten zugreifen (nach dem Prinzip LIFO: „last in, first
out”). Die Datenstruktur der wartenden Biber ist eine „Schlange”. Die
Datenstruktur der Teller ist ein „Stapel”.
\end{bAntwort}

\end{document}
