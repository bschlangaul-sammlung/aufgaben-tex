\documentclass{bschlangaul-aufgabe}
\bLadePakete{mathe}
\begin{document}
\bAufgabenMetadaten{
  Titel = {Aufgabe zur Komplexität},
  Thematik = {mehrere Funktionen},
  Referenz = AUD.Algorithmische-Komplexitaet.mehrere-Funktionen,
  RelativerPfad = Module/30_AUD/50_Algorithmische-Komplexitaet/Aufgabe_mehrere-Funktionen.tex,
  ZitatSchluessel = aud:pu:2,
  ZitatBeschreibung = {Seite 1-2, Aufgabe 2},
  BearbeitungsStand = mit Lösung,
  Korrektheit = unbekannt,
  Ueberprueft = {unbekannt},
  Stichwoerter = {Algorithmische Komplexität (O-Notation)},
}

\noindent
Geben Sie die Komplexität folgender Funktionen in der
$\mathcal{O}$-Notation an!\index{Algorithmische Komplexität (O-Notation)}
\footcite[Seite 1-2, Aufgabe 2]{aud:pu:2}

\begin{enumerate}
\item $x(n) = 4 \cdot n$

\begin{bAntwort}
$\mathcal{O}(n)$
\end{bAntwort}

\item $a(n) = n^2$

\begin{bAntwort}
$\mathcal{O}(n^2)$
\end{bAntwort}

\item $k(n) = 5 + n$

\begin{bAntwort}
$\mathcal{O}(n)$
\end{bAntwort}

\item $p(n) = 4$

\begin{bAntwort}
$\mathcal{O}(1)$
\end{bAntwort}

\item $j(n) = 4^n$

\begin{bAntwort}
$\mathcal{O}(4^n)$
\end{bAntwort}

\item $b(n) = \frac{n}{8}$

\begin{bAntwort}
$\mathcal{O}(n)$
\end{bAntwort}

\item $m(n) = \frac{1}{n}$

\begin{bAntwort}
$\mathcal{O}(1)$
\end{bAntwort}
\end{enumerate}
\end{document}
