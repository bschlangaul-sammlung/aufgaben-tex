\documentclass{bschlangaul-aufgabe}

\begin{document}
\bAufgabenMetadaten{
  Titel = {Aufgabe 1: Terminologie},
  Thematik = {Terminologie},
  Referenz = DB.Uebersicht.Terminologie,
  RelativerPfad = Module/10_DB/10_Uebersicht/Aufgabe_Terminologie.tex,
  ZitatSchluessel = winter,
  ZitatBeschreibung = {13},
  BearbeitungsStand = mit Lösung,
  Korrektheit = unbekannt,
  Ueberprueft = {unbekannt},
  Stichwoerter = {Datenbanksystem, Datenbank, Datenbankmanagementsystem},
}

\noindent
Beschreiben die Begriffe \emph{„Datenbank“} und \emph{„Datenbanksystem“}
das Gleiche? Kurze Begründung!
\footcite[13]{winter}
\index{Datenbanksystem}
\index{Datenbank}
\index{Datenbankmanagementsystem}

\begin{bAntwort}
Nein. \emph{Datenbanksystem} ist der Oberbegriff. Zu
\emph{Datenbankensystem} (DBS) gehört die \emph{Datenbank} (DB) und das
\emph{Datenbankmanagementsystem} (DBMS). Unter dem Begriff
\emph{Datenbank} versteht man die Menge der gespeicherten Daten. (Die
zweite Komponente eines Datenbanksystems ist das
Datenbankmanagementsystems, mit Hilfe dessen die Daten in der Datenbank
verwaltet werden können.)\footcite[Seite 1]{db:ab:1}
\end{bAntwort}
\end{document}
