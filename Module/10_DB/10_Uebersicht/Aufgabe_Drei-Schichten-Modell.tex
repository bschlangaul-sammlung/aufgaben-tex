\documentclass{bschlangaul-aufgabe}

\begin{document}
\bAufgabenMetadaten{
  Titel = {Schicht auf Schicht},
  Thematik = {Drei-Schichten-Modell},
  Referenz = DB.Uebersicht.Drei-Schichten-Modell,
  RelativerPfad = Module/10_DB/10_Uebersicht/Aufgabe_Drei-Schichten-Modell.tex,
  ZitatSchluessel = db:ab:klausurvorbereitung,
  BearbeitungsStand = mit Lösung,
  Korrektheit = unbekannt,
  Ueberprueft = {unbekannt},
  Stichwoerter = {Drei-Schichten-Modell},
}

Das Drei-Schichten-Modell trägt den verschiedenen Sichten auf eine
Datenbank Rechnung. Geben Sie zu den unten (unter a bis e) genannten
Vorgängen jeweils an, welche
der folgenden Aussagen zutreffen:
\index{Drei-Schichten-Modell}
\footcite{db:ab:klausurvorbereitung}

\begin{enumerate}
\item Änderungen in bestehenden Anwendungsprogrammen notwendig
\item Änderungen im externen Schema notwendig
\item Änderungen im konzeptionellen Schema notwendig
\item Änderungen im internen Schema notwendig
\end{enumerate}

\bPseudoUeberschrift{Vorgänge}

\begin{enumerate}

%%
% (a)
%%

\item Ein neues Anwendungsprogramm wird geschrieben, das bestehende
Daten nutzt.

%%
% (b)
%%

\item Der Datentyp eines Attributs wird geändert, \zB wird statt
VARCHAR(20) VARCHAR(30) verwendet.

%%
% (c)
%%

\item Ein neues Anwendungsprogramm wird entwickelt, das neue
(zusätzliche) Datenstrukturen benötigt.

%%
% (d)
%%

\item Es werden neue Daten eingespeichert bzw. bestehende gelöscht.

%%
% (e)
%%

\item Der Zugriff auf die Daten wird optimiert.
\end{enumerate}

\noindent
Sie könnten dazu folgende Tabelle ausfüllen: Kreuzen Sie das
entsprechende Feld an, wenn die Aussage diesbezüglich wahr ist, oder
lassen Sie es andernfalls leer. Ist die Aussage nicht eindeutig und
situationsbedingt wahr, setzen sie ein eingeklammertes Kreuz (x) ein!

\begin{bAntwort}
\begin{tabular}{lllll}
  &  1  &  2  & 3   & 4 \\
a &     & (x) &     &   \\
b & (x) & (x) & (x) & x \\
c & (x) & (x) &  x  & x \\
d &     &     &     &   \\
e &     &     &     & x
\end{tabular}

\bPseudoUeberschrift{Hinweis}

\begin{itemize}
\item Zu b) Bei der vorgegebenen Änderung des Datentyps ist sicher das
interne Schema betroffen, da sich die Speicherstruktur ändert. Wird der
Datentyp „stark“ geändert, \dh beispielsweise char durch integer
ersetzt, kann das auch Änderungen bei (1), (2) und (3) nach sich ziehen.

\item Zu d) Schemata beschreiben Strukturen. Das Speichern bzw. Löschen
von Daten kann damit keinen Einfluss auf ein Schema haben.
\end{itemize}

\end{bAntwort}
\end{document}
