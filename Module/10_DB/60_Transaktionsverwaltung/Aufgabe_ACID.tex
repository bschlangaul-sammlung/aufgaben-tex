\documentclass{bschlangaul-aufgabe}

\begin{document}
\bAufgabenMetadaten{
  Titel = {ACID und Transaktionen},
  Thematik = {ACID},
  Referenz = DB.Transaktionsverwaltung.ACID,
  RelativerPfad = Module/10_DB/60_Transaktionsverwaltung/Aufgabe_ACID.tex,
  ZitatSchluessel = db:ab:6,
  ZitatBeschreibung = {Aufgabe 4},
  BearbeitungsStand = mit Lösung,
  Korrektheit = unbekannt,
  Ueberprueft = {unbekannt},
  Stichwoerter = {Transaktionen, ACID},
}

Beurteilen Sie kurz folgende Aussagen oder Fragen unter
ACID-Gesichtspunkten!\index{Transaktionen}
\footcite[Aufgabe 4]{db:ab:6}
\footnote{\url{http://wwwlgis.informatik.uni-kl.de/archiv/wwwdvs.informatik.uni-kl.de/courses/DBSREAL/SS2003/Uebungen/Blatt.01.half.pdf}}\index{ACID}
% https://studylibde.com/doc/6009293/%C3%BCbungsblatt-1-%E2%80%93-l%C3%B6sung
\begin{enumerate}

%%
% (a)
%%

\item Seit dem Abort meiner Transaktion sind deren Änderungen überhaupt
nicht mehr vorhanden!

\begin{bAntwort}
Das entspricht der Forderung \emph{Atomicity}, da entweder alle Aktionen
oder keine Aktion einer Transaktion ausgeführt werden soll(en). Bei
Abbruch einer Transaktion werden die bereits abgearbeiteten Aktionen
dieser Transaktion zurückgesetzt.
\end{bAntwort}

%%
% (b)
%%

\item Leider wurde die erfolgreich abgeschlossene Transaktion
zurückgesetzt, da das DBS abgestürzt ist.

\begin{bAntwort}
Das widerspricht der Forderung \emph{Durability}, da erfolgreich
abgeschlossene Transaktionen permanent, \dh dauerhaft, abgespeichert
werden müssen.
\end{bAntwort}

%%
% (c)
%%

\item Eine andere Transaktion hat Änderungen meiner Transaktion
überschrieben. Darf ich jetzt meine Transaktion überhaupt noch beenden
oder muss ich sie abbrechen?

\begin{bAntwort}
Das widerspricht der Forderung \emph{Isolation}. Parallel ablaufende
Transaktionen dürfen sich nicht beeinflussen, wenn dies also wie hier
geschildert der Fall ist, dann muss die Transaktion abgebrochen werden.
\end{bAntwort}

\end{enumerate}

\end{document}
