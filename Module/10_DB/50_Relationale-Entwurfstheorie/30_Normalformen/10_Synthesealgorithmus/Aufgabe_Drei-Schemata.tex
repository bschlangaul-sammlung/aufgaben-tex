\documentclass{bschlangaul-aufgabe}
\bLadePakete{normalformen,synthese-algorithmus}
\begin{document}
\bAufgabenMetadaten{
  Titel = {Normalformen Einstieg},
  Thematik = {Drei-Schemata},
  Referenz = DB.Relationale-Entwurfstheorie.Normalformen.Synthesealgorithmus.Drei-Schemata,
  RelativerPfad = Module/10_DB/50_Relationale-Entwurfstheorie/30_Normalformen/10_Synthesealgorithmus/Aufgabe_Drei-Schemata.tex,
  ZitatSchluessel = db:pu:4,
  ZitatBeschreibung = {Seite 1 Normalformen Einstieg, Aufgabe 1},
  BearbeitungsStand = mit Lösung,
  Korrektheit = unbekannt,
  Ueberprueft = {unbekannt},
  Stichwoerter = {Boyce-Codd-Normalform, Dritte Normalform, Zweite Normalform, Synthese-Algorithmus},
}

\let\FA=\bFunktionaleAbhaengigkeiten
\let\schrittE=\bSyntheseUeberErklaerung
\let\r=\bRelation
\let\u=\underline

Es seien folgende Relationenschemata mit den jeweiligen Mengen
funktionaler Abhängigkeiten gegeben:
\footcite[Seite 1 Normalformen Einstieg, Aufgabe 1]{db:pu:4}

\begin{description}
\item \r[S_1]{P, Q, R} mit

\FA[$F_1$]{%
  P, Q -> R;
  P, R -> Q;
  Q, R -> P;
}

\item \r[S2]{P, R, S, T} mit

\FA[$F_2$]{P, S -> T}

\item \r[S3]{P, S, U}  mit

\FA[$F_3$]{}
\end{description}

\begin{enumerate}

%%
% (a)
%%

\item Welche der drei Schemata sind in
BCNF\index{Boyce-Codd-Normalform}, welche in 3NF\index{Dritte
Normalform}, welche in 2NF\index{Zweite Normalform}? Begründe!

\begin{bAntwort}
\begin{description}
\item[$S_1$:] BCNF

\item[$S_2$:] 1NF aber nicht 2NF

\item[$S_3$:] BCNF
\end{description}

($S_1$, $F_1$) und ($S_3$, $F_3$) sind offenbar in BCNF und
daher auch in 3NF und 2NF. ($S_2$, $F_2$) ist offenbar nicht in
2NF, da der Schlüsselkandidat PRS ist und T von einem Teil dieser
Schlüsselkandidaten, nämlich PS, abhängig ist und daher auch nicht in
3NF oder BCNF.
\end{bAntwort}

%%
% (b)
%%

\item Wenden Sie auf ($S_2$, $F_2$) den
Synthesealgorithmus\index{Synthese-Algorithmus} an, und bestimmen Sie
auch die Mengen aller nichttrivialen einfachen funktionalen
Abhängigkeiten, die über den erhaltenen Teilrelationen gelten. Ihr
Lösungsweg muss nachvollziehbar sein.

\begin{bAntwort}
\begin{enumerate}
\item \schrittE{1}

\FA[$F_2$]{P, S -> T} (ist schon in der kanonische Überdeckung)

\item \schrittE{2}

\r[R_{21}]{P, S, T}

\item \schrittE{3}

\r[R_{21}]{\u{P, S}, T} mit \FA[$F_{21}$]{P S-> T}

\bigskip

\r[R_{22}]{\u{P, S, R}} mit \FA[$F_{22}$]{}

\item \schrittE{4}

\bNichtsZuTun

\end{enumerate}
\end{bAntwort}

\end{enumerate}
\end{document}
