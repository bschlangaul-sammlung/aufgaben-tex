\documentclass{bschlangaul-aufgabe}
\bLadePakete{normalformen,synthese-algorithmus}
\begin{document}
\bAufgabenMetadaten{
  Titel = {Synthese-Algorithmus},
  Thematik = {Supermarkt},
  Referenz = DB.Relationale-Entwurfstheorie.Normalformen.Synthesealgorithmus.Supermarkt,
  RelativerPfad = Module/10_DB/50_Relationale-Entwurfstheorie/30_Normalformen/10_Synthesealgorithmus/Aufgabe_Supermarkt.tex,
  ZitatSchluessel = db:ab:6,
  ZitatBeschreibung = {Aufgabe 5},
  BearbeitungsStand = mit Lösung,
  Korrektheit = unbekannt,
  Ueberprueft = {unbekannt},
  Stichwoerter = {Zweite Normalform, Schlüsselkandidat, Update-Anomalie, Delete-Anomalie, Synthese-Algorithmus},
}

\let\FA=\bFunktionaleAbhaengigkeiten
\let\fa=\bFunktionaleAbhaengigkeit
\let\m=\bAttributMenge
\let\schrittE=\bSyntheseUeberErklaerung
\let\ah=\bAttributHuelle

Ein Supermarkt speichert seine Bestellungen in nachfolgender Tabelle:
\footcite[Aufgabe 5]{db:ab:6}
\bigskip

\noindent
\bRelation[ARTBEST]{
  ArtNr,
  ArtName,
  ArtArt,
  Hersteller,
  HerstTel,
  Lieferant,
  LiefTel,
  BestDat,
  Anzahl,
  EP
}

\bigskip

\noindent
Es existieren folgende funktionale Abhängigkeiten:

\bigskip

\FA[F]{
  ArtNr -> ArtName, ArtArt, Hersteller, HerstTel;
  Hersteller -> HerstTel;
  Lieferant -> LiefTel;
  ArtNr, Lieferant, BestDat -> Anzahl;
}

\bigskip
\noindent
„Bestdat“ steht für Bestelldatum, „EP“ für Einkaufspreis

\begin{enumerate}

%%
% (a)
%%

\item Erläutern Sie, warum nur Relationen mit einem zusammengesetzten
Schlüsselkandidaten die 2. Normalform\index{Zweite Normalform} verletzen
können!

\begin{bAntwort}
Eine Relation ist genau dann in der zweiten Normalform, wenn kein
Nichtprimärattribut funktional von einer echten Teilmenge eines
Schlüsselkandidaten abhängt.

Anders gesagt: Jedes nicht-primäre Attribut ist jeweils von
allen ganzen Schlüsseln abhängig, nicht nur von einem Teil
eines Schlüssels.

Bei nicht zusammengesetzten Schlüsselkandidaten, d. h.
Schlüsselkandidaten mit nur einem Attribut, können Nichtprimärattribute
nur von diesem einen Schlüsselkandidaten abhängen,
sonst wäre es ja kein Schlüsselkandidat / Primär\-schlüssel.
\end{bAntwort}

%%
% (b)
%%

\item Finden Sie den einzigen
Schlüsselkandidaten\index{Schlüsselkandidat} von ARTBEST.

\begin{bAntwort}
Ich wähle \m{ArtNr, Lieferant, BestDat} aus, da diese Attribute auf
keiner rechten Seite einer FD vorkommen. Außerdem wähle ich \m{EP}
aus, da \m{EP} in keiner FD vorkommt.

\begin{liAHuelle}
\ah{F, \m{ArtNr, Lieferant, BestDat, EP}} = \\
\m{
  ArtNr,
  Lieferant,
  BestDat,
  EP,
  ArtName,
  ArtArt,
  Hersteller,
  HerstTel,
  LiefTel,
  Anzahl
} = R
\end{liAHuelle}

Damit ist gezeigt, dass \m{ArtNr, Lieferant, BestDat, EP} ein
Superschlüssel ist.

Ich teste mit Hilfe der Attributhülle, ob man den Superschlüssel noch
weiter verkleinern kann.

\bPseudoUeberschrift{ohne \texttt{ArtNr}}

\begin{liAHuelle}
\ah{F, \m{Lieferant, BestDat, EP}} = \\
\m{
  Lieferant,
  BestDat,
  EP,
  LiefTel
} \neq R
\end{liAHuelle}
\bPseudoUeberschrift{ohne \texttt{Lieferant}}

\begin{liAHuelle}
\ah{F, \m{ArtNr, BestDat, EP}} = \\
\m{
  ArtNr,
  BestDat,
  EP,
  ArtName,
  ArtArt,
  Hersteller,
  HerstTel
} \neq R
\end{liAHuelle}

\bPseudoUeberschrift{ohne \texttt{BestDat}}

\begin{liAHuelle}
\ah{F, \m{ArtNr, Lieferant, EP}} = \\
\m{
  ArtNr,
  Lieferant,
  EP,
  ArtName,
  ArtArt,
  Hersteller,
  HerstTel,
  LiefTel
} \neq R
\end{liAHuelle}

\bPseudoUeberschrift{ohne \texttt{EP}}

\begin{liAHuelle}
\ah{F, \m{ArtNr, Lieferant, BestDat}} = \\
\m{
  ArtNr,
  Lieferant,
  BestDat,
  ArtName,
  ArtArt,
  Hersteller,
  HerstTel,
  LiefTel,
  Anzahl
} \neq R
\end{liAHuelle}

Der Superschlüssel \m{ArtNr, Lieferant, BestDat, EP} kann nicht
mehr weiter verkleinert werden. Er ist bereits minimal. \m{ArtNr,
Lieferant, BestDat, EP} ist der einzige Schlüsselkandidat und damit
der Primärschlüssel.

\end{bAntwort}

%%
% (c)
%%

\item Erläutern Sie, inwiefern obiges Schema die 3. Normalform verletzt!
Zeigen Sie anhand obiger Relation \texttt{ARTBEST} zwei mögliche
Anomalien auf, die bei fehlender Normalisierung auftreten können.

\begin{bAntwort}
In der dritten Normalform darf kein Nichtschlüsselattribut von einem
Schlüsselkandidaten transitiv abhängen. In der Relation \texttt{ARTBEST}
hängt \m{HerstTel} funkional von \m{Hersteller} und
\m{Hersteller} hängt wiederum funktional von dem Primärschlüssel /
Schlüsselkandidaten \m{ArtNr, Lieferant, BestDat, EP} ab.

\bigskip

\m{ArtNr, Lieferant, BestDat, EP}
$\rightarrow$ \m{Hersteller}
$\rightarrow$ \m{HerstTel}

\bPseudoUeberschrift{Update-Anomalie}\index{Update-Anomalie}

Es kann zur Update-Anomalie kommen. Ändert sich zum Beispiel die
Telefonnummer eines Herstellers, so müssen in allen
Datensätzen die Telefonnummer geändert werden.

\bPseudoUeberschrift{Delete-Anomalie}\index{Delete-Anomalie}

Wird die Datenbank aufgeräumt, d. h. alte Bestellungen gelöscht,
so verschwindet auch die Hersteller-Telefonnumer von manchen
Herstelllern.
\end{bAntwort}

%%
% (d)
%%

\item Überführen Sie das obige Relationenschema schrittweise in die 3.
Normalform! Erläutern Sie die dazu durchzuführenden Schritte jeweils
kurz!\index{Synthese-Algorithmus}

\begin{bAntwort}

\begin{enumerate}
\item \schrittE{1}

\begin{enumerate}
\item \schrittE{1-1}

Die einzige FD mit einer Determinante bestehtend aus mehrere Attributen,
ist \fa{ArtNr, Lieferant, BestDat -> Anzahl}

\begin{itemize}
\item ohne \m{ArtNr}

\begin{liAHuelle}
\m{Anzahl} \notin \ah{F, \m{Lieferant, BestDat}} = \\
\m{
  Lieferant,
  BestDat,
  LiefTel
}
\end{liAHuelle}

\item ohne \m{Lieferant}

\begin{liAHuelle}
Anzahl \notin \ah{F, \m{ArtNr, BestDat}} = \\
\m{
  ArtNr,
  BestDat,
  ArtName,
  ArtArt,
  Hersteller,
  HerstTel
}
\end{liAHuelle}

\item ohne \m{BestDat}

\begin{liAHuelle}
Anzahl \notin \ah{F, \m{ArtNr, Lieferant}} = \\
\m{
  ArtNr,
  Lieferant,
  ArtName,
  ArtArt,
  Hersteller,
  HerstTel,
  LiefTel
}
\end{liAHuelle}

\end{itemize}

Die linke Seiten der FDs können nicht reduziert werden.

\item \schrittE{1-2}

Das einzige Attribut, dass auf der rechten Seite der FDs doppelt vorkommt
ist \m{HerstTel}

\begin{liAHuelle}
\m{HerstTel} \in \ah{F - \fa{ArtNr -> HerstTel}, \m{ArtNr}} = \\
\m{
  ArtNr,
  ArtName,
  ArtArt,
  Hersteller,
  HerstTel
}
\end{liAHuelle}

\item \schrittE{1-3}

\bNichtsZuTun

\item \schrittE{1-4}

\bNichtsZuTun
\end{enumerate}

Kanonische Überdeckung:

\FA[$F_c$]{
ArtNr -> ArtName, ArtArt, Hersteller;
Hersteller -> HerstTel;
Lieferant -> LiefTel;
ArtNr, Lieferant, BestDat -> Anzahl;
}

\item \schrittE{2}

\bRelation[R1]{ArtNr, ArtName, ArtArt, Hersteller}\\
\bRelation[R2]{Hersteller, HerstTel}\\
\bRelation[R3]{Lieferant, LiefTel}\\
\bRelation[R4]{ArtNr, Lieferant, BestDat, Anzahl}

\item \schrittE{3}

Es muss noch eine Relation hinzugefügt werden, nämlich kommt das
Attribut \m{EP} bisher in keiner Relation vor.

\bRelation[R1]{ArtNr, ArtName, ArtArt, Hersteller}\\
\bRelation[R2]{Hersteller, HerstTel}\\
\bRelation[R3]{Lieferant, LiefTel}\\
\bRelation[R4]{ArtNr, Lieferant, BestDat, Anzahl}\\
\bRelation[R5]{ArtNr, Lieferant, BestDat, EP}

\item \schrittE{4}

Nicht zu tun.

\end{enumerate}

Ergebnis:

\bRelation[R1]{ArtNr, ArtName, ArtArt, Hersteller}\\
\bRelation[R2]{Hersteller, HerstTel}\\
\bRelation[R3]{Lieferant, LiefTel}\\
\bRelation[R4]{ArtNr, Lieferant, BestDat, Anzahl}\\
\bRelation[R5]{ArtNr, Lieferant, BestDat, EP}

\end{bAntwort}

%%
% (e)
%%

\item Erläutern Sie, inwiefern sich eine vollständige Normalisierung
nachteilig auf die Geschwindigkeit der Anfragebearbeitung auswirken kann
und wie darauf reagiert werden kann!

\begin{bAntwort}
Eine vollständige Normalisierung hat den Effekt, dass die Daten auf mehr
Relation bzw. Tabellen aufgeteilt werden. In der Regel geht damit
einher, dass bei Abfragen mehr Joins durchgeführt werden müssen, was
in der Regel mit mehr Speicherbedarf und Rechenzeit der Anfragen
einhergeht.

Man könnte auf eine Normalisierung verzichten, oder nur teilweise
normalisieren und somit zwischen Performance und Redundanz abwägen.
\end{bAntwort}
\end{enumerate}
\end{document}
