\documentclass{bschlangaul-aufgabe}
\bLadePakete{normalformen,rmodell}
\begin{document}
\bAufgabenMetadaten{
  Titel = {Aufgabe 6: Normalformen},
  Thematik = {Arbeitsvermittler},
  Referenz = DB.Relationale-Entwurfstheorie.Normalformen.Synthesealgorithmus.Arbeitsvermittler,
  RelativerPfad = Module/10_DB/50_Relationale-Entwurfstheorie/30_Normalformen/10_Synthesealgorithmus/Aufgabe_Arbeitsvermittler.tex,
  ZitatSchluessel = db:ab:7,
  ZitatBeschreibung = {Aufgabe 6},
  BearbeitungsStand = mit Lösung,
  Korrektheit = unbekannt,
  Ueberprueft = {unbekannt},
  Stichwoerter = {Normalformen, Funktionale Abhängigkeiten, Schlüsselkandidat, Synthese-Algorithmus},
}

\let\fa=\bFunktionaleAbhaengigkeit
\let\m=\bAttributMenge
\let\r=\bRelationMenge

Ein privater Arbeitsvermittler legt eine relationale Datenbank an, in
der u.a. die folgenden Informationen gespeichert werden:
\index{Normalformen}
\footcite[Aufgabe 6]{db:ab:7}

Zu jeder gemeldeten offenen Stelle werden der Name und die Adresse des
Arbeitgebers gespeichert, ebenso die genaue Stellenbezeichnung, das
Jahresgehalt und das Datum, ab dem die Stelle zu besetzen ist. Außerdem
wird vermerkt, ob es sich um eine befristete oder unbefristete
Anstellung handelt. Verschiedene Arbeitgeber können Stellen mit
derselben Stellenbezeichnung anbieten, ebenfalls kann ein Arbeitgeber
mehrere Niederlassungen haben.

Aus der Adresse kann nicht auf den Namen des Arbeitgebers geschlossen
werden (Beispiel Bürohochhaus). Ein Arbeitgeber hat jedoch nicht mehrere
Niederlassungen am selben Ort. Außerdem sei der Einfachheit halber
vorausgesetzt, dass jeder Arbeitgeber pro Niederlassung nur eine Stelle
mit einer bestimmten Bezeichnung zu vergeben hat, und jeder Bewerber nur
einmal vermittelt wird. Aus der Stellenbezeichnung lässt sich bereits
ersehen, ob die Stelle befristet ist oder nicht. Des Weiteren weist eine
Stellenbezeichnung darauf hin, welcher Branchenbezeichnung sie
zuzuordnen ist.

Von jedem Arbeitsuchenden werden der Name und die Adresse, die
Telefonnummer, der erlernte Beruf und das Geburtsdatum des Bewerbers
gespeichert. Zusätzlich soll direkt abrufbar sein, ob der Bewerber
bereits älter als 25 Jahre (schwer zu vermitteln!) ist. Als Kriterium
gilt dabei der Zeitpunkt der Meldung beim Arbeitsvermittler und nicht
der Zeitpunkt eines möglichen Stellenantritts.

Aus diesen Vorgaben ergibt sich (beispielsweise) folgendes relationales
Schema (Relationen mindestens in 1.NF):

\begin{itemize}
\item \r{Stellen}{Stellenbezeichnung, AG-Name, AG-Adresse, Gehalt,
Einstellungsdatum, befristet}

\item \r{Bewerber}{Name, Adresse, Gebdatum, Beruf, TelNr,
Antrittsdatum, aelter25}

\item \r{Branche}{Branchenbezeichnung, Bedarf}

\item \r{gehoert\_zu}{Stellenbezeichnung, AG-Name, AG-Adresse,
Branchenbezeichnung}

\item \r{vermittelt}{Stellenbezeichnung, AG-Name, AG-Adresse,
Name, Adresse}
\end{itemize}

\begin{enumerate}

%%
% (a)
%%

\item Welche funktionalen Abhängigkeiten\index{Funktionale
Abhängigkeiten} gibt es bzgl. der einzelnen Relationen?

\begin{bAntwort}
\begin{itemize}
\item \textbf{Stellen}: \\
\fa{Stellenbezeichnung, AG-Name, AG-Adresse -> Gehalt} \\
\fa{Stellenbezeichnung, AG-Name, AG-Adresse -> Einstellungsdatum} \\
\fa{Stellenbezeichnung -> befristet}

\item \textbf{Bewerber}: \\
\fa{Name, Adresse -> Gebdatum} \\
\fa{Name, Adresse -> Beruf} \\
\fa{Name, Adresse -> TelNr} \\
\fa{Name, Adresse -> Antrittsdatum} \\
\fa{Name, Adresse -> aelter25} \\
\fa{Name, Gebdatum -> aelter25}

\item \textbf{Branche}: \\
\fa{Branchenbezeichnung -> Bedarf}

\item \textbf{gehoert\_zu}: \\
\fa{Stellenbezeichnung -> Branchenbezeichnung}

\item \textbf{vermittelt}: \\
\fa{Stellenbezeichnung, AG-Name, AG-Adresse -> Name} \\
\fa{Stellenbezeichnung, AG-Name, AG-Adresse -> Adresse} \\
\fa{Name, Adresse -> Stellenbezeichnung} \\
\fa{Name, Adresse -> AG-Name} \\
\fa{Name, Adresse -> AG-Adresse}
\end{itemize}

\end{bAntwort}

%%
% (b)
%%

\item Wie lauten die Schlüsselkandidaten\index{Schlüsselkandidat} der
einzelnen Relationen?

\begin{bAntwort}
\begin{itemize}
\item \textbf{Stellen}: \\
\m{ Stellenbezeichnung, AG-Name, AG-Adresse }

\item \textbf{Bewerber}: \\
\m{ Name, Adresse }

\item \textbf{Branche}: \\
\m{ Branchenbezeichnung }

\item \textbf{gehoert\_zu}: \\
\m{ Stellenbezeichnung, AG-Name, AG-Adresse }

\item \textbf{vermittelt}: \\
\m{ Stellenbezeichnung, AG-Name, AG-Adresse }, \m{ Name, Adresse } \\
\end{itemize}

\end{bAntwort}

%%
% (c)
%%

\item Überprüfen Sie, welche Normalformen bei den einzelnen
Relationenschemata vorliegen!

%%
% (d)
%%

\item Überführen Sie die Relationenschemata in die 3. Normalform!
\index{Synthese-Algorithmus}
\end{enumerate}
\end{document}
