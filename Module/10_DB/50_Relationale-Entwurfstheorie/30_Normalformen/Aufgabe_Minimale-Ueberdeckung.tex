\documentclass{bschlangaul-aufgabe}
\bLadePakete{normalformen}
\begin{document}
\bAufgabenMetadaten{
  Titel = {Minimale Überdeckung},
  Thematik = {Minimale Überdeckung},
  Referenz = DB.Relationale-Entwurfstheorie.Normalformen.Minimale-Ueberdeckung,
  RelativerPfad = Module/10_DB/50_Relationale-Entwurfstheorie/30_Normalformen/Aufgabe_Minimale-Ueberdeckung.tex,
  ZitatSchluessel = db:ab:5,
  ZitatBeschreibung = {Seite 1, Aufgabe 2},
  BearbeitungsStand = mit Lösung,
  Korrektheit = unbekannt,
  Ueberprueft = {unbekannt},
  Stichwoerter = {Kanonische Überdeckung},
}

\let\fa=\bFunktionaleAbhaengigkeit
\let\FA=\bFunktionaleAbhaengigkeiten
\let\ah=\bAttributHuelle
\let\m=\bAttributMenge

Gegeben ist die Menge \FA[$F$]{
  A -> B, C;
  C -> D, A;
  E -> A, C;
  C, D -> B, E;
}. Bestimmen Sie eine minimale Überdeckung von $F$.
\index{Kanonische Überdeckung}
\footcite[Seite 1, Aufgabe 2]{db:ab:5}

\begin{bAntwort}

\begin{enumerate}
\item Linksreduktion

$\ah{F, \m{D}} = \m{D}$ \\
$\ah{F, \m{C}} = \m{C, D, A, B, E}$ \\

\FA[$F'$]{
  A -> B, C;
  C -> D, A;
  E -> A, C;
  C -> B, E;
}

\item Rechtsreduktion

$\ah{F - \fa{A -> B, C}, \m{A}} = \m{A}$ \\
$\ah{F - \m{C \rightarrow DA}, \m{C}} = \m{C, B, E, A}$ \\
$\ah{F - \m{E \rightarrow AC}, \m{E}} = \m{E}$ \\
$\ah{F - \m{C \rightarrow BE}, \m{C}} = \m{C, D, A, B}$ \\
Keine Rechtsreduktion möglich

schon minimal

\end{enumerate}

\end{bAntwort}
\end{document}
