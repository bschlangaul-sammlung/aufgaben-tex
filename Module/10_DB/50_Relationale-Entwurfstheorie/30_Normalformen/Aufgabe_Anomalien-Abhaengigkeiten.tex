\documentclass{bschlangaul-aufgabe}
\bLadePakete{normalformen}

\begin{document}
\bAufgabenMetadaten{
  Titel = {Anomalien und Abhängigkeiten},
  Thematik = {Anomalien Abhängigkeiten},
  Referenz = DB.Relationale-Entwurfstheorie.Normalformen.Anomalien-Abhaengigkeiten,
  RelativerPfad = Module/10_DB/50_Relationale-Entwurfstheorie/30_Normalformen/Aufgabe_Anomalien-Abhaengigkeiten.tex,
  ZitatSchluessel = db:ab:5,
  ZitatBeschreibung = {Seite 1, Aufgabe 1: Anomalien und Abhängigkeiten, Aufgabe 1},
  BearbeitungsStand = mit Lösung,
  Korrektheit = unbekannt,
  Ueberprueft = {unbekannt},
  Stichwoerter = {Update-Anomalie, Delete-Anomalie, Insert-Anomalie, Funktionale Abhängigkeiten, Attributhüllen-Algorithmus, Attributhülle, Superschlüssel},
}

\let\fa=\bFunktionaleAbhaengigkeit
\let\FA=\bFunktionaleAbhaengigkeiten
\let\ah=\bAttributHuelle
\let\m=\bAttributMenge

Gegeben ist die Relation \emph{Abteilungsmitarbeiter}, repräsentiert
durch folgende Tabelle. Es sei angenommen, dass innerhalb einer
Abteilung keine Mitarbeiter mit identischem Namen existieren. Die
Abteilungsnummer ist eindeutig, es kann aber durchaus sein, dass mehrere
Abteilungen die gleiche Bezeichnung tragen.
\footcite[Seite 1, Aufgabe 1: Anomalien und Abhängigkeiten, Aufgabe 1]{db:ab:5}

\begin{center}
\begin{tabular}{|l|l|l|l|l|}
\hline
\textbf{Name}  & \textbf{Straße} & \textbf{Ort} & \textbf{AbtNr} & \textbf{Bezeichnung} \\\hline
Schweizer      & Hauptstraße     & Zürich       & A3             & Finanzen   \\\hline
Deutscher      & Lindenstraße    & Passau       & A4             & Informatik \\\hline
Österreicher   & Nebenstraße     & Wien         & A4             & Informatik \\\hline
\end{tabular}
\end{center}

\begin{enumerate}

%%
% (a)
%%

\item Geben Sie - orientiert an der obigen Tabelle - ein Beispiel für
eine mögliche Änderungsanomalie an!

\begin{bAntwort}
\begin{description}
\item[Update-Anomalie]
\index{Update-Anomalie}

Die Abteilung \emph{A4} wird umbenannt, beispielsweise in
\emph{Softwareabteilung}. Die Änderung wird aus Versehen nicht in allen
Tupeln mit \emph{AbtNr = A4} vollzogen.

\item[Delete-Anomalie]
\index{Delete-Anomalie}

Herr \emph{Schweizer} (aus der Abteilung A4) verlässt die Firma und wird aus
der Datenbank gelöscht. Damit gehen auch die Daten über die Abteilung \emph{A3}
verloren.

\item[Insert-Anomalie]
\index{Insert-Anomalie}

Es wird eine neue Abteilung \emph{A5} (\emph{Hardwareabteilung})
geschaffen, der aber noch keine Mitarbeiter zugeteilt sind. Damit müsste
ein Tupel (NULL, NULL, NULL, A5, Hardwareabteilung) in die Datenbank
eingefügt werden. Da aber das Attribut \emph{Name} sicher in jedem
Schlüsselkandidaten enthalten sein muss, kann der Wert von Name keinen
Nullwert enthalten. Das Tupel kann nicht eingefügt werden.

\end{description}
\end{bAntwort}

%%
% (b)
%%

\item Bestimmen Sie eine Menge $F$ der funktionalen Abhängigkeiten, die
sich aus Ihrer Analyse des Anwendungsbereiches ergeben. (Triviale
Abhängigkeiten brauchen nicht angegeben werden.) Begründen Sie Ihre
Entscheidung kurz.
\index{Funktionale Abhängigkeiten}

\begin{bAntwort}
\begin{itemize}
\item \fa{AbtNr -> Bezeichnung}

Die \emph{Abteilungsnummer} ist eindeutig (als „künstliches“
Unterscheidungsmerkmal für \emph{Abteilungen}) und legt damit die
\emph{Abteilung}
eindeutig fest.

\item \fa{Name, AbtNr -> Strasse, Ort}

Da der \emph{Name} innerhalb der \emph{Abteilung} eindeutig ist, ist
damit der \emph{Mitarbeiter} und folglich auch die \emph{Adressdaten}
eindeutig festgelegt. Da es sich bei dieser Attributkombination um den
Primärschlüssel handelt, bestimmt diese Attributkombination auch das
Attribut \emph{Bezeichnung}, allerdings darf es nicht in diese
Funktionale Abhängigkeit aufgenommen werden, da die
Abteilungsbezeichnung nicht von der Kombination aus \emph{Name} \&
\emph{AbtNr} abhängig, sondern nur von der \emph{AbtNr} allein, somit
muss dies als einzelne Funktionale Abhängigkeit formuliert werden und
kann hier nicht aufgenommen werden

$\rightarrow$ der Rückschluss daraus wäre nämlich, dass sich die
\emph{Bezeichnung} der \emph{Abteilung} nur aus der Kombination von
\emph{Mitarbeiter} und \emph{AbtNr} erkennen lässt und nicht allein aus
der \emph{AbtNr} und das wäre ja nicht korrekt. Grundsätzlich gilt:
Primärschlüssel und Funktionale Abhängigkeiten müssen getrennt
betrachtet werden.

\FA[F]{
  Name, AbtNr -> Strasse, Ort;
  AbtNr -> Bezeichnung;
}
\end{itemize}

\end{bAntwort}

%%
% (c)
%%

\item Bestimmen Sie \zB mit Hilfe des Attributhüllen-Algorithmus die
Attributhülle
\index{Attributhüllen-Algorithmus}\index{Attributhülle}

\begin{enumerate}

%%
% 1.
%%

\item $\ah{F, \m{Name, Bezeichnung}}$

\begin{bAntwort}
$\ah{F, \m{Name, Bezeichnung}} = \m{Name, Bezeichnung}$
\end{bAntwort}

%%
% 2.
%%

\item $\ah{F, \m{Name, AbtNr}}$

\begin{bAntwort}
$\ah{F, \m{Name, AbtNr}} = \m{Name, AbtNr, Strasse, Ort, Bezeichnung}$
\end{bAntwort}
\end{enumerate}

%%
% (d)
%%

\item Ist \m{Name, Bezeichnung} bzw. \m{Name, AbtNr} ein Superschlüssel
der Relation Abteilungsmitarbeiter? Kurze Begründung!
\index{Superschlüssel}

\begin{bAntwort}
\m{Name, AbtNr}, da die Attributhülle von \m{Name, AbtNr} alle Attribute
der Relation umfasst und \m{Name, Bezeichnung} nicht.
\end{bAntwort}

\end{enumerate}
\end{document}
