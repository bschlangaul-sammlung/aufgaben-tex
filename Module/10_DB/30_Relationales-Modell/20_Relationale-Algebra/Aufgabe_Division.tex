\documentclass{bschlangaul-aufgabe}
\bLadePakete{spalten}
\begin{document}
\bAufgabenMetadaten{
  Titel = {Aufgabe 1: Division},
  Thematik = {Division},
  Referenz = DB.Relationales-Modell.Relationale-Algebra.Division,
  RelativerPfad = Module/10_DB/30_Relationales-Modell/20_Relationale-Algebra/Aufgabe_Division.tex,
  ZitatSchluessel = db:ab:4,
  ZitatBeschreibung = {Aufgabe 1},
  BearbeitungsStand = mit Lösung,
  Korrektheit = unbekannt,
  Ueberprueft = {unbekannt},
  Stichwoerter = {Relationale Algebra, Division},
}

\def\r#1{\textcolor{red}{#1}}
\def\g#1{\textcolor{green}{#1}}
\def\o#1{\textcolor{orange}{#1}}
\def\b#1{\textcolor{blue}{#1}}

Gegeben sind zwei Relationen, repräsentiert als Tabellen. Bestimmen Sie
die Ergebnisrelation $R_1 \div R_2$!\index{Relationale Algebra}
\index{Division}
\footcite[Aufgabe 1]{db:ab:4}

\begin{multicols}{2}

\bPseudoUeberschrift{$R_1$}

\begin{tabular}{lllll}
A & B & C & D & E \\
a & r & 4 & 3 & t \\
c & r & 2 & 3 & t \\%
b & w & d & 4 & s \\
a & b & r & 3 & t \\
b & r & d & 3 & t \\
a & b & w & 4 & s \\
a & w & 4 & 4 & s \\%
b & k & d & 2 & s \\
c & w & 3 & 4 & s
\end{tabular}

\bPseudoUeberschrift{$R_2$}

\begin{tabular}{lll}
B & D & E \\
r & 3 & t \\
w & 4 & s
\end{tabular}
\end{multicols}

\begin{bAntwort}
\begin{multicols}{2}
\bPseudoUeberschrift{$R_1$}

\begin{tabular}{lllll}
A & B & C & D & E \\
\r a & \b r & \r 4 & \b 3 & \b t \\
c & \b r & 2! & \b 3 & \b t \\%
\g b & \o w & \g d & \o 4 & \o s \\
a & b & r & 3 & t \\
\g b & \b r & \g d & \b 3 & \b t \\
a & b & w & 4 & s \\
\r a & \o w & \r 4 & \o 4 & \o s \\%
b & k & d & 2 & s \\
c & \o w & 3! & \o 4 & \o s
\end{tabular}

\bPseudoUeberschrift{$R_2$}

\begin{tabular}{lll}
B & D & E \\
\b r & \b 3 & \b t \\
\o w & \o 4 & \o s
\end{tabular}
\end{multicols}

\bPseudoUeberschrift{$R_1 \div R_2$}

\begin{center}
\begin{tabular}{ll}
\textbf{A} & \textbf{C} \\
\r a & \r 4 \\
\g b & \g d
\end{tabular}
\end{center}
\end{bAntwort}

\end{document}
