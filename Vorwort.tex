\documentclass{bschlangaul-basis}

\begin{document}
\section{Vorwort}

\subsection{Abkürzungen der Modulnamen}

\begin{tabular}{|l|l|}
Abk.  & Modulename                                          \\\hline
MATHE & Mathematische Grundlagen                            \\
DB    & Datenbanksysteme                                    \\
OOMUP & Objektorientierte Modellierung und Programmierung   \\
AUD   & Algorithmen und Datenstrukturen                     \\
FUMUP & Funktionale Modellierung und Programmierung         \\
SOSY  & Softwaresysteme und Softwareentwicklungspraktikum   \\
TECH  & Technische Informatik                               \\
THEO  & Theoretische Informatik                             \\
DDI   & Didaktik der Informatik                             \\
\end{tabular}

\subsection{Repositories}

\def\TmpRepo#1#2{\item[#1] \strut \par

{\footnotesize \url{https://github.com/bschlangaul-sammlung/#1}}

\par #2

}

\begin{description}
\TmpRepo{examens-aufgaben-tex}{Haupt-Repository des Projekts. LaTeX-Quelltexte. Sammlung von Examensaufgaben und weiteren, zusätzlichen Übungsaufgaben mit Lösungen für das Studium „Lehramt Informatik“ in Bayern.}
\TmpRepo{examens-aufgaben-pdf}{}
\TmpRepo{examen-scans}{}
\TmpRepo{latex-vorlagen}{}
\TmpRepo{java-fuer-examens-aufgaben}{Java-Code zum Einbetten in die LaTeX-Quelltexte der Aufgaben, Implementation von einigen Datenstrukturen (z. B. AVL-Baum), Kommandozeilen-Werkzeug}
\TmpRepo{kommandozeilen-werkzeug}{}
\TmpRepo{logo-grafiken}{}
\end{description}
\end{document}
