\documentclass{bschlangaul-aufgabe}
\bLadePakete{vollstaendige-induktion}
\begin{document}
\bAufgabenMetadaten{
  Titel = {Aufgabe 5},
  Thematik = {drei hoch},
  Referenz = 66112-2003-H.T2-A5,
  RelativerPfad = Staatsexamen/66112/2003/09/Thema-2/Aufgabe-5.tex,
  ZitatSchluessel = examen:66112:2003:09,
  ZitatBeschreibung = {Thema 2 Aufgabe 5},
  BearbeitungsStand = mit Lösung,
  Korrektheit = unbekannt,
  Ueberprueft = {unbekannt},
  Stichwoerter = {Vollständige Induktion},
  EinzelpruefungsNr = 66112,
  Jahr = 2003,
  Monat = 09,
  ThemaNr = 2,
  AufgabeNr = 5,
}

\let\m=\bInduktionMarkierung
\let\e=\bInduktionErklaerung

Zeigen Sie mit Hilfe vollständiger Induktion, dass das folgende
Programm bzgl. der Vorbedingung $x > 0$ und der Nachbedingung drei\_hoch
$x =  3^x$ partiell korrekt ist!\index{Vollständige Induktion}
\footcite[Thema 2 Aufgabe 5]{examen:66112:2003:09}

\begin{minted}{lisp}
(define (drei_hoch x)
  (cond ((= x 0) 1)
    (else (* 3 (drei_hoch (- x 1))))
  )
)
\end{minted}

\def\drei{\text{drei\_hoch}\,}

\begin{bAntwort}

%%
%
%%

\bInduktionAnfang

$\drei 1 = 3 \cdot (\drei 0) = 3 \cdot 1 = 3$

%%
%
%%

\bInduktionVoraussetzung

für alle $x < x 0$ gilt $\drei x = 3 x$

%%
%
%%

\bInduktionSchritt

x->x+1

\begin{align*}
\drei (x + 1)
& = 3 \cdot \drei (- (x + 1) 1))\\
& = 3 \cdot (\drei x)\\
& = 3 \cdot 3^x\\
& = 3^{x+1}
\end{align*}
\end{bAntwort}
\end{document}
