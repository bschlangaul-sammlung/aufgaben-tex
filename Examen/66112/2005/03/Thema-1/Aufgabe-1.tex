\documentclass{bschlangaul-aufgabe}
\bLadePakete{java,uml}
\begin{document}
\bAufgabenMetadaten{
  Titel = {Aufgabe 14},
  Thematik = {Klasse „DoublyLinkedList“},
  Referenz = 66112-2005-F.T1-A1,
  RelativerPfad = Staatsexamen/66112/2005/03/Thema-1/Aufgabe-1.tex,
  ZitatSchluessel = aud:pu:7,
  ZitatBeschreibung = {entnommen aus Algorithmen und Datenstrukturen,
Präsenzübung 4, Universität Bayreuth, Aufgabe 14},
  BearbeitungsStand = mit Lösung,
  Korrektheit = unbekannt,
  Ueberprueft = {unbekannt},
  Stichwoerter = {Klassendiagramm},
  EinzelpruefungsNr = 66112,
  Jahr = 2005,
  Monat = 03,
  ThemaNr = 1,
  AufgabeNr = 1,
}

Betrachten\footcite[entnommen aus Algorithmen und Datenstrukturen,
Präsenzübung 4, Universität Bayreuth, Aufgabe 14]{aud:pu:7} Sie
folgendes Klassendiagramm\index{Klassendiagramm}, das doppelt-verkettete
Listen spezifiziert. Die Assoziation \bJavaCode{head} zeigt auf das
erste Element der Liste. Die Assoziationen \bJavaCode{previous} und
\bJavaCode{next} zeigen auf das vorherige bzw. folgende Element.
\footcite{examen:66112:2005:03}

\begin{center}
\begin{tikzpicture}[scale=0.8, transform shape]
\tiny
\umlclass
{DoublyLinkedList}{}
{
  insert(i: Integer)\\
  check(): Boolean
}

\umlclass[x=5]
{ListElem}{
  data: Integer
}
{
  insert(i: Integer)\\
}

\umluniassoc[arg=head,mult=1,pos=0.7]{DoublyLinkedList}{ListElem}

\umlassoc[%
  pos1=0.3,
  arg1=previous,
  arg2=next,
  mult1=1,
  mult2=1,
  angle1=-30,
  angle2=30,
  loopsize=3cm,
]{ListElem}{ListElem}
\end{tikzpicture}
\end{center}

\noindent
Implementieren Sie die doppelt-verketteten Listen in einer geeigneten
objektorientierten Sprache (\zB Java oder C++), das heißt:

\begin{enumerate}

%%
% (a)
%%

\item Implementieren Sie die Klasse \bJavaCode{ListElem}. Die Methode
\bJavaCode{insert} ordnet eine ganze Zahl \bJavaCode{i} in eine
aufsteigend geordnete doppelt-verkettete Liste \bJavaCode{l} an die
korrekte Stelle ein. Sei \zB das Objekt \bJavaCode{l} eine
Repräsentation der Liste \bJavaCode{[0, 2, 2, 6, 8]} dann liefert
\bJavaCode{l.insert(3)} eine Repräsentation der Liste \bJavaCode{[0,
2, 2, 3, 6, 8]}.

\begin{bAntwort}
\bJavaExamen{66112}{2005}{03}{ListElem}
\end{bAntwort}

%%
% (b)
%%

\item Implementieren Sie die Klasse \bJavaCode{DoublyLinkedList}, wobei
die Methode \bJavaCode{insert} eine Zahl \bJavaCode{i} in eine
aufsteigend geordnete Liste einordnet. Die Methode \bJavaCode{check}
überprüft, ob eine Liste korrekt verkettet ist, \dh ob für jedes
ListElem-Objekt \bJavaCode{o}, das über den \bJavaCode{head} der Liste
erreichbar ist, der Vorgänger des Nachfolgers von \bJavaCode{o} gleich
\bJavaCode{o} ist.
\end{enumerate}

\begin{bAntwort}
\bJavaExamen{66112}{2005}{03}{DoublyLinkedList}
\end{bAntwort}
\end{document}
