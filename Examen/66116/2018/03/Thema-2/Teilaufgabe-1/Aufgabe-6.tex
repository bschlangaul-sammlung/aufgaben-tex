
\documentclass{bschlangaul-aufgabe}
\bLadePakete{normalformen, synthese-algorithmus}
\begin{document}
\bAufgabenMetadaten{
  Titel = {Aufgabe 6},
  Thematik = {Synthese-Algorithmus bei Relationenschema A-F},
  Referenz = 66116-2018-F.T2-TA1-A6,
  RelativerPfad = Examen/66116/2018/03/Thema-2/Teilaufgabe-1/Aufgabe-6.tex,
  ZitatSchluessel = db:pu:wh,
  BearbeitungsStand = unbekannt,
  Korrektheit = unbekannt,
  Ueberprueft = {unbekannt},
  Stichwoerter = {Synthese-Algorithmus, Dritte Normalform},
  EinzelpruefungsNr = 66116,
  Jahr = 2018,
  Monat = 03,
  ThemaNr = 2,
  TeilaufgabeNr = 1,
  AufgabeNr = 6,
}

\let\ah=\bAttributHuelle
\let\ahl=\bLinksReduktionInline
\let\ahr=\bRechtsReduktionInline
\let\FA=\bFunktionaleAbhaengigkeiten
\let\fa=\bFunktionaleAbhaengigkeit
\let\m=\bAttributMenge
\let\r=\bRelation
\let\schrittE=\bSyntheseUeberErklaerung
\let\u=\underline

Gegeben sei das Relationenschema \r{A,B,C,D,E,F}, sowie die Menge der
zugehörigen funktionalen Abhängigkeiten F.\footcite{db:pu:wh}

% C -> B
% B -> A
% C, E -> D
% E -> F
% C, E -> F
% C -> A

\FA{
  C -> B;
  B -> A;
  C, E -> D;
  E -> F;
  C, E -> F;
  C -> A;
}

\begin{enumerate}
\item Bestimmen Sie den Schlüsselkandidaten der Relation $R$ und
begründen Sie, warum es keine weiteren Schlüsselkandidaten gibt.

\begin{bAntwort}
$C$ und $E$ kommen auf keiner rechten Seite vor. Sie müssen deshalb
immer Teil des Schlüsselkandidaten sein.

\ah{F, \m{C, E}} = \m{A,B,C,D,E,F}

Daraus folgt, dass \m{C, E} ein Superschlüssel ist.

\ahl{C, E}{E}{A, B, C} $\neq R$\\
\ahl{C, E}{C}{E, F} $\neq R$

\m{C, E} kann nicht weiter minimiert werden.
\end{bAntwort}

\item Überführen Sie das Relationenschema R mit Hilfe des
Synthesealgorithmus\index{Synthese-Algorithmus} in die dritte
Normalform\index{Dritte Normalform}. Führen Sie hierfür jeden der vier
Schritte durch und kennzeichnen Sie Stellen, bei denen nichts zu tun
ist.

\begin{bAntwort}
\begin{itemize}

\item \schrittE{1}

\begin{itemize}

\item \schrittE{1-1}

\bPseudoUeberschrift{\fa{C, E -> D}}

$D \notin$ \ahl{C, E}{E}{A, C, B}\\
$D \notin$ \ahl{C, E}{C}{E, F}

\bPseudoUeberschrift{\fa{C, E -> F}}

$F \notin$ \ahl{C, E}{E}{A, C, B}\\
$F \in$ \ahl{C, E}{C}{E, \textbf{F}}

\FA{
  C -> B;
  B -> A;
  C, E -> D;
  E -> F;
  E -> F;
  C -> A;
}

\item \schrittE{1-2}

\bPseudoUeberschrift{A}

$A \notin$ \ahr{B -> A}{}{B}{B}\\
$A \in$ \ahr{C -> A}{}{C}{\textbf{A},B,C}

\FA{
  C -> B;
  B -> A;
  C, E -> D;
  E -> F;
  E -> F;
  C -> NICHTS;
}

\bPseudoUeberschrift{F}

$F \in$ \ahr{E -> F}{}{E}{E, \textbf{F}}

\FA{
  C -> B;
  B -> A;
  C, E -> D;
  E -> NICHTS;
  E -> F;
  C -> NICHTS;
}

\item \schrittE{1-3}

\FA{
  C -> B;
  B -> A;
  C, E -> D;
  E -> F;
}

\item \schrittE{1-4}

\bNichtsZuTun
\end{itemize}

\item \schrittE{2}

\r[R1]{\u{C}, B}\\
\r[R2]{\u{B}, A}\\
\r[R3]{\u{C, E}, D}\\
\r[R4]{\u{E}, F}\\

\item \schrittE{3}

\bNichtsZuTun

\item \schrittE{4}
\bNichtsZuTun

\end{itemize}

\end{bAntwort}

\end{enumerate}

\end{document}
