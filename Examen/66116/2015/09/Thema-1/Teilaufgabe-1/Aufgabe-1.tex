\documentclass{bschlangaul-aufgabe}

\begin{document}
\bAufgabenMetadaten{
  Titel = {1. Modellierung},
  Thematik = {Online-Auktionshaus},
  Referenz = 66116-2015-H.T1-TA1-A1,
  RelativerPfad = Staatsexamen/66116/2015/09/Thema-1/Teilaufgabe-1/Aufgabe-1.tex,
  ZitatSchluessel = examen:66116:2015:09,
  BearbeitungsStand = mit Lösung,
  Korrektheit = unbekannt,
  Ueberprueft = {unbekannt},
  Stichwoerter = {Entity-Relation-Modell},
  EinzelpruefungsNr = 66116,
  Jahr = 2015,
  Monat = 09,
  ThemaNr = 1,
  TeilaufgabeNr = 1,
  AufgabeNr = 1,
}

Wir wollen eine relationale Datenbankstruktur für ein
Online-Auktionshaus modellieren.\index{Entity-Relation-Modell}
\footcite{examen:66116:2015:09}

Das Auktionshaus hat Mitglieder. Diese Mitglieder haben Kundennummern,
Namen und Adressen. Sie können Verkäufer und/oder Käufer sein. Verkäufer
können eine Laden-URL (eine Subdomain) erhalten, die zu einer Seite mit
ihren aktuellen Auktionen führt. Verkäufer können neue Auktionen
starten. Die Auktionen eines Verkäufers werden durchnummeriert. Diese
Auktionsnummer ist nur je Verkäufer eindeutig. Jede Auktion hat ein
Mindestgebot und eine Ablaufzeit.

Auf Auktionen können Gebote abgegeben werden. Die Gebote auf eine
Auktion werden nach ihrem Eintreffen nummeriert. Diese Nummer ist nur
innerhalb einer Auktion eindeutig. Zu einem Gebot werden noch die Zeit
des Gebots sowie der gebotene Geldbetrag angegeben. Die Gebote werden
von Käufern abgegeben. Jedes Gebot muss einem Käufer zugeordnet sein.

Jede Auktion besteht aus einer Menge von Artikeln, aber aus mindestens
einem. Artikel haben eine Beschreibung und können und über ihre
Artikel-ID identifiziert werden. Zur Katalogisierung der Artikel gibt es
Kategorien. Kategorien haben eine eindeutige ID und einen Namen. Jede
Kategorie kann Subkategorien besitzen. Auch Subkategorien sind
Kategorien (und können damit weitere Subkategorien besitzen). Jeder
Artikel kann beliebig vielen Kategorien zugeordnet werden.

Entwerfen Sie für das beschriebene Szenario ein ER-Diagramm. Bestimmen
Sie hierzu:

\begin{itemize}
\item die Entity-Typen, die Relationship-Typen und jeweils deren
Attribute,

\item ein passendes ER-Diagramm,

\item die Primärschlüssel der Entity-Typen, welche Sie anschließend in
das ER-Diagramm eintragen,

\item die Funktionalitäten der Relationship-Typen, welche Sie ebenfalls
in das ER-Diagramm eintragen.
\end{itemize}

\end{document}
