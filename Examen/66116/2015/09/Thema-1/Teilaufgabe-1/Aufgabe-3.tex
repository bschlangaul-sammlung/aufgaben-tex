\documentclass{bschlangaul-aufgabe}
\bLadePakete{syntax,rmodell}
\begin{document}
\bAufgabenMetadaten{
  Titel = {Aufgabe 3},
  Thematik = {Vater und Mutter},
  Referenz = 66116-2015-H.T1-TA1-A3,
  RelativerPfad = Staatsexamen/66116/2015/09/Thema-1/Teilaufgabe-1/Aufgabe-3.tex,
  ZitatSchluessel = examen:66116:2015:09,
  BearbeitungsStand = mit Lösung,
  Korrektheit = unbekannt,
  Ueberprueft = {unbekannt},
  Stichwoerter = {SQL, SQL mit Übungsdatenbank},
  EinzelpruefungsNr = 66116,
  Jahr = 2015,
  Monat = 09,
  ThemaNr = 1,
  TeilaufgabeNr = 1,
  AufgabeNr = 3,
}

\let\a=\bAttribut
\let\f=\bFremd
\let\p=\bPrimaer
\let\r=\bRelationMenge

\noindent
Gegeben seien folgende Relationen:
\index{SQL}
\footcite{examen:66116:2015:09}

\begin{liRelationenSchemaFormat}
Mensch(ID*, MutterID, VaterID)
Mann(ID*)
Frau(ID*)
\end{liRelationenSchemaFormat}

\begin{bRelationenModell}
\r{Mensch}{\p{ID}, MutterID, VaterID}
\r{Mann}{\p{ID}}
\r{Frau}{\p{ID}}
\end{bRelationenModell}

\begin{bAdditum}[Übungsdatenbank]
% Datenbankname: vater_mutter
\begin{minted}{sql}
CREATE TABLE Mensch (
  ID INTEGER PRIMARY KEY,
  MutterID INTEGER,
  VaterID INTEGER
);

CREATE TABLE Mann (
  ID INTEGER PRIMARY KEY
);

CREATE TABLE Frau (
  ID INTEGER PRIMARY KEY
);

INSERT INTO Mensch VALUES
  (1, 42, 41),
  (2, 42, 41),
  (3, 42, 41),
  (4, 42, 41),
  (42, NULL, 1),
  (41, NULL, NULL);

INSERT INTO Mann VALUES
  (1),
  (3),
  (41);

INSERT INTO Frau VALUES
  (2),
  (4),
  (42);

CREATE VIEW VaterKind AS
SELECT Mensch.VaterID, Mensch.ID as KindID
FROM Mensch
WHERE
  Mensch.VaterID IS NOT NULL;
\end{minted}
\index{SQL mit Übungsdatenbank}
\end{bAdditum}

\noindent
Das zugehörige ER-Modell für dieses relationale Datenbankschema sieht
folgendermaßen aus:

\bigskip

\noindent
Bearbeiten Sie folgende Teilaufgaben:

\begin{enumerate}

%%
% a)
%%

\item Finden Sie die Töchter der Frau mit ID 42.

\begin{bAntwort}
\begin{minted}{sql}
SELECT Mensch.ID
FROM Mensch, Frau
WHERE
  Mensch.MutterID = Frau.id AND
  Frau.ID = 42;
\end{minted}
\end{bAntwort}

%%
% b)
%%

\item Gibt es Männer, die ihre eigenen Großväter sind? Formulieren Sie
eine geeignete SQL-Anfrage.

\begin{bAntwort}
\begin{minted}{sql}
SELECT Mensch.ID
FROM Mann, Mensch
WHERE
  Mensch.ID = Mann.id AND (
    Mensch.VaterID IN (SELECT v.ID FROM Mensch v WHERE v.VaterID = Mensch.ID)
    OR
    Mensch.MutterID IN (SELECT v.ID FROM Mensch v WHERE v.VaterID = Mensch.ID)
  );
\end{minted}
\end{bAntwort}

%%
% c)
%%

\item Definieren Sie eine View VaterKind (VaterID; KindID), die allen
Vätern (VaterID) ihre Kinder (KinderID) zuordnet. Diese View darf keine
NULL-Werte enthalten.

\begin{bAntwort}

\begin{minted}{sql}
-- Wir erzeuge bereits beim Erstellen der Datenbank diese View, damit
-- sie für spätere Aufgaben zur Verfügung steht.
DROP VIEW IF EXISTS VaterKind;
CREATE VIEW VaterKind AS
SELECT Mensch.VaterID, Mensch.ID as KindID
FROM Mensch
WHERE
  Mensch.VaterID IS NOT NULL;
SELECT * FROM VaterKind;
\end{minted}
\end{bAntwort}

%%
% d)
%%

\item Verwenden Sie die View aus c), um alle Väter zurückzugeben,
absteigend geordnet nach der Anzahl ihrer Kinder.

\begin{bAntwort}
\begin{minted}{sql}
SELECT VaterID, COUNT(VaterID) as Anzahl
FROM VaterKind
GROUP BY VaterID
ORDER BY Anzahl DESC;
\end{minted}
\end{bAntwort}

%%
% e)
%%

\item Hugo möchte mit folgender Anfrage auf Basis der View aus c) alle
kinderlosen Männer erhalten:

\begin{minted}{sql}
SELECT VaterID
FROM VaterKind
GROUP BY VaterID
HAVING COUNT(KindID) = 0
\end{minted}

\begin{enumerate}

%%
% i.
%%

\item Was ist das Ergebnis von Hugos Anfrage und warum?

\begin{bAntwort}
Die Anfrage liefert kein Ergebnis. Da die View laut Angabe keine
Null-Werte enthalten darf, sind in der View nur Männer verzeichnet, die
wirklich Väter sind.
\end{bAntwort}

%%
% ii.
%%

\item Formulieren Sie eine Anfrage, die tatsächlich alle kinderlosen
Männer zurückliefert.

\begin{bAntwort}
\begin{minted}{sql}
SELECT * FROM Mann
EXCEPT
SELECT VaterID
FROM VaterKind
GROUP BY VaterID;
\end{minted}
\end{bAntwort}

\end{enumerate}
\end{enumerate}

Hinweis: Denken Sie daran, dass SQL auch Mengenoperationen kennt.

\end{document}
