\documentclass{bschlangaul-aufgabe}
\bLadePakete{mathe,java,wpkalkuel}
\begin{document}
\bAufgabenMetadaten{
  Titel = {Aufgabe 3},
  Thematik = {Methode „doubleFac()“: wp-Kalkül und Schleifeninvariante},
  Referenz = 66116-2015-H.T2-TA2-A3,
  RelativerPfad = Staatsexamen/66116/2015/09/Thema-2/Teilaufgabe-2/Aufgabe-3.tex,
  ZitatSchluessel = examen:66116:2015:09,
  ZitatBeschreibung = {Thema 2 Teilaufgabe 2 Aufgabe 3},
  BearbeitungsStand = mit Lösung,
  Korrektheit = unbekannt,
  Ueberprueft = {unbekannt},
  Stichwoerter = {wp-Kalkül, Invariante, Terminierungsfunktion},
  EinzelpruefungsNr = 66116,
  Jahr = 2015,
  Monat = 09,
  ThemaNr = 2,
  TeilaufgabeNr = 2,
  AufgabeNr = 3,
}

\let\wp=\bWpKalkuelOhneMathe

Gegeben Sei folgendes Programm:
\footcite[Thema 2 Teilaufgabe 2 Aufgabe 3]{examen:66116:2015:09}

\begin{bJavaAngabe}
long doubleFac (long n) {
  /* P */ long df = 1;
  for (long x = n; x > 1; x -= 2) {
    df *= x;
  } /* Q */
  return df;
}
\end{bJavaAngabe}

\noindent
sowie die Vorbedingung $P \equiv n \geq 0$ und Nachbedingung $Q \equiv
(\text{df} = n!!)$ wobei gilt\index{wp-Kalkül}
\footcite{sosy:ab:8}

\begin{equation*}
n!! :=
\begin{cases}
2^k \cdot k! & n\text{ gerade, }k := \frac{n}{2} \\
\frac{(2k)!}{2^k \cdot k!}& n\text{ ungerade, }k := \frac{n+1}{2}
\end{cases}
\end{equation*}

%%
%
%%

\begin{bExkurs}[Fakultät]
Die Fakultät ist eine Funktion, die einer natürlichen Zahl
das Produkt aller natürlichen Zahlen (ohne Null) kleiner und gleich
dieser Zahl zuordnet.
%
Für alle natürlichen Zahlen $n$ ist

\begin{displaymath}
 n! = 1 \cdot 2 \cdot 3 \dotsm n=\prod_{k=1}^{n}k
\end{displaymath}

\end{bExkurs}

%%
%
%%

\begin{bExkurs}[Doppelfakultät]
Die seltener verwendete „Doppelfakultät“ oder „doppelte Fakultät“ ist
für gerade $n$ das Produkt aller geraden Zahlen kleiner gleich $n$. Für
ungerade $n$ ist es das Produkt aller ungeraden Zahlen kleiner gleich
$n$.
%
Sie ist definiert als:

\begin{equation*}
n!! = \begin{cases}
n \cdot (n-2) \cdot (n-4) \dotsm 2 & \text{für } n \text{ gerade und } n > 0, \\
n \cdot (n-2) \cdot (n-4) \dotsm 1 & \text{für } n \text{ ungerade und } n > 0, \\
1 & \text{für } n \in \{ -1, 0\}
\end{cases}
\end{equation*}

\noindent
Häufig werden anstelle der Doppelfakultät Ausdrücke mit der gewöhnlichen
Fakultät verwendet. Es gilt $(2k)!! = 2^k k!$  und $(2k-1)!! =
\frac{(2k)!}{2^k k!}$
\end{bExkurs}

\noindent
Zur Vereinfachung nehmen Sie im Folgenden an, dass die verwendeten
Datentypen unbeschränkt sind und daher keine Überläufe auftreten können.

\begin{enumerate}

%%
% (a)
%%

\item Welche der folgenden Bedingungen ist eine zum Beweisen der
Korrektheit der Methode mittels wp-Kalkül (Floyd-Hoare-Kalkül) sinnvolle
Schleifeninvariante?
\index{Invariante}

\begin{enumerate}
\item $\text{df} = n!! - x!! \land x \geq 1$
\item $\text{df} = (n - x)!! \land x \geq 1$
\item $\text{df} \cdot x!! = n!! \land x \geq 0$
\item $(\text{df} + x)!! = n!! \land x \geq 0$
\end{enumerate}

\begin{bAntwort}
Zunächst wird der Code in einen äquivalenten Code mit while-Schleife
umgewandelt:

\begin{bJavaAngabe}
long doubleFac (long n) {
  /* P */ long df = 1;
  long x = n ;
  while (x > 1) {
    df = df * x;
    x = x - 2;
  } /* Q */
  return df;
}
\end{bJavaAngabe}

\begin{enumerate}
\item $\text{df} = n!! - x!! \land x \geq 1$
\item $\text{df} = (n - x)!! \land x \geq 1$ \\

Die ersten beiden Bedingungen sind unmöglich, da \zB für $n = 2$ nach
der Schleife $x = 0$ gilt und daher $x \geq 1$ verletzt wäre.

\item $\text{df} \cdot x!! = n!! \land x \geq 0$ \\

Nach dem Ausschlussprinzip ist es daher die dritte Bedingung: $I \equiv
(\text{df} + x)!! = n!! \land x \geq 0$.

\item $(\text{df} + x)!! = n!! \land x \geq 0$ \\

Die letzte kann es auch nicht sein, da vor der Schleife $\text{df} = 1$
und $x = n$ gilt, \dh $(\text{df} + x)!! = (1 + n)!!$. Jedoch ist
offenbar $(1 + n)!! \not = n!!$.
\end{enumerate}

$\Rightarrow$ Die Schleifeninvariante lautet:
$\text{df} \cdot x!! = n!! \land x \geq 0$

\end{bAntwort}

%%
% (b)
%%
\item Zeigen Sie formal mittels wp-Kalkül, dass die von Ihnen gewählte
Bedingung unmittelbar vor Beginn der Schleife gilt, wenn zu Beginn der
Methode die Anfangsbedingung $P$ gilt.

\begin{bAntwort}
Zu zeigen $P \Rightarrow \wp{Code vor der Schleife}{I}$

\begin{align*}
\wp{Code vor der Schleife}{I}
& \equiv \wp{df = 1; x = n;}{(\text{df} \cdot x)!! = n!! \land x \geq 0} \\
& \equiv \wp{df = 1;}{(\text{df} \cdot n)!! = n!! \land n \geq 0} \\
& \equiv \wp{}{(1 \cdot n)!! = n!! \land n \geq 0} \\
& \equiv n!! = n!! \land n \geq 0 \\
& \equiv n \geq 0 \\
& \equiv P \\
\end{align*}

Insbesondere folgt damit die Behauptung.
\end{bAntwort}

%%
% (c)
%%

\item Zeigen Sie formal mittels wp-Kalkül, dass die von Ihnen gewählte
Bedingung tatsächlich eine Invariante der Schleife ist.

\begin{bAntwort}
zu zeigen: $I \land \text{Schleifenbedingung} \Rightarrow \wp{Code in der Schleife}{I}$

Bevor wir dies beweisen, zeigen wir erst $x \cdot (x - 2)!! = x!!$.

\begin{itemize}
\item Fall $x$ ist gerade ($n!! = 2^k \cdot k!$ für $k := \frac{n}{2}$):

$x \cdot (x - 2)!! =
x \cdot 2^\frac{x - 2}{2} \cdot (\frac{x - 2}{2})! =
x \cdot \frac{1}{2} \cdot 2^\frac{x}{2} \cdot (\frac{x}{2} - 1)! =
2^\frac{x}{2} \cdot (\frac{x}{2})! =
x!!
$

Nebenrechnung (Division mit gleicher Basis: $x^{a-b} = \frac{x^a}{x^b}$):

$2^\frac{x - 2}{2} =
2^{(\frac{x}{2} - \frac{2}{2})} =
\frac{2^\frac{x}{2}}{2^\frac{2}{2}} =
\frac{2^\frac{x}{2}}{2^1} =
\frac{2^\frac{x}{2}}{2} =
\frac{1}{2} \cdot 2^\frac{x}{2}
$

Nebenrechnung ($n! = (n - 1)! \cdot n$):

$x \cdot \frac{1}{2} \cdot (\frac{x}{2} - 1)! =
\frac{x}{2} \cdot (\frac{x}{2} - 1)! =
\frac{x}{2}!
$

\item Fall $x$ ist ungerade:
\end{itemize}

Dies benutzen wir nun, um den eigentlichen Beweis zu führen:

\begin{align*}
\wp{Code vor der Schleife}{I}
& \equiv \wp{df = df * x; x = x - 2;}{(\text{df} \cdot x)!! = n!! \land x \geq 0} \\
& \equiv \wp{df = df * x;}{(\text{df} \cdot (x - 2)))!! = n!! \land x - 2 \geq 0} \\
& \equiv \wp{}{(\text{df} \cdot x \cdot (x - 2)))!! = n!! \land x - 2 \geq 0} \\
& \equiv (\text{df} \cdot x)!! = n!! \land x \geq 2 \\
& \equiv (\text{df} \cdot x)!! = n!! \land x > 1 \\
& \equiv I \land x > 1 \\
& \equiv I \land \text{Schleifenbedingung} \\
\end{align*}
\end{bAntwort}

%%
% (d)
%%

\item Zeigen Sie formal mittels wp-Kalkül, dass am Ende der Methode die
Nachbedingung $Q$ erfüllt wird.

\begin{bAntwort}
z.z. $I \land \neg \text{Schleifenbedingung} \Rightarrow \wp{Code nach der Schleife}{Q}$

Wir vereinfachen den Ausdruck $I \land \neg \text{Schleifenbedingung}$:

$I \land \neg \text{Schleifenbedingung}
\equiv I \land (x \leq 1)
\equiv I \land ((x = 0) \lor (x = 1))
\equiv (I \land (x = 0)) \lor (I \land (x = 1))
\equiv (df \cdot 1 = n!!)  \lor  (df \cdot 1 = n!!)
\equiv df = n!!$

Damit gilt:

$\wp{Code nach der Schleife}{Q}
\equiv \wp{}{df = n!!}
\equiv df = n!!
\equiv I \land \neg \text{Schleifenbedingung}$
\end{bAntwort}

%%
% (e)
%%

\item Beweisen Sie, dass die Methode immer terminiert. Geben Sie dazu
eine Terminierungsfunktion\index{Terminierungsfunktion} an und begründen
Sie kurz ihre Wahl.

\begin{bAntwort}
Sei $T (x) := x$. $T$ ist offenbar ganzzahlig. Da $x$ in jedem
Schleifendurchlauf um $2$ verringert wird, ist $T$ streng monoton
fallend. Aus der Schleifeninvariante folgt $x \geq 0$ und daher ist $x$
auch nach unten beschränkt. Damit folgt $I \Rightarrow T \geq 0$ und $T$
ist eine gültige Terminierungsfunktion.
\end{bAntwort}

\end{enumerate}
\end{document}
