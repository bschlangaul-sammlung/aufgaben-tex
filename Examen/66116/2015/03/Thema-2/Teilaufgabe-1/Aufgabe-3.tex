\documentclass{bschlangaul-aufgabe}
\bLadePakete{baum}
\begin{document}
\bAufgabenMetadaten{
  Titel = {Aufgabe 3},
  Thematik = {Indexstrukturen},
  Referenz = 66116-2015-F.T2-TA1-A3,
  RelativerPfad = Staatsexamen/66116/2015/03/Thema-2/Teilaufgabe-1/Aufgabe-3.tex,
  ZitatSchluessel = examen:66116:2015:03,
  BearbeitungsStand = mit Lösung,
  Korrektheit = unbekannt,
  Ueberprueft = {unbekannt},
  Stichwoerter = {B-Baum},
  EinzelpruefungsNr = 66116,
  Jahr = 2015,
  Monat = 03,
  ThemaNr = 2,
  TeilaufgabeNr = 1,
  AufgabeNr = 3,
}

Als Indexstruktur einer Datenbank sei folgender B-Baum ($k = 2$)
gegeben:\index{B-Baum}
\footcite{examen:66116:2015:03}

\begin{center}
\begin{tikzpicture}[
  b bbaum,
  scale=0.8,
  transform shape,
  level 1/.style={level distance=15mm,sibling distance=80mm},
  level 2/.style={level distance=15mm,sibling distance=20mm}
]
\node{40} [->]
  child {
    node {33 \nodepart{two} 37}
      child {
        node {31 \nodepart{two} 32}
      }
      child {
        node {34 \nodepart{two} 35 \nodepart{three} 36}
      }
      child {
        node {38 \nodepart{two} 39}
      }
  }
  child {
    node {43 \nodepart{two} 46 \nodepart{three} 49 \nodepart{four} 53 }
      child {
        node {41  \nodepart{two} 42}
      }
      child {
        node {44 \nodepart{two} 45}
      }
      child {
        node {47 \nodepart{two} 48}
      }
      child {
        node {50 \nodepart{two} 51}
      }
      child {
        node {54 \nodepart{two} 55 \nodepart{three} 57 \nodepart{four} 58}
      }
};
\end{tikzpicture}
\end{center}

Führen Sie nacheinander die folgenden Operationen aus. Geben Sie die
auftretenden Zwischenergebnisse an. Teilbäume, die sich in einem Schritt
nicht verändern, müssen nicht erneut gezeichnet werden. Sollten
Wahlmöglichkeiten auftreten, so sind größere Schlüsselwerte bzw. weiter
rechts liegende Knoten zu bevorzugen.

\begin{enumerate}

%%
% a)
%%

\item Einfügen des Wertes 56

\begin{bAntwort}

\begin{center}
\begin{tikzpicture}[
  b bbaum,
  scale=0.6,
  transform shape,
  level 1/.style={level distance=15mm,sibling distance=60mm},
  level 2/.style={level distance=15mm,sibling distance=20mm}
]
\node{40 \nodepart{two} 49} [->]
  child {
    node {33 \nodepart{two} 37}
      child {
        node {31 \nodepart{two} 32}
      }
      child {
        node {34 \nodepart{two} 35 \nodepart{three} 36}
      }
      child {
        node {38 \nodepart{two} 39}
      }
  }
  child {
    node {43 \nodepart{two} 46}
      child {
        node {41  \nodepart{two} 42}
      }
      child {
        node {44 \nodepart{two} 45}
      }
      child {
        node {47 \nodepart{two} 48}
      }
  }
  child {
    node {53 \nodepart{two} 56}
      child {
        node {50 \nodepart{two} 51}
      }
      child {
        node {54 \nodepart{two} 55}
      }
      child {
        node {57 \nodepart{two} 58}
      }
  };
\end{tikzpicture}
\end{center}

\end{bAntwort}

%%
% b)
%%

\item Löschen des Wertes 37

\begin{center}
\begin{tikzpicture}[
  b bbaum,
  scale=0.8,
  transform shape,
  level 1/.style={level distance=15mm,sibling distance=80mm},
  level 2/.style={level distance=15mm,sibling distance=20mm}
]
\node{40} [->]
  child {
    node {33 \nodepart{two} 37}
      child {
        node {31 \nodepart{two} 32}
      }
      child {
        node {34 \nodepart{two} 35 \nodepart{three} 36}
      }
      child {
        node {38 \nodepart{two} 39}
      }
  }
  child {
    node {43 \nodepart{two} 46 \nodepart{three} 49 \nodepart{four} 53 }
      child {
        node {41  \nodepart{two} 42}
      }
      child {
        node {44 \nodepart{two} 45}
      }
      child {
        node {47 \nodepart{two} 48}
      }
      child {
        node {50 \nodepart{two} 51}
      }
      child {
        node {54 \nodepart{two} 55 \nodepart{three} 57 \nodepart{four} 58}
      }
};
\end{tikzpicture}
\end{center}
\end{enumerate}
\end{document}
