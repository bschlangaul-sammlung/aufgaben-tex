\documentclass{bschlangaul-aufgabe}

\begin{document}
\bAufgabenMetadaten{
  Titel = {Aufgabe 1},
  Thematik = {Gantt und CPM},
  Referenz = 66116-2017-H.T2-TA2-A1,
  RelativerPfad = Examen/66116/2017/09/Thema-2/Teilaufgabe-2/Aufgabe-1.tex,
  ZitatSchluessel = examen:66116:2017:09,
  ZitatBeschreibung = {Thema 2 Teilaufgabe 2 Aufgabe 1},
  BearbeitungsStand = nur Angabe,
  Korrektheit = unbekannt,
  Ueberprueft = {unbekannt},
  Stichwoerter = {Gantt-Diagramm},
  EinzelpruefungsNr = 66116,
  Jahr = 2017,
  Monat = 09,
  ThemaNr = 2,
  TeilaufgabeNr = 2,
  AufgabeNr = 1,
}

Gegeben ist das folgende Gantt-Diagramm\index{Gantt-Diagramm} zur Planung eines
hypothetischen Softwareprojekts:
\footcite[Thema 2 Teilaufgabe 2 Aufgabe 1]{examen:66116:2017:09}

\begin{enumerate}

%%
% a)
%%

\item Konvertieren Sie das Gantt-Diagramm in ein CPM-Netzwerk, das die
Aktivitäten und Abhängigkeiten äquivalent beschreibt. Gehen Sie von der
Zeiteinheit „Monate“ aus. Definieren Sie im CPM-Netzwerk je einen
globalen Start- und Endknoten. Der Start jeder Aktivität hängt dabei vom
Projektstart ab, das Projektende hängt vom Ende aller Aktivitäten ab.

%%
% b)
%%

\item Berechnen Sie für jedes Ereignis (\dh für jeden Knoten Ihres
CPM-Netzwerks) die früheste Zeit, die späteste Zeit sowie die
Pufferzeit. Beachten Sie, dass die Berechnungsreihenfolge einer
topologischen Sortierung des Netzwerks entsprechen sollte.

%%
% c)
%%

\item Geben Sie einen kritischen Pfad durch das CPM-Netzwerk an.
Welche Aktivität darf sich demnach wie lange verzögern?
\end{enumerate}
\end{document}
