\documentclass{bschlangaul-aufgabe}
\bLadePakete{cpm}
\begin{document}
\bAufgabenMetadaten{
  Titel = {Aufgabe 1},
  Thematik = {Projektmanagement},
  Referenz = 66116-2021-F.T2-TA1-A1,
  RelativerPfad = Staatsexamen/66116/2021/03/Thema-2/Teilaufgabe-1/Aufgabe-1.tex,
  ZitatSchluessel = examen:66116:2021:03,
  BearbeitungsStand = mit Lösung,
  Korrektheit = unbekannt,
  Ueberprueft = {unbekannt},
  Stichwoerter = {CPM-Netzplantechnik},
  EinzelpruefungsNr = 66116,
  Jahr = 2021,
  Monat = 03,
  ThemaNr = 2,
  TeilaufgabeNr = 1,
  AufgabeNr = 1,
}

\let\f=\footnotesize
\let\FZ=\bCpmFruehI
\let\SZ=\bCpmSpaetI

Gegeben seien folgende Tätigkeiten mit ihren Abhängigkeiten und Dauern:
\index{CPM-Netzplantechnik}
\footcite{examen:66116:2021:03}

\begin{center}
\begin{tabular}{lll}
\hline
Task & Dauer (in h) & Abhängigkeiten \\\hline
T1   & 3            & / \\
T2   & 6            & / \\
T3   & 2            & T1 \\
T4   & 2            & T2 \\
T5   & 5            & T1 \\
T6   & 3            & T4, T5 \\
T7   & 6            & T3 \\
T8   & 7            & T4 \\
T9   & 4            & T6, T8 \\
T10  & 1            & T7, T9 \\\hline
\end{tabular}
\end{center}
\begin{enumerate}

%%
% a)
%%

\item Zeichnen Sie ein CPM-Diagramm basierend auf der gegebenen
Aufgabenliste. Benutzen Sie explizite Start- und Endknoten.

\begin{bAntwort}
\bPseudoUeberschrift{Abkürzungen}

\begin{description}
\item[S] Start
\item[1S] Start von T1
\item[1E] Ende von T1
\item[E] Ende
\end{description}

\begin{center}
\begin{tikzpicture}[x=1.2cm,y=1.5cm,scale=0.9,transform shape]
\bCpmEreignis{S}{0}{0}

\bCpmEreignis{1S}{1}{1}
\bCpmEreignis{1E}{2}{1}

\bCpmEreignis{2S}{1}{-1}
\bCpmEreignis{2E}{2}{-1}

\bCpmEreignis{3S}{3}{1}
\bCpmEreignis{3E}{4}{1}

\bCpmEreignis{4S}{3}{-2}
\bCpmEreignis{4E}{4}{-2}

\bCpmEreignis{5S}{3}{0}
\bCpmEreignis{5E}{4}{0}

\bCpmEreignis{6S}{4}{-1}
\bCpmEreignis{6E}{5}{-1}

\bCpmEreignis{7S}{5}{1}
\bCpmEreignis{7E}{6}{1}

\bCpmEreignis{8S}{5}{-2}
\bCpmEreignis{8E}{6}{-2}

\bCpmEreignis{9S}{6}{-1}
\bCpmEreignis{9E}{7}{-1}

\bCpmEreignis{10S}{7}{0}
\bCpmEreignis{10E}{8}{0}

\bCpmEreignis{E}{9}{0}

\bCpmVorgang{1S}{1E}{3}
\bCpmVorgang{2S}{2E}{6}
\bCpmVorgang{3S}{3E}{2}
\bCpmVorgang{4S}{4E}{2}
\bCpmVorgang{5S}{5E}{5}
\bCpmVorgang{6S}{6E}{3}
\bCpmVorgang{7S}{7E}{6}
\bCpmVorgang{8S}{8E}{7}
\bCpmVorgang{9S}{9E}{4}
\bCpmVorgang{10S}{10E}{1}

\bCpmVorgang[schein]{S}{1S}{}
\bCpmVorgang[schein]{S}{2S}{}

\bCpmVorgang[schein]{1E}{3S}{}

\bCpmVorgang[schein]{2E}{4S}{}

\bCpmVorgang[schein]{1E}{5S}{}

\bCpmVorgang[schein]{4E}{6S}{}
\bCpmVorgang[schein]{5E}{6S}{}

\bCpmVorgang[schein]{3E}{7S}{}

\bCpmVorgang[schein]{4E}{8S}{}

\bCpmVorgang[schein]{6E}{9S}{}
\bCpmVorgang[schein]{8E}{9S}{}

\bCpmVorgang[schein]{7E}{10S}{}
\bCpmVorgang[schein]{9E}{10S}{}

\bCpmVorgang[schein]{10E}{E}{}
\end{tikzpicture}
\end{center}

Teilen wir einen Task in zwei Knoten auf, so wird das Diagramm sehr
unübersichtlich. Wir verwenden pro Task nur einen Knoten. Es gibt zwei
Möglichkeiten:

\bPseudoUeberschrift{Knoten sind Anfang der Tasks}

\begin{center}
\begin{tikzpicture}[x=1.5cm,y=1.5cm,scale=0.9,transform shape]
\bCpmEreignis{S}{0}{0}

\bCpmEreignis{T1}{1}{1}
\bCpmEreignis{T2}{1}{-1}
\bCpmEreignis{T3}{2}{2}
\bCpmEreignis{T4}{2}{-1}
\bCpmEreignis{T5}{2}{0.5}
\bCpmEreignis{T6}{3}{0}
\bCpmEreignis{T7}{4}{2}
\bCpmEreignis{T8}{4}{-1}
\bCpmEreignis{T9}{5}{0}
\bCpmEreignis{T10}{6}{0}
\bCpmEreignis{E}{7}{0}

\bCpmVorgang[schein]{S}{T1}{}
\bCpmVorgang[schein]{S}{T2}{}
% 3
\bCpmVorgang{T1}{T3}{3}

% 4
\bCpmVorgang{T2}{T4}{6}

% 5
\bCpmVorgang{T1}{T5}{3}

% 6
\bCpmVorgang{T4}{T6}{2}
\bCpmVorgang{T5}{T6}{5}

% 7
\bCpmVorgang{T3}{T7}{2}

% 8
\bCpmVorgang{T4}{T8}{2}

% 9
\bCpmVorgang{T6}{T9}{3}
\bCpmVorgang{T8}{T9}{7}

% 10
\bCpmVorgang{T7}{T10}{6}
\bCpmVorgang{T9}{T10}{4}
\bCpmVorgang{T10}{E}{1}
\end{tikzpicture}
\end{center}

\bPseudoUeberschrift{Knoten sind Ende der Tasks}

\begin{center}
\begin{tikzpicture}[x=1.5cm,y=1.5cm,scale=0.9,transform shape]
\bCpmEreignis{S}{0}{0}

\bCpmEreignis{T1}{1}{1}
\bCpmEreignis{T2}{1}{-1}
\bCpmEreignis{T3}{2}{2}
\bCpmEreignis{T4}{2}{-1}
\bCpmEreignis{T5}{2}{0.5}
\bCpmEreignis{T6}{3}{0}
\bCpmEreignis{T7}{4}{2}
\bCpmEreignis{T8}{4}{-1}
\bCpmEreignis{T9}{5}{0}
\bCpmEreignis{T10}{6}{0}
\bCpmEreignis{E}{7}{0}

\bCpmVorgang{S}{T1}{3}
\bCpmVorgang{S}{T2}{6}
\bCpmVorgang{T1}{T3}{2}
\bCpmVorgang{T1}{T5}{5}
\bCpmVorgang{T2}{T4}{2}
\bCpmVorgang{T3}{T7}{6}
\bCpmVorgang{T4}{T6}{3}
\bCpmVorgang{T4}{T8}{7}
\bCpmVorgang{T5}{T6}{3}
\bCpmVorgang{T6}{T9}{4}
\bCpmVorgang{T7}{T10}{1}
\bCpmVorgang{T8}{T9}{4}
\bCpmVorgang{T9}{T10}{1}
\bCpmVorgang[schein]{T10}{E}{}

\end{tikzpicture}
\end{center}
\end{bAntwort}

%%
% b)
%%

\item Als \emph{Slack} bezeichnet man die Zeit, um die eine Aufgabe
bezüglich ihres frühesten Startzeitpunktes verzögert werden kann, ohne
dass es Probleme bei der fristgerechten Fertigstellung des Projektes
gibt. Berechnen Sie den Slack für alle Aktivitäten und ergänzen Sie ihn
in Ihrem Diagramm.

\begin{bAntwort}
\bPseudoUeberschrift{Knoten sind Anfang der Tasks}

\bCpmFruehErklaerung

\begin{tabular}{|l|l|r|}
\hline
$i$ & Nebenrechnung            & \FZ \\\hline
T1  &                          & 0   \\
T2  &                          & 0   \\
T3  &                          & 3   \\
T4  &                          & 6   \\
T5  &                          & 3   \\
T6  & $\max(8_{T4}, 8_{T5})$   & 8   \\
T7  &                          & 5   \\
T8  &                          & 8   \\
T9  & $\max(11_{T6}, 15_{T4})$ & 15  \\
T10 & $\max(19_{T9}, 11_{T7})$ & 19  \\
E   &                          & 20  \\\hline
\end{tabular}

\bCpmSpaetErklaerung

\begin{tabular}{|l|l|r|}
\hline
$i$ & Nebenrechnung            & \SZ \\\hline
E   &                          & 20  \\
T10 &                          & 19  \\
T9  &                          & 15  \\
T8  &                          & 8   \\
T7  &                          & 13  \\
T6  &                          & 12  \\
T5  &                          & 7   \\
T4  & $\min(12_{T6}, 6_{T8})$  & 6   \\
T3  &                          & 11  \\
T2  &                          & 0   \\
T1  & $\min(8_{T3}, 4_{T5})$   & 4   \\\hline
\end{tabular}

\begin{tabular}{|l|l|l|l|l|l|l|l|l|l|l|l|}
\hline
$i$ & T1 & T2 & T3  & T4 & T5 & T6 & T7 & T8 & T9 & T10 & E  \\\hline\hline
\FZ & 0  & 0  & 3   & 6  & 3  & 8  & 5  & 8  & 15 & 19  & 20 \\\hline
\SZ & 4  & 0  & 11  & 6  & 7  & 12 & 13 & 8  & 15 & 19  & 20  \\\hline
GP  & 4  & 0  & 8   & 0  & 4  & 4  & 8  & 0  & 0  & 0   & 0  \\\hline
\end{tabular}

\bPseudoUeberschrift{Knoten sind Ende der Tasks}

\bCpmFruehErklaerung

\begin{tabular}{|l|l|r|}
\hline
$i$ & Nebenrechnung            & \FZ \\\hline
T1  &                          & 3   \\
T2  &                          & 6   \\
T3  &                          & 5   \\
T4  &                          & 8   \\
T5  &                          & 8   \\
T6  & $\max(11_{T4}, 11_{T5})$ & 11  \\
T7  &                          & 11  \\
T8  &                          & 15  \\
T9  & $\max(15_{T6}, 19_{T8})$ & 19  \\
T10 & $\max(20_{T9}, 12_{T7})$ & 20  \\
E   &                          & 20  \\\hline
\end{tabular}

\bCpmSpaetErklaerung

\begin{tabular}{|l|l|r|}
\hline
$i$ & Nebenrechnung            & \SZ \\\hline
E   &                          & 20  \\
T10 &                          & 20  \\
T9  &                          & 19  \\
T8  &                          & 15  \\
T7  &                          & 19  \\
T6  &                          & 15  \\
T5  &                          & 12  \\
T4  & $\min(12_{T6}, 8_{T8})$  & 8   \\
T3  &                          & 13  \\
T2  &                          & 6   \\
T1  & $\min(11_{T3}, 7_{T5})$  & 7   \\\hline
\end{tabular}

\begin{tabular}{|l|l|l|l|l|l|l|l|l|l|l|l|}
\hline
$i$ & T1 & T2 & T3 & T4 & T5 & T6 & T7 & T8 & T9 & T10 & E  \\\hline\hline
\FZ & 3  & 6  & 5  & 8  & 8  & 11 & 11 & 15 & 19 & 20  & 20 \\\hline
\SZ & 7  & 6  & 13 & 8  & 12 & 15 & 19 & 15 & 19 & 20  & 20  \\\hline
GP  & 4  & 0  & 8  & 0  & 4  & 4  & 8  & 0  & 0  & 0   & 0  \\\hline
\end{tabular}

\end{bAntwort}

%%
% c)
%%

\item Zeichnen Sie den kritischen Pfad in Ihr Diagramm ein oder geben
Sie die Tasks des kritischen Pfades in der folgenden Form an:
\textbf{Start} ! $\dots$ ! \textbf{Ende}. Sollte es mehrere kritische
Pfade geben, geben Sie auch diese an. Wie lange ist die Dauer des
kritischen Pfades bzw. der kritischen Pfade?

\begin{bAntwort}
Kritischer Pfad: \textbf{Start} ! T2 ! T4 ! T8 ! T9 ! T10 ! \textbf{Ende}

Dauer: 20 h
\end{bAntwort}

\end{enumerate}
\end{document}
