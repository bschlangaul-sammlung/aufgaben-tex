\documentclass{bschlangaul-aufgabe}
\bLadePakete{normalformen}
\begin{document}
\bAufgabenMetadaten{
  Titel = {Aufgabe 6},
  Thematik = {Relation Prüfung},
  Referenz = 66116-2021-F.T2-TA2-A6,
  RelativerPfad = Examen/66116/2021/03/Thema-2/Teilaufgabe-2/Aufgabe-6.tex,
  ZitatSchluessel = examen:66116:2021:03,
  BearbeitungsStand = mit Lösung,
  Korrektheit = unbekannt,
  Ueberprueft = {unbekannt},
  Stichwoerter = {Normalformen},
  EinzelpruefungsNr = 66116,
  Jahr = 2021,
  Monat = 03,
  ThemaNr = 2,
  TeilaufgabeNr = 2,
  AufgabeNr = 6,
}

\let\FA=\bFunktionaleAbhaengigkeiten
\let\t=\texttt

Gegeben ist die Relation Prüfung (Prüfungsnummer, Fakultät,
Prüfungsname, Dozent, Prüfungstyp, ECTS) mit den beiden
Schlüsselkandidaten (Prüfungsnummer, Fakultät) und (Fakultät,
Prüfungsname, Dozent).
\index{Normalformen}
\footcite{examen:66116:2021:03}

% N Prüfungsnummer
% F Fakultät
% P Prüfungsname
% D Dozent
% T Prüfungstyp
% E ECTS

% NFPDTE
% NF->PDTE
% FPD->NTE

Alle Attributwerte sind atomar. Es gelten nur die durch die
Schlüsselkandidaten implizierten funktionalen Abhängigkeit.

Geben Sie die höchste Normalform an, die die Relation-Prüfung erfüllt.
Zeigen Sie, dass alle Bedingungen für diese Normalform erfüllt sind
und dass mindestens eine Bedingung der nächsthöheren Normalform verletzt
ist. Beziehen Sie sich bei der Begründung auf die gegebene Relation und
nennen Sie nicht nur die allgemeinen Definitionen der Normalformen.

\begin{bAntwort}
\FA{
  Prüfungsnummer, Fakultät -> Prüfungsname, Dozent, Prüfungstyp, ECTS;
  Fakultät, Prüfungsname, Dozent -> Prüfungsnummer, Prüfungstyp, ECTS;
}

Höchste Normalform: 4NF

Siehe Taschenbuch Seite 449

\begin{description}
\item[1NF]

Alle Werte sind atomar.

\item[2NF]

Ist in 1NF und jedes Attribut ist Teil des Schlüsselkandidaten
(\t{Prüfungsnummer, Fakultät} oder \t{Fakultät, Prüfungsname, Dozent})
oder das Attribut ist von einem Schlüsselkandidaten voll funktional
abhängig (\t{Prüfungsname, Dozent, Prüfungstyp, ECTS} oder
\t{Prüfungsnummer, Prüfungstyp, ECTS}).

\item[3NF]

Ist in 2NF und ein Nichtschlüsselattribut darf nur vom
Schlüsselkandidaten abhängen (\t{Prüfungsname, Dozent, Prüfungstyp,
ECTS} hängt von \t{Prüfungsnummer, Fakultät} ab) und (\t{Prüfungsnummer,
Prüfungstyp, ECTS} hängt von \t{Fakultät, Prüfungsname, Dozent} ab).

\item[BCNF]

Jede Determinate ist Schlüsselkandidat (\t{Prüfungsnummer, Fakultät} und
\t{Fakultät, Prüfungsname, Dozent}).

\item[4NF]

keine paarweise Unabhängigkeiten mehrwertigen Abhängigkeiten zwischen
ihren Attributen.
\end{description}
\end{bAntwort}
\end{document}
