\documentclass{bschlangaul-aufgabe}
\bLadePakete{syntax,relationale-algebra}

\begin{document}
\bAufgabenMetadaten{
  Titel = {Aufgabe 7},
  Thematik = {Optimierung},
  Referenz = 66116-2021-F.T2-TA2-A7,
  RelativerPfad = Examen/66116/2021/03/Thema-2/Teilaufgabe-2/Aufgabe-7.tex,
  ZitatSchluessel = examen:66116:2021:03,
  BearbeitungsStand = mit Lösung,
  Korrektheit = unbekannt,
  Ueberprueft = {unbekannt},
  Stichwoerter = {DB},
  EinzelpruefungsNr = 66116,
  Jahr = 2021,
  Monat = 03,
  ThemaNr = 2,
  TeilaufgabeNr = 2,
  AufgabeNr = 7,
}

\begin{enumerate}

%%
% a)
%%

\item Erläutern Sie kurz, was Indizes sind und warum diese in
Datenbanksystemen verwendet werden.\index{DB}
\footcite{examen:66116:2021:03}

\begin{bAntwort}
Ein Datenbankindex ist eine von der Datenstruktur getrennte
Indexstruktur in einer Datenbank, die die Suche und das Sortieren nach
bestimmten Feldern beschleunigt.

Ein Index besteht aus einer Ansammlung von Zeigern (Verweisen), die eine
Ordnungsrelation auf eine oder mehrere Spalten in einer Tabelle
definieren. Wird bei einer Abfrage eine indizierte Spalte als
Suchkriterium herangezogen, sucht das Datenbankmanagementsystem (DBMS)
die gewünschten Datensätze anhand dieser Zeiger. In der Regel finden
hier B+-Bäume Anwendung. Ohne Index müsste die Spalte sequenziell
durchsucht werden, während eine Suche mit Hilfe des Baums nur
logarithmische Komplexität hat.
\bFussnoteUrl{https://de.wikipedia.org/wiki/Datenbankindex}
\end{bAntwort}

%%
% b)
%%

\item Übertragen Sie folgendes SQL-Statement in einen nicht optimierten
algebraischen Term oder in einen Anfragegraphen.

\begin{minted}{sql}
SELECT Kunde.Name, Kunde.Geburtsdatum
FROM Kunde, Rechnung
WHERE Kunde.ID = Rechnung.Kunde
AND Rechnung.Summe < 100;
\end{minted}

\begin{bAntwort}
\begin{center}
\begin{tikzpicture}
\node
  (pi) {$\pi_{\text{Kunde.Name, Kunde.Geburtsdatum}}$};

\node[below=of pi]
  (sigma) {$\sigma_{\text{Kunde.ID = Rechnung.Kunde} \land \text{Rechnung.Summe < 100}}$}
  edge (pi);

\node[below=of sigma]
  (x) {$\times$}
  edge (sigma);

\node[below left=of x]
  {Kunde}
  edge(x);

\node[below right=of x]
  {Rechnung}
  edge(x);
\end{tikzpicture}
\end{center}
\end{bAntwort}

%%
% c)
%%

\item Nennen Sie zwei Möglichkeiten, den algebraischen Term bzw. den
Anfragegraphen aus der vorhergehenden Teilaufgabe logisch (d. h.
algebraisch) zu optimieren. Beziehen Sie sich auf konkrete Stellen und
Operatoren des von Ihnen aufgestellten algebraischen Ausdrucks.

\begin{bAntwort}
Zuerst Selection und dann Join

Theta-Join

\begin{center}
\begin{tikzpicture}
\node
  (pi) {$\pi_{\text{Kunde.Name, Kunde.Geburtsdatum}}$};

\node[below=of pi]
  (theta join) {$\bowtie_{\text{Kunde.ID = Rechnung.Kunde}}$}
  edge (pi);

\node[below left=of theta join]
  {Kunde}
  edge(theta join);

\node[below right=of theta join]
  (sigma rechnung) {$\sigma_{\text{Summe < 100}}$}
  edge (theta join);

\node[below=of sigma rechnung]
  {Rechnung}
  edge(sigma rechnung);
\end{tikzpicture}
\end{center}
\end{bAntwort}

\end{enumerate}
\end{document}
