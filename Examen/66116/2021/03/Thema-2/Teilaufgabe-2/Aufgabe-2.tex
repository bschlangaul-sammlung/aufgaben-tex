\documentclass{bschlangaul-aufgabe}
\bLadePakete{er}
\usepackage{fontawesome}
\begin{document}
\bAufgabenMetadaten{
  Titel = {Aufgabe 2},
  Thematik = {Online-Marktplatze},
  Referenz = 66116-2021-F.T2-TA2-A2,
  RelativerPfad = Staatsexamen/66116/2021/03/Thema-2/Teilaufgabe-2/Aufgabe-2.tex,
  ZitatSchluessel = examen:66116:2021:03,
  BearbeitungsStand = mit Lösung,
  Korrektheit = unbekannt,
  Ueberprueft = {unbekannt},
  Stichwoerter = {Entity-Relation-Modell},
  EinzelpruefungsNr = 66116,
  Jahr = 2021,
  Monat = 03,
  ThemaNr = 2,
  TeilaufgabeNr = 2,
  AufgabeNr = 2,
}

\let\a=\bErMpAttribute
\let\d=\bErDatenbankName
\let\e=\bErMpEntity
\let\r=\bErMpRelationship

Im Folgenden finden Sie die Beschreibung eines Online-Marktplatzes.
Erstellen Sie zu dieser Beschreibung ein erweitertes ER-Diagramm.
Kennzeichnen Sie die Primärschlüssel durch passendes Unterstreichen
und geben Sie die Kardinalitäten in Chen-Notation (= Funktionalitäten)
an. Kennzeichnen Sie auch die totale Teilnahme von Entity-Typen an
Beziehungstypen.\index{Entity-Relation-Modell}
\footcite{examen:66116:2021:03}

Es gibt \e{Produkte}. Diese haben eine eindeutige \a{Bezeichnung}, einen
\a{Beschreibungstext} und eine \a{Bewertung}. Außerdem gibt es
\e{Personen}, die entweder \e{Kunde}, \e{Händler} oder beides sind. Jede
Person hat einen \a{Nachnamen}, einen oder mehrere \a{Vornamen}, ein
\a{Geburtsdatum} und eine \a{E-Mail-Adresse}, mit der diese eindeutig
identifiziert werden kann.

Das System verwaltet außerdem \e{Zahlungsmittel}. Jedes Zahlungsmittel
ist entweder eine \e{Kreditkarte} oder eine \e{Bankverbindung} für
Lastschriften. Für das Lastschriftverfahren wird die international
eindeutige \a{IBAN} und der Name des \a{Kontoinhabers} erfasst, bei
Zahlung mit Kreditkarte der Name des \a{Karteninhabers}, die eindeutige
\a{Kartennummer}, das \a{Ablaufdatum} sowie der \a{Kartenanbieter}. Es
gibt \r{Transaktionen}. Jede Transaktion bezieht sich stets auf ein
Produkt, einen Kunden, einen Händler und auf ein Zahlungsmittel, das für
die Transaktion verwendet wird. Jede Transaktion enthält außerdem den
\a{Preis}, auf den sich Kunde und Händler geeinigt haben, das
\a{Abschlussdatum} sowie eine \a{Lieferadresse}, an die das Produkt
versandt wird.

\begin{bAntwort}
\begin{center}
\begin{tikzpicture}[er2,scale=0.7,transform shape]

% Person
\node[entity] (Person) {Person};
\node[attribute,right=1cm of Person] {\key{E-Mail}} edge (Person);
\node[multi attribute,above left=1cm of Person] {Vornamen} edge (Person);
\node[attribute,left=1cm of Person] {Nachnamen} edge (Person);
\node[attribute,above right=1cm of Person] {Geburtsdatum} edge (Person);

% Kunde
\node[entity,below left=1cm of Person] (Kunde) {Kunde};

% Händler
\node[entity,below right=1cm of Person] (Händler) {Händler};

\node[specialization,below=0.2cm of Person]{is-a}
  edge (Kunde) edge (Händler) edge (Person);

% Transaktion
\node[relationship,below=2cm of Person] (Transaktion) {Transaktion}
  edge node[auto]{1} (Kunde)
  edge node[auto]{1} (Händler);
\node[attribute,below=1cm of Transaktion] {Preis} edge (Transaktion);
\node[attribute,left=1cm of Transaktion,text width=2cm] {Abschlussdatum} edge (Transaktion);
\node[attribute,right=1cm of Transaktion] {Lieferadresse} edge (Transaktion);

% Zahlungsmittel
\node[entity,below=4cm of Händler] (Zahlungsmittel) {Zahlungsmittel}
  edge node[auto]{1} (Transaktion);
\node[attribute,right=1cm of Zahlungsmittel] {Inhaber} edge (Zahlungsmittel);

% Bankverbindung
\node[entity,below left=1cm and 0cm of Zahlungsmittel] (Bankverbindung) {Bankverbindung};
\node[attribute,below left=1cm of Bankverbindung] {\key{IBAN}} edge (Bankverbindung);

% Kreditkarte
\node[entity,below right=3cm and -2cm of Zahlungsmittel]
(Kreditkarte) {Kreditkarte};

\node[attribute,below left=1cm of Kreditkarte]
{\key{Nummer}} edge (Kreditkarte);

\node[attribute,below right=1cm of Kreditkarte,text width=2cm]
{Ablaufdatum} edge (Kreditkarte);

\node[attribute,right=1cm of Kreditkarte]
{Anbieter} edge (Kreditkarte);

\node[generalization,below=0.8cm of Zahlungsmittel]{is-a}
  edge (Zahlungsmittel) edge (Bankverbindung) edge (Kreditkarte);

% Produkt
\node[entity,below=4cm of Kunde] (Produkt) {Produkt}
  edge node[auto]{1} (Transaktion);
\node[attribute,left=1cm of Produkt] {\key{Bezeichnung}} edge (Produkt);
\node[attribute,below left=1cm of Produkt,text width=2cm] {Beschreibungstext} edge (Produkt);
\node[attribute,above left=1cm of Produkt] {Bewertung} edge (Produkt);
\end{tikzpicture}
\end{center}
\end{bAntwort}

% Person(Nachname, Vorname, Geburtsdatum, E-Mail-Adresse)

% Produkt(Bezeichnung, Beschreibungstext, Bewertung)

% Transaktion(Käufer, Händler, Produkt, Zahlungsmittel)

% Kreditkarte()

% Bankverbindung()
\end{document}
