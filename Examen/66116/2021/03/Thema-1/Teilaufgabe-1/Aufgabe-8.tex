\documentclass{bschlangaul-aufgabe}
\begin{document}
\bAufgabenMetadaten{
  Titel = {Aufgabe 8},
  Thematik = {Client-Server-Modell},
  Referenz = 66116-2021-F.T1-TA1-A8,
  RelativerPfad = Staatsexamen/66116/2021/03/Thema-1/Teilaufgabe-1/Aufgabe-8.tex,
  ZitatSchluessel = examen:66116:2021:03,
  BearbeitungsStand = mit Lösung,
  Korrektheit = unbekannt,
  Ueberprueft = {unbekannt},
  Stichwoerter = {Client-Server-Modell},
  EinzelpruefungsNr = 66116,
  Jahr = 2021,
  Monat = 03,
  ThemaNr = 1,
  TeilaufgabeNr = 1,
  AufgabeNr = 8,
}

Das Client-Server-Modell ist ein Architekturmuster. Nennen Sie zwei
Vorteile einer nach diesem Muster gestalteten Architektur.
\index{Client-Server-Modell}
\footcite{examen:66116:2021:03}

\begin{bAntwort}
\begin{enumerate}
\item Einfache Integration weiterer Clients

\item Prinzipiell uneingeschränkte Anzahl der Clients
\bFussnoteUrl{https://www.karteikarte.com/card/164928/vorteile-und-nachteile-des-client-server-modells}

\item Es muss  nur ein Server gewartet werden. Dies gilt \zB für
Updates, die einmalig und zentral auf dem Server durchgeführt werden und
danach für alle Clients verfügbar sind.
\bFussnoteUrl{https://www.eoda.de/wissen/blog/client-server-architekturen-performance-und-agilitaet-fuer-data-science/}
\end{enumerate}
\end{bAntwort}

\end{document}
