\documentclass{bschlangaul-aufgabe}

\begin{document}
\bAufgabenMetadaten{
  Titel = {Aufgabe 1},
  Thematik = {private, package-private und protected},
  Referenz = 66116-2021-F.T1-TA1-A1,
  RelativerPfad = Staatsexamen/66116/2021/03/Thema-1/Teilaufgabe-1/Aufgabe-1.tex,
  ZitatSchluessel = examen:66116:2021:03,
  BearbeitungsStand = mit Lösung,
  Korrektheit = unbekannt,
  Ueberprueft = {unbekannt},
  Stichwoerter = {Sichtbarkeit},
  EinzelpruefungsNr = 66116,
  Jahr = 2021,
  Monat = 03,
  ThemaNr = 1,
  TeilaufgabeNr = 1,
  AufgabeNr = 1,
}

% F21-ST-TA1

% Aufgabe 1a, Sichbarkeiten:

%

\begin{enumerate}

%%
% a)
%%

\item Definieren Sie die Bedeutung der Sichtbarkeiten \emph{private},
\emph{package-private} und \emph{protected} von Feldern in Java-Klassen.
Erklären Sie diese kurz (je ein Satz).\index{Sichtbarkeit}
\footcite{examen:66116:2021:03}

\begin{bAntwort}
\begin{description}
\item[private]
Das Feld ist nur innerhalb der eigenen Klasse zugreifbar.

\item[package-private]
Bei \emph{package private} (kein \emph{modifier}) wird die Sichtbarkeit
auf das Paket, dem die Klasse angehört, ausgedehnt
(https://docs.oracle.com/javase/tutorial/java/javaOO/accesscontrol.html).

\item[protected]
Die Sichtbarkeit wird gegenüber \emph{package-private} auf alle
Subklassen
(auch Subklassen, die nicht dem Paket angehören) ausgedehnt.

\end{description}

\begin{tabular}{llllll}
Modifier & Class & Package & Sub (same pkg) & Sub (diff pkg) & World\\\hline
public & ja & ja & ja & ja & ja \\
protected & ja & ja & ja & ja & nein\\
\emph{no modifier} & ja & ja & ja & nein & nein\\
private & ja & nein & nein & nein & nein\\
\end{tabular}

https://stackoverflow.com/a/215505
\end{bAntwort}

%%
% b)
%%

\item Benennen Sie jeweils einen Grund für den Einsatz der
Sichtbarkeiten \emph{private}, \emph{package-private} und
\emph{protected} von Feldern in Java-Klassen.

\begin{bAntwort}
\begin{description}
\item[private]

Datenkapselung, Verbergen der internen Implementation. So kann die
interne Implementation geändert werden, ohne dass die öffentlichen
Schnittstellen sich ändern.

\item[package-private]

Um die Felder zwar innerhalb deines Pakets für alle Klassen zugänglich
zu machen aber nicht außerhalb, \zB in einem größeren Projekt.

\item[protected]

Wenn eine Klasse vererbt werden soll, \zB abstrakte Klassen.
\end{description}
\end{bAntwort}
\end{enumerate}
\end{document}
