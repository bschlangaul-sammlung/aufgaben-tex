\documentclass{bschlangaul-aufgabe}
\bLadePakete{entwurfsmuster}
\begin{document}
\bAufgabenMetadaten{
  Titel = {Aufgabe 6},
  Thematik = {Wissen Erbauer},
  Referenz = 66116-2021-F.T1-TA1-A6,
  RelativerPfad = Staatsexamen/66116/2021/03/Thema-1/Teilaufgabe-1/Aufgabe-6.tex,
  ZitatSchluessel = examen:66116:2021:03,
  BearbeitungsStand = mit Lösung,
  Korrektheit = unbekannt,
  Ueberprueft = {unbekannt},
  Stichwoerter = {Erbauer (Builder)},
  EinzelpruefungsNr = 66116,
  Jahr = 2021,
  Monat = 03,
  ThemaNr = 1,
  TeilaufgabeNr = 1,
  AufgabeNr = 6,
}

\begin{enumerate}

%%
% a)
%%

\item Erläutern Sie den Zweck (Intent) des Erzeugungsmusters Erbauer in
max. drei Sätzen, ohne dabei auf die technische Umsetzung einzugehen.
\index{Erbauer (Builder)}
\footcite{examen:66116:2021:03}

\begin{bAntwort}
Die Erzeugung komplexer Objekte wird vereinfacht, indem der
Konstruktionsprozess in eine spezielle Klasse verlagert wird. Er wird so
von der Repräsentation getrennt und kann sehr unterschiedliche
Repräsentationen zurückliefern.
\footcite[Seite 29]{eilebrecht}
\end{bAntwort}

%%
% b)
%%

\item Erklären Sie, wie das Erzeugungsmuster Erbauer umgesetzt werden
kann (Implementierung). Die Angabe von Code ist hierbei NICHT notwendig!

\begin{bAntwort}
\bMetaNochKeineLoesung
\end{bAntwort}

%%
% c)
%%

\item Nennen Sie jeweils einen Vor- und einen Nachteil des
Erzeugungsmusters Erbauer im Vergleich zu einer Implementierung ohne
dieses Muster.

\begin{bAntwort}
\begin{description}
\item[Vorteil]

Die Implementierungen der Konstruktion und der Repräsentationen werden
isoliert. Die Erbauer verstecken ihre interne Repräsentation vor dem
Direktor.

\item[Nachteil]

Es besteht eine enge Kopplung zwischen Produkt, konkretem Erbauer und
den am Konstruktionsprozess beteiligten Klassen.\footcite{wiki:erbauer}
\end{description}
\end{bAntwort}

\end{enumerate}

\begin{bExkurs}[Erbauer (Builder)]
\bEntwurfsErbauerUml

\bEntwurfsErbauerAkteure
\end{bExkurs}
\end{document}
