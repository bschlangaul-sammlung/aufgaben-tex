\documentclass{bschlangaul-aufgabe}
\bLadePakete{java}
\let\j=\bJavaCode
\begin{document}
\bAufgabenMetadaten{
  Titel = {Aufgabe 4},
  Thematik = {Klasse Box},
  Referenz = 66116-2021-F.T1-TA1-A4,
  RelativerPfad = Examen/66116/2021/03/Thema-1/Teilaufgabe-1/Aufgabe-4.tex,
  ZitatSchluessel = examen:66116:2021:03,
  BearbeitungsStand = mit Lösung,
  Korrektheit = unbekannt,
  Ueberprueft = {unbekannt},
  Stichwoerter = {Implementierung in Java},
  EinzelpruefungsNr = 66116,
  Jahr = 2021,
  Monat = 03,
  ThemaNr = 1,
  TeilaufgabeNr = 1,
  AufgabeNr = 4,
}

\begin{enumerate}

%%
% a)
%%

\item Schreiben Sie ein Programm in einer objektorientierten
Programmiersprache Ihrer Wahl, das den folgenden Anweisungen entspricht.
\index{Implementierung in Java}
\footcite{examen:66116:2021:03}

\begin{enumerate}

\item Es gibt eine Klasse mit dem Namen \j{Box}.

\item Alle Zahlen sind Fließkommazahlen.

\item \j{Box} wird mit einem Argument instanziiert, dessen Wert einer
Variable namens \j{length} zugewiesen wird.

\begin{bAntwort}
\bJavaExamen[firstline=6,lastline=8]{66116}{2021}{03}{Box}
\end{bAntwort}

\item \j{Box} hat eine Methode ohne Argumente namens \j{size}, welche
den Wert von \j{length} zurückgibt.

\begin{bAntwort}
\bJavaExamen[firstline=10,lastline=12]{66116}{2021}{03}{Box}
\end{bAntwort}

\item Eine weitere Methode namens \j{size} hat genau ein Argument namens
\j{width}. Diese zweite Methode namens \j{size} gibt das Produkt aus
\j{width} und \j{length} zurück. Eine weitere Methode namens \j{size}
hat genau zwei Argumente, nämlich eine Zahl \j{num} und einen Faktor
\j{f}. Es wird \j{length} minus das Produkt aus \j{num} und \j{f}
zurückgegeben.

\begin{bAntwort}
\bJavaExamen[firstline=14,lastline=20]{66116}{2021}{03}{Box}
\end{bAntwort}

\item Schreiben Sie eine \j{main}-Methode, die eine Box namens
\j{example} mit einer Länge von 15 anlegt.

\item Führen Sie die Methode \j{size} in der \j{main}-Methode wie unten
angegeben drei Mal aus.

\item Speichern Sie hierbei das Ergebnis jeweils in einer Variable
\j{mysize}. Geben Sie das Ergebnis jeweils in einer eigenen Zeile des
Ausgabemediums \j{System.out} aus.

\begin{itemize}
\item Mit keinen Argumenten
\item Mit dem Argument \j{10}
\item Mit den beiden Argumenten \j{5} und \j{2}
\end{itemize}

\begin{bAntwort}
\bJavaExamen[firstline=22,lastline=33]{66116}{2021}{03}{Box}
\end{bAntwort}

\end{enumerate}

\begin{bAntwort}
\bJavaExamen{66116}{2021}{03}{Box}
\end{bAntwort}

%%
% b)
%%

\item Notieren Sie die Ausgabe der \j{main}-Methode.

\begin{bAntwort}
\begin{itemize}
\item 15
\item 150
\item 5
\end{itemize}
\end{bAntwort}
\end{enumerate}

\end{document}
