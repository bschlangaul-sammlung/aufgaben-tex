\documentclass{bschlangaul-aufgabe}

\begin{document}
\bAufgabenMetadaten{
  Titel = {Aufgabe 3},
  Thematik = {Relation X und Y},
  Referenz = 66116-2019-F.T2-TA1-A3,
  RelativerPfad = Staatsexamen/66116/2019/03/Thema-2/Teilaufgabe-1/Aufgabe-3.tex,
  ZitatSchluessel = examen:66116:2019:03,
  BearbeitungsStand = mit Lösung,
  Korrektheit = unbekannt,
  Ueberprueft = {unbekannt},
  Stichwoerter = {Relationale Algebra, RelaX - relational algebra calculator},
  EinzelpruefungsNr = 66116,
  Jahr = 2019,
  Monat = 03,
  ThemaNr = 2,
  TeilaufgabeNr = 1,
  AufgabeNr = 3,
}

Gegeben seien folgende Relationen:
\index{Relationale Algebra}
\footcite{examen:66116:2019:03}

\bPseudoUeberschrift{X}

\begin{tabular}{|l|l|l|}
\hline
A & B & C \\ \hline
1 & 2 & 3 \\ \hline
2 & 3 & 4 \\ \hline
4 & 3 & 3 \\ \hline
1 & 2 & 2 \\ \hline
2 & 3 & 2 \\ \hline
3 & 3 & 2 \\ \hline
3 & 1 & 2 \\ \hline
2 & 2 & 1 \\ \hline
1 & 1 & 1 \\ \hline
\end{tabular}

\bPseudoUeberschrift{Y}

\begin{tabular}{|l|l|l|}
\hline
B & C & D \\ \hline
2 & 3 & 1 \\ \hline
1 & 1 & 3 \\ \hline
2 & 1 & 3 \\ \hline
3 & 2 & 3 \\ \hline
2 & 2 & 3 \\ \hline
1 & 3 & 2 \\ \hline
\end{tabular}

Geben Sie die Ergebnisrelationen folgender Ausdrücke der relationalen
Algebra als Tabellen an; machen Sie Ihren Rechenweg kenntlich.
\index{RelaX - relational algebra calculator}
% https://dbis-uibk.github.io/relax/calc/local/uibk/local/0

% X = {
%   A B C
%   1 2 3
%   2 3 4
%   4 3 3
%   1 2 2
%   2 3 2
%   3 3 2
%   3 1 2
%   2 2 1
%   1 1 1
% }

% Y = {
%   B C D
%   2 3 1
%   1 1 3
%   2 1 3
%   3 2 3
%   2 2 3
%   1 3 2
% }

\begin{enumerate}
\item $\sigma_{A=2}(X) \bowtie Y$
% sigma A = 2 (X) natural join Y

\begin{bAntwort}
\begin{tabular}{|l|l|l|l|l|}
\hline
A & B & C & D & E \\ \hline
2 & 3 & 2 & 3 & 1 \\ \hline
2 & 3 & 2 & 1 & 3 \\ \hline
2 & 3 & 2 & 2 & 3 \\ \hline
2 & 2 & 1 & 1 & 3 \\ \hline
2 & 2 & 1 & 3 & 2 \\ \hline
\end{tabular}

\end{bAntwort}

\item $(\pi_{B,C}(X) - \pi_{B,C}(Y)) \bowtie X$

% pi B,C X

% X = {
%   B C
%   2 3
%   3 4
%   3 3
%   2 2
%   3 2
%   1 2
%   2 1
%   1 1
% }

% pi B,C Y

% Y = {
%   B C
%   2 3
%   1 1
%   2 1
%   3 2
%   2 2
%   1 3
% }

% pi B,C X - pi B,C Y

\item

% pi A X

% X = {
%   A
%   1
%   2
%   4
%   3
% }

% sigma C=1 Y:

% Y = {
%   B C D
%   1 1 3
%   2 1 3
% }

% (pi A X) ⨝ (sigma C=1 Y)

%   A B C D
%   1 1 1 3
%   1 2 1 3
%   2 1 1 3
%   2 2 1 3
%   3 1 1 3
%   3 2 1 3
%   4 1 1 3
%   4 2 1 3

% (pi A X) ⨝ A > D (sigma C=1 Y)

%   A B C D
%   4 1 1 3
%   4 2 1 3

\begin{bAntwort}
\item
%pi A,C (X) ÷ pi C (Y)

\begin{tabular}{|l|}
\hline
A \\ \hline
1 \\ \hline
2 \\ \hline
3 \\ \hline
\end{tabular}
\end{bAntwort}

\end{enumerate}

\end{document}
