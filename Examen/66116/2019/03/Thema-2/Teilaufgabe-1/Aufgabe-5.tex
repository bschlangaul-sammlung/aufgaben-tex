\documentclass{bschlangaul-aufgabe}
\bLadePakete{sql}
\begin{document}
\bAufgabenMetadaten{
  Titel = {Aufgabe 5},
  Thematik = {App für konfizierte Schüler-Smartphones},
  Referenz = 66116-2019-F.T2-TA1-A5,
  RelativerPfad = Staatsexamen/66116/2019/03/Thema-2/Teilaufgabe-1/Aufgabe-5.tex,
  ZitatSchluessel = examen:66116:2019:03,
  BearbeitungsStand = mit Lösung,
  Korrektheit = unbekannt,
  Ueberprueft = {unbekannt},
  Stichwoerter = {SQL},
  EinzelpruefungsNr = 66116,
  Jahr = 2019,
  Monat = 03,
  ThemaNr = 2,
  TeilaufgabeNr = 1,
  AufgabeNr = 5,
}

Wir betrachten erneut das gegebene Schema aus Aufgabe 4. Primärschlüssel
sind unterstrichen, Fremdschlüssel sind überstrichen und der Text in den
darauf folgenden eckigen Klammern benennt die Relation, auf die
verwiesen wird.
\index{SQL}
\footcite{examen:66116:2019:03}

Der Notenschnitt, der Preis und die Bewertung werden als Kommazahl
dargestellt, wobei die Bewertung die Anzahl der Sterne angibt, also
maximal 5 und mindestens 0. Die Modellnummer kann sowohl aus Zahlen und
Buchstaben bestehen, ist jedoch nie länger als 50 Zeichen. ID, RAM,
Bildschirmdiagonale und Datum sind ganze Zahlen. Die restlichen
Attribute sind Strings der Länge 15.

Lehrer (Name, Fachl, Fach2, Fach3)

Schüler (Vorname, Nachname, Notenschnitt)

Smartphone (ID, Modellnr, RAM, Bildschirmdiagonale)

App (Name, Bewertung, Preis)

Eingesammelt (Vorname [Schüler], Nachname [Schüler], Name [Lehrer], ID [Smartphone], Datum)

Installiert (ID [Smartphone], Name [App])

\begin{enumerate}

%%
% a)
%%

\item Geben Sie die Anweisung in SOL-DDL an, die notwendig ist, um die
Relation ’App’ zu erzeugen.

\begin{bAntwort}
\begin{minted}{sql}
CREATE TABLE App(
Name VARCHAR(50) PRIMARY KEY,
Bewertung VARCHAR(255),
Preis INTEGER);
\end{minted}
\end{bAntwort}

%%
% b)
%%

\item Geben Sie die Anweisung in SOL-DDL an, die notwendig ist, um die
Relation ’Installiert’ zu erzeugen.

\begin{bAntwort}
\begin{minted}{sql}
CREATE TABLE Installiert(
ID INTEGER REFERENCES Smartphone(ID),
Name VARCHAR(50) App(Name),
PRIMARY KEY(ID,Name));
\end{minted}
\end{bAntwort}

%%
% c)
%%

\item Formulieren Sie die folgende Anfrage in SQL: Gesucht sind die
Namen der Apps zusammen mit ihrer Bewertung, die auf den Smartphones
installiert sind, die Lehrer Keating eingesammelt hat. Sortieren Sie das
Ergebnis nach Bewertung absteigend. Hinweis: Achten Sie auf gleichnamige
Attribute.

\begin{bAntwort}
\begin{minted}{sql}
SELECT DISTINCT a.Name, a.Bewertung
FROM Eingesammelt e, Installiert i, App a
WHERE e.ID =i.ID AND i.Name = a.Name AND e.Name = "Keating"
ORDER BY a.Bewertung DESC;
\end{minted}
\end{bAntwort}

%%
% d)
%%

\item Formulieren Sie die folgende Anfrage in SQL:

Gesucht ist der durchschnittliche Notenschnitt der Schüler, denen ein
iPhone 65 abgenommen wurde. Ein iPhone 6s kann A1633 als Modellnummer
haben oder A1688.

\begin{bAntwort}
\begin{minted}{sql}
SELECT DISTINCT AVG(s.Notenschnitt)
FROM Schüler s, Eingesammelt e, Smartphone sm
WHERE s.Vorname = e.Vorname AND s.Nachname = e.Nachname
AND e.ID=sm.ID AND(sm.Modellnr = 'A1633' OR sm.Modellnr = 'A1688');
\end{minted}
\end{bAntwort}

%%
% e)
%%

\item Formulieren Sie die folgende Anfrage in SQL:

Gesucht ist die Modellnummer der Smartphones, die durchschnittlich die
meisten Apps installiert haben.

Tipp: Die Verwendung von Views kann die Aufgabe vereinfachen.

\begin{bAntwort}
\begin{minted}{sql}
CREATE VIEW NumberApps AS
SELECT s.ID, s.Modellnr, COUNT(i.Name) AS number
FROM Smartphone s, Installiert i
WHERE s.ID = i.ID
GROUP BY s.ID
SELECT ModellNr, AVG(number)
FROM NumberApps
GROUP BY ModellNr
\end{minted}
\end{bAntwort}
\end{enumerate}
\end{document}
