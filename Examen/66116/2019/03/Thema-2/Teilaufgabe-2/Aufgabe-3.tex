\documentclass{bschlangaul-aufgabe}
\bLadePakete{uml}
\begin{document}
\bAufgabenMetadaten{
  Titel = {Aufgabe 3},
  Thematik = {Umfrage},
  Referenz = 66116-2019-F.T2-TA2-A3,
  RelativerPfad = Staatsexamen/66116/2019/03/Thema-2/Teilaufgabe-2/Aufgabe-3.tex,
  ZitatSchluessel = examen:66116:2019:03,
  BearbeitungsStand = TeX-Fehler,
  Korrektheit = unbekannt,
  Ueberprueft = {unbekannt},
  Stichwoerter = {Anwendungsfalldiagramm},
  EinzelpruefungsNr = 66116,
  Jahr = 2019,
  Monat = 03,
  ThemaNr = 2,
  TeilaufgabeNr = 2,
  AufgabeNr = 3,
}

\begin{enumerate}

%%
% a)
%%

\item Nehmen Sie an, dass in einem System zur Durchführung von Umfragen
die drei Anwendungsfälle \emph{„Teilnahme“}, \emph{„Fragen beantworten“}
und \emph{„Registrierung für Verlosung“} gegeben seien. Alle
Anwendungsfälle sind einem Teilnehmer (an einer Umfrage) zugänglich.
Stellen Sie die drei Anwendungsfälle in einem UML-Anwendungsfalldiagramm
(„use case diagram“) dar. Verwenden Sie dabei sowohl die <<extend>> -
als auch die <<include>>-Beziehung und beschreiben Sie die Bedeutung der
Beziehungen für die Existenz des „Teilnahme“-Anwendungsfalls.
\index{Anwendungsfalldiagramm}
\footcite{examen:66116:2019:03}

\begin{bAntwort}
\begin{tikzpicture}
\begin{umlsystem}{Umfrage}
\umlusecase[y=-2]{Teilnahme}
\umlusecase[y=0,x=2]{Fragen beantworten}
\umlusecase[y=-4,x=2]{Registrierung für Verlosung}
\end{umlsystem}

\umlactor[x=-3,y=-2]{user}
\umlinclude{usecase-1}{usecase-2}
\umlextend{usecase-3}{usecase-1}

\umlassoc{user}{usecase-1}
\umlassoc{user}{usecase-2}
\umlassoc{user}{usecase-3}
\end{tikzpicture}

\begin{description}
\item[<<include>>]

Der/die Anwender/in kann nur an der Umfrage teilnehmen, wenn er/sie die
Fragen beantwortet hat.

\item[<<extend>>]

Der/die Anwender/in kann zusätzlich zur Teilnahme an der Umfrage sich
für die Verlosung registrieren.
\end{description}
\end{bAntwort}

%%
% b)
%%

\item Welche Informationen sollte eine typische Spezifikation eines
Anwendungsfalls enthalten?

\begin{bAntwort}
Ein Use-Case-Diagramm zeigt das externe Verhalten eines Systems aus der
Sicht der Nutzer, indem es die Nutzer, die Use-Cases und deren
Beziehungen zueinander darstellt.
\footcite[Seite 241]{rupp}
\end{bAntwort}
\end{enumerate}
\end{document}
