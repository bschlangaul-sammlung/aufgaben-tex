\documentclass{bschlangaul-aufgabe}

\begin{document}
\bAufgabenMetadaten{
  Titel = {Aufgabe 3},
  Thematik = {Roboter in einer Montagehalle},
  Referenz = 66116-2019-F.T1-TA2-A3,
  RelativerPfad = Staatsexamen/66116/2019/03/Thema-1/Teilaufgabe-2/Aufgabe-3.tex,
  ZitatSchluessel = examen:66116:2019:03,
  BearbeitungsStand = mit Lösung,
  Korrektheit = unbekannt,
  Ueberprueft = {unbekannt},
  Stichwoerter = {Testen},
  EinzelpruefungsNr = 66116,
  Jahr = 2019,
  Monat = 03,
  ThemaNr = 1,
  TeilaufgabeNr = 2,
  AufgabeNr = 3,
}

\begin{enumerate}
%%
% a)
%%

\item Erläutern Sie kurz, was man unter der Methode der testgetriebenen
Entwicklung versteht.\index{Testen}
\footcite{examen:66116:2019:03}

\begin{bAntwort}
Bei der testgetriebenen Entwicklung erstellt der/die ProgrammiererIn
Softwaretests konsequent vor den zu testenden Komponenten.
\footcite{wiki:testgetriebene-entwicklung}
\end{bAntwort}

%%
% b)
%%

\item Geben Sie für obige Aufgabenstellung (Abarbeitung der Aufträge
durch den Roboter) einen Testfall für eine typische Situation an (d. h.
das Einsammeln von 4 Objekten an unterschiedlichen Positionen).
Spezifizieren Sie als Input alle für die Abarbeitung eines Auftrages
relevanten Eingabe- und Klassen-Parameter sowie den vollständigen und
korrekten Output.

\begin{bAntwort}
\bMetaNochKeineLoesung
\end{bAntwort}
\end{enumerate}
\end{document}
