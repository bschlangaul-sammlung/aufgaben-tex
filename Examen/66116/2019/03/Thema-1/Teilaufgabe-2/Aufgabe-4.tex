\documentclass{bschlangaul-aufgabe}

\begin{document}
\bAufgabenMetadaten{
  Titel = {Aufgabe 4},
  Thematik = {Roboter in einer Montagehalle},
  Referenz = 66116-2019-F.T1-TA2-A4,
  RelativerPfad = Examen/66116/2019/03/Thema-1/Teilaufgabe-2/Aufgabe-4.tex,
  ZitatSchluessel = examen:66116:2019:03,
  BearbeitungsStand = nur Angabe,
  Korrektheit = unbekannt,
  Ueberprueft = {unbekannt},
  Stichwoerter = {Aktivitätsdiagramm},
  EinzelpruefungsNr = 66116,
  Jahr = 2019,
  Monat = 03,
  ThemaNr = 1,
  TeilaufgabeNr = 2,
  AufgabeNr = 4,
}

\begin{enumerate}

%%
% a)
%%

\item Geben Sie für alle Methoden, die zur Abarbeitung des Auftrages für
den Roboter erforderlich sind, Pseudocode an. Nutzen Sie (im
Klassendiagramm definierte) Hilfsmethoden, um den Code übersichtlicher
zu gestalten. (Verwenden Sie statt einer komplizierten Methode mehrere
einfachere Methoden, die sich aufrufen.)
\index{Aktivitätsdiagramm}
\footcite{examen:66116:2019:03}

Achten Sie auf die Konsistenz zum Klassendiagramm (Inkonsistenzen führen
zu Punktabzügen).

%%
% b)
%%

\item Überführen Sie den Pseudocode der Hauptmethode zur Abarbeitung
eines Auftrags in ein syntaktisch korrektes UML-Aktivitätsdiagramm.
Schleifen und Verzweigungen müssen durch strukturierte Knoten
dargestellt werden (d. h. kein graphisches "goto").

\end{enumerate}
\end{document}
