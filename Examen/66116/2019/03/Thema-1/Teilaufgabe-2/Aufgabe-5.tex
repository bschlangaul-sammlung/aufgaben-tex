\documentclass{bschlangaul-aufgabe}
\bLadePakete{java}
\begin{document}
\bAufgabenMetadaten{
  Titel = {Aufgabe 5},
  Thematik = {Roboter in einer Montagehalle},
  Referenz = 66116-2019-F.T1-TA2-A5,
  RelativerPfad = Staatsexamen/66116/2019/03/Thema-1/Teilaufgabe-2/Aufgabe-5.tex,
  ZitatSchluessel = examen:66116:2019:03,
  BearbeitungsStand = mit Lösung,
  Korrektheit = unbekannt,
  Ueberprueft = {unbekannt},
  Stichwoerter = {Implementierung in Java},
  EinzelpruefungsNr = 66116,
  Jahr = 2019,
  Monat = 03,
  ThemaNr = 1,
  TeilaufgabeNr = 2,
  AufgabeNr = 5,
}

Geben Sie korrekten Code für die Abarbeitung eines Auftrages durch den
Roboter in einer objektorientierten Programmiersprache Ihrer Wahl an.
Sie sollen nur den Code für die Methoden angeben. Sie brauchen keinen
Code für Klassendefinitionen angeben, sondern können sich auf das
UML-Klassendiagramm aus Aufgabe 2 beziehen.
\index{Implementierung in Java}
\footcite{examen:66116:2019:03}

\bJavaExamen{66116}{2019}{03}{roboter/Auftrag}
\bJavaExamen{66116}{2019}{03}{roboter/AuftragsPosition}
\bJavaExamen{66116}{2019}{03}{roboter/LagerObjekt}
\bJavaExamen{66116}{2019}{03}{roboter/Material}
\bJavaExamen{66116}{2019}{03}{roboter/ObjektTyp}
\bJavaExamen{66116}{2019}{03}{roboter/Roboter}
\bJavaExamen{66116}{2019}{03}{roboter/Standort}
\bJavaExamen{66116}{2019}{03}{roboter/Werkzeug}

\end{document}
