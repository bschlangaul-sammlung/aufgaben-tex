\documentclass{bschlangaul-aufgabe}
\bLadePakete{relationale-algebra}
\begin{document}
\bAufgabenMetadaten{
  Titel = {Aufgabe 4},
  Thematik = {Game of Thrones},
  Referenz = 66116-2019-H.T1-TA2-A4,
  RelativerPfad = Staatsexamen/66116/2019/09/Thema-1/Teilaufgabe-2/Aufgabe-4.tex,
  ZitatSchluessel = examen:66116:2019:09,
  BearbeitungsStand = unbekannt,
  Korrektheit = unbekannt,
  Ueberprueft = {unbekannt},
  Stichwoerter = {Relationale Algebra},
  EinzelpruefungsNr = 66116,
  Jahr = 2019,
  Monat = 09,
  ThemaNr = 1,
  TeilaufgabeNr = 2,
  AufgabeNr = 4,
}

\let\t=\text

Übertragen Sie die folgenden Ausdrücke in die relationale Algebra.
Beschreiben Sie diese Ausdrücke umgangssprachlich, bevor Sie die
Umformung durchführen. Das Schema der Datenbank entspricht dem Schema
aus Aufgabe 3.\index{Relationale Algebra}
\footcite{examen:66116:2019:09}

\begin{enumerate}

%%
% 1.
%%

\item $\{ f \, | \, f \in \t{Figur} \land \neg \exists g \in \t{gehört\_zu}(f.\t{Id} = g.\t{Id})\}$

\begin{bAntwort}
Gesucht sind alle Figuren, die keiner Familie angehören.

\begin{displaymath}
\t{Figur} - \pi_{\t{Id}, \t{Name}, \t{Schwertkunst}, \t{Lebendig}, \t{Titel}}(\t{Figur} \bowtie \t{gehört\_zu})
\end{displaymath}
\end{bAntwort}

%%
% 2.
%%

\item $\{ f \, | \,
  f \in \t{Figur} \land
  \exists l \in \t{lebt}(l.\t{Id} = f.\t{Id}) \land
  \exists f_2 \in \t{Festung}(l.\t{Festung} = f.\t{Name}) \land
  \exists b \in \t{besetzt}(f_2.\t{Name} = b.\t{Festung}) \land
  \exists f_3 \in \t{Familie}(b.\t{Familie} = f_3.Id) \land
  f_3.\t{Name} = \t{Stark})
\}$

\begin{bAntwort}
In der Angabe steht $\t{lebt}(l.\t{Id} = c.\t{Id})$ wurde abgeändert in
$\t{lebt}(l.\t{Id} = \textbf{f}.\t{Id})$ Außerdem kommt $f$ mehrmals
vor. Wir wandeln $f_2$ in $f_3$ um und führen ein neues $f_2$ ein. Diese
schließende Klammer muss weg: $\t{Familie}(b.\t{Familie} =
f_3.Id\textbf{)}$ Übersichtlicher geschrieben

\begin{align*}
\{ f \, | \,
  & f                 & \in & \, \t{Figur} \land \\
  & \exists l   & \in & \, \t{lebt}(l.\t{Id} = f.\t{Id}) \land \\
  & \exists f_2 & \in & \, \t{Festung}(l.\t{Festung} = f.\t{Name})\land \\
  & \exists b   & \in & \, \t{besetzt}(f_2.\t{Name} = b.\t{Festung}) \land \\
  & \exists f_3 & \in & \,\t{Familie}(b.\t{Familie} = f_3.Id \land f_3.\t{Name} = \t{Stark})\\
\}
\end{align*}

Gesucht sind alle Figuren, die in einer von der Familie „Stark“
besetzten Festung leben.
\end{bAntwort}
\end{enumerate}

\end{document}
