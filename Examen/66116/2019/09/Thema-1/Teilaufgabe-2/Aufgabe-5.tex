\documentclass{bschlangaul-aufgabe}
\bLadePakete{normalformen,synthese-algorithmus}
\begin{document}
\bAufgabenMetadaten{
  Titel = {Aufgabe 5},
  Thematik = {Jedi-Ritter},
  Referenz = 66116-2019-H.T1-TA2-A5,
  RelativerPfad = Examen/66116/2019/09/Thema-1/Teilaufgabe-2/Aufgabe-5.tex,
  ZitatSchluessel = examen:66116:2019:09,
  BearbeitungsStand = unbekannt,
  Korrektheit = unbekannt,
  Ueberprueft = {unbekannt},
  Stichwoerter = {Normalformen},
  EinzelpruefungsNr = 66116,
  Jahr = 2019,
  Monat = 09,
  ThemaNr = 1,
  TeilaufgabeNr = 2,
  AufgabeNr = 5,
}

\let\schrittE=\bSyntheseUeberErklaerung
\let\b=\textbf
\let\ahl=\bLinksReduktionInline
\let\ahr=\bRechtsReduktionInline
\let\FA=\bFunktionaleAbhaengigkeiten
\let\fa=\bFunktionaleAbhaengigkeit
\let\m=\bAttributMenge
\begin{enumerate}

%-----------------------------------------------------------------------
% 1.
%-----------------------------------------------------------------------

\item Gegeben sind folgende funktionale Abhängigkeiten.
\index{Normalformen}
\footcite{examen:66116:2019:09}

\FA{
  X -> Y, Z;
  Z -> W, X;
  Q -> X, Y, Z;
  V -> Z, W;
  Z, W -> Y, Q, V;
}

% https://normalizer.db.in.tum.de/index.py

% QVWXYZ

% X -> YZ
% Z -> WX
% Q -> XYZ
% V -> ZW
% ZW -> YQV

% Q
% V
% W
% X
% Y
% Z

% Q V W X Y Z

Berechnen Sie die kanonische Überdeckung.

\begin{bAntwort}
\begin{enumerate}
\item \schrittE{1-1}

\bPseudoUeberschrift{\fa{Z, W -> Y, Q, V}}

$\m{Y, Q, V} \in$ \ahl{Z, W}{W}{\b{Q}, \b{V}, W, \b{Y}, Z}

\FA{
  X -> Y, Z;
  Z -> W, X;
  Q -> X, Y, Z;
  V -> Z, W;
  Z -> Y, Q, V;
}

\item \schrittE{1-2}

Auf der rechten Seite kommen mehrfach vor: W, X, Y, Z

\bPseudoUeberschrift{Y}

$Y \in$ \ahr{X -> Y, Z}{X -> Z}{X}{Q, V, W, X, \b{Y}, Z}

\FA{
  X -> Z;
  Z -> W, X;
  Q -> X, Y, Z;
  V -> Z, W;
  Z -> Y, Q, V;
}

$Y \in$ \ahr{Q -> X, Y, Z}{Q -> X, Z}{Q}{Q, V, W, X, \b{Y}, Z}

\FA{
  X -> Z;
  Z -> W, X;
  Q -> X, Z;
  V -> Z, W;
  Z -> Y, Q, V;
}

\bPseudoUeberschrift{Z}

$Z \notin$ \ahr{X -> Z}{}{X}{X}\\
$Z \in$ \ahr{Q -> X, Z}{Q -> X}{Q}{Q, V, W, X, Y, \b{Z}}

\FA{
  X -> Z;
  Z -> W, X;
  Q -> X;
  V -> Z, W;
  Z -> Y, Q, V;
}

$Z \notin$ \ahr{V -> Z, W}{V -> W}{V}{V, W}

\bPseudoUeberschrift{W}

$W \in$ \ahr{Z -> W, X}{Z -> X}{Z}{Q, V, \b{W}, X, Y, Z}

\FA{
  X -> Z;
  Z -> X;
  Q -> X;
  V -> Z, W;
  Z -> Y, Q, V;
}

\bPseudoUeberschrift{X}

% Q V W X Y Z
$X \in$ \ahr{Z -> X}{}{Z}{Q, V, W, \b{X}, Y, Z}

\FA{
  X -> Z;
  Z -> NICHTS;
  Q -> X;
  V -> Z, W;
  Z -> Y, Q, V;
}

\item \schrittE{1-3}

\FA{
  X -> Z;
  Q -> X;
  V -> Z, W;
  Z -> Y, Q, V;
}

\item \schrittE{1-4}

\bNichtsZuTun
\end{enumerate}
\end{bAntwort}

%-----------------------------------------------------------------------
% 2.
%-----------------------------------------------------------------------

\item Gegeben ist folgende Tabelle.

\begin{center}
\begin{tabular}{|l|l|l|l|l|}
\hline
\underline{JedID} & Name             & Rasse        & Lichtschwert & Seite der Macht\\\hline\hline
2     & Yoda             & Unbekannt & Grün         & Gute Seite\\\hline
3     & Anakin Skywalker & Mensch    & Blau, Rot    & Gute Seite, Dunkle Seite\\\hline
4     & Mace Windou      & Mensch    & Lila         & Gute Seite\\\hline
5     & Count Dooku      & Mensch    & Rot          & Dunkle Seite\\\hline
6     & Ahsoka Tano      & Togruta   & Grün         & Gute Seite\\\hline
7     & Yoda             & Mensch    & Rot          & Dunkle Seite\\\hline
\end{tabular}
\end{center}

Geben Sie zuerst die funktionalen Abhängigkeiten in der Tabelle an.

\begin{bAntwort}
\bFunktionaleAbhaengigkeiten{
JedID -> Name, Rasse, Lichtschwert, Seite der Macht;
Lichtschwert -> Seite der Macht;
}

JedID ist ein Surrogat-Schlüssel\bFussnoteUrl{https://de.wikipedia.org/wiki/Surrogatschlüssel}, \dh ein künstlich eingeführter
Primärschlüssel, von dem alle Attribute abhängen.

Ein rotes Lichtschwert zeigt an, dass der Jedi-Ritter zur dunklen Seite
der Macht gehört. Grüne, blaue und lila Lichtschwerter zeigen an, dass
der Jedi-Ritter zu guten Seite gehört. Wir können die Funktionale
Abhängigkeit nicht umdrehen, weil wir nicht von der guten Seite der
Macht auf die Farbe schließen können.
\end{bAntwort}

%-----------------------------------------------------------------------
% 3.
%-----------------------------------------------------------------------

\item \strut

\begin{enumerate}

%%
%  a)
%%

\item Geben Sie die zentrale Eigenschaft der 1. NF an.

\begin{bAntwort}
Eine Relation befindet sich in erster Normalform (1NF), wenn sie
ausschließlich atomare Attributwerte enthält.\footcite[Seite 448]{schneider}
\end{bAntwort}

%%
% b)
%%

\item Nennen Sie alle Stellen, an denen das Schema die 1. NF verletzt.

\begin{bAntwort}
Im Tupel (JedID = 3) haben die Attribute \emph{Lichtschwert} und
\emph{Seite der Macht} mehrwertige Attribute.
\end{bAntwort}

%%
% c)
%%

\item Überführen Sie die Tabelle in die 1. NF.

\begin{bAntwort}
\begin{center}
\begin{tabular}{|l|l|l|l|l|}
\hline
\underline{JedID} & Name             & Rasse        & Lichtschwert & Seite der Macht\\\hline\hline
2     & Yoda             & Unbekannt & Grün         & Gute Seite\\\hline
3     & Anakin Skywalker & Mensch    & Blau         & Gute Seite\\\hline
4     & Mace Windou      & Mensch    & Lila         & Gute Seite\\\hline
5     & Count Dooku      & Mensch    & Rot          & Dunkle Seite\\\hline
6     & Ahsoka Tano      & Togruta   & Grün         & Gute Seite\\\hline
7     & Yoda             & Mensch    & Rot          & Dunkle Seite\\\hline
\b{8}     & \b{Darth Vader}      & \b{Mensch}    & \b{Rot}          & \b{Dunkle Seite}\\\hline
\end{tabular}
\end{center}
\end{bAntwort}

\end{enumerate}

%-----------------------------------------------------------------------
% 4.
%-----------------------------------------------------------------------

\item \strut

\begin{enumerate}

%%
% a)
%%

\item Geben Sie die Definition der 2. NF an.

\begin{bAntwort}
Eine Relation in in zweiter Normalform (2NF), wenn sie in 1NF und jedes
Nichtschlüsselattribut von jedem Schlüsselkandidaten voll funktional
abhängig ist.\footcite[Seite 449]{schneider}
\end{bAntwort}

%%
% b)
%%

\item Arbeiten Sie bitte mit folgender, nicht korrekten Zwischenlösung weiter.
Erläutern Sie, inwiefern dieses Schema die 2. NF verletzt.

\begin{center}
\begin{tabular}{|l|l|l|l|l|}
\hline
JedID & Name      & Rasse     & Lichtschwert & Seite der Macht\\\hline\hline
2     & Yoda      & Unbekannt & Grün         & Gute Seite\\\hline
3     & Skywalker & Mensch    & Blau         & Gute Seite\\\hline
4     & Windou    & Mensch    & Lila         & Gute Seite\\\hline
5     & Dooku     & Mensch    & Rot          & Dunkle Seite\\\hline
6     & Tano      & Togruta   & Grün         & Gute Seite\\\hline
2     & Yoda      & Mensch    & Rot          & Dunkle Seite\\\hline
\end{tabular}
\end{center}

%%
% c)
%%

\item Überführen Sie die Tabelle in die 2. NF.
\end{enumerate}

%-----------------------------------------------------------------------
% 5.
%-----------------------------------------------------------------------

\item 5.

\begin{enumerate}

%%
% a)
%%

\item Geben Sie die Definition der 3. NF an.

\begin{bAntwort}
Eine Relation befindet sind in der dritten Normalform (3NF), wenn keine
transitiven Abhängigkeiten der Nichtschlüsselattribute existieren.
\footcite[Seite 449]{schneider}
\end{bAntwort}

%%
% b)
%%

\item Erläutern Sie, ob und wenn ja, wie das von Ihnen in 3c) neu
erstellte Schema die 3. NF verletzt.

\end{enumerate}
\end{enumerate}
\end{document}
