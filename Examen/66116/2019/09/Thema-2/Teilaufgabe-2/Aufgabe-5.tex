\documentclass{bschlangaul-aufgabe}

\begin{document}
\bAufgabenMetadaten{
  Titel = {Aufgabe 5},
  Thematik = {Relationen „Professor“ und „Vorlesung“},
  Referenz = 66116-2019-H.T2-TA2-A5,
  RelativerPfad = Staatsexamen/66116/2019/09/Thema-2/Teilaufgabe-2/Aufgabe-5.tex,
  ZitatSchluessel = examen:66116:2019:09,
  BearbeitungsStand = mit Lösung,
  Korrektheit = unbekannt,
  Ueberprueft = {unbekannt},
  Stichwoerter = {Anfrageoptimierung},
  EinzelpruefungsNr = 66116,
  Jahr = 2019,
  Monat = 09,
  ThemaNr = 2,
  TeilaufgabeNr = 2,
  AufgabeNr = 5,
}

Gegeben seien die beiden Relationen Professor und Vorlesung mit folgendem Umfang:
Relation Professor: 5 Spalten, 164 Datensätze
Relation Vorlesung: 10 Spalten, 333 Datensätze.
\index{Anfrageoptimierung}
\footcite{examen:66116:2019:09}

Optimieren Sie auf geeignete Weise folgende SQL-Anweisung möglichst gut und berechnen Sie wie
stark sich die Datenmenge durch jede Optimierung reduziert (nehmen Sie für Ihre Berechnung an, dass
Herr Mustermann genau zwei Vorlesungen hält). Geben Sie jeweils den Operatorbaum vor und nach
Ihren jeweiligen Optimierungen an.

SELECT Titel FROM Professor, Vorlesung WHERE Name = Mustermann’ AND PersNr =
gelesenVon;

\end{document}
