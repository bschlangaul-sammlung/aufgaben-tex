\documentclass{bschlangaul-aufgabe}

\begin{document}
\bAufgabenMetadaten{
  Titel = {Aufgabe 1},
  Thematik = {Sportverein},
  Referenz = 66116-2019-H.T2-TA2-A1,
  RelativerPfad = Staatsexamen/66116/2019/09/Thema-2/Teilaufgabe-2/Aufgabe-1.tex,
  ZitatSchluessel = examen:66116:2019:09,
  BearbeitungsStand = mit Lösung,
  Korrektheit = unbekannt,
  Ueberprueft = {unbekannt},
  Stichwoerter = {Entity-Relation-Modell},
  EinzelpruefungsNr = 66116,
  Jahr = 2019,
  Monat = 09,
  ThemaNr = 2,
  TeilaufgabeNr = 2,
  AufgabeNr = 1,
}

Erstellen Sie ein möglichst einfaches ER-Schema, das alle gegebenen
Informationen enthält. Attribute von Entitäten und Beziehungen sind
anzugeben, Schlüsselattribute durch Unterstreichen zu kennzeichnen.
Verwenden Sie für die Angabe der Kardinalitäten von Beziehungen die
Min-MaxNotation. Führen Sie Surrogatschlüssel nur dann ein, wenn es
nötig ist und modellieren Sie nur die im Text vorkommenden Elemente.
\index{Entity-Relation-Modell}
\footcite{examen:66116:2019:09}

Ein örtlicher Sportverein möchte seine Vereinsangelegenheiten mittels
einer Datenbank verwalten. Der Verein besteht aus verschiedenen
Abteilungen, welche eine eindeutige Nummer und einen aussagekräftigen
Namen besitzen. Für jede Abteilung soll zudem automatisch die Anzahl der
Mitglieder gespeichert werden, wobei ein Mitglied zu mehreren
Abteilungen gehören kann. Die Mitglieder des Vereins können keine, eine
oder mehrere Rollen (auch Ämter genannt) einnehmen. So gibt es die
Ämter: 1. Vorstand, 2. Vorstand, Kassier, Jugendleiter, Trainer sowie
einen Abteilungsleiter für jede Abteilung. Es ist dabei auch möglich,
dass ein Abteilungsleiter mehrere Abteilungen leitet oder ein Mitglied
mehrere Aufgaben übernimmt, mit der Einschränkung, dass die
Vorstandsposten und Kassier nicht von der gleichen Person ausgeübt
werden dürfen. Zu jedem Trainer wird eine Liste von Lizenzen
gespeichert. Jeder Trainer ist zudem in mindestens einer Abteilung eine
bestimmte Anzahl von Stunden tätig. Zu allen Mitgliedern werden
Mitgliedsnummer, Name (bestehend aus Vor- und Nachname), Geburtsdatum,
E-Mail, Eintrittsdatum, Adresse (bestehend aus PLZ, Ort, Straße,
Hausnummer), IBAN und die Vereinszugehörigkeit in Jahren gespeichert.

Im Verein fallen Finanztransaktionen an. Zu jeder Transaktion wird ein
Zeitstempel, der Betrag und eine eindeutige Transaktionsnummer
gespeichert. Die Mitglieder leisten Zahlungen an den Verein. Umgekehrt
erstattet der Verein auch bestimmte Kosten. Im Verein existieren drei
verschiedene Mitgliedsbeiträge. So gibt es einen
Kinder-und-Jugendlichen-Tarif, einen Erwachsenentarif und einen
Familientarif.

Mitgliedern entstehen des Öfteren Fahrtkosten. Für jede
Fahrtkostenabrechnung werden das Datum, die gefahrenen Kilometer und
Start und Ziel, der Zweck sowie das Mitglied gespeichert, welches den
Antrag gestellt hat. Zu jeder Fahrtkostenabrechnung existiert genau eine
Erstattung.

Durch die Teilnahme an verschiedenen Wettbewerben besteht die
Notwendigkeit die von einem Team (also mehreren Mitgliedern zusammen)
oder Mitgliedern erzielten sportlichen Erfolge, d.h. Platzierungen, zu
verwalten. Jeder Wettkampf besitzt eine eindeutige ID, ein Datum und
eine Kurzbeschreibung.

Das Vereinsleben besteht aus zahlreichen Terminen, die durch Datum und
Uhrzeit innerhalb einer

Abteilung eindeutig identifiziert werden können. Zu jedem Termin wird
zusätzlich eine Kurzbeschreibung gespeichert.

\end{document}
