\documentclass{bschlangaul-aufgabe}
\bLadePakete{normalformen,synthese-algorithmus}
\begin{document}
\bAufgabenMetadaten{
  Titel = {Aufgabe 4},
  Thematik = {R (A,B,C,D,E,F)},
  Referenz = 66116-2019-H.T2-TA2-A4,
  RelativerPfad = Staatsexamen/66116/2019/09/Thema-2/Teilaufgabe-2/Aufgabe-4.tex,
  ZitatSchluessel = examen:66116:2019:09,
  BearbeitungsStand = unbekannt,
  Korrektheit = unbekannt,
  Ueberprueft = {unbekannt},
  Stichwoerter = {Normalformen},
  EinzelpruefungsNr = 66116,
  Jahr = 2019,
  Monat = 09,
  ThemaNr = 2,
  TeilaufgabeNr = 2,
  AufgabeNr = 4,
}

\let\ah=\bAttributHuelle
\let\ahl=\bLinksReduktionInline
\let\ahr=\bRechtsReduktionInline
\let\b=\textbf
\let\fa=\bFunktionaleAbhaengigkeit
\let\FA=\bFunktionaleAbhaengigkeiten
\let\m=\bAttributMenge
\let\r=\bRelation
\let\schrittE=\bSyntheseUeberErklaerung
\let\u=\underline

Gegeben sei das Relationenschema R (A, B, C, D, E, F) sowie die Menge
der zugehörigen funktionalen Abhängigkeiten F:
\index{Normalformen}
\footcite{examen:66116:2019:09}

\FA[F]{
  A, B -> C;
  A -> D;
  F -> B;
  D, E -> B;
  B -> A;
}

% https://normalizer.db.in.tum.de/index.py

% ABCDEF

% AB -> C
% A -> D
% F -> B
% DE -> B
% B -> A

\begin{enumerate}
%%
% a)
%%

\item Bestimmen Sie sämtliche Schlüsselkandidaten der Relation R und
begründen Sie, warum es keine weiteren Schlüsselkandidaten geben kann.

\begin{bAntwort}
Die Attribute E, F kommen auf keiner rechten Seite vor.

$\ah{F, \m{E, F}} = \m{A, B, C, D, E, F} = R$

Der Superschlüssel kann nicht weiter minimiert werden:

$\ah{F, \m{E}} = \m{E} \neq R$\\
$\ah{F, \m{F}} = \m{A, B, C, D, F} \neq R$

Der Schlüsselkandidat ist \m{E, F}

\end{bAntwort}

%%
% b)
%%

\item Ist die gegebene Menge an funktionalen Abhängigkeiten minimal?
Fall sie minimal ist begründen Sie diese Eigenschaft ausführlich,
anderenfalls minimieren Sie FD schrittweise. Vergessen Sie nicht die
einzelnen Schritte entsprechend zu begründen.

\begin{bAntwort}
\begin{enumerate}
\item \schrittE{1-1}

\bPseudoUeberschrift{\fa{A, B -> C}}

$C \in$ \ahl{A, B}{A}{A, B, \b{C}, D}

\FA[F]{
  B -> C;
  A -> D;
  F -> B;
  D, E -> B;
  B -> A;
}

\bPseudoUeberschrift{\fa{D, E -> B}}

$B \notin$ \ahl{D, E}{D}{E}\\
$B \notin$ \ahl{D, E}{E}{D}

\item \schrittE{1-2}

\bPseudoUeberschrift{B}

$B \notin$ \ahr{F -> B}{}{F}{F}\\
$B \notin$ \ahr{D, E -> B}{}{D, E}{D, E}

\FA[F]{
  B -> C;
  A -> D;
  F -> B;
  D, E -> B;
  B -> A;
}

\item \schrittE{1-3}

\bNichtsZuTun

\item \schrittE{1-4}

\FA[F]{
  A -> D;
  F -> B;
  D, E -> B;
  B -> A, C;
}
\end{enumerate}
\end{bAntwort}

%%
% c)
%%

\item Überführen Sie falls nötig das Schema in dritte Normalform. Ist
die dritte Normalform bereits erfüllt, begründen Sie dies ausführlich.

\begin{bAntwort}
\begin{enumerate}
\item \schrittE{1}

\FA[F]{
  A -> D;
  F -> B;
  D, E -> B;
  B -> A, C;
}

\item \schrittE{2}

\r[R1]{\u{A}, D}\\
\r[R2]{\u{F}, B}\\
\r[R3]{\u{D, E}, B}\\
\r[R3]{\u{B}, A, C}

\item \schrittE{3}

\r[R1]{\u{A}, D}\\
\r[R2]{\u{F}, B}\\
\r[R3]{\u{D, E}, B}\\
\r[R4]{\u{B}, A, C}\\
\r[R5]{\u{E, F}}
\item \schrittE{4}

\bNichtsZuTun

\end{enumerate}
\end{bAntwort}
\end{enumerate}
\end{document}
