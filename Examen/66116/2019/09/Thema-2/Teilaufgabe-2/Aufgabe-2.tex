\documentclass{bschlangaul-aufgabe}
\bLadePakete{er}
\begin{document}
\bAufgabenMetadaten{
  Titel = {Aufgabe 2},
  Thematik = {Mitarbeiterverwaltung},
  Referenz = 66116-2019-H.T2-TA2-A2,
  RelativerPfad = Staatsexamen/66116/2019/09/Thema-2/Teilaufgabe-2/Aufgabe-2.tex,
  ZitatSchluessel = examen:66116:2019:09,
  BearbeitungsStand = unbekannt,
  Korrektheit = unbekannt,
  Ueberprueft = {unbekannt},
  Stichwoerter = {Entity-Relation-Modell},
  EinzelpruefungsNr = 66116,
  Jahr = 2019,
  Monat = 09,
  ThemaNr = 2,
  TeilaufgabeNr = 2,
  AufgabeNr = 2,
}

In einer Datenbank zur Mitarbeiterverwaltung werden die Mitarbeiter über
ihr Ausscheiden aus dem Betrieb hinaus (z. B. Ruhestand oder
Arbeitsplatzwechsel) gespeichert. Im Folgenden ist ein Ausschnitt aus
dem ER-Diagramm dargestellt. Erweitern Sie das Diagramm um genau eine
Entität, welche die derzeit aktiven Mitarbeiter aus allen Unterklassen
umfasst. Benennen und erläutern Sie das von Ihnen verwendete
Modellierungskonstrukt.
\index{Entity-Relation-Modell}
\footcite{examen:66116:2019:09}

\begin{center}
\begin{tikzpicture}[er2]
\node[entity] (Mitarbeiter) {Mitarbeiter};

\node[below=1cm of Mitarbeiter,circle,draw] (disjunkt) {d}
  edge[weak] node {$\bigcup$} (Mitarbeiter);

\node[entity,below left=2cm of disjunkt] (Sekretär) {Sekretär}
  edge (disjunkt);

\node[entity,below=1.3cm of disjunkt] (Forscher) {Forscher}
  edge (disjunkt);

\node[entity,below right=2cm of disjunkt] (Techniker) {Techniker}
  edge (disjunkt);
\end{tikzpicture}
\end{center}

\begin{bAntwort}
EER-Notation nach
Elmasri/Navathe

\begin{center}
\begin{tikzpicture}[er2]
\node[entity] (Mitarbeiter) {Mitarbeiter};

\node[below=1cm of Mitarbeiter,circle,draw] (disjunkt) {d}
  edge[weak] node {$\bigcup$} (Mitarbeiter);

\node[entity,below left=2cm of disjunkt] (Sekretär) {Sekretär}
  edge (disjunkt);

\node[entity,below=1.3cm of disjunkt] (Forscher) {Forscher}
  edge (disjunkt);

\node[entity,below right=2cm of disjunkt] (Techniker) {Techniker}
  edge (disjunkt);

\node[below=1cm of Forscher,circle,draw] (union) {u}
  edge (Sekretär) edge (Forscher) edge (Techniker);

\node[below=1cm of union,entity] (AktiveMitarbeiter) {Aktive Mitarbeiter}
edge node {$\bigcup$} (union);
\end{tikzpicture}
\end{center}
U = Vereinigungsmenge
\end{bAntwort}
\end{document}
