\documentclass{bschlangaul-aufgabe}

\begin{document}
\bAufgabenMetadaten{
  Titel = {Aufgabe 1},
  Thematik = {Wetterdienst},
  Referenz = 66116-2020-F.T2-TA2-A1,
  RelativerPfad = Examen/66116/2020/03/Thema-2/Teilaufgabe-2/Aufgabe-1.tex,
  ZitatSchluessel = examen:66116:2020:03,
  BearbeitungsStand = mit Lösung,
  Korrektheit = unbekannt,
  Ueberprueft = {unbekannt},
  Stichwoerter = {Entity-Relation-Modell},
  EinzelpruefungsNr = 66116,
  Jahr = 2020,
  Monat = 03,
  ThemaNr = 2,
  TeilaufgabeNr = 2,
  AufgabeNr = 1,
}

Hinweis: Bei Wahl dieser Aufgabe wird Wissen über das erweiterte Entity-
Relationship-Modell (beispielsweise schwache Entity-Typen, Vererbung)
sowie die Verfeinerung eines relationalen Sche- mas vorausgesetzt.
\index{Entity-Relation-Modell}
\footcite{examen:66116:2020:03}

Gegeben seien folgende Informationen:
\begin{itemize}

\item Ein Wetterdienst - identifiziert durch einen eindeutigen Namen -
betreibt mehrere Wetterstationen, die mit einer eindeutigen Nummer je
Wetterdienst identifiziert werden können. Jede Wetterstation hat zudem
mehrere Messgeräte, die wiederum pro Wetterstation einen eindeutigen
Code besitzen und zudem eine Betriebsdauer.

\item Ein Wetterdienst hat eine Adresse.

\item Bei den Messgeräten wird unter anderem zwischen manuellen und
digitalen Messgeräten unterschieden. Dabei kann ein Messgerät immer nur
zu einer Kategorie gehören.

\item Meteorologen sind Mitarbeiter eines Wetterdienstes, haben einen
Namen und werden über eine Personalnummer identifiziert. Zudem soll
gespeichert werden, in welcher Wetterstation welcher Mitarbeiter zu
welchem Zeitpunkt arbeitet.

\item Meteorologen können für eine Menge an Messgeräten verantwortlich
sein. Manuelle Messgeräte werden von Meteorologen abgelesen.

\item Ein Wettermoderator präsentiert das vorhergesagte Wetter eines
Wetterdienstes für einen Fernsehsender.

\item Für einen Wettermoderator wird der eindeutige Name und die Größe
gespeichert. Ein Fernsehsender wird ebenfalls über den eindeutigen
Namen identifiziert.

\end{itemize}

\begin{enumerate}
\item Erstellen Sie für das oben gegebene Szenario ein geeignetes ER-Diagramm.

Verwenden Sie dabei - wenn angebracht - das Prinzip der
Spezialisierung. Kennzeichnen Sie die Primärschlüssel der Entity-Typen,
totale Teilnahmen (existenzabhängige Beziehungen) und schwache
Entity-Typen.

Zeichnen Sie die Funktionalitäten der Relationship-Typen in das Diagramm
ein.

\item Überführen Sie das in Teilaufgabe a) erstellte ER-Modell in ein
verfeinertes relationales Schema. Kennzeichnen Sie die Schlüssel durch
Unterstreichen. Datentypen müssen nicht angegeben werden. Die einzelnen
Schritte müssen angegeben werden.

\begin{bAntwort}
1. Schritt: Starke Entity-Typen einfügen
Wetterdienst { Name, Adresse }
Meteorologe { Personalnummer, Name }
Wettermoderator { Name, Größe }
Fernsehsender { Name }
2. Schritt: Schwache Entity-Typen einfügen
Wetterstation { Name [Wetterdienst], Nummer }
Messgeraet { Name [Wetterdienst], Nummer [Wetterstation], Code, Betriebsdauer }
3. Schritt: Is-A-Beziehung einfügen
ManuellesMessgeraet { Name [Wetterdienst], Nummer [Wetterstation], Code [Messgeraet] }
DigitalesMessgeraet { Name [Wetterdienst], Nummer [Wetterstation], Code [Messgeraet] }
4. Schritt: Beziehungen einfügen
Betreibt { Name [Wetterdienst], Nummer }
Hat { Name [Wetterdienst], Nummer [Wetterstation], Code [Messgeraet] }
ArbeitetIn { Personalnummer [Meteorologe], Name [Wetterdienst], Nummer [Wetterstation], Zeitpunkt }
IstMitarbeiterVon { Personalnummer [Meteorologe], Name [Wetterdienst] }
IstVerantwortlichFür { Personalnummer [Meteorologe], Name [Wetterdienst], Nummer [Wetterstation],
Code [Messgeraet] }
LiestAb { Personalnummer [Meteorologe], Name [Wetterdienst], Nummer [Wetterstation], Code
[Messgeraet] }
PraesentiertWetter { ModeratorenName [Wettermoderator], FernsehsenderName [Fernsehsender],
WetterdienstName [Wetterdienst] }
(Bei PraesentiertWetter sind nur zwei Felder Teil des Primärschlüssels, da es sich um eine 1 : 1 : n-
Beziehung handelt.)
5. Schritt: Verfeinerung
Wetterdienst { Name, Adresse }
Wetterstation { Name [Wetterdienst], Nummer [Wetterstation] }
Messgeraet { Name [Wetterdienst], Nummer [Wetterstation], Code [Messgeraet], Betriebsdauer,
VerantwortlicherMitarbeiterPersonalnummer [Meteorologe] }
ManuellesMessgeraet { Name [Wetterdienst], Nummer [Wetterstation], Code [Messgeraet] }
DigitalesMessgeraet { Name [Wetterdienst], Nummer [Wetterstation], Code [Messgeraet] }
Meteorologe { Personalnummer, Name, WetterdienstName [Wetterdienst] }
ArbeitetIn { Personalnummer [Meteorologe], Name [Wetterdienst], Nummer [Wetterstation], Zeitpunkt }
LiestAb { Personalnummer [Meteorologe], Name [Wetterdienst], Nummer [Wetterstation], Code
[Messgeraet] }
Wettermoderator { Name, Größe }
Fernsehsender { Name }
PraesentiertWetter { ModeratorenName [Wettermoderator], FernsehsenderName [Fernsehsender],
WetterdienstName [Wetterdienst] }
\end{bAntwort}
\end{enumerate}
\end{document}
