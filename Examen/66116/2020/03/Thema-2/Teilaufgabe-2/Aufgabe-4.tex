\documentclass{bschlangaul-aufgabe}
\bLadePakete{mathe}
\begin{document}
\bAufgabenMetadaten{
  Titel = {Aufgabe 4},
  Thematik = {Schedule S},
  Referenz = 66116-2020-F.T2-TA2-A4,
  RelativerPfad = Staatsexamen/66116/2020/03/Thema-2/Teilaufgabe-2/Aufgabe-4.tex,
  ZitatSchluessel = examen:66116:2020:03,
  BearbeitungsStand = mit Lösung,
  Korrektheit = unbekannt,
  Ueberprueft = {unbekannt},
  Stichwoerter = {Transaktionen},
  EinzelpruefungsNr = 66116,
  Jahr = 2020,
  Monat = 03,
  ThemaNr = 2,
  TeilaufgabeNr = 2,
  AufgabeNr = 4,
}

\def\l#1#2#3{$\text{#1lock}_#2(#3)$}

\begin{enumerate}
\item Betrachten Sie den folgenden Schedule S:
\index{Transaktionen}
\footcite{examen:66116:2020:03}

\begin{center}
\begin{tabular}{l|l|l}
$T_1$ & $T_2$ & $T_3$ \\\hline
         & $r_2(z)$ & \\
         &          & $w_3(y)$ \\
         & $r_2(x)$ & \\
$w_1(x)$ &          & \\
         & $w_2(x)$ & \\
         &          & $r_3(z)$ \\
         &          & $c_3$ \\
         & $w_2(z)$ & \\
$w_1(y)$ &          & \\
$c_1$    &          & \\
         & $c_2$    & \\
\end{tabular}
\end{center}

Geben Sie den Ausgabeschedule (einschließlich der Operationen zur
Sperranforderung und -freigabe) im rigorosen Zweiphasen-Sperrprotokoll
für den obigen Eingabeschedule S an.

\begin{bAntwort}
\begin{center}
\begin{tabular}{l|l|l}
$T_1$         & $T_2$         & $T_3$ \\\hline
              & \l r2z        & \\
              & $r_2(z)$      & \\
              &               & \l x3y \\
              &               & $w_3(y)$ \\
              & \l r2x        & \\
              & $r_2(x)$      & \\
              & \l x2x        & \\
              & $w_2(x)$      & \\
              &               & \l r3z \\
              &               & $r_3(z)$ \\
              &               & $c_3$ \\
              &               & \l {un}2{y,z}\\
              & \l x2z        & \\
              & $w_2(z)$      & \\
              & $c_2$         & \\
              & \l {un}2{x,z} & \\
\l x1x        &               & \\
$w_1(x)$      &               & \\
\l x1y        &               & \\
$w_1(y)$      &               & \\
$c_1$         &               & \\
\l {un}2{x,z} &               & \\

\end{tabular}
\end{center}
\end{bAntwort}

\item Beschreiben Sie den Unterschied zwischen dem herkömmlichen
Zweiphasen-Sperrprotokoll (2PL) und dem rigorosen
Zweiphasen-Sperrprotokoll. Warum wird in der Praxis häufiger das
rigorose Zweiphasen-Sperrprotokoll verwendet?

\begin{bAntwort}
Bei beiden Protokollen fordert die Transaktion erst alle Sperren an
(Anforderungsphase) und gibt sie später frei (Freigabephase).

\begin{itemize}
\item Beim strengen oder rigorosen 2PL werden die Sperren dann
angefordert, wenn sie benötigt werden, beim konservativen 2PL werden
alle Sperren zu Beginn gemeinsam angefordert.

\item Beim strengen oder rigorosen 2PL werden die Lesesperren bis zum
Commit, die Schreibsperren sogar bis nach dem Commit gehalten. Beim
konservativen 2PL werden dagegen die Sperren freigegeben, wenn sie nicht
mehr benötigt werden.

\end{itemize}

Das rigorose 2PL wird der Praxis häufiger verwendet, weil dabei nicht zu
Beginn der Transaktion bekannt sein muss, welche Sperren benötigt
werden, und durch die schrittweise Anforderung der Sperren unter
Umständen ein höheres Maß an Parallelität erreicht werden kann.
\end{bAntwort}

\end{enumerate}
\end{document}
