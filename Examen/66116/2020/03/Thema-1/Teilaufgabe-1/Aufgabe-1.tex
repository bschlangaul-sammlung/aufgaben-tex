\documentclass{bschlangaul-aufgabe}
\bLadePakete{uml}
\begin{document}
\bAufgabenMetadaten{
  Titel = {Aufgabe 1},
  Thematik = {Bankautomat},
  Referenz = 66116-2020-F.T1-TA1-A1,
  RelativerPfad = Staatsexamen/66116/2020/03/Thema-1/Teilaufgabe-1/Aufgabe-1.tex,
  ZitatSchluessel = examen:66116:2020:03,
  BearbeitungsStand = mit Lösung,
  Korrektheit = unbekannt,
  Ueberprueft = {unbekannt},
  Stichwoerter = {Zustandsdiagramm zeichnen},
  EinzelpruefungsNr = 66116,
  Jahr = 2020,
  Monat = 03,
  ThemaNr = 1,
  TeilaufgabeNr = 1,
  AufgabeNr = 1,
}

\begin{enumerate}

%%
% a)
%%

\item Basisfunktion\index{Zustandsdiagramm zeichnen}
\footcite{examen:66116:2020:03}

Erstellen Sie ein Zustandsdiagramm für einen Bankautomat, welcher den im
Folgenden beschriebenen Authentifizierungsvorgang von Bankkunden
realisiert. Modellieren Sie dazu soweit nötig sowohl Zustände und
Transitionsbedingungen als auch die Aktionen der Zustände.

Der Bankautomat startet im Grundzustand und wartet auf das Einlegen
einer Bankkarte. Wird eine Karte eingelegt, wird diese eingezogen und
der Automat startet die Überprüfung der Bankkarte. Ist die Karte
ungültig, wird die Karte ausgeworfen und der Automat wechselt in den
Grundzustand. Ist die Karte gültig, kann die vierstellige PIN eingelesen
werden. Nach der Bestätigung der Eingabe wird diese überprüft. Ist die
PIN gültig, so stoppt der Automat und zeigt eine erfolgreiche
Authentifizierung an. Ist die PIN ungültig, zeigt der Automat einen
Fehler an und erlaubt eine erneute Eingabe der PIN.

\begin{bAntwort}
% vergleiche Rupp Seite 334
\begin{tikzpicture}

\umlbasicstate[x=0,y=0]{wartend}

\umlbasicstate[x=8,y=0,do=überprüfe die Karte]{überprüfend}

\umlbasicstate[x=0,y=-4]{Pin einlesend}

\umlbasicstate[x=5,y=-4]{Pin überprüfen}

\umltrans[arg={[Karte eingelegt] / Karte einziehed}]{wartend}{überprüfend}

\umltrans[arg={[ungültig] / Karte auswerfen}]{überprüfend}{wartend}

\umltrans[arg={[gültig]}]{überprüfend}{Pin einlesend}

\umltrans[arg=Bestätigung der Eingabe]{Pin einlesend}{Pin einlesend}

\end{tikzpicture}
\end{bAntwort}

%%
% b)
%%

\item Erweiterung

Der Bankautomaten aus Aufgabe a) soll nun so verändert werden, dass ein
Bankkunde nach dem ersten fehlerhaften Eingeben der PIN die PIN erneut
eingeben muss. Bei erneuter Falscheingabe, wird eine dritte Eingabe
möglich. Bei der dritten Falscheingabe der PIN wird die Karte vom
Automaten eingezogen und der Automat geht wieder in den Ausgangszustand
über.

Hinweis: Für diese Aufgabe dürfen Sie Ihr Zustandsdiagramm aus a) weiter
verwenden, wenn Sie eindeutig, z. B. durch den Einsatz von Farben
kennzeichnen, was nur zur Aufgabe a) gehört und was Abänderungen des
Zustandsdiagramms aus a) sind. Sie können, falls Sie einen neuen
Automaten zeichnen, Zustände und Übergänge, die inhaltsgleich zur Lösung
des Aufgabenteils a) sind mit einem “W“ markieren, statt sie zu
beschriften. In diesem Fall wird der Text aus der Lösung zu Aufgabenteil
a) an dieser Stelle wiederholt gedacht.
\end{enumerate}
\end{document}
