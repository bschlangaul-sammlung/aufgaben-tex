\documentclass{bschlangaul-aufgabe}
\bLadePakete{syntax,normalformen}
\begin{document}
\bAufgabenMetadaten{
  Titel = {Aufgabe 5},
  Thematik = {Schlüssel},
  Referenz = 66116-2020-F.T1-TA2-A5,
  RelativerPfad = Staatsexamen/66116/2020/03/Thema-1/Teilaufgabe-2/Aufgabe-5.tex,
  ZitatSchluessel = examen:66116:2020:03,
  BearbeitungsStand = mit Lösung,
  Korrektheit = unbekannt,
  Ueberprueft = {unbekannt},
  Stichwoerter = {Schlüssel},
  EinzelpruefungsNr = 66116,
  Jahr = 2020,
  Monat = 03,
  ThemaNr = 1,
  TeilaufgabeNr = 2,
  AufgabeNr = 5,
}

Gegeben sei die Relation \bRelation{A,B,C}
\index{Schlüssel}
\footcite{examen:66116:2020:03}

\begin{enumerate}
\item Schreiben Sie eine SQL-Anfrage, mit der sich zeigen lässt, ob das
Paar $A$, $B$ ein Superschlüssel der Relation $R$ ist. Beschreiben Sie
ggf. textuell - falls nicht eindeutig ersichtlich - wie das Ergebnis
Ihrer Anfrage interpretiert werden muss, um zu erkennen ob $A$, $B$ ein
Superschlüssel ist.

\begin{bAntwort}
Diese Anfrage darf keine Ergebnisse liefern, dann ist das Paar $A$, $B$ ein
Superschlüssel.

\begin{minted}{sql}
SELECT *
FROM R
GROUP BY A, B
HAVING COUNT(*) > 1;
\end{minted}
\end{bAntwort}

\item Erläutern Sie den Unterschied zwischen einem Superschlüssel und
einem Kandidatenschlüssel. Tipp: Was muss gelten, damit $A$, $B$ ein
Kandidatenschlüssel ist und nicht nur ein Superschlüssel?

\begin{bAntwort}
Ein Superschlüssel ist ein Attribut oder
eine Attributkombination, von der \emph{alle Attribute} einer Relation
funktional \emph{abhängen}.
\footcite[Seite 181 Kapitel 6.2 „Superschlüssel“]{kemper}

Ein Kandidatenschlüssel ist ein \emph{minimaler} Superschlüssel. Keine
Teilmenge dieses Superschlüssels ist ebenfalls Superschlüssels.

\end{bAntwort}

\item Sei $A$, $B$ der Kandidatenschlüssel für die Relation $R$. Geben
Sie eine minimale Ausprägung der Relation $R$ an, die diese Eigenschaft
erfüllt.

\begin{bAntwort}
\begin{tabular}{|l|l|l|}
\hline
A & B & C \\\hline\hline
1 & 2 & 3 \\\hline
2 & 1 & 4 \\\hline
1 & 1 & 5 \\\hline
2 & 2 & 5 \\\hline
\end{tabular}
\end{bAntwort}

\end{enumerate}
\end{document}
