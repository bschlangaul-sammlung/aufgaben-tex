\documentclass{bschlangaul-aufgabe}
\bLadePakete{mathe}
\begin{document}
\bAufgabenMetadaten{
  Titel = {Aufgabe 3},
  Thematik = {Universitätsschema},
  Referenz = 66116-2020-F.T1-TA2-A3,
  RelativerPfad = Examen/66116/2020/03/Thema-1/Teilaufgabe-2/Aufgabe-3.tex,
  ZitatSchluessel = examen:66116:2020:03,
  BearbeitungsStand = mit Lösung,
  Korrektheit = unbekannt,
  Ueberprueft = {unbekannt},
  Stichwoerter = {Relationale Algebra, Tupelkalkül},
  EinzelpruefungsNr = 66116,
  Jahr = 2020,
  Monat = 03,
  ThemaNr = 1,
  TeilaufgabeNr = 2,
  AufgabeNr = 3,
}

Gegeben sei ein Universitätsschema.
\index{Relationale Algebra}
\footcite{examen:66116:2020:03}

\begin{enumerate}

%%
% a)
%%

\item Finden Sie alle Studierenden, die keine Vorlesung hören.
Formulieren Sie die Anfrage im Tupelkalkül.
\index{Tupelkalkül}

\begin{bAntwort}
$\{ s \in \text{Studierende} \land h \in \text{hören} \, | \, \neg \exists \text{s.MatrNr} = \text{h.MatrNr} \}$
\end{bAntwort}

%%
% b)
%%

\item Geben Sie einen Ausdruck an, der die Relation \neg \texttt{hören}
erzeugt. Diese enthält für jeden Studierenden und jede Vorlesung, die
der Studierende \textbf{nicht} hört, einen Eintrag mit Matrikelnummer
und Vorlesungsnummer. Formulieren Sie die Anfrage in
\textbf{relationaler Algebra}.

\begin{bAntwort}
$\rho_{\neg \text{hören}} \Bigl(
\bigl(
  \pi_{\text{MatrNr}}(\text{Studierende})
  \times
  \pi_{\text{VorlNr}}(\text{Vorlesungen})
\bigr) - \text{hören}
\Bigr)
$
\end{bAntwort}

%%
% c)
%%

\item Finden Sie alle Studierenden, die \textbf{keine} Vorlesung hören.
Formulieren Sie die Anfrage in \textbf{relationaler Algebra}.

\begin{bAntwort}
$\pi_{\text{MatrNr}}(\text{Studierende}) - \pi_{\text{MatrNr}}(\text{hören})$
\end{bAntwort}
\end{enumerate}
\end{document}
