\documentclass{bschlangaul-aufgabe}
\bLadePakete{baum}
\begin{document}
\bAufgabenMetadaten{
  Titel = {Aufgabe 6},
  Thematik = {Physische Datenstrukturen)},
  Referenz = 66116-2020-F.T1-TA2-A6,
  RelativerPfad = Staatsexamen/66116/2020/03/Thema-1/Teilaufgabe-2/Aufgabe-6.tex,
  ZitatSchluessel = examen:66116:2020:03,
  BearbeitungsStand = mit Lösung,
  Korrektheit = unbekannt,
  Ueberprueft = {unbekannt},
  Stichwoerter = {B-Baum},
  EinzelpruefungsNr = 66116,
  Jahr = 2020,
  Monat = 03,
  ThemaNr = 1,
  TeilaufgabeNr = 2,
  AufgabeNr = 6,
}

Fügen Sie die Zahl $4$ in den folgenden B-Baum ein.
\index{B-Baum}
\footcite{examen:66116:2020:03}

\begin{center}
\begin{tikzpicture}[
  b bbaum,
  level 1/.style={level distance=15mm,sibling distance=28mm},
  level 2/.style={level distance=10mm,sibling distance=20mm},
]
\node {10 \nodepart{two} 20 \nodepart{three} . \nodepart{four} .} [->]
  child {
    node {5 \nodepart{two} 6 \nodepart{three} 7 \nodepart{four} 8}
  }
  child {
    node {12 \nodepart{two} 14 \nodepart{three} 16 \nodepart{four} 18}
  }
  child {
    node {25 \nodepart{two} 30 \nodepart{three} . \nodepart{four} .}
  };
\end{tikzpicture}
\end{center}

\noindent
Zeichnen Sie den vollständigen, resultierenden Baum.

\begin{bAntwort}
\begin{center}
\begin{tikzpicture}[
  b bbaum,
  level 1/.style={level distance=15mm,sibling distance=28mm},
  level 2/.style={level distance=10mm,sibling distance=20mm},
]
\node {6 \nodepart{two} 10 \nodepart{three}  20 \nodepart{four} .} [->]
  child {
    node {4 \nodepart{two} 5 \nodepart{three} . \nodepart{four} .}
  }
  child {
    node {7 \nodepart{two} 8 \nodepart{three} . \nodepart{four} .}
  }
  child {
    node {12 \nodepart{two} 14 \nodepart{three} 16 \nodepart{four} 18}
  }
  child {
    node {25 \nodepart{two} 30 \nodepart{three} . \nodepart{four} .}
  };
\end{tikzpicture}
\end{center}
\end{bAntwort}

\end{document}
