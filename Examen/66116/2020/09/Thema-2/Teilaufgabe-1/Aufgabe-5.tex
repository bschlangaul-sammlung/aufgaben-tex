\documentclass{bschlangaul-aufgabe}
\bLadePakete{java}
\begin{document}
\bAufgabenMetadaten{
  Titel = {Aufgabe 5},
  Thematik = {Terme über die Rechenarten},
  Referenz = 66116-2020-H.T2-TA1-A5,
  RelativerPfad = Staatsexamen/66116/2020/09/Thema-2/Teilaufgabe-1/Aufgabe-5.tex,
  ZitatSchluessel = examen:66116:2020:09,
  BearbeitungsStand = mit Lösung,
  Korrektheit = unbekannt,
  Ueberprueft = {unbekannt},
  Stichwoerter = {Entwurfsmuster},
  EinzelpruefungsNr = 66116,
  Jahr = 2020,
  Monat = 09,
  ThemaNr = 2,
  TeilaufgabeNr = 1,
  AufgabeNr = 5,
}

% https://student.cs.uwaterloo.ca/~cs446/1171/Arch_Design_Activity/Composite.pdf
% https://docplayer.org/9839515-Grundkurs-programmieren-in-java.html

Wir betrachten Terme über die Rechenarten $op \in \{ +, -, \cdot, \div \}$, die
rekursiv definiert sind:
\index{Entwurfsmuster}
\footcite{examen:66116:2020:09}

\begin{itemize}
\item Jedes Literal ist ein Term, \zB „$4$“.

\item Jedes Symbol ist ein Term, \zB „$x$“.

\item Ist $t$ ein Term, so ist „$(t)$“ ein (geklammerter) Term.

\item Sind $t_1$, $t_2$ Terme, so ist „$t_1 \, op \, t_2$“ ebenso ein
Term.

\end{itemize}
Beispiele für gültige Terme sind also „$4 + 8$“, „$4 \cdot x$“ oder „$4
+ (8 \cdot x)$“.
\begin{enumerate}

%%
% a)
%%

\item Welches Design-Pattern eignet sich hier am besten zur Modellierung
dieses Sachverhalts?

\begin{bAntwort}
Kompositum
\end{bAntwort}

%%
% b)
%%

\item Nennen Sie drei wesentliche Vorteile von Design-Pattern im
Allgemeinen.

%%
% c)
%%

\item Modellieren Sie eine Klassenstruktur in UML, die diese rekursive
Struktur von \emph{Termen} abbildet. Sehen Sie mindestens einzelne
Klassen für die \emph{Addition} und \emph{Multiplikation} vor, sowie
weitere Klassen für \emph{geklammerte Terme} und \emph{Literale}, welche
ganze Zahlen repräsentieren. Gehen Sie bei der Modellierung der
Klassenstruktur davon aus, dass eine objektorientierte
Programmiersprache wie Java zu benutzen ist.

%%
% d)
%%

\item Erstellen Sie ein Objektdiagramm, welches den Term $t := 4 + (3
\cdot 2) + (12 \cdot y / (8 \cdot x))$ entsprechend Ihres
Klassendiagramms repräsentiert.

%%
% e)
%%

\item Überprüfen Sie, ob das Objektdiagramm für den in Teilaufgabe d)
gegebenen Term eindeutig definiert ist. Begründen Sie Ihre Entscheidung.

%%
% f)
%%

\item Die gegebene Klassenstruktur soll mindestens folgende Operationen
unterstützen:

\begin{itemize}
\item das Auswerten von Termen,
\item das Ausgeben in einer leserlichen Form,
\item das Auflisten aller verwendeten Symbole.
\end{itemize}

Welches Design-Pattern ist hierfür am besten geeignet?

%%
% g)
%%

\item Erweitern Sie Ihre Klassenstruktur um die entsprechenden Methoden,
Klassen und Assoziationen, um die in Teilaufgabe f) genannten
zusätzlichen Operationen gemäß dem von Ihnen genannten Design Pattern zu
unterstützen.

\begin{bAntwort}
\bJavaExamen{66116}{2020}{09}{rechenarten/Addition}
\bJavaExamen{66116}{2020}{09}{rechenarten/Divison}
\bJavaExamen{66116}{2020}{09}{rechenarten/GeklammerterTerm}
\bJavaExamen{66116}{2020}{09}{rechenarten/Literal}
\bJavaExamen{66116}{2020}{09}{rechenarten/Multiplikation}
\bJavaExamen{66116}{2020}{09}{rechenarten/Rechenart}
\bJavaExamen{66116}{2020}{09}{rechenarten/Subtraktion}
\bJavaExamen{66116}{2020}{09}{rechenarten/Symbol}
\bJavaExamen{66116}{2020}{09}{rechenarten/Term}
\end{bAntwort}

\end{enumerate}
\end{document}
