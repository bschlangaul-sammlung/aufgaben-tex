\documentclass{bschlangaul-aufgabe}

\begin{document}
\bAufgabenMetadaten{
  Titel = {Aufgabe 1},
  Thematik = {Wissensfragen},
  Referenz = 66116-2020-H.T2-TA1-A1,
  RelativerPfad = Staatsexamen/66116/2020/09/Thema-2/Teilaufgabe-1/Aufgabe-1.tex,
  ZitatSchluessel = examen:66116:2020:09,
  BearbeitungsStand = mit Lösung,
  Korrektheit = unbekannt,
  Ueberprueft = {unbekannt},
  Stichwoerter = {Wasserfallmodell, Prozessmodelle, Kontinuierliche Integration (Continuous Integration), EXtreme Programming, Softwaremaße, SCRUM},
  EinzelpruefungsNr = 66116,
  Jahr = 2020,
  Monat = 09,
  ThemaNr = 2,
  TeilaufgabeNr = 1,
  AufgabeNr = 1,
}

\begin{enumerate}

%%
% a)
%%

\item Nennen Sie fünf Phasen, die im Wasserfallmodell durchlaufen werden
sowie deren jeweiliges Ziel bzw.
Ergebnis(-dokument).\index{Wasserfallmodell}
\index{Prozessmodelle}
\footcite{examen:66116:2020:09}

\begin{bAntwort}
\begin{description}
\item[Anforderung]

Lasten- und Pflichtenheft

\item[Entwurf]

Softwarearchitektur in Form von Struktogrammen, UML-Diagrammen etc.

\item[Implementierung]

Software

\item[Überprüfung]

überarbeitete Software

\item[Einsatz und Wartung]

erneut überarbeitete Software, verbesserte Wartung
\end{description}
\end{bAntwort}

%%
% b)
%%

\item Nennen Sie drei Arten der Softwarewartung und geben Sie jeweils
eine kurze Beschreibung an.

\begin{bAntwort}
\begin{description}
\item[korrektive Wartung]

Korrektur von Fehlern, die schon beim Kunden in Erscheinung
getreten sind

\item[präventive Wartung]

Korrektur von Fehlern, die beim Kunden
noch nicht in Erscheinung getreten sind

\item[adaptive Wartung]

Anpassung der Software an neue Anforderungen

\item[perfektionierende Wartung]

Verbesserung von Performance und Wartbarkeit und Behebung von technical
depts\bFussnoteUrl{https://de.wikipedia.org/wiki/Softwarewartung}
\end{description}
\end{bAntwort}

%%
% c)
%%

\item Eine grundlegende Komponente des \emph{Extreme Programming} ist
\emph{„Continuous Integration“}. Erklären Sie diesen Begriff und warum
man davon einen Vorteil erwartet.
\index{Kontinuierliche Integration (Continuous Integration)}
\index{EXtreme Programming}

\begin{bAntwort}
Die Software wird nach jeder Änderung (push) automatisch kompiliert und
auch getestet. Man erwartet sich davon einen Vorteil, weil Fehler
schneller erkannt werden und die aktuelle Version des Systems ständig
verfügbar ist.
\end{bAntwort}

%%
% d)
%%

\item Nennen Sie zwei Softwaremetriken und geben Sie jeweils einen Vor-
und Nachteil an, der im Projektmanagement von Bedeutung
ist.\index{Softwaremaße}

\begin{bAntwort}
\bPseudoUeberschrift{Anzahl der Code-Zeilen}

\begin{description}
\item[Vorteil:]

Ist sehr einfach zu bestimmen.

\item[Nachteil:]

Entwickler können die Zeilenzahl manipulieren (zusätzliche Leerzeilen,
Zeilen aufteilen), ohne dass sich die Qualität des Codes verändert.
\end{description}

\bPseudoUeberschrift{Zyklomatische Komplexität}

\begin{description}
\item[Vorteil:]

Trifft eine Aussage über die Qualität des Algorithmus.

\item[Nachteil:]

Ist für Menschen nicht intuitiv zu erfassen. Zwei Codes, die dasselbe
Problem lösen können die gleiche Zyklomatische Komplexität haben, obwohl
der eine wesentlich schlechter zu verstehen ist (Spaghetticode!).
\end{description}
\end{bAntwort}

%%
% e)
%%

\item Nennen und beschreiben Sie kurz drei wichtige Aktivitäten, welche
innerhalb einer Sprint-Iteration von Scrum durchlaufen werden.
\index{SCRUM}

\begin{bAntwort}
\begin{description}
\item[Sprint Planning]

Im Sprint Planning werden zwei Fragen beantwortet:

\begin{itemize}
\item Was kann im kommenden Sprint entwickelt werden?
\item Wie wird die Arbeit im kommenden Sprint erledigt?
\end{itemize}

Die Sprint-Planung wird daher häufig in zwei Teile geteilt. Sie dauert
in Summe maximal 2 Stunden je Sprint-Woche, beispielsweise maximal acht
Stunden für einen 4-Wochen-Sprint.

\item[Daily Scrum]

Zu Beginn eines jeden Arbeitstages trifft sich das Entwicklerteam zu
einem max. 15-minütigen Daily Scrum. Zweck des Daily Scrum ist der
Informationsaustausch.

\item[Sprint Review]

Das Sprint Review steht am Ende des Sprints. Hier überprüft das
Scrum-Team das Inkrement, um das Product Backlog bei Bedarf anzupassen.
Das Entwicklungsteam präsentiert seine Ergebnisse und es wird überprüft,
ob das zu Beginn gesteckte Ziel erreicht wurde.
\footcite{wiki:scrum}
\end{description}
\end{bAntwort}

%%
% f)
%%

\item Erläutern Sie den Unterschied zwischen dem Product-Backlog und dem
Sprint-Backlog.

\begin{bAntwort}
\begin{description}
\item[Product Backlog]

Das Product Backlog ist eine geordnete Auflistung der Anforderungen an
das Produkt.

\item[Sprint Backlog]

Das Sprint Backlog ist der aktuelle Plan der für einen Sprint zu
erledigenden Aufgaben.
\footcite{wiki:scrum}
\end{description}
\end{bAntwort}

%%
% g)
%%

\item Erläutern Sie, warum eine agile Entwicklungsmethode nur für
kleinere Teams (max. 10 Personen) gut geeignet ist, und zeigen Sie einen
Lösungsansatz, wie auch größere Firmen agile Methoden sinnvoll einsetzen
können.

\begin{bAntwort}
Bei agilen Entwicklungsmethoden finden daily scrums statt, die in der
intendierten Kürze nur in kleinen Teams umsetzbar sind. Außerdem ist in
größeren Teams oft eine Hierarchie vorhanden, die eine schnelle
Entscheidungsfindung verhindert. Man könnte die große Firma in viele
kleine Teams einteilen, die jeweils an kleinen (Teil-)Projekten
arbeiten. Eventuell können die product owner der einzelnen Teams nach
agilen Methoden zusammenarbeiten.
\end{bAntwort}

\end{enumerate}
\end{document}
