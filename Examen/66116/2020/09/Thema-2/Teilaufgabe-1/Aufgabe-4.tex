\documentclass{bschlangaul-aufgabe}

\begin{document}
\bAufgabenMetadaten{
  Titel = {Aufgabe 4},
  Thematik = {Objektorientierte Analyse},
  Referenz = 66116-2020-H.T2-TA1-A4,
  RelativerPfad = Staatsexamen/66116/2020/09/Thema-2/Teilaufgabe-1/Aufgabe-4.tex,
  ZitatSchluessel = examen:66116:2020:09,
  BearbeitungsStand = mit Lösung,
  Korrektheit = unbekannt,
  Ueberprueft = {unbekannt},
  Stichwoerter = {Objektorientierung, Sequenzdiagramm},
  EinzelpruefungsNr = 66116,
  Jahr = 2020,
  Monat = 09,
  ThemaNr = 2,
  TeilaufgabeNr = 1,
  AufgabeNr = 4,
}

Betrachten Sie das folgende Szenario:

Entwickeln Sie für einen Kunden eine einheitliche Online-Plattform, in
welcher mehrere Restaurants angebunden sind. Die Nutzer des Systems
sollen Essen wie Pizzen, Burger oder Pasta bestellen und dabei aus einer
Liste von verschiedenen Gerichten auswählen können. Die Plattform soll
zusätzliche Optionen (\zB Lieferung durch einen Lieferdienst oder
direkte Abholung, inkl. Salat oder einer Flasche Wein) ermöglichen. Die
Bestellung soll dann von der Plattform an den jeweiligen
Gaststättenbetreiber gesendet werden. Die Besteller sollen zudem eine
Bestätigung als Nachricht erhalten. Die jeweiligen Gerichte und Optionen
haben unterschiedliche Preise, die dem Internetnutzer angezeigt werden
müssen, bevor er diese auswählt. Der Endpreis muss vor der endgültigen
Zahlung des Auftrags angezeigt werden. Kunden können (optional) einen
Benutzeraccount anlegen und erhalten bei häufigen Bestellungen einen
Rabatt.\index{Objektorientierung}
\footcite{examen:66116:2020:09}

\begin{enumerate}

%%
% a)
%%

\item Beschreiben Sie kurz ein Verfahren, wie Sie aus der
Szenariobeschreibung mögliche Kandidaten für Klassen erhalten können.

\begin{bAntwort}
Verfahren nach Abbott;
\bFussnoteUrl{http://info.johpie.de/stufe_q1/info_01_verfahren_abbott.pdf}

Objektorientierte Analyse und Design (OOAD)
\end{bAntwort}

%%
% b)
%%

\item Beschreiben Sie kurz ein Verfahren, um Vererbungshierarchien zu
identifizieren.

\begin{bAntwort}
Eine Begriffshierachie mit den Substantiven des Textes bilden.
\end{bAntwort}

%%
% c)
%%

\item Geben Sie fünf geeignete Klassen für das obige Szenario an. Nennen
Sie dabei keine Klassen, welche durch Basisdatentypen wie Integer oder
String abgedeckt werden können.

\begin{bAntwort}
\begin{itemize}
\item Benutzer (Kunde, Gaststättenbetreiber)
\item Restaurant
\item Gericht (Pizza, Burger, Pasta)
\item ZusatzOption (Lieferung, Salat, Wein)
\item Bestellung
\end{itemize}
\end{bAntwort}

%%
% d)
%%

\item Nennen Sie drei Klassen für das obige Szenario, die direkt durch
Basisdatentypen wie Integer oder String abgedeckt werden können.

\begin{bAntwort}
\begin{itemize}
\item Nachricht
\item Preis
\item Rabatt
\end{itemize}
\end{bAntwort}

%%
% e)
%%

\item Erstellen Sie ein Sequenzdiagramm für das gegebene System mit
folgendem Anwendungsfall: Ein Nutzer bestellt zwei Pizzen und eine
Flasche Wein. Beginnen Sie mit der „Auswahl des Gerichts“ bis hin zur
„Bestätigung der Bestellung“ sowie „Lieferung an die Haustür“. Das
Diagramm soll mindestens je einen Akteur für Benutzer (Browser),
Applikation (Webserver) und Restaurant (Koch und Lieferdienst)
vorsehen. Die Bezahlung selbst muss nicht modelliert werden.
\index{Sequenzdiagramm}

\begin{bAntwort}
\bMetaNochKeineLoesung
\end{bAntwort}

\end{enumerate}
\end{document}
