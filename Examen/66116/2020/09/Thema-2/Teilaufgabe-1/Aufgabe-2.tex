\documentclass{bschlangaul-aufgabe}
\bLadePakete{gantt}
\begin{document}
\bAufgabenMetadaten{
  Titel = {Aufgabe 2},
  Thematik = {Projektmanagement},
  Referenz = 66116-2020-H.T2-TA1-A2,
  RelativerPfad = Examen/66116/2020/09/Thema-2/Teilaufgabe-1/Aufgabe-2.tex,
  ZitatSchluessel = examen:66116:2020:09,
  BearbeitungsStand = mit Lösung,
  Korrektheit = unbekannt,
  Ueberprueft = {unbekannt},
  Stichwoerter = {Gantt-Diagramm},
  EinzelpruefungsNr = 66116,
  Jahr = 2020,
  Monat = 09,
  ThemaNr = 2,
  TeilaufgabeNr = 1,
  AufgabeNr = 2,
}

Betrachten Sie die folgende Tabelle zum Projektmanagement:
\index{Gantt-Diagramm}
\footcite{examen:66116:2020:09}

\begin{center}
\begin{tabular}{|c|c|l|}
\hline
\textbf{Arbeitspaket} & \textbf{Dauer (Tage)} & \textbf{abhängig von}\\\hline
A1 & 5  & \\\hline
A2 & 5  & A1 \\\hline
A3 & 10 & A2 \\\hline
A4 & 25 & A1 \\\hline
A5 & 10 & A1, A3\\\hline
A6 & 10 & \\\hline
A7 & 5  & A2, A3 \\\hline
A8 & 10 & A4, A5, A6 \\\hline
\end{tabular}
\end{center}

\begin{enumerate}
%%
% a)
%%

\item Erstellen Sie ein Gantt-Diagramm, das die in der Tabelle
angegebenen Abhängigkeiten berücksichtigt. Das Diagramm muss nicht
maßstabsgetreu sein, jedoch jede Information aus der gegebenen Tabelle
enthalten.

\begin{bAntwort}
\begin{center}
\begin{ganttchart}[
  x unit=0.25cm,
  y unit chart=0.8cm,
  vgrid
]{1}{40}
\ganttbar[name=4]{A4}{6}{30} \\
\ganttbar[name=3]{A3}{11}{20} \\
\ganttbar[name=7]{A7}{21}{25} \\
\ganttbar[name=2]{A2}{6}{10} \\
\ganttbar[name=1]{A1}{1}{5} \\
\ganttbar[name=5]{A5}{21}{30} \\
\ganttbar[name=6]{A6}{1}{10}\\
\ganttbar[name=8]{A8}{31}{40}
\gantttitlelist[
  title list options={var=\y, evaluate={} as \x}
]{1,...,40}{1}\\
\gantttitlelist[
  title list options={var=\i, evaluate={int(\i * 5)} as \x}
]{1,...,8}{5}\\

%
\ganttlink[link type=f-s]{1}{2}
\ganttlink[link type=f-s]{2}{3}
%
\ganttlink[link type=f-s]{1}{4}
%
\ganttlink[link type=f-s]{1}{5}
\ganttlink[link type=f-s]{3}{5}
%
\ganttlink[link type=f-s]{2}{7}
\ganttlink[link type=f-s]{3}{7}
%
\ganttlink[link type=f-s]{4}{8}
\ganttlink[link type=f-s]{5}{8}
\ganttlink[link type=f-s]{6}{8}
\end{ganttchart}
\end{center}
\end{bAntwort}

%%
% b)
%%

\item Wie lange dauert das Projekt mindestens?

\begin{bAntwort}
Es dauert mindestens 40 Tage.
\end{bAntwort}

%%
% c)
%%

\item Geben Sie alle kritischen Pfade an.

\begin{bAntwort}
A1 $\rightarrow$ A4 $\rightarrow$ A8

A1 $\rightarrow$ A2 $\rightarrow$ A3 $\rightarrow$ A5 $\rightarrow$ A8
\end{bAntwort}

%%
% d)
%%

\item Bewerten Sie folgende Aussage eines Projektmanagers: „Falls unser
Projekt in Verzug gerät, bringen uns neue Programmierer auch nicht
weiter.“

\begin{bAntwort}
Ist der Verzug in einem Arbeitspaket, in dem genügend Pufferzeit
vorhanden ist (hier A6), so hilft ein neuer Programmierer nicht
unbedingt weiter. Geht allerdings die Pufferzeit zu Ende oder ist erst
gar nicht vorhanden (\zB im kritischen Pfad), so kann ein/e neue/r
ProgrammerIn helfen, falls deren/dessen Einarbeitsungszeit gering ist.
Muss sich der/die Neue erste komplett einarbeiten, so wird er/sie wohl
auch keine große Hilfe sein.
\end{bAntwort}

\end{enumerate}
\end{document}
