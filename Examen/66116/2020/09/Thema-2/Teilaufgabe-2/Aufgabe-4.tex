\documentclass{bschlangaul-aufgabe}
\bLadePakete{normalformen}
\begin{document}
\bAufgabenMetadaten{
  Titel = {Aufgabe 4},
  Thematik = {Entwurfstheorie},
  Referenz = 66116-2020-H.T2-TA2-A4,
  RelativerPfad = Examen/66116/2020/09/Thema-2/Teilaufgabe-2/Aufgabe-4.tex,
  ZitatSchluessel = examen:66116:2020:09,
  BearbeitungsStand = mit Lösung,
  Korrektheit = unbekannt,
  Ueberprueft = {unbekannt},
  Stichwoerter = {Normalformen},
  EinzelpruefungsNr = 66116,
  Jahr = 2020,
  Monat = 09,
  ThemaNr = 2,
  TeilaufgabeNr = 2,
  AufgabeNr = 4,
}

\let\FA=\bFunktionaleAbhaengigkeiten

Gegeben ist das folgende Relationenschema R in erster Normalform.
\index{Normalformen}
\footcite{examen:66116:2020:09}

R:{[A,B, C, D, E, F]}

Für R gelte folgende Menge FD funktionaler Abhängigkeiten:

\FA{
  AC -> DE;
  ACE -> B;
  E -> B;
  D -> F;
  AC -> F;
  AD -> F;
}

\begin{enumerate}

%%
% a)
%%

\item R mit FD hat genau einen Kandidatenschlüssel X. Bestimmen Sie
diesen und begründen Sie Ihre Antwort.

\begin{bAntwort}
AC ist der Kandidatenschlüssel. AC kommt in keiner rechten Seite der
Funktionalen Abhängigkeiten vor.
\end{bAntwort}

%%
% b)
%%

\item Berechnen Sie Schritt für Schritt die Hülle $X^+$ von $X := \{ K
\}$.

\begin{bAntwort}
\begin{enumerate}
\item $AC \cup DE$
\item $ACDE \cup B$ (ACE -> B)
\item $ACDEB$ (E -> B)
\item $ACDEB \cup F$ (D -> F)
\item $ACDEBF$ (AC -> F)
\item $ACDEBF$ (AD -> F)
\end{enumerate}
\end{bAntwort}

%%
% c)
%%

\item Nennen Sie alle primen und nicht-primen Attribute.

\begin{bAntwort}
prim: AC

nicht-prim: BDEF
\end{bAntwort}

%%
% d)
%%

\item Geben Sie die höchste Normalform an, in der sich die Relation
befindet. Begründen Sie.

\begin{bAntwort}
2NF

D --> F hängt transitiv von AC ab: AC -> D, D-> F
\end{bAntwort}

%%
% e)
%%

\item Gegeben ist die folgende Zerlegung von R:

R1 (A, C, D, E)
R2 (B, E)
R3 (D, F)

Weisen Sie nach, dass es sich um eine verlustfreie Zerlegung handelt.

\end{enumerate}
\end{document}
