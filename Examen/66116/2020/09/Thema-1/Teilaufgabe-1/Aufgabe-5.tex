\documentclass{bschlangaul-aufgabe}

\begin{document}
\bAufgabenMetadaten{
  Titel = {Aufgabe 5},
  Thematik = {Lebenszyklus},
  Referenz = 66116-2020-H.T1-TA1-A5,
  RelativerPfad = Examen/66116/2020/09/Thema-1/Teilaufgabe-1/Aufgabe-5.tex,
  ZitatSchluessel = examen:66116:2020:09,
  BearbeitungsStand = mit Lösung,
  Korrektheit = unbekannt,
  Ueberprueft = {unbekannt},
  Stichwoerter = {Projektplanung, SCRUM},
  EinzelpruefungsNr = 66116,
  Jahr = 2020,
  Monat = 09,
  ThemaNr = 1,
  TeilaufgabeNr = 1,
  AufgabeNr = 5,
}

\begin{enumerate}

%%
% a)
%%

\item Nennen Sie fünf kritische Faktoren, die bei der Auswahl eines
Vorgehensmodells helfen können und ordnen Sie plangetriebene und agile
Prozesse entsprechend ein.\index{Projektplanung}
\footcite{examen:66116:2020:09}

\begin{bAntwort}
\begin{enumerate}
\item Vollständigkeit der Anforderungen
\begin{compactitem}
\item bei vollständiger Kenntnis der Anforderungen: \emph{plangetrieben}
\item bei teilweiser Kenntnis der Anforderungen: \emph{agil}
\end{compactitem}

\item Möglichkeit der Rücksprache mit dem Kunden
\begin{compactitem}
\item keine Möglichkeit: \emph{plangetrieben}
\item Kunde ist partiell immer wieder involviert: \emph{agil}
\end{compactitem}

\item Teamgröße
\begin{compactitem}
\item kleine Teams (max. 10 Personen): \emph{agil}
\item größere Teams: \emph{plangetrieben}
\end{compactitem}

\item Bisherige Arbeitsweise des Teams
\begin{compactitem}
\item bisher feste Vorgehensmodelle: \emph{plangetrieben}
\item flexible Arbeitsweisen: \emph{agil}
\end{compactitem}

\item Verfügbare Zeit
\begin{compactitem}
\item kurze Zeitvorgabe: \emph{plangetrieben}
\item möglichst schnell funktionierender Prototyp verlangt: \emph{agil}
\item beide Vorgehensmodelle sind allerdings zeitlich festgelegt
\end{compactitem}
\end{enumerate}

Mögliche weitere Faktoren: Projektkomplexität, Dokumentation
\end{bAntwort}

%%
% b)
%%

\item Nennen und beschreiben Sie kurz die Rollen im Scrum.
\index{SCRUM}

\begin{bAntwort}
\begin{description}
\item[Product Owner]

Der Product Owner ist für die Eigenschaften und den wirtschaftlichen
Erfolg des Produkts verantwortlich.

\item[Entwickler]
Die Entwickler sind für die Lieferung der Produktfunktionalitäten in der
vom Product Owner gewünschten Reihenfolge verantwortlich.

\item[Scrum Master]

Der Scrum Master ist dafür verantwortlich, dass Scrum als Rahmenwerk
gelingt. Dazu arbeitet er mit dem Entwicklungsteam zusammen, gehört aber
selbst nicht dazu.
\footcite{wiki:scrum}
\end{description}
\end{bAntwort}

%%
% c)
%%

\item Nennen und beschreiben Sie drei Scrum Artefakte. Nennen Sie die
verantwortliche Rolle für jedes Artefakt.

\begin{bAntwort}
\begin{description}
\item[Product Backlog]

Das Product Backlog ist eine geordnete Auflistung der Anforderungen an
das Produkt.

\item[Sprint Backlog]

Das Sprint Backlog ist der aktuelle Plan der für einen Sprint zu
erledigenden Aufgaben.

\item[Product Increment]

Das Inkrement ist die Summe aller Product-Backlog-Einträge, die während
des aktuellen und allen vorangegangenen Sprints fertiggestellt wurden.
\footcite{wiki:scrum}
\end{description}
\end{bAntwort}

%%
% d)
%%

\item Beschreiben Sie kurz, was ein Sprint ist. Wie lange sollte ein
Sprint maximal dauern?

\begin{bAntwort}
Ein Sprint ist ein Arbeitsabschnitt, in dem ein Inkrement einer
Produktfunktionalität implementiert wird. Ein Sprint umfasst ein
Zeitfenster von ein bis vier Wochen.
\end{bAntwort}

\end{enumerate}
\end{document}
