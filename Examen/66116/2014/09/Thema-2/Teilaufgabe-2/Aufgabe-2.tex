\documentclass{bschlangaul-aufgabe}

\begin{document}
\bAufgabenMetadaten{
  Titel = {Aufgabe 4},
  Thematik = {Wahlsystem},
  Referenz = 66116-2014-H.T2-TA2-A2,
  RelativerPfad = Staatsexamen/66116/2014/09/Thema-2/Teilaufgabe-2/Aufgabe-2.tex,
  ZitatSchluessel = examen:66116:2014:09,
  BearbeitungsStand = mit Lösung,
  Korrektheit = unbekannt,
  Ueberprueft = {unbekannt},
  Stichwoerter = {Modell-Präsentation-Steuerung (Model-View-Controller), Implementierung in Java},
  EinzelpruefungsNr = 66116,
  Jahr = 2014,
  Monat = 09,
  ThemaNr = 2,
  TeilaufgabeNr = 2,
  AufgabeNr = 2,
}

Sie sollen das Design für ein einfaches Wahlsystem
entwerfen.\footcite{examen:66116:2014:09} Das System soll dabei die
Verteilung der Stimmen auf die einzelnen Parteien ermöglichen.
Zusätzlich soll es verschiedene Darstellungen dieser Daten erlauben:
Eine Tabelle, in der die Daten gelesen und auch eingegeben werden
können, und ein Diagramm als alternative Darstellung der Informationen.
Das System soll mit dem \emph{Model-View-Controller} Muster modelliert
werden.
\footcite[StEx H14, T2, TA2, A2 (geänderte Aufgabenstellung), Aufgabe 4]{sosy:ab:6}

\begin{enumerate}

%%
% 1.
%%

\item Beschreiben Sie das \emph{Model-View-Controller} Muster:
\index{Modell-Präsentation-Steuerung (Model-View-Controller)}

\begin{enumerate}

%%
% (a)
%%

\item Beschreiben Sie das Problem, welches das Muster adressiert.

\begin{bAntwort}
Das MVC-Muster wird verwendet, um spätere Änderungen bzw. Erweiterungen
zu vereinfachen. Dies unterstützt somit auch die Wiederverwendbarkeit.
\end{bAntwort}

%%
% (b)
%%

\item Beschreiben Sie die Aufgaben der Komponenten, die im Muster
verwendet werden.

\begin{bAntwort}
Im Modell werden die Daten verwaltet, die View ist für die Darstellung
der daten sowie die Benutzerinteraktion zuständig und der Controller
übernimmt die Steue- rung zwischen View und Modell.

Das MVC-Muster ist aus den drei Entwurfsmustern Beobachter, Kompositum
und Strategie zusammengesetzt.
\end{bAntwort}

\end{enumerate}

%%
% 2.
%%

\item Modellieren Sie das System unter Anwendung des Musters:

\begin{enumerate}

%%
% a.
%%

\item Entwerfen Sie ein UML Klassendiagramm.

%%
% b.
%%

\item Implementieren Sie eine setup-Methode, die das Objektmodell
erstellt.\index{Implementierung in Java}
\end{enumerate}

\end{enumerate}
\end{document}
