\documentclass{bschlangaul-aufgabe}
\bLadePakete{mathe,java,kontrollflussgraph}
\begin{document}
\bAufgabenMetadaten{
  Titel = {Aufgabe 2: Kontrollflussorientiertes Testen},
  Thematik = {Methode „specialSums()“},
  Referenz = 66116-2014-H.T2-TA2-A3,
  RelativerPfad = Staatsexamen/66116/2014/09/Thema-2/Teilaufgabe-2/Aufgabe-3.tex,
  ZitatSchluessel = examen:66116:2014:09,
  BearbeitungsStand = mit Lösung,
  Korrektheit = unbekannt,
  Ueberprueft = {unbekannt},
  Stichwoerter = {Kontrollflussorientieres Testen, Kontrollflussgraph, C0-Test Anweisungsüberdeckung (Statement Coverage), C1-Test Zweigüberdeckung (Branch Coverage)},
  EinzelpruefungsNr = 66116,
  Jahr = 2014,
  Monat = 09,
  ThemaNr = 2,
  TeilaufgabeNr = 2,
  AufgabeNr = 3,
}

\let\c=\bKontrollCode
\let\b=\bBedingung

Im\index{Kontrollflussorientieres Testen}
\footcite{examen:66116:2014:09} Folgenden ist ein Algorithmus angegeben,
der für eine positive Zahl \bJavaCode{until} die Summe aller Zahlen
bildet, die kleiner als \bJavaCode{until} und Vielfache von $4$ oder
$6$ sind. Für nicht positive Zahlen soll $0$ zurückgegeben werden. Der
Algorithmus soll also folgender Spezifikation genügen:
\footcite[Kontrollflussorientiertes Testen, Aufgabe 2]{sosy:ab:7}

\bigskip

{\footnotesize
\noindent
$\texttt{until} > 0 \Rightarrow \texttt{specialSums(until)} =
\sum \{y \, | \, 0 < y < \texttt{until} \land (y\%4 = 0 \lor y\%6 = 0)\}$

\noindent
$\texttt{until} \leq 0 \Rightarrow \texttt{specialSums(until)} = 0$
}

\bigskip\noindent
wobei $\%$ den Modulo-Operator bezeichnet.

\bJavaExamen[firstline=4,lastline=14]{66116}{2014}{09}{SpecialSum}

\noindent
Beachten Sie, dass der Algorithmus nicht der Spezifikation genügt. Der
Fehler liegt in der Bedingung der for-Schleife. Der Fehler kann jedoch
einfach korrigiert werden indem die Bedingung

\begin{center}
$i \leq \texttt{until}$ in $i < \texttt{until}$
\end{center}

geändert würde.

\begin{enumerate}

%%
% (a)
%%

% public static long specialSums(int until) {
%   long sum = 0; // 0
%   if (until > 0) { // 1
%     for (int i = 1; i <= until; i++) { // 2 // 5
%       if (i % 4 == 0 || i % 6 == 0) { // 3
%         sum += i; // 4
%       }
%     }
%   }
%   return sum; // 6
% }

\item Zeichnen Sie das zum Programm gehörige Ablaufdiagramm.
\index{Kontrollflussgraph}

\begin{bAntwort}
\begin{bKontrollflussgraph}[xscale=1,yscale=-1.5]
\node[pin=\c{long sum = 0;}] at (0,0.5) (0) {0};
\node[pin=\c{if (until > 0)}] at (0,1) (1) {1};
\node[pin=\c{for (int i = 1; i <= until; i++)}] at (1,2) (2) {2};
\node[pin=\c{if (i \% 4 == 0 || i \% 6 == 0)}] at (2,3) (3) {3};
\node[pin=\c{sum += i;}] at (3,4) (4) {4};
\node[pin=\c{i++}] at (2,5) (5) {5};
\node[pin=\c{return sum;}] at (0,5.5) (6) {6};

\path (0) -- (1);
\path[wahr] (1) -- (2) \b{right}{until > 0};
\path[falsch] (1) -- (6) \b{left}{until <= 0};
\path[wahr] (2) -- (3) \b{right}{i <= until};
\path[falsch] (2) -- (6);
\path[wahr] (3) -- (4) \b{right}{i \% 4 == 0 || i \% 6 == 0};
\path[falsch] (3) -- (5);
\path (4) -- (5);
\path (5) -- (2);
\end{bKontrollflussgraph}
\end{bAntwort}

%%
% (b)
%%

\item Schreiben Sie einen Testfall, der das Kriterium „100\%
Anweisungsüberdeckung“ erfüllt, aber den Fehler trotzdem nicht aufdeckt.
\index{C0-Test Anweisungsüberdeckung (Statement Coverage)}

\begin{bAntwort}
Der Fehler fällt nur dann auf, wenn \texttt{until} durch $4$ oder $6$
ohne Rest teilbar ist. \texttt{until = 0\%4} oder \texttt{until = 0\%6}.
Wähle daher den Testfall $\{(1, 0)\}$. Alternativ kann für die Eingabe
auch 2, 3, 5, 7, 9, 10, 11, 13, ... gewählt werden.
\end{bAntwort}

%%
% (c)
%%

\item Schreiben Sie einen Testfall, der das Kriterium „100\%
Zweigüberdeckung“ erfüllt, aber den Fehler trotzdem nicht aufdeckt.
\index{C1-Test Zweigüberdeckung (Branch Coverage)}

\begin{bAntwort}
Betrachte den Testfall $\{(0, 0), (5, 4)\}$.

\begin{description}
\item[erste if-Bedingung]
Das erste Tupel mit $\text{until} = 0$ stellt sicher, dass
die erste \bJavaCode{if}-Bedingung \bJavaCode{false} wird.

\item[Bedingung der for-Schleife]
Für die zweite Eingabe $\text{until} = 5$ werden für $i$ die Werte $1, 2,
3, 4, 5, 6$ angenommen. Wobei für $i = 6$ die Bedingung der for-Schleife
\bJavaCode{false} ist.

\item[Innere if-Bedingung]
Für $i = 1, 2, 3, 5$ wird die innere if-Bedingung jeweils \bJavaCode{false},
für $i = 4$ wird sie \bJavaCode{true}.
\end{description}

\end{bAntwort}

%%
% (d)
%%

\item Schreiben Sie einen Testfall, der den Fehler aufdeckt. Berechnen
Sie Anweisungsüberdeckung und Zweigüberdeckung ihres Testfalls.

\begin{bAntwort}
\begin{description}
\item[Anweisungsüberdeckung]
Wähle $\{(4, 0)\}$. Durch die fehlerhafte Bedingung in der for-Schleife
wird der Wert $i = 4$ akzeptiert. Da alle Anweisungen ausgeführt werden,
wird eine Anweisungsüberdeckung mit 100\% erreicht.

\item[Verzweigungsüberdeckung]
Da die erste Verzweigung nur zur Hälfte überdeckt wird und die anderen
beiden vollständig, gilt für die Verzweigungsüberdeckung:

$\frac{1+2+2}{2+2+2} = \frac{5}{6}$
\end{description}
\end{bAntwort}

%%
% (e)
%%

\item Es ist nicht immer möglich vollständige Pfadüberdeckung zu
erreichen. Geben Sie einen gültigen Pfad des Programmes an, der nicht
erreichbar ist. Ein Testfall kann als Menge von Paaren dargestellt
werden, wobei jedes Paar $(I, O)$ die Eingabe $I$ und die zu dieser
erwartete Ausgabe $O$ darstellt.

\begin{bAntwort}
Ein gültiger Pfad im Kontrollflussgraphen wäre
\bKontrollTextzeileKnoten{0} - \bKontrollTextzeileKnoten{1} -
\bKontrollTextzeileKnoten{2} - \bKontrollTextzeileKnoten{3} -
\bKontrollTextzeileKnoten{4} - \bKontrollTextzeileKnoten{5} -
\bKontrollTextzeileKnoten{2} - \bKontrollTextzeileKnoten{6}. Der
Übergang von \bKontrollTextzeileKnoten{3} auf
\bKontrollTextzeileKnoten{4} ist hier aber nicht möglich, da beim
ersten Durchlaufen der for-Schleife (\bKontrollTextzeileKnoten{2}) $i$
immer $1$ ist und $1$ weder durch $4$ noch durch $6$ teilbar ist. Somit
kann die Bedingung des inneren ifs (\bKontrollTextzeileKnoten{3}) beim
ersten Durchlauf nie \emph{wahr} sein, womit immer der Übergang
\bKontrollTextzeileKnoten{3} - \bKontrollTextzeileKnoten{5} zu Beginn
genommen werden muss. Alle Pfade, die zu Beginn
\bKontrollTextzeileKnoten{3} - \bKontrollTextzeileKnoten{4} enthalten,
sind somit nicht überdeckbar.
\end{bAntwort}

\end{enumerate}
\end{document}
