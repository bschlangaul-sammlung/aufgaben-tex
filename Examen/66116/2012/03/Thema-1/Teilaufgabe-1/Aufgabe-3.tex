\documentclass{bschlangaul-aufgabe}
\bLadePakete{rmodell,syntax}
\begin{document}
\bAufgabenMetadaten{
  Titel = {SQL},
  Thematik = {Gebrauchtwagen},
  Referenz = 66116-2012-F.T1-TA1-A3,
  RelativerPfad = Staatsexamen/66116/2012/03/Thema-1/Teilaufgabe-1/Aufgabe-3.tex,
  ZitatSchluessel = examen:66116:2012:03,
  BearbeitungsStand = mit Lösung,
  Korrektheit = unbekannt,
  Ueberprueft = {unbekannt},
  Stichwoerter = {SQL, SQL mit Übungsdatenbank},
  EinzelpruefungsNr = 66116,
  Jahr = 2012,
  Monat = 03,
  ThemaNr = 1,
  TeilaufgabeNr = 1,
  AufgabeNr = 3,
}

\let\a=\bAttribut
\let\f=\bFremd
\let\p=\bPrimaer
\let\r=\bRelationMenge

\begin{liRelationenSchemaFormat}
Fahrzeug (
  MNR[Modell] {INTEGER},
  FZGNR* {char(12)},
  Baujahr {INTEGER},
  KMStand {INTEGER},
  Preis {INTEGER}
)

Modell (
  MNR* {INTEGER},
  HNR[Hersteller] {INTEGER},
  Typ {char(20)},
  Neupreis {INTEGER},
  ps {INTEGER}
)

Hersteller (
  HNR* {INTEGER},
  Name {char(20)}
)
\end{liRelationenSchemaFormat}

\noindent
Gegeben sei das folgende Relationenschema:\index{SQL}
\footcite{examen:66116:2012:03}

\begin{bRelationenModell}
\r{Fahrzeug}{\f{MNR[Modell]}, \p{FZGNR}, Baujahr, KMStand, Preis}
\r{Modell}{\p{MNR}, \f{HNR[Hersteller]}, Typ, Neupreis, ps}
\r{Hersteller}{\p{HNR}, Name}
\end{bRelationenModell}

\begin{bAdditum}[Übungsdatenbank]
% Datenbankname: gebrauchtwagen
\begin{minted}{sql}
CREATE TABLE Hersteller (
  HNR INTEGER PRIMARY KEY,
  Name CHAR(20)
);

CREATE TABLE Modell (
  MNR INTEGER PRIMARY KEY,
  HNR INTEGER REFERENCES Hersteller(HNR),
  Typ CHAR(20),
  Neupreis INTEGER,
  ps INTEGER
);

CREATE TABLE Fahrzeug (
  MNR INTEGER REFERENCES Modell(MNR),
  FZGNR CHAR(12) PRIMARY KEY,
  Baujahr INTEGER,
  KMStand INTEGER,
  Preis INTEGER
);

INSERT INTO Hersteller VALUES
  (1, 'BMW'),
  (2, 'VW'),
  (3, 'Daimler');

INSERT INTO Modell VALUES
  (1, 1, '1er', 30134, 122),
  (2, 1, '2er', 42134, 180),
  (3, 2, 'Gold', 19278, 90);

INSERT INTO Fahrzeug VALUES
  (1, 1, 2010, 60134, 11154),
  (2, 2, 2017, 82134, 19130),
  (3, 3, 2002, 119278, 9278);
\end{minted}
\index{SQL mit Übungsdatenbank}
\end{bAdditum}

Dabei sind die Schlüsselattribute jeweils unterstrichen und zusätzlich
für alle Attribute die Typen angegeben. Formulieren Sie die folgenden
Anfragen bzw. Anweisungen in SQL.

\begin{enumerate}

%%
% a)
%%

\item Geben Sie die Anweisungen in SQL-DDL an, die notwendig sind, um
die Relationen „Fahrzeug“, „Modell“ und „Hersteller“ zu erzeugen. Achten
Sie dabei darauf, die Primärschlüssel der Relationen zu kennzeichnen.

\begin{bAntwort}
\begin{minted}{sql}
CREATE TABLE IF NOT EXISTS Hersteller (
  HNR INTEGER PRIMARY KEY,
  Name CHAR(20)
);

CREATE TABLE IF NOT EXISTS Modell (
  MNR INTEGER PRIMARY KEY,
  HNR INTEGER REFERENCES Hersteller(HNR),
  Typ CHAR(20),
  Neupreis INTEGER,
  ps INTEGER
);

CREATE TABLE IF NOT EXISTS Fahrzeug (
  MNR INTEGER REFERENCES Modell(MNR),
  FZGNR CHAR(12) PRIMARY KEY,
  Baujahr INTEGER,
  KMStand INTEGER,
  Preis INTEGER
);
\end{minted}
\end{bAntwort}

%%
% b)
%%

\item Bestimmen Sie die Typen aller Modelle des Herstellers mit Namen
BMW.

\begin{bAntwort}
\begin{minted}{sql}
SELECT m.Typ
FROM Modell m, Hersteller h
WHERE h.HNR = m.HNR AND h.Name = 'BMW'
GROUP BY m.Typ;
\end{minted}
\end{bAntwort}

%%
% c)
%%

\item Bestimmen Sie den Mindestpreis, bezogen auf das Attribut „Preis“,
der Fahrzeuge eines jeden Herstellers.

\begin{bAntwort}
\begin{minted}{sql}
SELECT h1.Name AS Hersteller, (
  SELECT MIN(f.Preis)
  FROM Fahrzeug f, Modell m, Hersteller h2
  WHERE
    f.MNR = m.MNR AND
    m.HNR = h2.HNR AND
    H2.HNR = h1.HNR
) AS Mindestpreis
FROM Hersteller h1;
\end{minted}
\end{bAntwort}

%%
% d)
%%

\item Bestimmen Sie die Namen der Hersteller, für die von jedem ihrer
Modelle mindestens ein Fahrzeug in der Datenbank gespeichert ist.

\begin{bAntwort}
\begin{minted}{sql}
SELECT h.Name AS Hersteller
FROM Fahrzeug f, Modell m, Hersteller h
WHERE
  f.MNR = m.MNR AND
  m.HNR = h.HNR
GROUP BY h.Name;
\end{minted}
\end{bAntwort}

%%
% e)
%%

\item Bestimmen Sie die Namen aller Hersteller, von denen mindestens
fünf Fahrzeuge eines beliebigen Modells in der Datenbank gespeichert
sind.
\end{enumerate}

\end{document}
