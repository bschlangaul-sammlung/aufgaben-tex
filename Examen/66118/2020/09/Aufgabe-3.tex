\documentclass{bschlangaul-aufgabe}

\begin{document}
\bAufgabenMetadaten{
  Titel = {Thema Nr. 3},
  Thematik = {Datenschutz, Datensicherheit, Views, Lernziele, Gruppenarbeit},
  Referenz = 66118-2020-H.A3,
  RelativerPfad = Examen/66118/2020/09/Aufgabe-3.tex,
  ZitatSchluessel = examen:66118:2020:03,
  BearbeitungsStand = mit Lösung,
  Korrektheit = unbekannt,
  Ueberprueft = {unbekannt},
  Stichwoerter = {DDI},
  EinzelpruefungsNr = 66118,
  Jahr = 2020,
  Monat = 09,
  AufgabeNr = 3,
}
\index{DDI}
\footcite{examen:66118:2020:03}

Die Themen Datenschutz und Datensicherheit erhalten im LehrplanPLUS
zunehmend Bedeutung. So beschäftigen sich die Schülerinnen und Schüler
unter anderem im Kontext relationaler Datenbanksysteme in der 9. und
10. Jahrgangsstufe mit diesem Thema.

\begin{enumerate}
\item Definieren Sie die Begriffe \emph{personenbezogene Daten},
\emph{personenbeziehbare Daten}, \emph{Datenschutz} und
\emph{Datensicherheit} in einer
altersgerechten Form!

\item Überlegen Sie sich für ein Unterrichtsbeispiel zum Thema
Datenschutz eine angemessene, nicht-triviale Datenbank mit Bezug zur
Lebenswirklichkeit der Schülerinnen und Schüler und zeichnen Sie ein
Klassendiagramm davon! Erläutern Sie, welche Daten in Ihrem Beispiel
\emph{personenbezogen} bzw. \emph{personenbeziehbar} sind!

\item Um Datenschutz bei der Mehrbenutzerproblematik relationaler
Datenbanksysteme zu gewährleisten, werden unter anderem zwei
Mechanismen verwendet: Benutzerrechte und Sichten (Views). Beschreiben
Sie jeweils ein Beispiel, wie mit diesen beiden Mechanismen der
Datenschutz Ihrer Datenbank verbessert werden kann!
\end{enumerate}

\noindent
Nachfolgend soll eine Doppelstunde zum Thema „Einführung von Views in
Datenbanksystemen“ entwickelt werden. Ziel ist es, diese Doppelstunde
möglichst schülerorientiert zu gestalten und mindestens eine Phase mit
strukturierter Gruppenarbeit zu integrieren. Gehen Sie dabei von einer
Klasse der neunten Jahrgangsstufe aus, in der deutliche Unterschiede
bzgl. der Leistungsfähigkeit der Schülerinnen und Schüler bestehen und
ein Schüler von vielen anderen Schülern gemobbt wird.

\begin{enumerate}

%%
% d)
%%

\item Formulieren Sie drei beobachtbare Feinziele zu der genannten
Doppelstunde! Ordnen Sie die Feinziele mit kurzer Begründung in eine
Lernzieltaxonomie (z. B. Bloom, Anderson und Krathwohl) ein!

%%
% e)
%%

\item Beschreiben Sie eine zu diesen Lernzielen passende Doppelstunde
(textuelle Form, ca. zwei Seiten)! Gliedern Sie den Text deutlich
erkennbar in Unterrichtsphasen! Gehen Sie kurz darauf ein, wie Sie die
Gruppenarbeit in diesem Kontext sinnvoll strukturieren können!

%%
% f)
%%

\item Vergleichen Sie die Art der Gruppenbildung durch Zufall (Los),
durch Selbstorganisation, anhand der Sitzordnung bzw. durch Vorgabe der
Lehrperson! Wählen Sie dazu sinnvolle Kriterien für den Vergleich aus
und untersuchen Sie, welche Organisationsformen diese Kriterien jeweils
gut bzw. schlecht erfüllen (Stichpunkte reichen)! Wählen Sie danach
begründet eine Form der Einteilung für Ihre Gruppenarbeit aus! Denken
Sie dabei an die Beschreibung der Klasse!
\end{enumerate}
\end{document}
