\documentclass{bschlangaul-aufgabe}
\bLadePakete{syntax}
\begin{document}
\bAufgabenMetadaten{
  Titel = {Aufgabe 3: SQL},
  Thematik = {Universitätsverwaltung},
  Referenz = 66113-2003-F.T2-A3,
  RelativerPfad = Staatsexamen/66113/2003/03/Thema-2/Aufgabe-3.tex,
  ZitatSchluessel = db:pu:2,
  ZitatBeschreibung = {Aufgabe 3: SQL},
  BearbeitungsStand = mit Lösung,
  Korrektheit = unbekannt,
  Ueberprueft = {unbekannt},
  Stichwoerter = {SQL},
  EinzelpruefungsNr = 66113,
  Jahr = 2003,
  Monat = 03,
  ThemaNr = 2,
  AufgabeNr = 3,
}

Gegeben sei folgendes relationales Schema, das eine
Universitätsverwaltung modelliert:
\footcite[Aufgabe 3: SQL]{db:pu:2}

\begin{minted}{md}
Studenten {[MatrNr:integer, Name:string, Semester:integer]}
Vorlesungen {[VorlNr:integer, Titel:string, SWS:integer, gelesenVon:integer]}
Professoren {[PersNr:integer, Name:string, Rang:string, Raum:integer] }
hoeren {[MatrNr:integer, VorlNr:integer]}
voraussetzen {[VorgaengerVorlNr:integer, NachfolgerVorlNr:integer]}
pruefen {[MatrNr:integer, VorlNr:integer, PrueferPersNr:integer, Note:decimal]}
\end{minted}

\noindent
Formulieren Sie die folgenden Anfragen in SQL:\index{SQL}
\footcite[Thema 2 Aufgabe 3]{66113:2003:03}

\begin{enumerate}

%%
% (a)
%%

\item Alle Studenten, die den Professor \emph{Kant} aus einer Vorlesung
kennen.

\begin{bAntwort}
\begin{minted}{sql}
SELECT DISTINCT s.Name
FROM Studenten s, Professoren p, hoeren h, Vorlesungen v
WHERE
  p.Name = 'Kant' AND
  v.gelesenVon = p.PersNr AND
  s.MatrNr = h.MatrNr AND
  h.VorlNr = h.VorlNr
\end{minted}
\end{bAntwort}

%%
% (b)
%%

\item Geben Sie eine Liste der Professoren (Name, PersNr) mit ihrem
Lehrdeputat (Summe der SWS der gelesenen Vorlesungen) aus. Ordnen Sie
diese Liste so, dass sie absteigend nach Lehrdeputat sortiert ist! Bei
gleicher Lehrtätigkeit dann noch aufsteigend nach dem Namen des
Professors/der Professorin.

\begin{bAntwort}
\begin{minted}{sql}
SELECT p.Name, p.PersNr, SUM(v.SWS) AS Lehrdeputat
FROM Vorlesung v, Professoren p
WHERE v.gelesenVon = p.PersNr
GROUP BY p.Name, p.PersNr
ORDER BY Lehrdeputat DESC, p.Name ASC;
\end{minted}
\end{bAntwort}

%%
% (c)
%%

\item Geben Sie eine Liste der Studenten (Name, MatrNr, Semester) aus,
die mindestens zwei Vorlesungen bei \emph{Kant} gehört haben.

\begin{bAntwort}
\subsection*{Mit einer VIEW}

\begin{minted}{sql}
CREATE VIEW hoertKant AS
  SELECT s.Name, s.MatrNr, s.Semester, v.VorlNr
  FROM Studenten s, hoeren h, Vorlesungen v, Professoren p
  WHERE
    s.MatrNr = h.MatrNr AND
    h.VorlNr = v.VorlNr AND
    v.gelesenVon = p.PersNr AND
    p.Name = 'Kant';
\end{minted}

\begin{minted}{sql}
SELECT DISTINCT h1.Name, h2.MatrNr, h1.Semester
FROM hoertKant h1, hoertKant h2
WHERE h1.MatrNr = h2.MatrNr AND h1.VorlNr <> h2.VorlNr;
\end{minted}

oder:

\begin{minted}{sql}
SELECT DISTINCT Name, MatrNr, Semester
FROM hoertKant
GROUP BY Name, MatrNr, Semester
HAVING COUNT(VorlNr) > 1;
\end{minted}

\subsection*{In einer Abfrage}

\begin{minted}{sql}
SELECT s.Name, s.MatrNr, s.Semester
FROM Studenten s, hoeren h, Vorlesungen v, Professoren p
WHERE
  s.MatrNr = h.MatrNr AND
  h.VorlNr = v.VorlNr AND
  v.gelesenVon = p.PersNr AND
  p.Name = 'Kant'
GROUP BY s.MatrNr, s.Name, s.Semster
HAVING COUNT(s.MatrNr) > 1;
\end{minted}
\end{bAntwort}

%%
% (d)
%%

\item Geben Sie eine Liste der Semesterbesten (MatrNr und
Notendurchschnitt) aus.

\begin{bAntwort}
\begin{minted}{sql}
CREATE VIEW Notenschnitte AS (
  SELECT p.MatrNr, s.Name, s.Semester, AVG(Note) AS Durchschnitt
  FROM Studenten s, pruefen p
  WHERE s.MatrNr = p.MatrNr
  GROUP BY p.MatrNr, s.Name, s.Semester
);

SELECT a.Durchschnitt, a.MatrNr, a.Semester
FROM Notenschnitte a, Notenschnitte b
WHERE
  a.Durchschnitt >= b.Durchschnitt
  a.Semster = b.Semster
GROUP BY a.Durchschnitt, a.MatrNr, a.Semester
HAVING COUNT(*) < 2;
\end{minted}
\end{bAntwort}

\end{enumerate}
\end{document}
