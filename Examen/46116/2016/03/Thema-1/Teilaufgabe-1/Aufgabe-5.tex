\documentclass{bschlangaul-aufgabe}
\usepackage{tikz}
\begin{document}
\bAufgabenMetadaten{
  Titel = {Aufgabe 5},
  Thematik = {Transaktionen},
  Referenz = 46116-2016-F.T1-TA1-A5,
  RelativerPfad = Staatsexamen/46116/2016/03/Thema-1/Teilaufgabe-1/Aufgabe-5.tex,
  ZitatSchluessel = db:pu:5,
  ZitatBeschreibung = {Aufgabe 1: Transaktionen,
Schedules},
  BearbeitungsStand = mit Lösung,
  Korrektheit = unbekannt,
  Ueberprueft = {unbekannt},
  Stichwoerter = {Transaktionen, ACID, Serialisierbarkeitsgraph, Zwei-Phasen-Sperrprotokoll},
  EinzelpruefungsNr = 46116,
  Jahr = 2016,
  Monat = 03,
  ThemaNr = 1,
  TeilaufgabeNr = 1,
  AufgabeNr = 5,
}

\begin{enumerate}

%%
% a)
%%

\item Nennen\index{Transaktionen} \footcite[Aufgabe 1: Transaktionen,
Schedules]{db:pu:5} Sie die vier wesentlichen Eigenschaften\index{ACID}
einer Transaktion und erläutern Sie jede Eigenschaft kurz (ein Satz pro
Eigenschaft).
\footcite[Thema 1 Teilaufgabe 1 Aufgabe 5]{examen:46116:2016:03}

\begin{bAntwort}
\begin{description}
\item[Atomicity]

Eine Transaktion ist atomar, \dh von den vorgesehenen
Änderungsoperationen auf die Datenbank haben entweder alle oder keine
eine Wirkung auf die Datenbank.

\item[Consistency]

Eine Transaktion überführt einen korrekten (konsistenten)
Datenbankzustand wieder in einen korrekten (konsistenten)
Datenbankzustand.

\item[Isolation]

Eine Transaktion bemerkt das Vorhandensein anderer (parallel
ablaufender) Transaktionen nicht und beeinflusst auch andere
Transaktionen nicht.

\item[Durability]

Die durch eine erfolgreiche Transaktion vorgenommenen Änderungen sind
dauerhaft (persistent).\footcite[Kapitel 9.5 „Eigenschaften von Transaktionen“, Seite 305]{kemper}
\end{description}
\end{bAntwort}

%%
% b)
%%

\item Gegeben ist die folgende Historie (Schedule) dreier Transaktionen:

\bigskip

$
r_1 (B) \rightarrow
w_1 (C) \rightarrow
r_3 (C) \rightarrow
r_1 (A) \rightarrow
c_1 \rightarrow
r_2 (C) \rightarrow
r_3 (C) \rightarrow
r_2 (C) \rightarrow
w_2 (B) \rightarrow
c_2 \rightarrow
c_3
$

\bigskip

Zeichnen Sie den
Serialisierbarkeitsgraphen\index{Serialisierbarkeitsgraph} zu dieser
Historie und begründen Sie, warum die Historie serialisierbar ist oder
nicht.

\begin{bExkurs}[Historie]
In Transaktionssystemen existiert ein Ausführungsplan für die parallele
Ausführung mehrerer Transaktionen. Der Plan wird auch Historie genannt
und gibt an, in welcher Reihenfolge die einzelnen Operationen der
Transaktion ausgeführt werden. Als serialisierbar bezeichnet man eine
Historie, wenn sie zum selben Ergebnis führt wie eine nacheinander
(seriell) ausgeführte Historie über dieselben Transaktionen.
\end{bExkurs}

\begin{bAntwort}
Der Algorithmus geht schrittweise durch den Ablaufplan unten und hebt
die Abhängigkeiten der aktiven Transaktion zu allen anderen
Transaktionen hervor. Hierfür werden in allen nachfolgenden Schritten
solche Operationen gesucht, die einen Konflikt mit der aktuellen
Operation hervorrufen. Konflikt-Operationen sind: \,  \texttt{read after
write}, \, \texttt{write after read} \,  und \,  \texttt{write after
write} \, auf denselben Datenobjekt.

\begin{center}
\begin{tabular}{|c|c|c|}
\hline
$T_1$     & $T_2$     & $T_3$     \\\hline\hline
$r_1 (B)$ &           &           \\\hline
$w_1 (C)$ &           &           \\\hline
          &           & $r_3 (C)$ \\\hline
$r_1 (A)$ &           &           \\\hline
$c_1$     &           &           \\\hline
          & $r_2 (C)$ &           \\\hline
          &           & $r_3 (C)$ \\\hline
          & $r_2 (C)$ &           \\\hline
          & $w_2 (B)$ &           \\\hline
          & $c_2$     &           \\\hline
          &           & $c_3$     \\\hline
\end{tabular}
\end{center}

\bPseudoUeberschrift{Konfliktoperation}

\begin{itemize}
\item $r_1 (B) < w_2 (B)$: Kante von $T_1$ nach $T_2$
\item $w_1 (C) < r_3 (C)$: Kante von $T_1$ nach $T_3$
\item $w_1 (C) < r_2 (C)$: Kante von $T_1$ nach $T_2$
\end{itemize}

\bPseudoUeberschrift{Serialisierbarkeitsgraph}

\begin{center}
\begin{tikzpicture}
\node (t1) at (0,0) {$T_1$};
\node (t2) at (-1,-1) {$T_2$};
\node (t3) at (1,-1) {$T_3$};

\draw[->] (t1) -- (t2);
\draw[->] (t1) -- (t3);
\end{tikzpicture}
\end{center}

Es gibt keinen Zyklus im Graph. Er ist deshalb serialisierbar. Wenn ein
Zyklus auftreten würde, dann wäre er nicht serialisierbar.
\end{bAntwort}

%%
% c)
%%

\item Geben Sie an, wodurch die erste und die zweite Phase des
Zwei-Phasen-Sperrprotokolls\index{Zwei-Phasen-Sperrprotokoll} jeweils
charakterisiert sind (ein Satz pro Phase).

\begin{bAntwort}
Die zwei Phasen des Protokolls bestehen aus einer \emph{Sperrphase}, in der
alle benötigten Objekte für die Transaktion gesperrt werden. In der
zweiten Phase werden die \emph{Sperren wieder freigegeben}, sodass die
Objekte von anderen Transaktionen genutzt werden können.
\end{bAntwort}

\end{enumerate}
\end{document}
