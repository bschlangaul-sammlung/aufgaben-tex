\documentclass{bschlangaul-aufgabe}
\bLadePakete{petri}
\begin{document}

\bAufgabenMetadaten{
  Titel = {Aufgabe 2},
  Thematik = {Petri-Netz},
  Referenz = 46116-2016-F.T2-TA1-A2,
  RelativerPfad = Staatsexamen/46116/2016/03/Thema-2/Teilaufgabe-1/Aufgabe-2.tex,
  ZitatSchluessel = examen:46116:2016:03,
  ZitatBeschreibung = {Seite 9},
  BearbeitungsStand = mit Lösung,
  Korrektheit = unbekannt,
  Ueberprueft = {unbekannt},
  Stichwoerter = {Petri-Netz, Erreichbarkeitsgraph},
  EinzelpruefungsNr = 46116,
  Jahr = 2016,
  Monat = 03,
  ThemaNr = 2,
  TeilaufgabeNr = 1,
  AufgabeNr = 2,
}

\let\t=\bPetriTransitionsName
\let\tp=\bPetriTransPfeile
\let\k=\bPetriErreichKnotenDrei

\index{Petri-Netz}

Gegeben sei das folgende Petri-Netz:\footcite[Seite 9]{examen:46116:2016:03}

\begin{center}
\begin{tikzpicture}[li petri,x=2cm,y=2cm]
  \node[place,label=A,label=south east:1,tokens=1] at (1,2) (A) {};
  \node[place,label=south:B] at (0,0) (B) {};
  \node[place,label=south:C,tokens=1] at (2,0) (C) {};

  \node[transition] at (1,0) {\t1}
    edge[pre] node[auto]{2} (B)
    edge[post] (C);

  \node[transition] at (0,2) {\t2}
    edge[pre] (A)
    edge[post] node[auto]{2} (B);

  \node[transition] at (2,2) {\t3}
    edge[pre] (C)
    edge[post] (A)
    edge[inhibitor,bend right=50] (A);

  \node[transition] at (1,1) {\t4}
    edge[pre] (A)
    edge[pre] node[auto]{2} (B)
    edge[post] (C);
\end{tikzpicture}
\end{center}

\begin{enumerate}

%%
% a)
%%

\item Erstellen Sie den zum Petri-Netz gehörenden
Erreichbarkeitsgraphen\index{Erreichbarkeitsgraph}. Die Belegungen sind
jeweils in der Form [A, B, C] anzugeben. Beschriften Sie auch jede Kante
mit der zugehörigen Transition. Beachten Sie die auf 1 beschränkte
Kapazität von Stelle A oder alternativ die Inhibitor-Kante von A zu
\t3 (beides ist hier semantisch äquivalent).

\begin{bAntwort}
\begin{center}
\begin{tabular}{lll}
\k001 & \tp3 & \k100 \\
\k002 & \tp3 & \k101 \\
\k020 & \tp1 & \k001 \\
\k021 & \tp1 & \k002 \\
\k021 & \tp3 & \k120 \\
\k100 & \tp2 & \k020 \\
\k101 & \tp2 & \k021 \\
\k120 & \tp1 & \k101 \\
\k120 & \tp2 & \k040 \\
\k120 & \tp4 & \k001 \\
\k040 & \tp2 & \k040 \\
\end{tabular}
\end{center}

\def\K #1#2#3 (#4,#5){\node at (#4,#5) (#1#2#3) {\k#1#2#3};}
\def\T#1 > #2 > #3{\path[draw,->] (#1) -- node[auto]{\t#2} (#3) ;}
\def\L#1 > #2{\path[draw,->] (#1) edge[loop] node[auto]{\t#2} (#1) ;}

\begin{tikzpicture}[li petri]
\K 001 (0,0)
\K 002 (3,7)
\K 020 (4,-3)
\K 021 (0,4)
\K 100 (5,0)
\K 101 (-3,7)
\K 120 (0,2)
\K 040 (5,2)

\T {001} > 3 > {100}
\T {002} > 3 > {101}
\T {020} > 1 > {001}
\T {021} > 1 > {002}
\T {021} > 3 > {120}
\T {100} > 2 > {020}
\T {101} > 2 > {021}
\T {120} > 1 > {101}
\T {120} > 2 > {040}
\T {120} > 4 > {001}
\L {040} > 2
\end{tikzpicture}
\end{bAntwort}

%%
% b)
%%

\item Wie kann man mit Hilfe des Erreichbarkeitsgraphen feststellen, ob
ein Petri-Netz lebendig ist?

%%
% c)
%%

\item Aufgrund von Transition \t4 ist das gegebene Petri-Netz nicht
stark lebendig. Wie müssten die Pfeilgewichte der Transition \t4
verändert werden, damit das Petri-Netz mit der gegebenen Startmarkierung
beschränkt bleibt und lebendig wird?

\begin{bAntwort}
\t4 nach C mit Gewicht 2 versehen
\end{bAntwort}

\end{enumerate}

\end{document}
