\documentclass{bschlangaul-aufgabe}

\begin{document}
\bAufgabenMetadaten{
  Titel = {Aufgabe 3: Entwurfsmuster},
  Thematik = {Entwurfsmuster bei Bankkonten, Hockeyspiel, Dateisystem},
  Referenz = 46116-2017-H.T2-TA1-A3,
  RelativerPfad = Staatsexamen/46116/2017/09/Thema-2/Teilaufgabe-1/Aufgabe-3.tex,
  ZitatSchluessel = sosy:pu:4,
  BearbeitungsStand = mit Lösung,
  Korrektheit = unbekannt,
  Ueberprueft = {unbekannt},
  Stichwoerter = {Entwurfsmuster, Klassendiagramm, Abstrakte Fabrik (Abstract Factory), Beobachter (Observer), Kompositum (Composite)},
  EinzelpruefungsNr = 46116,
  Jahr = 2017,
  Monat = 09,
  ThemaNr = 2,
  TeilaufgabeNr = 1,
  AufgabeNr = 3,
}

Verwenden\index{Entwurfsmuster}
\index{Klassendiagramm}
\footcite{sosy:pu:4} Sie geeignete \textbf{Entwurfsmuster}, um die
folgenden Sachverhalte mit Hilfe von \textbf{UML-Klassendiagrammen} zu
beschreiben. Nennen Sie das zu verwendende Entwurfsmuster namentlich,
wenden Sie es zur Lösung derjeweiligen \emph{Fragestellung} an und
erstellen Sie damit das problemspezifische UML-Klassendiagramm.
Beschränken Sie sich dabei auf die statische Sicht, \dh definieren Sie
keinerlei Verhalten mit Ausnahme der Definition geeigneter Operationen.
\footcite[Seite 11]{examen:46116:2017:09}

\begin{enumerate}
\item Es gibt unterschiedliche Arten von Bankkonten: Girokonto,
Bausparkonto vmd Kreditkarte. Bei allen Konten ist der Name des Inhabers
hinterlegt. Girokonten haben eine IBAN. Kreditkarten sind immer mit
einem Girokonto verknüpft. Bei Bausparkonten werden ein Sparzins sowie
ein Darlehenszins festgelegt. Es gibt eine \emph{zentrale} Klasse, die
die \emph{Erzeugung} unterschiedlicher Typen von Bankkonten steuert.
\index{Abstrakte Fabrik (Abstract Factory)}

\item Beim Ticker für ein Hockeyspiel können sich verschiedene Geräte
registrieren und wieder abmelden, um auf \emph{Veränderungen} des
Spielstands \emph{zu reagieren}. Hierzu werden im Ticker die Tore der
Heim- vmd Gastmannschaft sowie die aktuelle Spielminute vermerkt. Als
konkrete Geräte sind eine Smartphone-App sowie eine Stadionuhr bekannt.
\index{Beobachter (Observer)}

\item Dateisysteme sind \emph{baumartig} strukturiert. Verzeichnisse
können wiederum selbst Verzeichnisse und/oder Dateien beinhalten. Sowohl
Dateien als auch Verzeichnisse haben einen Namen.Das jeweilige
Eltemverzeichnis ist eindeutig. Bei Dateien wird die Art(Binär, Text
oder andere) sowie die Größe in Byte, bei Verzeichnissen die Anzahl
enthaltener Dateien hinterlegt.
\index{Kompositum (Composite)}

\end{enumerate}
\end{document}
