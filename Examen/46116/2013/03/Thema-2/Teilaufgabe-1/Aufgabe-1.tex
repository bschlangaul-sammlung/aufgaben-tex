\documentclass{bschlangaul-aufgabe}
\bLadePakete{java}
\begin{document}
\bAufgabenMetadaten{
  Titel = {Aufgabe 1},
  Thematik = {CreditCard, Order},
  Referenz = 46116-2013-F.T2-TA1-A1,
  RelativerPfad = Examen/46116/2013/03/Thema-2/Teilaufgabe-1/Aufgabe-1.tex,
  ZitatSchluessel = examen:46116:2013:03,
  BearbeitungsStand = mit Lösung,
  Korrektheit = unbekannt,
  Ueberprueft = {unbekannt},
  Stichwoerter = {Klassendiagramm},
  EinzelpruefungsNr = 46116,
  Jahr = 2013,
  Monat = 03,
  ThemaNr = 2,
  TeilaufgabeNr = 1,
  AufgabeNr = 1,
}

Gegeben sei folgendes Klassendiagramm:
\index{Klassendiagramm}
\footcite{examen:46116:2013:03}

\begin{enumerate}

%%
% a)
%%

\item Implementieren Sie die Klassen „Order“, „Payment“, „Check“,
„OrderDetail“ und „Item“ in einer geeigneten objektorientierten Sprache
Ihrer Wahl. Beachten Sie dabei insbesondere Sichtbarkeiten, Klassen- vs.
Instanzzugehörigkeiten und Schnittstellen bzw. abstrakte
Klassen.

\begin{bAntwort}
\bJavaExamen{46116}{2013}{03}{t2_ta1_a1/Check.java}
\bJavaExamen{46116}{2013}{03}{t2_ta1_a1/Item.java}
\bJavaExamen{46116}{2013}{03}{t2_ta1_a1/Order.java}
\bJavaExamen{46116}{2013}{03}{t2_ta1_a1/OrderDetail.java}
\bJavaExamen{46116}{2013}{03}{t2_ta1_a1/Payment.java}
\end{bAntwort}

%%
% b)
%%

\item Erstellen Sie ein Sequenzdiagramm (mit konkreten Methodenaufrufen und
Nummerierung der Nachrichten) für folgendes Szenario:

\begin{enumerate}

%%
% i.
%%

\item „Erika Mustermann“ aus „Rathausstraße 1, 10178 Berlin“ wird als
neue Kundin angelegt.

%%
% ii,
%%

\item Frau Mustermann bestellt heute 1 kg Gurken und 2 kg Kartoffeln.

%%
% ii.
%%

\item Sie bezahlt mit ihrer Visa-Karte, die im August 2014 abläuft und
die Nummer „1234 567891 23456“ hat — die Karte erweist sich bei der
Prüfung als gültig.

%%
% iv.
%%

\item Am Schluss möchte sie noch wissen, wie viel ihre Bestellung kostet
— dabei wird der Anteil der Mehrwertsteuer extra ausgewiesen.

\end{enumerate}

\end{enumerate}

\end{document}
