\documentclass{bschlangaul-aufgabe}

\begin{document}
\bAufgabenMetadaten{
  Titel = {Aufgabe 2},
  Thematik = {modernen Softwaretechnologie. 3 Begriffe in 3 Sätzen},
  Referenz = 46116-2013-F.T1-TA1-A2,
  RelativerPfad = Staatsexamen/46116/2013/03/Thema-1/Teilaufgabe-1/Aufgabe-2.tex,
  ZitatSchluessel = examen:46116:2013:03,
  BearbeitungsStand = mit Lösung,
  Korrektheit = unbekannt,
  Ueberprueft = {unbekannt},
  Stichwoerter = {Agile Methoden},
  EinzelpruefungsNr = 46116,
  Jahr = 2013,
  Monat = 03,
  ThemaNr = 1,
  TeilaufgabeNr = 1,
  AufgabeNr = 2,
}

Erläutern Sie in jeweils ca. 3 Sätzen die folgenden Begriffe aus der
modernen Softwaretechnologie: \emph{Pair Programming},
\emph{Refactoring}, \emph{agile Softwareentwicklung}.
\index{Agile Methoden}
\footcite{examen:46116:2013:03}

\begin{bAntwort}
\begin{description}
\item[Pair Programming]
Beim Pair Programming entwickeln zwei Personen die Software zusammen. Es
gibt einen sogenannten Driver\footcite[Seite 22]{sosy:fs:agile-methoden}
- die Person, die tippt - und einen sogenannten Navigator - die Person,
die kritisch den Code überprüft.\footcite[Seite 49]{kleuker} Das Pair
Programming ist wichtiger in agilen Entwicklungsmethoden wie z. B dem
Exterme Programming (XP).

\item[Refactoring] Unter Refactoring versteht man die Verbesserung der
Code- und Systemstruktur mit dem Ziel einer besseren
Wartbarkeit.\footcite{sosy:ab:9} Dabei sollen Lesbarkeit,
Verständlichkeit, Wartbarkeit und Erweiterbarkeit verbessert werden, mit
dem Ziel, den jeweiligen Aufwand für Fehleranalyse und funktionale
Erweiterungen deutlich zu senken. Im Refactoring werden keine neuen
Funktionalitäten programmiert.\footcite{wiki:refactoring}

\item[agile Softwareentwicklung]
Agile Entwicklungsprozesse gehen auf das Agile Manifest, das im Jahr
2001 veröffentlicht wurde, zurück. Dieses Agile Manifest bildet die
Basis für agile Entwicklungsprozesse wie eXtreme Programming oder SCRUM.
Das Manifest beinhaltet 12 Grundprinzipien für die moderne agile
Software-Entwicklung. Hauptzielsetzung ist, rasch eine funktionsfähige
Software-Lösung auszuliefern, um den Kundenwunsch bestmöglich zu
erfüllen. Dabei spielt die Interaktion mit dem Kunden eine zentrale
Rolle.
\footcite[Seite 62]{schatten}
\end{description}
\end{bAntwort}
\end{document}
