\documentclass{bschlangaul-aufgabe}
\bLadePakete{syntax,mathe}
\begin{document}
\bAufgabenMetadaten{
  Titel = {2. Anfragen},
  Thematik = {Browser-Online-Spiele},
  Referenz = 46116-2013-F.T1-TA2-A2,
  RelativerPfad = Staatsexamen/46116/2013/03/Thema-1/Teilaufgabe-2/Aufgabe-2.tex,
  ZitatSchluessel = examen:46116:2013:03,
  ZitatBeschreibung = {Thema 1 Teilaufgabe 2 Aufgabe 2},
  BearbeitungsStand = mit Lösung,
  Korrektheit = unbekannt,
  Ueberprueft = {unbekannt},
  Stichwoerter = {Relationale Algebra, SQL, Tupelkalkül},
  EinzelpruefungsNr = 46116,
  Jahr = 2013,
  Monat = 03,
  ThemaNr = 1,
  TeilaufgabeNr = 2,
  AufgabeNr = 2,
}

\section{2. Anfragen
\index{Relationale Algebra}
\footcite[Thema 1 Teilaufgabe 2 Aufgabe 2]{examen:46116:2013:03}
}

Zu einer Website, auf der Besucher im Browser Online-Spiele spielen
können, liegt das folgende relationale Schema einer Datenbank vor:

\bigskip

{
\ttfamily
\noindent
Team : \{[\underline{TNr}, Name, Teamfarbel]\}\\
Spieler : \{[\underline{SNr}, Name, Icon, TNr, EMail]\}\\
Minispiel : \{[\underline{MNr}, Name, Kategorie, Schwierigkeit]\}\\
Wettkampf : \{[\underline{WNr}, Sieger, Geschlagener, MNr, Dauer]\}\\
}

\bigskip

Auf der Website treten jeweils zwei Spieler gegeneinander in Minispielen
an. In diesen ist es das Ziel, den gegnerischen Spieler in möglichst
kurzer Zeit zu besiegen. Minispiele gibt es dabei in verschiedenen
Schwierigkeitsstufen („leicht“, „mittel“, „schwer“, „sehr schwer“) und
verschiedenen Kategorien („Denkspiel“, „Geschicklichkeitsspiel“, usw.).
Die Attribute \emph{Sieger} und \emph{Geschlagener} sind jeweils
Fremdschlüsselattribute, die auf das Attribut \emph{SNr} der Relation
\emph{Spieler} verweisen. Beachten Sie, dass das Dauer-Attribut der
Wettkampf-Relation die Dauer eines Wettkampfes in der Einheit Sekunden
speichert.
\footcite{db:ab:examen-rs-2013-03}

\begin{enumerate}

%%
% a)
%%

\item Formulieren Sie geeignete Anfragen in relationaler Algebra für
die folgenden Teilaufgaben:

\begin{enumerate}
%%
% i.
%%

\item Geben Sie die \emph{Namen} der \emph{Minispiele} zurück, die zur
\emph{Kategorie} \emph{„Denkspiele“} zählen.

\begin{bAntwort}
$
\pi_{\text{Name}}(
  \sigma_{\text{Kategorie} = \mlq \text{Denkspiele} \mrq}(\text{Minispiele})
)
$
\end{bAntwort}

%%
% ii.
%%

\item Geben Sie die \emph{Namen} und \emph{E-Mail-Adressen} aller
Spieler zurück, die in einem Minispiel des Typs
\emph{„Geschicklichkeitsspiel“} gewonnen haben.
\end{enumerate}

\begin{bAntwort}
$
\pi_{\text{Spieler.Name,Spieler.Email}}(
  \sigma_{\text{Kategorie} = \mlq \text{Geschicklichkeitsspiel} \mrq}(\text{Minispiele})
  \bowtie_{\text{Minispiel.MNr} = \text{Wettkampf.MNr}}
  \text{Wettkampf}
  \bowtie_{\text{Wettkampf.Sieger} = \text{Spieler.SNr}}
  \text{Spieler}
)
$
\end{bAntwort}

%%
% b)
%%

\item Formulieren Sie geeignete SQL-Anfragen\index{SQL} für die
folgenden Teilaufgaben. Beachten Sie dabei, dass Ihre Ergebnisrelationen
keine Duplikate enthalten sollen.

\begin{enumerate}

%%
% i.
%%

\item Geben Sie jede Spielekategorie aus, für die ein Minispiel der
Schwierigkeitsstufe sehr schwer vorhanden ist.

\begin{bAntwort}
\begin{minted}{sql}
SELECT DISTINCT Kategorie
FROM Minispiel
WHERE Schwierigkeit = 'sehr schwer';
\end{minted}
\end{bAntwort}

%%
% ii.
%%

\item Geben Sie die Wettkämpfe aus, deren Dauer unter der
durchschnittlichen Dauer der Wettkämpfe liegt.

\begin{bAntwort}
\begin{minted}{sql}
SELECT WNr
FROM Wettkampf
WHERE Dauer < (
  SELECT AVG(Dauer) FROM Wettkampf
);
\end{minted}
\end{bAntwort}

%%
% iii.
%%

\item Geben Sie für jeden \emph{Spieler} seine \emph{SNr}, seinen
\emph{Namen}, die Anzahl seiner Siege, die durchschnittliche Dauer
seiner siegreichen Wettkämpfe und die Anzahl der Teams, aus denen er
bereits mindestens einen Spieler besiegt hat, zurück.

\begin{minted}{sql}
SELECT
  SNr,
  Name,
  (
    SELECT COUNT(*)
    FROM Wettkampf
    WHERE Wettkampf.Sieger = SNr
  ) AS Anzahl_Siege,
  (
    SELECT AVG(Dauer)
    FROM Wettkampf
    WHERE Wettkampf.Sieger = SNr
  ),
  (
    SELECT COUNT(DISTINCT Team.TNr)
    FROM Team, Spieler, Wettkampf
    WHERE
      Team.TNr = Spieler.TNr AND
      Wettkampf.Geschlagener = Spieler.SNr AND
      Wettkampf.Sieger = SNr
  )
FROM Spieler;
\end{minted}

\begin{bAntwort}
\begin{minted}{sql}
SELECT
  s.SNR,
  s.Name,
  COUNT(*) AS AnzahlSiege,
  AVG(w.Dauer) AS DurchschnittlicheWettkampfdauer,
  COUNT(DISTINCT g.TNr) AS TeamsBesiegt
FROM Spieler s, Wettkampf w, Spieler g
WHERE
  s.SNr = w.Sieger AND
  g.SNr = w.Geschlagener
GROUP BY s.SNR, s.Name;
\end{minted}
\end{bAntwort}
\end{enumerate}

%%
% c)
%%

\item Verwenden Sie den relationalen Tupelkalkül\index{Tupelkalkül}, um
die folgenden Anfragen zu formulieren:

\begin{enumerate}

%%
% i.
%%

\item Finden Sie die Namen der Spieler des Teams Dream Team.

%%
% ii.
%%

\item Geben Sie die Namen der Minispiele zurück, bei denen bereits
Spieler gegeneinander angetreten sind, deren Teams dieselbe Teamfarbe
haben.
\end{enumerate}
\end{enumerate}

Hinweis: Die Anfragen im relationalen Tupelkalkül dürfen auch nicht
sicher sein.
\end{document}
