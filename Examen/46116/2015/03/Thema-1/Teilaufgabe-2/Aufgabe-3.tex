\documentclass{bschlangaul-aufgabe}
\bLadePakete{er}
\begin{document}
\bAufgabenMetadaten{
  Titel = {Aufgabe 3},
  Thematik = {Konsulat},
  Referenz = 46116-2015-F.T1-TA2-A3,
  RelativerPfad = Staatsexamen/46116/2015/03/Thema-1/Teilaufgabe-2/Aufgabe-3.tex,
  ZitatSchluessel = examen:46116:2015:03,
  BearbeitungsStand = mit Lösung,
  Korrektheit = unbekannt,
  Ueberprueft = {unbekannt},
  Stichwoerter = {Entity-Relation-Modell, Relationenmodell},
  EinzelpruefungsNr = 46116,
  Jahr = 2015,
  Monat = 03,
  ThemaNr = 1,
  TeilaufgabeNr = 2,
  AufgabeNr = 3,
}

\let\a=\bErMpAttribute
\let\d=\bErDatenbankName
\let\e=\bErMpEntity
\let\r=\bErMpRelationship

Es\index{Entity-Relation-Modell}
\index{Relationenmodell}
\footcite{examen:46116:2015:03} sind folgende Informationen zu einer Datenbank für Konsulate
gegeben:
\footcite[Aufgabe 5: ER-Modell und Relationenmodell (Check-Up)]{db:ab:2}

\begin{itemize}

\item Jedes Konsulat hat einen Sitz in einer Stadt

\item Zu einem \e{Konsulat} soll ein eindeutiger
\a{Name} \d{KonsulatName} (\zB Konsulat Bayern), die
\a{Adresse} und der \a{Vor-} \d{KVorname} bzw.
\a{Nachname} \d{KNachname} des Konsuls gespeichert werden.

\item Für jede \e{Stadt} sollen der \a{Name}
\d{StadtName}, die \a{Anzahl der Einwohner}
\d{EinwohnerAnzahl}, sowie das Land in dem es \r{liegt},
festgehalten werden. Gehen Sie davon aus, dass eine Stadt nur in
Zusammenhang mit dem zugehörigen Land identifizierbar ist.

\item Für ein \e{Land} soll der Name in
\a{Landessprache}, der \a{Name des
Staatspräsidenten} \d{Staatspräsident} und eine eindeutige
\a{ID} \d{LandesID} gespeichert werden.
\end{itemize}

\begin{enumerate}

%%
% (a)
%%

\item Entwerfen Sie für das obige Szenario ein ER-Diagramm in
Chen-Notation. Bestimmen Sie hierzu:

\begin{itemize}

\item Die Entity-Typen, die Relationship-Typen und jeweils deren
Attribute,

\item Die Primärschlüssel der Entity-Typen, welche Sie anschließend in
das ER-Diagramm eintragen, und

\item Die Funktionalitäten der Relationship-Typen.
\end{itemize}

Hinweis: Achten Sie darauf, alle Totalitäten einzutragen.

\begin{bAntwort}
\begin{center}
\begin{tikzpicture}[er2,scale=0.7,transform shape]
% Land
\node[entity] (Land) {Land};
\node[attribute,above=1cm of Land] {Staatspräsident} edge (Land);
\node[attribute,above left=0.5cm of Land] {Landessprache} edge (Land);
\node[attribute,left=0.5cm of Land] {\key{LandesID}} edge (Land);

% Stadt
\node[weak entity,right=3cm of Land] (Stadt) {Stadt};
\node[attribute,above right=0.5cm of Stadt] {EinwohnerAnzahl} edge (Stadt);
\node[attribute,right=0.5cm of Stadt] {\discriminator{StadtName}} edge (Stadt);

% liegen
\node[ident relationship,right=0.4cm of Land]{liegen}
  edge (Land) edge[weak] (Stadt);

% Konsulat
\node[entity,below right=1cm of Land] (Konsulat) {Konsulat};
\node[attribute,left=0.5cm of Konsulat] {Adresse} edge (Konsulat);
\node[attribute,below left=0.5cm of Konsulat] {KNachname} edge (Konsulat);
\node[attribute,below=1cm of Konsulat] {KVorname} edge (Konsulat);
\node[attribute,below right=0.5cm of Konsulat] {\key{KonsulatName}} edge (Konsulat);
\end{tikzpicture}
\end{center}
\end{bAntwort}

%%
% (b)
%%

\item Überführen Sie das ER-Modell aus Aufgabe a) in ein verfeinertes
relationales Modell. Geben Sie hierfür die verallgemeinerten
Relationenschemata an. Achten Sie dabei insbesondere darauf, dass die
Relationenschemata keine redundanten Attribute enthalten.

\begin{bAntwort}
Konsulat(\underline{KonsulatName}, KVorname, KNachname, Adresse, StadtName, LandesID)

Stadt(\underline{LandesID, StadtName}, EinwohnerAnzahl)

Land(\underline{LandesID}, Landessprache, Staatspraesident)
\end{bAntwort}
\end{enumerate}

\end{document}
