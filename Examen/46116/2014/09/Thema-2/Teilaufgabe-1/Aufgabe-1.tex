\documentclass{bschlangaul-aufgabe}
\bLadePakete{checkbox}
\begin{document}
\bAufgabenMetadaten{
  Titel = {Aufgabe 1},
  Thematik = {Multiple-Choice: Allgemeine SWT, Vorgehensmodelle und Requirements},
  Referenz = 46116-2014-H.T2-TA1-A1,
  RelativerPfad = Examen/46116/2014/09/Thema-2/Teilaufgabe-1/Aufgabe-1.tex,
  ZitatSchluessel = examen:46116:2014:09,
  BearbeitungsStand = mit Lösung,
  Korrektheit = unbekannt,
  Ueberprueft = {unbekannt},
  Stichwoerter = {EXtreme Programming, V-Modell, SCRUM, Prototyping, Funktionale Anforderungen, Nicht-funktionale Anforderungen},
  EinzelpruefungsNr = 46116,
  Jahr = 2014,
  Monat = 09,
  ThemaNr = 2,
  TeilaufgabeNr = 1,
  AufgabeNr = 1,
}

\section{Aufgabe 1: Allgemeine SWT, Vorgehensmodelle und Requirements
Engineering
}

Kreuzen Sie für die folgenden Multiple-Choice-Fragen genau die richtigen
Antworten deutlich an. Es kann mehr als eine Antwort richtig sein.

Jedes korrekt gesetzte oder korrekt nicht gesetzte Kreuz wird mit 1
Punkt gewertet. Jedes falsch gesetzte oder falsch nicht gesetzte Kreuz
wird mit -1 Punkt gewertet. Eine Frage kann entwertet werden, dann wird
sie nicht in der Korrektur berücksichtigt. Einzelne Antworten können
nicht entwertet werden. Entwerten Sie eine Frage wie
folgt\footcite{examen:46116:2014:09}

Die gesamte Aufgabe wird nicht mit weniger als 0 Punkten gewertet.

\begin{enumerate}

%%
% 1.
%%

\item Welche Aussage ist wahr?

\begin{itemize}
\bCheckboxLeer Je früher ein Fehler entdeckt wird, umso teurer ist seine
Korrektur.

\bCheckboxLeer Je später ein Fehler entdeckt wird, umso teurer ist seine
Korrektur.

\bCheckboxLeer Der Zeitpunkt der Entdeckung hat keinen Einfluss auf die
Kosten.
\end{itemize}

\begin{bAntwort}
2 ist richtig: Je später der Fehler entdeckt wird, desto mehr wurde er
schon in das Projekt „eingearbeitet“, daher dauert das Beseitigen des
Fehlers länger und das kostet mehr Geld.
\end{bAntwort}

%%
% 2.
%%

\item  Mit welcher Methodik können Funktionen spezifiziert werden?

\begin{itemize}
\bCheckboxLeer Als Funktionsvereinbarung in einer Programmiersprache
\bCheckboxLeer Mit den Vor- und Nachbedingungen von Kontrakten
\bCheckboxLeer Als Zustandsautomaten
\end{itemize}

\begin{bAntwort}
2 und 3 ist richtig: Die Spezifikation soll unabhängig von einer
Programmiersprache sein.
\end{bAntwort}

%%
% 3.
%%

\item Welche Vorgehensmodelle sind für Projekte mit häufigen Änderungen
geeignet?

\begin{itemize}
\bCheckboxLeer Extreme Programming (XP)\index{EXtreme Programming}
\bCheckboxLeer Das V-Modell 97\index{V-Modell}
\bCheckboxLeer Scrum\index{SCRUM}
\end{itemize}

\begin{bAntwort}
1 und 3 ist richtig. Das V-Modell ist ein starres Vorgehensmodell, bei
dem alle Anforderungen zu Beginn vorhanden sein müssen.
\end{bAntwort}

%%
% 4.
%%

\item Welche der folgenden Aussagen ist korrekt?

\begin{itemize}
\bCheckboxLeer Mittels Prototyping\index{Prototyping} versucht man die
Anzahl an nötigen Unit-Tests zu reduzieren.

\bCheckboxLeer Ein Ziel von Prototyping ist die Erhöhung der Qualität
während der Anforderungsanalyse.

\bCheckboxLeer Mit Prototyping versucht man sehr früh Feedback von
Stakeholdern zu erhalten.
\end{itemize}

\begin{bAntwort}
2 und 3 ist richtig: Prototypen müssen auch getestet werden. Es kann
nicht an Tests gespart werden. Durch das häufige Feedback des Kunden /
der Stakeholder können die Anforderungen immer genauer und klarer
erfasst werden.
\end{bAntwort}

%%
% 5.
%%

\item Welche der folgenden Aussagen ist korrekt?

\begin{itemize}
\bCheckboxLeer Bei der Architektur sollten funktionale\index{Funktionale
Anforderungen} und nicht-funktionale
Anforderungen\index{Nicht-funktionale Anforderungen} beachtet werden.

\bCheckboxLeer Bei der Architektur soliten nur funktionale Anforderungen
beachtet werden.

\bCheckboxLeer Bei der Architektur sollten nur nicht-funktionale
Anforderungen beachtet werden.

\bCheckboxLeer Bei der Architektur sollte auf die mögliche Änderungen
von Komponenten geachtet werden.
\end{itemize}

\begin{bAntwort}
1 und 4 ist richtig: Mögliche Änderungen werden durch klar definierte
Schnittstellen und wenig Kopplung der Komponenten erleichtert. (Kopplung
handelt von Abhängigkeiten zwischen Modulen. Kohäsion handelt von
Abhängigkeiten zwischen Funktionen innerhalb eines Moduls.)
\end{bAntwort}
\end{enumerate}
\end{document}
