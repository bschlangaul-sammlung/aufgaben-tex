\documentclass{bschlangaul-aufgabe}

\begin{document}
\bAufgabenMetadaten{
  Titel = {Aufgabe 3},
  Thematik = {Fertigung},
  Referenz = 66111-1997-H.A3,
  RelativerPfad = Staatsexamen/66111/1997/09/Aufgabe-3.tex,
  ZitatSchluessel = examen:66111:1997:09,
  ZitatBeschreibung = {Aufgabe
3},
  BearbeitungsStand = mit Lösung,
  Korrektheit = unbekannt,
  Ueberprueft = {unbekannt},
  Stichwoerter = {Entity-Relation-Modell, Relationenmodell},
  EinzelpruefungsNr = 66111,
  Jahr = 1997,
  Monat = 09,
  AufgabeNr = 3,
}

Für\index{Entity-Relation-Modell} \footcite[Aufgabe
3]{examen:66111:1997:09} ein Unternehmen soll eine Fertigungsdatenbank
aufgebaut werden. Der Erhebungsprozess liefert folgenden
Informationsbedarf:
\footcite[Aufgabe 4]{db:ab:7}

\subsection{Entity-Typen:}

\begin{compactitem}
\item ABTEILUNG mit den Attributen ANR, ANAME, AORT, MNR
\item PERSONAL mit den Attributen PNR, NAME, BERUF
\item MASCHINE mit den Attributen MANR, FABRIKAT, TYP, BEZ, LEISTUNG
\item TEILE mit den Attributen LNR, BEZ, GEWICHT, FARBE, PREIS
\end{compactitem}

\subsection{Relationship-Typen:}

\begin{compactitem}
\item ABT-PERS zwischen ABTEILUNG und PERSONAL
\item SETZT-EIN zwischen ABTEILUNG und MASCHINEN
\item KANN-BEDIENEN zwischen PERSONAL und MASCHINEN
\item GEEIGNET-FÜR-DIE-HERSTELLUNG-VON zwischen MASCHINEN und TEILE
\item PRODUKTION zwischen PERSONAL, TEILE und MASCHINEN mit den Attributen
DATUM und MENGE
\end{compactitem}

Dabei sollen folgende grundlegenden Bedingungen gelten:

\begin{compactitem}
\item Zu einer Abteilung gehört mindestens ein Beschäftigter

\item Eine Person ist immer nur genau einer Abteilung zugeordnet

\item Eine Maschine kann, wenn überhaupt, nur von einer Abteilung
eingesetzt werden

\item Alle anderen (Teil-)Beziehungen sind nicht weiter eingeschränkt.
\end{compactitem}

\begin{enumerate}
\item Zeichnen Sie zu dem obigen Szenario das zugehörige ER-Diagramm.

\item Legen Sie die Schlüsselkandidaten fest und zeichnen Sie diese in
das ER-Diagramm ein. Tragen Sie die oben genannten Bedingungen mit Hilfe
der (min, max) – Notation in das ER-Diagramm. Formulieren Sie weitere
sinnvolle Bedingungen und tragen Sie diese ebenfalls in das Diagramm
ein.
\end{enumerate}

Konvertieren Sie das folgende ER-Modell in ein (vereinfachtes)
relationales Schema\index{Relationenmodell}! Geben Sie dabei geeignete
Domänenattribute an!
\footcite[Aufgabe 5: Fertigungsdatenbank die Zweite (Fortsetzung Staatsexamen Herbst 1997)]{db:ab:7}
\end{document}
