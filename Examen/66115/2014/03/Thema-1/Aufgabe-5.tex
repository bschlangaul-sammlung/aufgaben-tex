\documentclass{bschlangaul-aufgabe}
\bLadePakete{formale-sprachen}
\begin{document}
\bAufgabenMetadaten{
  Titel = {Aufgabe 5},
  Thematik = {Halteproblem H m},
  Referenz = 66115-2014-F.T1-A5,
  RelativerPfad = Examen/66115/2014/03/Thema-1/Aufgabe-5.tex,
  ZitatSchluessel = examen:66115:2014:03,
  BearbeitungsStand = mit Lösung,
  Korrektheit = unbekannt,
  Ueberprueft = {unbekannt},
  Stichwoerter = {Entscheidbarkeit},
  EinzelpruefungsNr = 66115,
  Jahr = 2014,
  Monat = 03,
  ThemaNr = 1,
  AufgabeNr = 5,
}

\begin{enumerate}

%%
% a)
%%

\item Definieren\index{Entscheidbarkeit} \footcite{examen:66115:2014:03}
Sie die zum Halteproblem für Turing-Maschinen bei fester Eingabe $m \in
\mathbb{N}_0 = \{ 0, 1, 2, \dots \}$ gehörende Menge $H_m$.
\footcite[Seite 55]{theo:fs:3}

\begin{bAntwort}
\bAusdruck[H_m]{c(M) \in \mathbb{N}}{c(M)\text{ hält auf Eingabe }m},
wobei $c(M)$ der Codierung der Turingmaschine (Gödelnummer) entspricht.
\end{bAntwort}

%%
% b)
%%

\item Gegeben sei das folgende Problem E:

Entscheiden Sie, ob es für die deterministische Turing-Maschine mit der
Gödelnummer $n$ eine Eingabe $w \in \mathbb{N}_0$ gibt, so dass $w$ eine
gerade Zahl ist und die Maschine $n$ gestartet mit $w$ hält.

Zeigen. Sie, dass $E$ nicht entscheidbar ist. Benutzen Sie, dass $H_m$
aus (a) für jedes $m \in \mathbb{N}_0$ nicht entscheidbar ist.

\begin{bAntwort}
Wir zeigen dies durch Reduktion $H_2 \leq E$:
\begin{itemize}
\item Berechenbare Funktion $f$: lösche Eingabe, schreibe eine $2$ und
starte dich selbst.

\item $M‘$ ist eine Turingmaschine, die $E$ entscheidet.

\item $x \in H_2$ (Quellcode der Programme, die auf die Eingabe von $2$
halten)

\item $M_x$ (kompiliertes Programm, TM)

\item Für alle $x \in H_2$ gilt, $M_x$ hält auf Eingabe von $2
\Leftrightarrow f(x) = c(M‘) \in E$. Denn sofern die ursprüngliche
Maschine auf das Wort $2$ hält, hält $M‘$ auf alle Eingaben und somit
auch auf Eingaben gerader Zahlen. Hält die ursprüngliche Maschine $M$
nicht auf die Eingabe der Zahl $2$, so hält $M‘$ auf keine Eingabe.
\end{itemize}
\end{bAntwort}

%%
% c)
%%

\item Zeigen Sie, dass das Problem $E$ aus (b) partiell-entscheidbar (=
rekursiv aufzählbar) ist.

\end{enumerate}
\end{document}
