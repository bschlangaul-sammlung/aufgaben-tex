\documentclass{bschlangaul-aufgabe}
\bLadePakete{formale-sprachen}
\begin{document}
\bAufgabenMetadaten{
  Titel = {Aufgabe 4},
  Thematik = {Automaten und formale Sprachen},
  Referenz = 66115-2008-H.T2-A4,
  RelativerPfad = Examen/66115/2008/09/Thema-2/Aufgabe-4.tex,
  ZitatSchluessel = examen:66115:2008:09,
  BearbeitungsStand = nur Angabe,
  Korrektheit = unbekannt,
  Ueberprueft = {unbekannt},
  Stichwoerter = {Theoretische Informatik},
  EinzelpruefungsNr = 66115,
  Jahr = 2008,
  Monat = 09,
  ThemaNr = 2,
  AufgabeNr = 4,
}

Gesucht ist die Menge $L$ aller Dezimalzahlen über \bAlphabet{0,1,2,3}
(mit führenden Nullen), die durch 2 oder (logisches oder) durch 3
teilbar sind.
\index{Theoretische Informatik}
\footcite{examen:66115:2008:09}

Beschreiben Sie L durch einen Automaten oder durch eine Grammatik.

In welcher Klasse der Chomsky-Hierarchie liegt L? Geben Sie die
kleinstmögliche Klasse der Chomsky-Hierarchie an.
\end{document}
