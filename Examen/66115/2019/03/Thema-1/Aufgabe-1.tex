\documentclass{bschlangaul-aufgabe}
\bLadePakete{mathe,formale-sprachen}
\begin{document}
\bAufgabenMetadaten{
  Titel = {Aufgabe 1},
  Thematik = {Wissensfragen wahr / falsch},
  Referenz = 66115-2019-F.T1-A1,
  RelativerPfad = Staatsexamen/66115/2019/03/Thema-1/Aufgabe-1.tex,
  ZitatSchluessel = examen:66115:2019:03,
  BearbeitungsStand = mit Lösung,
  Korrektheit = unbekannt,
  Ueberprueft = {unbekannt},
  Stichwoerter = {Formale Sprachen},
  EinzelpruefungsNr = 66115,
  Jahr = 2019,
  Monat = 03,
  ThemaNr = 1,
  AufgabeNr = 1,
}

\let\m=\bMenge

Antworten Sie auf die folgenden Behauptungen mit Wahr/Falsch und geben
Sie eine kurze Begründung an.
\index{Formale Sprachen}
\footcite{examen:66115:2019:03}

\begin{enumerate}

%%
% (a)
%%

\item Wenn $L_2$ regulär ist und $L_1 \subseteq L_2$ gilt, dann ist
$L_1$ auch regulär.

\begin{bAntwort}
Falsch

Nein. Wähle $L = \Sigma^*$. Wähle eine beliebige nicht-reguläre Sprache
$L'$. $L'$ ist Teilmenge von $L$.
\bFussnoteUrl{https://www.c-plusplus.net/forum/topic/287036/theoretische-informatik-teilmenge-einer-regulären-sprache}

%%
%
%%

$L_2 = \m{a^* b^*}$ ...Regulär

$L_1 = \m{a^n
b^n}$ ...nicht regulär,

da man sich das $n$ merken muss, das bedeutet,
dass man beispielsweise einen Automaten bräuchte, der unendlich viele
Zustände besitzt. Das wiederum ist aber bei DEA’s nicht möglich. (Da sie
nur endlich viele Zustände haben können.) Für beide Sprachen gilt somit:
$L_1 \subseteq L_2$, aber nicht beide sind regulär.
\bFussnoteUrl{https://vowi.fsinf.at/images/3/3a/TU_Wien-Theoretische_Informatik_und_Logik_VU_(Fermüller,_Freund)-Übungen_SS13_-_Uebungsblatt_1.pdf}
\end{bAntwort}

%%
% (b)
%%

\item \bAusdruck{a^q}{\exists i \in \mathbb{N}. \, q = i^2} ist
bekanntlich nicht regulär. Behauptung: $Q^*$ ist ebenfalls nicht
regulär.

\begin{bAntwort}
Wahr.

Siehe Abschlusseigenschaften der Formalen Sprachen unter dem Kleene-Stern.
\end{bAntwort}

%%
% (c)
%%

\item Wenn $L \subseteq \Sigma^*$ entscheidbar ist, dann ist auch das
Komplement $\bar L = \Sigma^* \setminus L$ entscheidbar.

\begin{bAntwort}
Wahr

Zu jeder entscheidbaren Menge ist auch ihr Komplement
entscheidbar.\footcite{wiki:entscheidbar}

$L$ entscheidbar, dann auch $\overline{L}$ entscheidbar. Wir benutzen
eine DTM $\tau'$, die die DTM $\tau$ simuliert. Diese vertauscht die
beiden Ausgaben „JA“ und „NEIN“.
\bFussnoteUrl{http://www.informatikseite.de/theorie/node14.php\#SECTION00044100000000000000}
\end{bAntwort}

%%
% (d)
%%

\item Jedes $\mathcal{NP}$-vollständige Problem ist entscheidbar.

\begin{bAntwort}
Wahr

Definition von $\mathcal{NP}$-Vollständigkeit:

vollständig für die Klasse der Probleme, die sich nichtdeterministisch
in Polynomialzeit lösen lassen, wenn es zu den schwierigsten Problemen
in der Klasse NP gehört, also sowohl in NP liegt als auch NP-schwer
ist.\footcite{wiki:np-vollstaendig}
\end{bAntwort}
\end{enumerate}

\end{document}
