\documentclass{bschlangaul-aufgabe}
\bLadePakete{formale-sprachen,automaten,potenzmengen-konstruktion}
\begin{document}
\bAufgabenMetadaten{
  Titel = {Aufgabe 2},
  Thematik = {NEA nach DEA},
  Referenz = 66115-2019-F.T1-A2,
  RelativerPfad = Examen/66115/2019/03/Thema-1/Aufgabe-2.tex,
  ZitatSchluessel = examen:66115:2019:03,
  BearbeitungsStand = mit Lösung,
  Korrektheit = unbekannt,
  Ueberprueft = {unbekannt},
  Stichwoerter = {Reguläre Sprache},
  EinzelpruefungsNr = 66115,
  Jahr = 2019,
  Monat = 03,
  ThemaNr = 1,
  AufgabeNr = 2,
}

\def\z#1{
\bZustandsMengenSammlungNr{#1}{
    {
      {0} {0}
      {1} {0,1}
      {2} {0,1,2}
      {3} {0,2}
    }
  }
}
\let\s=\bZustandsnameGross

\begin{enumerate}
%%
% (a)
%%

\item Gegeben sei der nichtdeterministische endliche Automat $A$ über
dem Alphabet \bAlphabet{a, b}
wie folgt:\index{Reguläre Sprache}
\footcite{examen:66115:2019:03}

\begin{center}
\begin{tikzpicture}[li automat]
  \node[state,initial] (z0) at (3.71cm,-2.43cm) {$z_0$};
  \node[state] (z1) at (5.71cm,-2.43cm) {$z_1$};
  \node[state,accepting] (z2) at (8.14cm,-2.43cm) {$z_2$};

  \path (z0) edge[auto] node{$a$} (z1);
  \path (z0) edge[auto,loop above] node{$a,b$} (z0);
  \path (z1) edge[auto] node{$a,b$} (z2);
\end{tikzpicture}
\end{center}
\bFlaci{Arozq4rm2}

Konstruieren Sie einen deterministischen endlichen Automaten, der das
Komplement \bAusdruck[\bar L(A)]{w \in \Sigma^*}{w \notin L(A)} der von $A$
akzeptierten Sprache $L(A)$ akzeptiert.

\begin{bAntwort}
Wir konvertieren zuerst den nichtdeterministischen endlichen Automaten in
einen deterministischen endlichen Automaten mit Hilfe des
Potenzmengenalgorithmus.

\begin{tabular}{l|l|l}
Zustandsmenge & Eingabe $a$ & Eingabe $b$ \\\hline
\z0 & \z1 & \z0 \\
\z1 & \z2 & \z3 \\
\z2 & \z2 & \z3 \\
\z3 & \z1 & \z0 \\
\end{tabular}

\begin{center}
\begin{tikzpicture}[li automat]
  \node[state,initial] (z0) at (2.14cm,-2.14cm) {\s0};
  \node[state,accepting] (z012) at (5.29cm,-5.29cm) {\s1};
  \node[state,accepting] (z02) at (2.14cm,-5.29cm) {\s3};
  \node[state] (z01) at (5.29cm,-2.14cm) {\s2};

  \path (z0) edge[auto] node{$a$} (z01);
  \path (z0) edge[auto,loop above] node{$b$} (z0);
  \path (z012) edge[auto,loop right] node{$a$} (z012);
  \path (z012) edge[auto] node{$b$} (z02);
  \path (z02) edge[auto,bend left] node{$a$} (z01);
  \path (z02) edge[auto] node{$b$} (z0);
  \path (z01) edge[auto] node{$a$} (z012);
  \path (z01) edge[auto,bend left] node{$b$} (z02);
\end{tikzpicture}
\end{center}
\bFlaci{Arxujcbdg}

Wir vertauschen die End- und Nicht-End-Zustände, um das Komplement zu
erhalten:

\begin{center}
\begin{tikzpicture}[li automat]
  \node[state,initial,accepting] (z0) at (2.14cm,-2.14cm) {\s0};
  \node[state] (z012) at (5.29cm,-5.29cm) {\s1};
  \node[state] (z02) at (2.14cm,-5.29cm) {\s3};
  \node[state,accepting] (z01) at (5.29cm,-2.14cm) {\s2};

  \path (z0) edge[auto] node{$a$} (z01);
  \path (z0) edge[auto,loop above] node{$b$} (z0);
  \path (z012) edge[auto,loop right] node{$a$} (z012);
  \path (z012) edge[auto] node{$b$} (z02);
  \path (z02) edge[auto,bend left] node{$a$} (z01);
  \path (z02) edge[auto] node{$b$} (z0);
  \path (z01) edge[auto] node{$a$} (z012);
  \path (z01) edge[auto,bend left] node{$b$} (z02);
\end{tikzpicture}
\end{center}
\bFlaci{A5zqsonq2}

\end{bAntwort}

%%
% (b)
%%

\item Gegeben sei zudem der nichtdeterministische Automat $B$ über dem
Alphabet \bAlphabet{a, b}:

\begin{center}
\begin{tikzpicture}[li automat]
  \node[state,initial,accepting] (z3) at (5.29cm,-5.57cm) {$z_3$};
  \node[state] (z4) at (7.71cm,-5.57cm) {$z_4$};

  \path (z3) edge[auto,bend left] node{$a$} (z4);
  \path (z3) edge[auto,loop above] node{$b$} (z3);
  \path (z4) edge[auto,bend left] node{$a$} (z3);
\end{tikzpicture}
\end{center}
\bFlaci{Arafgk0h2}

Konstruieren Sie einen endlichen Automaten (möglicherweise mit
$\varepsilon$-Übergängen), der die Sprache $(L(A)L(B))^* \subseteq
\Sigma^*$ akzeptiert ($A$ aus der vorigen Aufgabe). Erläutern Sie auch
Ihre Konstruktionsidee.

\begin{bAntwort}
$L(A)L(B))^*$ ist die beliebige Konkatenation (Verknüpfung/Verkettung)
der Sprachen $L(A)$ und $L(B)$ mit dem leeren Wort. Wir führen einen
neuen Startzustand ($z_5$) ein, der zugleich Endzustand ist. Dadurch
wird das leere Wort akzeptiert. Dieser neue Startzustand führt über
$\varepsilon$-Übergängen zu den ehemaligen Startzuständen der Automaten
$A$ und $B$. Die Endzustände der Automaten $A$ und $B$ führen über
$\varepsilon$-Übergängen zu $z_5$. Dadurch sind beliebige
Konkatenationen möglich.

\begin{center}
\begin{tikzpicture}[li automat]
  \node[state] (z0) at (3.71cm,-2.43cm) {$z_0$};
  \node[state] (z1) at (5.71cm,-2.43cm) {$z_1$};
  \node[state,accepting] (z2) at (8.14cm,-2.43cm) {$z_2$};
  \node[state,initial,accepting] (z5) at (2.14cm,-5cm) {$z_5$};
  \node[state,accepting] (z3) at (5.29cm,-5.71cm) {$z_3$};
  \node[state] (z4) at (7.71cm,-5.57cm) {$z_4$};

  \path (z0) edge[auto] node{$a$} (z1);
  \path (z0) edge[auto,loop above] node{$a,b$} (z0);
  \path (z1) edge[auto] node{$a,b$} (z2);
  \path (z2) edge[auto] node{$\varepsilon$} (z5);
  \path (z5) edge[auto,bend left] node{$\varepsilon$} (z0);
  \path (z5) edge[auto,bend left] node{$\varepsilon$} (z3);
  \path (z3) edge[auto,bend left] node{$a$} (z4);
  \path (z3) edge[auto,bend left] node{$\varepsilon$} (z5);
  \path (z3) edge[auto,loop above] node{$b$} (z3);
  \path (z4) edge[auto,bend left] node{$a$} (z3);
\end{tikzpicture}
\end{center}
\bFlaci{Aro3uhzjz}

\end{bAntwort}

\end{enumerate}
\end{document}
