\documentclass{bschlangaul-aufgabe}
\bLadePakete{java,mathe}
\begin{document}
\bAufgabenMetadaten{
  Titel = {Aufgabe 5},
  Thematik = {Notation des Informatik-Duden},
  Referenz = 66115-2019-H.T1-A5,
  RelativerPfad = Staatsexamen/66115/2019/09/Thema-1/Aufgabe-5.tex,
  ZitatSchluessel = examen:66115:2019:09,
  BearbeitungsStand = mit Lösung,
  Korrektheit = unbekannt,
  Ueberprueft = {unbekannt},
  Stichwoerter = {Quicksort},
  EinzelpruefungsNr = 66115,
  Jahr = 2019,
  Monat = 09,
  ThemaNr = 1,
  AufgabeNr = 5,
}

In der folgenden Aufgabe soll ein Feld A von ganzen Zahlen aufsteigend
sortiert werden. Das Feld habe n Elemente A[0] bis A[n-1]. Der folgende
Algorithmus (in der Notation des Informatik-Duden) sei gegeben:
\index{Quicksort}
\footcite{examen:66115:2019:09}

\begin{minted}{python}
procedure quicksort(links, rechts : integer)
var i, j, x : integer;
begin
  i := links;
  j := rechts;
  if j > i then begin
    x := A[links];
    repeat
      while A[i] < x do i := i+1;
      while A[j] > x do j := j-1;
      if i < j then begin
        tmp := A[i]; A[i] := A[j]; A[j] := tmp;
        i := i+1; j := j-1;:
      end
    until i > j;
    quicksort(links, j);
    quicksort(i, rechts);
  end
end
\end{minted}

\bPseudoUeberschrift{Umsetzung in Java:}

\bJavaExamen[firstline=6,lastline=32]{66115}{2019}{09}{QuickSort}

Der initiale Aufruf der Prozedur lautet:

\begin{minted}{python}
quicksort(0,n-1)
\end{minted}

\begin{enumerate}

\item Sortieren Sie das folgende Feld der Länge 7 mittels des
Algorithmus. Notieren Sie jeweils alle Aufrufe der Prozedur quicksort
mit den konkreten Parameterwerten. Geben Sie zudem für jeden Aufruf der
Prozedur den Wert des in Zeile 7 gewählten Elements an.

\begin{center}
27 13 21 3 6 17 44 42
\end{center}

\begin{bAntwort}
\begin{minted}{md}
quicksort(0, 6)
27 32 3 6 17 44 42
x: 27
quicksort(0, 2)
17 6 3 32 27 44 42
x: 17
quicksort(0, 1)
3 6 17 32 27 44 42
x: 3
quicksort(0, -1)
3 6 17 32 27 44 42
quicksort(1, 1)
3 6 17 32 27 44 42
quicksort(2, 2)
3 6 17 32 27 44 42
quicksort(3, 6)
3 6 17 32 27 44 42
x: 32
quicksort(3, 3)
3 6 17 27 32 44 42
quicksort(4, 6)
3 6 17 27 32 44 42
x: 32
quicksort(4, 3)
3 6 17 27 32 44 42
quicksort(5, 6)
3 6 17 27 32 44 42
x: 44
quicksort(5, 5)
3 6 17 27 32 42 44
quicksort(6, 6)
3 6 17 27 32 42 44
3 6 17 27 32 42 44
\end{minted}

\end{bAntwort}

\item Angenommen, die Bedingung $j > i$ in Zeile 6 des Algorithmus wird
ersetzt durch die Bedingung $j \geq i$. Ist der Algorithmus weiterhin
korrekt? Begründen Sie Ihre Antwort.

\begin{bAntwort}
geht, dauert aber länger

\begin{minted}{md}
quicksort(0, 6)
27 32 3 6 17 44 42
x: 27
quicksort(0, 2)
17 6 3 32 27 44 42
x: 17
quicksort(0, 1)
3 6 17 32 27 44 42
x: 3
quicksort(0, -1)
3 6 17 32 27 44 42
quicksort(1, 1)
3 6 17 32 27 44 42
x: 6
quicksort(1, 0)
3 6 17 32 27 44 42
quicksort(2, 1)
3 6 17 32 27 44 42
quicksort(2, 2)
3 6 17 32 27 44 42
x: 17
quicksort(2, 1)
3 6 17 32 27 44 42
quicksort(3, 2)
3 6 17 32 27 44 42
quicksort(3, 6)
3 6 17 32 27 44 42
x: 32
quicksort(3, 3)
3 6 17 27 32 44 42
x: 27
quicksort(3, 2)
3 6 17 27 32 44 42
quicksort(4, 3)
3 6 17 27 32 44 42
quicksort(4, 6)
3 6 17 27 32 44 42
x: 32
quicksort(4, 3)
3 6 17 27 32 44 42
quicksort(5, 6)
3 6 17 27 32 44 42
x: 44
quicksort(5, 5)
3 6 17 27 32 42 44
x: 42
quicksort(5, 4)
3 6 17 27 32 42 44
quicksort(6, 5)
3 6 17 27 32 42 44
quicksort(6, 6)
3 6 17 27 32 42 44
x: 44
quicksort(6, 5)
3 6 17 27 32 42 44
quicksort(7, 6)
3 6 17 27 32 42 44
3 6 17 27 32 42 44
\end{minted}
\end{bAntwort}

\item Angenommen, die Bedingung $i \leq j$ in Zeile 11 des Algorithmus wird
ersetzt durch die Bedingung $i < j$. Ist der Algorithmus weiterhin
korrekt? Begründen Sie Ihre Antwort.

\begin{bAntwort}
bleibt hängen

\begin{minted}{md}
quicksort(0, 6)
27 32 3 6 17 44 42
x: 27
quicksort(0, 2)
17 6 3 32 27 44 42
x: 17
quicksort(0, 1)
3 6 17 32 27 44 42
x: 3
\end{minted}

\end{bAntwort}

\item Wie muss das Feld A gestaltet sein, damit der Algorithmus mit der
geringsten Anzahl von Schritten terminiert? Betrachten Sie dazu vor
allem Zeile 7. Begründen Sie Ihre Antwort und geben Sie ein Beispiel.

\begin{bAntwort}
Im Worst Case (schlechtesten Fall) wird das Pivotelement stets so
gewählt, dass es das größte oder das kleinste Element der Liste ist.
Dies ist etwa der Fall, wenn als Pivotelement stets das Element am Ende
der Liste gewählt wird und die zu sortierende Liste bereits sortiert
vorliegt. Die zu untersuchende Liste wird dann in jedem
Rekursionsschritt nur um eins kleiner und die Zeitkomplexität wird
beschrieben durch $\mathcal {O}(n^{2})$. Die Anzahl der Vergleiche ist
in diesem Fall ${\tfrac {n\cdot \left(n+1\right)}{2}}-1={\tfrac
{n^{2}}{2}}+{\tfrac {n}{2}}-1$.

Die Länge der jeweils längeren Teilliste beim rekursiven Aufrufe ist
nämlich im Schnitt  $\textstyle {\frac {2}{n}}\sum \limits _{i={\frac
{n}{2}}}^{n-1}i={\frac {3}{4}}n-{\frac {2}{4}}$ und die Tiefe der
Rekursion damit in  ${\mathcal {O}}(\log(n)).$ Im Average Case ist die
Anzahl der Vergleiche etwa $2\cdot \log(2)\cdot (n+1)\cdot \log
_{2}(n)\approx 1,39\cdot (n+1)\cdot \log _{2}(n)$.
\footcite{wiki:quicksort}
\end{bAntwort}

\item Die rekursiven Aufrufe in den Zeilen 16 und 17 des Algorithmus
werden zur Laufzeit des Computers auf dem Stack verwaltet. Die Anzahl
der Aufrufe von quicksort auf dem Stack abhängig von der Eingabegröße n
sei mit s(n) bezeichnet. Geben Sie die Komplexitätsklasse von s(n) für
den schlimmsten möglichen Fall an. Begründen Sie Ihre Antwort.
\end{enumerate}

\end{document}
