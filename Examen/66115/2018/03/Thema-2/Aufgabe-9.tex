\documentclass{bschlangaul-aufgabe}
\bLadePakete{graph}
\begin{document}
\bAufgabenMetadaten{
  Titel = {Aufgabe 9:},
  Thematik = {Negative Kantengewichte},
  Referenz = 66115-2018-F.T2-A9,
  RelativerPfad = Staatsexamen/66115/2018/03/Thema-2/Aufgabe-9.tex,
  ZitatSchluessel = examen:66115:2018:03,
  BearbeitungsStand = mit Lösung,
  Korrektheit = unbekannt,
  Ueberprueft = {unbekannt},
  Stichwoerter = {Algorithmus von Dijkstra},
  EinzelpruefungsNr = 66115,
  Jahr = 2018,
  Monat = 03,
  ThemaNr = 2,
  AufgabeNr = 9,
}

% delta oder dijkstra „schwarz“
\def\D#1#2#3{\textbf{(#1, #2, #3)}}
% „grau“
\def\d#1#2#3{(#1, #2, #3)}
% update
\def\u#1#2#3{(#1, \textcolor{blue}{\textbf{#2}}, \textcolor{blue}{\textbf{#3}})}
% select
\def\s#1{\textbf{[}#1\textbf{]}}

Gegeben sei folgender Graph $G$.
\index{Algorithmus von Dijkstra}
\footcite{examen:66115:2018:03}

\begin{bGraphenFormat}
s: -1 0
a: 6 0
b: 0 -1
c: 4 0
d: 1 2
e: 2 0
f: 6 2
g: 2 -2
h: -2 -2
a -- c: 4
a -- f: 9
b -- e: 6
b -- g: 4
b -- s: 3
c -- d: 7
c -- e: 5
c -- f: 1
d -- e: 3
d -- f: 4
d -- s: 8
e -- g: 4
e -- s: 13
g -- h: 3
h -- s: 5
\end{bGraphenFormat}

\begin{center}
\begin{tikzpicture}[li graph]
\node (a) at (6,0) {a};
\node (b) at (0,-1) {b};
\node (c) at (4,0) {c};
\node (d) at (1,2) {d};
\node (e) at (2,0) {e};
\node (f) at (6,2) {f};
\node (g) at (2,-2) {g};
\node (h) at (-2,-2) {h};
\node (s) at (-1,0) {s};

\path (a) edge node {4} (c);
\path (a) edge node {9} (f);
\path (b) edge node {6} (e);
\path (b) edge node {4} (g);
\path (b) edge node {3} (s);
\path (c) edge node {7} (d);
\path (c) edge node {5} (e);
\path (c) edge node {1} (f);
\path (d) edge node {3} (e);
\path (d) edge node {4} (f);
\path (d) edge node {8} (s);
\path (e) edge node {4} (g);
\path (e) edge node {13} (s);
\path (g) edge node {3} (h);
\path (h) edge node {5} (s);
\end{tikzpicture}

\end{center}

\begin{enumerate}

%%
% a)
%%

\item Berechnen Sie mithilfe des Algorithmus von Dijkstra die kürzesten
Wege vom Knoten $s$ zu allen anderen Knoten im Graphen $G$. Erstellen
Sie dazu eine Tabelle mit zwei Spalten und stellen Sie jeden einzelnen
Schritt des Verfahrens in einer eigenen Zeile dar. Geben Sie in der
ersten Spalte den jeweils als nächstes fertigzustellenden Knoten $v$
(wird sog. „schwarz“) als Tripel ($v$, $p$, $\delta$) mit $v$ als
Knotenname, $p$ als aktueller Vorgängerknoten und $\delta$ als aktuelle
Distanz von $s$ zu $v$ über $p$ an. Führen Sie in der zweiten Spalten
alle anderen bisher erreichten Knoten $v$ ebenfalls als Tripel ($v$,
$p$, $\delta$) auf, wobei diese sog. „grauen Randknoten“ in folgenden
Durchgängen erneut betrachtet werden müssen. Zeichnen Sie anschließend
den entstandenen Wegebaum, \dh den Graphen $G$, in dem nur noch
diejenigen Kanten vorkommen, die Teil der kürzesten Wege von $s$ zu
allen anderen Knoten sind.

\begin{bAntwort}
\begin{tabular}{l|l|l}
Nr & „schwarze“ Knoten & „graue“ Randknoten \\\hline
1 & \D s - 0 &
\scriptsize
% a
\s{\d b s 3}
% c
\d d s 8
\d e s {13}
% f
% g
\d h s 5
% \D s - 0
\\

2 & \D b s 3 &
\scriptsize
% a
% \D b s 3
% c
\d d s 8
\u e b 9
% f
\d g b 7
\s{\d h s 5}
% \D s - 0
\\

3 & \D h s 5 &
\scriptsize
% a
% \D b s 3
% c
\d d s 8
\d e b 9
% f
\s{\d g b 7}
% \D h s 5
% \D s - 0
\\

4 & \D g b 7 &
\scriptsize
% a
% \D b s 3
% c
\s{\d d s 8}
\d e b 9
% f
% \D g b 7
% \D h s 5
% \D s - 0
\\

5 & \D d s 8 &
\scriptsize
% a
% \D b s 3
\d c d {15}
% \D d s 8
\s{\d e b 9}
\d f d {12}
% \D g b 7
% \D h s 5
% \D s - 0
\\

6 & \D e b 9 &
\scriptsize
% a
% \D b s 3
\u c e {14}
% \D d s 8
% \D e b 9
\s{\d f d {12}}
% \D g b 7
% \D h s 5
% \D s - 0
\\

7 & \D f d {12} &
\scriptsize
\d a f {21}
% \D b s 3
\s{\u c f {13}}
% \D d s 8
% \D e b 9
% \D f d {12}
% \D g b 7
% \D h s 5
% \D s - 0
\\

8 & \D c f {13} &
\scriptsize
\s{\u a c {17}}
% \D b s 3
% \D c f {13}
% \D d s 8
% \D e b 9
% \D f d {12}
% \D g b 7
% \D h s 5
% \D s - 0
\\

9 & \D a c {17} &
\scriptsize
% \D a c {21}
% \D b s 3
% \D c f {13}
% \D d s 8
% \D e b 9
% \D f d {12}
% \D g b 7
% \D h s 5
% \D s - 0
\\

\end{tabular}

\begin{center}
\begin{tikzpicture}[li graph]
\node (a) at (6,0) {a};
\node (b) at (0,-1) {b};
\node (c) at (4,0) {c};
\node (d) at (1,2) {d};
\node (e) at (2,0) {e};
\node (f) at (6,2) {f};
\node (g) at (2,-2) {g};
\node (h) at (-2,-2) {h};
\node (s) at (-1,0) {s};

\path (a) edge node {4} (c);
% \path (a) edge node {9} (f);
\path (b) edge node {6} (e);
\path (b) edge node {4} (g);
\path (b) edge node {3} (s);
% \path (c) edge node {7} (d);
% \path (c) edge node {5} (e);
\path (c) edge node {1} (f);
% \path (d) edge node {3} (e);
\path (d) edge node {4} (f);
\path (d) edge node {8} (s);
% \path (e) edge node {4} (g);
% \path (e) edge node {13} (s);
% \path (g) edge node {3} (h);
\path (h) edge node {5} (s);
\end{tikzpicture}
\end{center}

\end{bAntwort}

\bPseudoUeberschrift{Alternativer Lösungsweg}

\begin{bAntwort}
\begin{tabular}{lllllllllll}
\bf{Nr.}     & \bf{besucht} & \bf{a}       & \bf{b}       & \bf{c}       & \bf{d}       & \bf{e}       & \bf{f}       & \bf{g}       & \bf{h}       & \bf{s}       \\
\hline
0            &              & $\infty$     & $\infty$     & $\infty$     & $\infty$     & $\infty$     & $\infty$     & $\infty$     & $\infty$     & 0            \\
1            & s            & $\infty$     & 3            & $\infty$     & 8            & 13           & $\infty$     & $\infty$     & 5            & \bf{0}       \\
2            & b            & $\infty$     & \bf{3}       & $\infty$     & 8            & 9            & $\infty$     & 7            & 5            & |            \\
3            & h            & $\infty$     & |            & $\infty$     & 8            & 9            & $\infty$     & 7            & \bf{5}       & |            \\
4            & g            & $\infty$     & |            & $\infty$     & 8            & 9            & $\infty$     & \bf{7}       & |            & |            \\
5            & d            & $\infty$     & |            & 15           & \bf{8}       & 9            & 12           & |            & |            & |            \\
6            & e            & $\infty$     & |            & 14           & |            & \bf{9}       & 12           & |            & |            & |            \\
7            & f            & 21           & |            & 13           & |            & |            & \bf{12}      & |            & |            & |            \\
8            & c            & 17           & |            & \bf{13}      & |            & |            & |            & |            & |            & |            \\
9            & a            & \bf{17}      & |            & |            & |            & |            & |            & |            & |            & |            \\
\end{tabular}

\begin{tabular}{llll}
\bf{nach}                                                         & \bf{Entfernung}                                                   & \bf{Reihenfolge}                                                  & \bf{Pfad}                                                         \\
\hline
s  $\rightarrow$  a                                               & 17                                                                & 9                                                                 & s $\rightarrow$ d $\rightarrow$ f $\rightarrow$ c $\rightarrow$ a \\
s  $\rightarrow$  b                                               & 3                                                                 & 2                                                                 & s $\rightarrow$ b                                                 \\
s  $\rightarrow$  c                                               & 13                                                                & 8                                                                 & s $\rightarrow$ d $\rightarrow$ f $\rightarrow$ c                 \\
s  $\rightarrow$  d                                               & 8                                                                 & 5                                                                 & s $\rightarrow$ d                                                 \\
s  $\rightarrow$  e                                               & 9                                                                 & 6                                                                 & s $\rightarrow$ b $\rightarrow$ e                                 \\
s  $\rightarrow$  f                                               & 12                                                                & 7                                                                 & s $\rightarrow$ d $\rightarrow$ f                                 \\
s  $\rightarrow$  g                                               & 7                                                                 & 4                                                                 & s $\rightarrow$ b $\rightarrow$ g                                 \\
s  $\rightarrow$  h                                               & 5                                                                 & 3                                                                 & s $\rightarrow$ h                                                 \\
s  $\rightarrow$  s                                               & 0                                                                 & 1                                                                 &                                                                   \\
\end{tabular}
\end{bAntwort}

%%
% b)
%%

\item Der Dijkstra-Algorithmus liefert bekanntlich auf Graphen mit
negativen Kantengewichten unter Umständen ein falsches Ergebnis.

\begin{enumerate}

%%
% i)
%%

\item Geben Sie einen Graphen mit negativen Kantengewichten an, sodass
der Dijkstra-Algorithmus ausgehend von einem von Ihnen ausgezeichneten
Startknoten ein korrektes Ergebnis liefert.

\begin{bAntwort}
Startknoten a

\begin{bGraphenFormat}
a: 0 0
b: 2 2
c: 4 0
a -- b: 10
b -- c: -9
a -- c: 2
\end{bGraphenFormat}
\begin{tikzpicture}[li graph]
\node (a) at (0,0) {a};
\node (b) at (2,2) {b};
\node (c) at (4,0) {c};

\path (a) edge node {10} (b);
\path (a) edge node {2} (c);
\path (b) edge node {-7} (c);
\end{tikzpicture}

\begin{tabular}{lllll}
\bf{Nr.}     & \bf{besucht} & \bf{a}       & \bf{b}       & \bf{c}       \\
\hline
0            &              & 0            & $\infty$     & $\infty$     \\
1            & a            & \bf{0}       & 10           & 2            \\
2            & c            & |            & 10           & \bf{2}       \\
3            & b            & |            & \bf{10}      & |            \\
\end{tabular}

Richtig: a - c: 2
\end{bAntwort}

%%
% ii)
%%

\item Geben Sie einen Graphen mit negativen Kantengewichten an, sodass
der Dijkstra-Algorithmus ausgehend von einem von Ihnen ausgezeichneten
Startknoten ein falsches Ergebnis liefert.

\begin{bAntwort}
Startknoten a

\begin{bGraphenFormat}
a: 0 0
b: 2 2
c: 4 0
a -- b: 10
b -- c: -9
a -- c: 2
\end{bGraphenFormat}
\begin{tikzpicture}[li graph]
\node (a) at (0,0) {a};
\node (b) at (2,2) {b};
\node (c) at (4,0) {c};

\path (a) edge node {10} (b);
\path (a) edge node {2} (c);
\path (b) edge node {-9} (c);
\end{tikzpicture}

\begin{tabular}{lllll}
\bf{Nr.}     & \bf{besucht} & \bf{a}       & \bf{b}       & \bf{c}       \\
\hline
0            &              & 0            & $\infty$     & $\infty$     \\
1            & a            & \bf{0}       & 10           & 2            \\
2            & c            & |            & 10           & \bf{2}       \\
3            & b            & |            & \bf{10}      & |            \\
\end{tabular}

falsch: a - c: müsste 1 (10 - 9) sein.
\end{bAntwort}
\end{enumerate}

Ein Beweis oder eine Begründung ist jeweils nicht erforderlich.

\end{enumerate}
\end{document}
