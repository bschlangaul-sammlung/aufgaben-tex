\documentclass{bschlangaul-aufgabe}
\begin{document}
\bAufgabenMetadaten{
  Titel = {Aufgabe 7},
  Thematik = {Quicksort},
  Referenz = 66115-2018-F.T2-A7,
  RelativerPfad = Staatsexamen/66115/2018/03/Thema-2/Aufgabe-7.tex,
  ZitatSchluessel = examen:66115:2018:03,
  BearbeitungsStand = mit Lösung,
  Korrektheit = unbekannt,
  Ueberprueft = {unbekannt},
  Stichwoerter = {Quicksort},
  EinzelpruefungsNr = 66115,
  Jahr = 2018,
  Monat = 03,
  ThemaNr = 2,
  AufgabeNr = 7,
}

\begin{enumerate}

%%
% a)
%%

\item Gegeben ist das folgende Array von Zahlen: $[23, 5, 4, 67, 30, 15, 25, 21]$.

Sortieren Sie das Array mittels Quicksort in-situ aufsteigend von links
nach rechts. Geben Sie die (Teil-)Arrays nach jeder Swap-Operation (auch
wenn Elemente mit sich selber getauscht werden) und am Anfang jedes
Aufrufs der rekursiven Methode an. Verwenden Sie als Pivotelement
jeweils das rechteste Element im Teilarray und markieren Sie dieses
entsprechend. Teilarrays der Länge $\leq 2$ dürfen im rekursiven Aufruf
durch direkten Vergleich sortiert werden. Geben Sie am Ende das
sortierte Array an.\index{Quicksort}
\footcite{examen:66115:2018:03}

%%
% b)
%%

\item Welche Worst-Case-Laufzeit (O-Notation) hat Quicksort für n
Elemente? Geben Sie ein Array mit fünf Elementen an, in welchem die
Quicksort-Variante aus (a) diese Wort-Case-Laufzeit benötigt (ohne
Begründung).

\end{enumerate}
\end{document}
