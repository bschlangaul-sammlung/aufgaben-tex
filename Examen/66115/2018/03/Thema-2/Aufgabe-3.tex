\documentclass{bschlangaul-aufgabe}
\bLadePakete{mathe,automaten}
\begin{document}
\bAufgabenMetadaten{
  Titel = {Aufgabe 3},
  Thematik = {Exponentieller Blow-Up},
  Referenz = 66115-2018-F.T2-A3,
  RelativerPfad = Examen/66115/2018/03/Thema-2/Aufgabe-3.tex,
  ZitatSchluessel = examen:66115:2018:03,
  BearbeitungsStand = mit Lösung,
  Korrektheit = unbekannt,
  Ueberprueft = {unbekannt},
  Stichwoerter = {Reguläre Sprache},
  EinzelpruefungsNr = 66115,
  Jahr = 2018,
  Monat = 03,
  ThemaNr = 2,
  AufgabeNr = 3,
}

Gesucht\index{Reguläre Sprache} \footcite{examen:66115:2018:03} ist eine
reguläre Sprache $C \subseteq \{a, b\}^*$, deren minimaler
deterministischer endlicher Automat (DEA) mindestens 4 Zustände mehr
besitzt als der minimale nichtdeterministische endliche Automat (NEA).
Gehen Sie wie folgt vor:
\footcite[Seite 23, Aufgabe 13]{theo:ab:1}

\begin{enumerate}

%%
% (a)
%%

\item Definieren Sie $C \subseteq \{a, b\}^*$ und erklären Sie kurz, warum es
bei dieser Sprache NEAs gibt, die deutlich kleiner als der minimale DEA
sind.

\begin{bAntwort}
Sprache mit exponentiellem Blow-Up:

Ein NEA der Sprache

\begin{align*}
L_k &= \{ xay \, | \, x, y \in \{a, b\}^* \land |y| = k - 1 \}\\
    &= \{ w \in \{a, b\}^* | \text{ der k-te Buchstabe von hinten ist ein } a\}\\
\end{align*}

kommt mit $k + 1$ Zuständen aus.

Jeder DEA $M$ mit $L(M) = L$ hat dann mindestens $2^k$ Zustände. Wir
wählen $k = 3$. Dann hat der zughörige NEA 4 Zustände und der zugehörige
DEA mindestens $8$. Sei also $L_k = \{ xay | x, y \in \{a, b\}^* \land
|y| = 2 \}$ die gesuchte Sprache.

Der informelle Grund, warum ein DEA für die Sprache $L_k$ groß sein muss, ist
dass er sich immer die letzten n Symbole merken muss.
\bFussnoteUrl{https://www.tcs.ifi.lmu.de/lehre/ss-2013/timi/handouts/handout-02}
\end{bAntwort}

%%
% (b)
%%

\item Geben Sie den minimalen DEA $M$ für $C$ an.
(Zeichnung des DEA genügt; die Minimalität muss nicht begründet werden.)

\begin{bAntwort}
\begin{center}
\begin{tikzpicture}[li automat]
  \node[state,initial] (z0) at (2.14cm,-2.14cm) {$z_0$};
  \node[state,accepting] (z1) at (2cm,-7cm) {$z_1$};
  \node[state,accepting] (z2) at (10.43cm,-2.14cm) {$z_2$};
  \node[state,accepting] (z3) at (8cm,-5cm) {$z_3$};
  \node[state] (z4) at (8cm,-2.14cm) {$z_4$};
  \node[state] (z5) at (5.14cm,-2.14cm) {$z_5$};
  \node[state,accepting] (z6) at (10.43cm,-7cm) {$z_6$};
  \node[state] (z7) at (5.14cm,-5cm) {$z_7$};

  \path (z0) edge[auto] node{$a$} (z5);
  \path (z0) edge[auto,loop above] node{$b$} (z0);
  \path (z1) edge[auto] node{$b$} (z0);
  \path (z1) edge[auto] node{$a$} (z5);
  \path (z2) edge[auto,loop above] node{$a$} (z2);
  \path (z2) edge[auto] node{$b$} (z6);
  \path (z3) edge[auto] node{$a$} (z4);
  \path (z3) edge[auto,bend left] node{$b$} (z7);
  \path (z4) edge[auto] node{$a$} (z2);
  \path (z4) edge[auto] node{$b$} (z6);
  \path (z5) edge[auto] node{$a$} (z4);
  \path (z5) edge[auto] node{$b$} (z7);
  \path (z6) edge[auto] node{$b$} (z1);
  \path (z6) edge[auto] node{$a$} (z3);
  \path (z7) edge[auto] node{$b$} (z1);
  \path (z7) edge[auto,bend left] node{$a$} (z3);
\end{tikzpicture}
\end{center}
\bFlaci{Ahhefpjir}
\end{bAntwort}

%%
% (c)
%%

\item Geben Sie einen NEA $N$ für $C$ an, der mindestens $4$ Zustände
weniger besitzt als $M$. (Zeichnung des NEA genügt)

\begin{bAntwort}
\begin{center}
\begin{tikzpicture}[->,node distance=2cm]
\node[state,initial] (0) {$z_0$};
\node[state,right of=0] (1) {$z_1$};
\node[state,right of=1] (2) {$z_2$};
\node[state,right of=2,accepting] (3) {$z_3$};

\path (0) edge[above] node{a} (1);
\path (1) edge[above] node{a,b} (2);
\path (2) edge[above] node{a,b} (3);
\path (0) edge[loop,above] node{a,b} (0);
\end{tikzpicture}
\end{center}
\bFlaci{Ajrz7h5r7}
\end{bAntwort}
\end{enumerate}
\end{document}
