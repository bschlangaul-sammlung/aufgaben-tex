\documentclass{bschlangaul-aufgabe}
\bLadePakete{java,mathe,master-theorem}
\begin{document}
\bAufgabenMetadaten{
  Titel = {Aufgabe 6},
  Thematik = {Methode „m()“},
  Referenz = 66115-2018-F.T2-A6,
  RelativerPfad = Staatsexamen/66115/2018/03/Thema-2/Aufgabe-6.tex,
  ZitatSchluessel = aud:fs:2,
  ZitatBeschreibung = {Seite 39},
  BearbeitungsStand = mit Lösung,
  Korrektheit = unbekannt,
  Ueberprueft = {unbekannt},
  Stichwoerter = {Master-Theorem},
  EinzelpruefungsNr = 66115,
  Jahr = 2018,
  Monat = 03,
  ThemaNr = 2,
  AufgabeNr = 6,
}

Der Hauptsatz der Laufzeitfunktionen ist bekanntlich folgendermaßen
definiert:\index{Master-Theorem}
\footcite[Seite 39]{aud:fs:2}
\footcite[Thema 2 Aufgabe 6]{examen:66115:2018:03}

\bMasterExkurs

\begin{enumerate}

\item Betrachten Sie die folgende Methode \bJavaCode{m} in Java, die
initial mit \bJavaCode{m(r, 0, r.length)} für das Array \bJavaCode{r}
aufgerufen wird. Geben Sie dazu eine Rekursionsgleichung $T(n)$ an,
welche die Anzahl an Rechenschritten von \bJavaCode{m} in Abhängigkeit
von der Länge \bJavaCode{n = r.length} berechnet.

\bJavaExamen[firstline=5,lastline=21]{66115}{2018}{03}{MasterTheorem}

\begin{bAntwort}
\bMasterVariablenDeklaration
{3} % a
{3} % b
{\mathcal{O}(1)} % f(n)
\end{bAntwort}

%%
% b)
%%

\item Ordnen Sie die rekursive Funktion $T(n)$ aus (a) einem der drei
Fälle des Mastertheorems zu und geben Sie die resultierende
Zeitkomplexität an. Zeigen Sie dabei, dass die Voraussetzung des Falles
erfüllt ist.

\begin{bAntwort}

\bMasterFallRechnung
% 1. Fall
{$f(n) \in \mathcal{O}\left(n^{\log_{3}3-\varepsilon}\right) =
\mathcal{O}\left(n^{1-\varepsilon}\right) =
\mathcal{O}\left(1\right) \text{ für } \varepsilon = 1
$}
% 2. Fall
{$f(n) \notin \Theta \left(n^{{\log_{3}3}}\right) =
\Theta \left(n^1\right)
$}
% 3. Fall
{$f(n) \notin \Omega \left(n^{\log_{3}3 + \varepsilon}\right) =
\Omega \left(n^{1 + \varepsilon}\right)$}

Also: $T(n)\in \Theta \left(n^{\log_{b}a}\right)$

\end{bAntwort}
\end{enumerate}

\end{document}
