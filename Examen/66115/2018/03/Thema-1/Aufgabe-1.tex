\documentclass{bschlangaul-aufgabe}
\bLadePakete{formale-sprachen,automaten,potenzmengen-konstruktion}
\begin{document}
\bAufgabenMetadaten{
  Titel = {Aufgabe 1},
  Thematik = {NEA},
  Referenz = 66115-2018-F.T1-A1,
  RelativerPfad = Examen/66115/2018/03/Thema-1/Aufgabe-1.tex,
  ZitatSchluessel = examen:66115:2018:03,
  BearbeitungsStand = mit Lösung,
  Korrektheit = unbekannt,
  Ueberprueft = {unbekannt},
  Stichwoerter = {Formale Sprachen},
  EinzelpruefungsNr = 66115,
  Jahr = 2018,
  Monat = 03,
  ThemaNr = 1,
  AufgabeNr = 1,
}

\let\m=\bMengeOhneMathe
\let\M=\bMenge

Gegeben ist der nichtdeterministische endliche Automat (NEA) $(\m{0,1},
Q,\delta ,q_0, F)$, wobei $Q = \m{A, B,C}$, $q_0 = A$, $F = \m{C}$ und
\index{Formale Sprachen}
\footcite{examen:66115:2018:03}

\begin{center}
\begin{tabular}{l||ll}
$\delta$ & 0 & 1 \\\hline
$A$ & \M{A, B} & \M{A, C} \\
$B$ & \M{B, C} & \M{B} \\
$C$ & \M{C} & \M{C} \\
\end{tabular}
\end{center}

\begin{bAntwort}
\begin{center}
\begin{tikzpicture}[->,node distance=4cm]
\node[state,initial]
(a) {$A$};

\node[state,above right of=a]
(b) {$B$};

\node[state,below right of=b,accepting]
(c) {$C$};

\path (a) edge[above] node{0} (b);
\path (a) edge[above,loop] node{0,1} (a);
\path (a) edge[above] node{1} (c);

\path (b) edge[above,loop] node{0,1} (b);
\path (b) edge[above] node{0} (c);

\path (c) edge[above,loop] node{0,1} (c);
\end{tikzpicture}
\bFlaci{Aiq3xxgi9}
\end{center}
\end{bAntwort}

\begin{enumerate}

%%
% a)
%%

\item Führen Sie für diesen NEA die Potenzmengenkonstruktion durch;
geben Sie alle acht entstehenden Zustände mit ihren Transitionen an,
nicht nur die erreichbaren.

\begin{bAntwort}
\def\z#1{
  \bZustandsMengenSammlung{#1}{
    {
      {0} {A}
      {1} {B}
      {2} {C}
      {3} {A,B}
      {4} {A,C}
      {5} {B,C}
      {6} {A,B,C}
      {7} {}
    }
  }
}
\let\s=\bZustandsnameGross

\def\g#1{\textcolor{gray}{#1}}

\g{nicht erreichbar}

\begin{tabular}{l|l|l|l}
Name & Zustandsmenge & Eingabe 0 & Eingabe 1 \\\hline\hline
\s{0} & \z{0} & \z{3} & \z{4} \\
\g{\s{1}} & \g{\z{1}} & \g{\z{5}} & \g{\z{1}} \\
\g{\s{2} }& \g{\z{2}} & \g{\z{2}} & \g{\z{2}} \\
\s{3} & \z{3} & \z{6} & \z{6} \\
\s{4} & \z{4} & \z{6} & \z{4} \\
\g{\s{5}} & \g{\z{5}} & \g{\z{5}} & \g{\z{5}} \\
\s{6} & \z{6} & \z{6} & \z{6} \\
\g{\s{7}} & \g{\z{7}} & \g{\z{7}} & \g{\z{7}} \\
\end{tabular}

\begin{center}
\begin{tikzpicture}[->,node distance=3cm]
\node[state,initial]
(0) {\s{0}};

\node[state,right of=0]
(3) {\s{3}};

\node[state,below of=0,accepting]
(4) {\s{4}};

\node[state,right of=4,accepting]
(6) {\s{6}};

\path (0) edge[above] node{0} (3);
\path (0) edge[left] node{1} (4);
\path (3) edge[right] node{0,1} (6);
\path (4) edge[above] node{0} (6);
\path (4) edge[above,loop left] node{1} (4);
\path (6) edge[above,loop below] node{0,1} (6);
\end{tikzpicture}
\bFlaci{Aigwcbsf7}
\end{center}
\end{bAntwort}

%%
% b)
%%

% \item Es sei L eine beliebige reguläre Sprache über dem Alphabet © =
% {a,b,c}. Die Sprache L’ über dem Alphabet W’ = {a,b} umfasst alle
% Wörter, die aus Wörtern in L durch Streichen aller cs entstehen. Ist zum
% Beispiel L = {acb, acbcc, abbaccc}, so ist L’ = {ab, abba}.

% Zeigen Sie, dass L’ regular ist.

%%
% c)
%%

% \item Die Sprache L, = {a"b" | n > 1} ist bekanntlich kontextfrei aber
% nicht regulär. Obwohl die

% kontextfreien Sprachen nicht unter Komplement abgeschlossen sind, ist Lı
% kontextfrei. Die Sprache Dz = {a"b"c” | n > 1} ist bekanntlich nicht
% kontextfrei.

% Geben Sie eine kontextfreie Grammatik für L, das Komplement von Ly, an.

%%
% d)
%%

% \item Die Sprache Ly = {a"b"c” | n > 1} kann bekanntlich als Schnitt
% zweier kontextfreier Sprachen dargestellt werden: Lg = Lic* Na*L wobei
% L = {b"c” | n > 1}. Zeigen Sie, dass die Komplemente von L,c* und at
% L{ kontextfrei sind.

%%
% e)
%%

% \item Leiten Sie daraus einen Beweis her, dass die kontextfreien
% Sprachen nicht unter Komplement abgeschlossen sind.
\end{enumerate}
\end{document}
