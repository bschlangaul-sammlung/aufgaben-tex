\documentclass{bschlangaul-aufgabe}
\bLadePakete{formale-sprachen}
\begin{document}
\bAufgabenMetadaten{
  Titel = {Verständnis formale Sprachen},
  Thematik = {Verständnis},
  Referenz = 66115-2016-F.T2-A1,
  RelativerPfad = Staatsexamen/66115/2016/03/Thema-2/Aufgabe-1.tex,
  ZitatSchluessel = examen:66115:2016:03,
  BearbeitungsStand = mit Lösung,
  Korrektheit = unbekannt,
  Ueberprueft = {unbekannt},
  Stichwoerter = {Formale Sprachen},
  EinzelpruefungsNr = 66115,
  Jahr = 2016,
  Monat = 03,
  ThemaNr = 2,
  AufgabeNr = 1,
}

% Info_2021-06-11-2021-06-11_13.34.15.mp4

Beantworten Sie kurz, präzise und mit Begründung folgende Fragen: (Die
Begründungen müssen keine formellen mathematischen Beweise sein).
\index{Formale Sprachen}
\footcite{examen:66115:2016:03}

\begin{enumerate}

%%
% a)
%%

\item Welche Möglichkeiten gibt es, eine formale Sprache vom Typ 3 zu
definieren?

\begin{bAntwort}
\begin{itemize}
\item reguläre Grammatik
\item endlicher nichtdeterministische und deterministischer Automat
\item regulärer Ausdruck
\end{itemize}
\end{bAntwort}

%%
% b)
%%

% 12 min

\item Was ist die Komplexität des Wortproblems für Typ-3 Sprachen und
wieso ist das so?\footcite[Aufgabe 5a)]{theo:ab:5}

\begin{bAntwort}
$P$, CYK-Algorithmus löst es ein Polynomialzeit.
\end{bAntwort}

%%
% c)
%%

% 19 min

\item Sind Syntaxbäume zu einer Grammatik immer eindeutig? Falls nicht,
geben Sie ein Gegenbeispiel.

\begin{bAntwort}
Nein. Syntaxbäume zu einer Grammtik sind nicht immer eindeutig.

\bPseudoUeberschrift{Gegenbeispiel}

\bGrammatik{
  variablen={S, A, B},
  alphabet={a}
}

\begin{bProduktionsRegeln}
S -> A A,
S -> B B,
A -> a,
B -> a,
\end{bProduktionsRegeln}

\bAbleitung{S -> A A -> a A -> a a}

\bAbleitung{S -> B B -> a B -> a a}\footcite{wiki:mehrdeutig}
\end{bAntwort}

%%
% d)
%%

\item Wie kann man die Äquivalenz zweier Typ-3 Sprachen nachweisen?

% 23 min

\begin{bAntwort}
Myhill-Nerode-Äquivalenz. Wir können von den auf Äquivalenz zu
überprüfenden Sprachen jeweils einen minimalen endlichen Automaten
bilden. Sind diese entstanden zwei Automaten äquivalent so sind auch die
Sprachen äquivalent.
\footcite{wiki:aequivalenzproblem}
\end{bAntwort}

%%
% e)
%%

\item Wie kann man das Wortproblem für das Komplement einer Typ-3
Sprache lösen?

% 25 min

\begin{bAntwort}
Da das Komplement einer regulären Sprache wieder eine reguläre Sprache
ergibt, kann das Wortproblem beim Komplement durch einen deterministisch
endlichen Automaten gelöst werden. Tausche akzeptierende mit nicht
akzeptierenden Zuständen des zugehörigen Automaten.

Alternativ: Ergebnis des CYK-Algorithmus invertieren oder CYK auf
Komplement, da reguläre Sprachen unter dem Komplement abgeschlossen
sind.
\end{bAntwort}

%%
% f)
%%

\item Weshalb gilt das Pumping-Lemma für Typ 3 Sprachen?

\begin{bAntwort}
endliche Anzahl von Zuständen $n$ im Automaten $\rightarrow$ für Wörter
$|\omega| > n$ muss Zyklus vorhanden sein.
\end{bAntwort}

%%
% g)
%%

\item Ist der Nachweis, dass das Typ-3 Pumping-Lemma für eine gegebene
Sprache gilt, ausreichend, um zu zeigen, dass die Sprache vom Typ 3 ist?
Falls nicht, geben Sie ein Gegenbeispiel, mit Begründung.

\begin{bAntwort}
Nein:

Die Sprache $L = \left\{a^mb^nc^n \mid m,n \ge 1\right\} \cup
\{b^mc^n \mid m,n \ge 0 \}$ ist nicht regulär. Allerdings erfüllt
$L$ die Eigenschaften des Pumping-Lemmas, denn jedes Wort
$z \in L$ lässt sich so zerlegen $z = uvw$, dass
auch für alle $i\ge0$ $uv^iw \in L$. Dazu kann
$v$ einfach als erster Buchstabe gewählt werden. Dieser ist
entweder ein $a$, die Anzahl von führenden $a$s
ist beliebig. Oder er ist ein $b$ oder $c$, ohne
führende $a$s ist aber die Anzahl von führenden
$b$s oder $c$s beliebig.
\bFussnoteUrl{https://de.wikipedia.org/wiki/Pumping-Lemma}
\end{bAntwort}

%%
% h)
%%

% 37 min

\item Geben Sie ein Beispiel, an dem deutlich wird, dass
deterministische und nichtdeterministische Typ-2 Sprachen
unterschiedlich sind.

\begin{bAntwort}
\begin{description}
\item[Deterministisch Kontextfrei]
\bAusdruck{0^n 1^n}{n \geq 0}

\item[Nichtdeterministisch Kontextfrei]
\bAusdruck{\omega \omega^R}{\omega \geq \{0, 1\}^*} (R steht für rückwärts)
\end{description}
\bFussnoteUrl{https://docplayer.org/19566652-Einfuehrung-in-die-theoretische-informatik.html}
\end{bAntwort}

%%
% i)
%%

% 39min

\item Worin macht sich der Unterschied zwischen Typ 0 und 1 bemerkbar,
wenn man Turingmaschinen benutzt, um das Wortproblem vom Typ 0 oder 1 zu
lösen. Warum ist das so?
\footcite[Aufgabe 5b)]{theo:ab:5}

\begin{bAntwort}
Typ 0: semi-entscheidbar, Typ 1: entscheidbar

Typ 0: Unendlichkeit des Band kann die unendlich lange Berechenbarkeit
zustande kommen.

Typ 1: Linear beschränkte Turingmaschine endlich, dadurch Anzahl an
Kombiniation

Typ 1 Sprachen sind monoton wachsend.

Da Typ 1 nur Wörter verlängert, kann daher in Polynomialzeit überprüft
werden, ob das Wort in der Sprache liegt, indem die Regeln angewendet
werden, bis das Wortende erreicht ist.
\end{bAntwort}

\end{enumerate}
\end{document}
