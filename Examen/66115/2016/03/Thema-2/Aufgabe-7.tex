\documentclass{bschlangaul-aufgabe}
\bLadePakete{mathe}
\begin{document}
\bAufgabenMetadaten{
  Titel = {Verständnis Suchbäume},
  Thematik = {Vergleich Suchbäume},
  Referenz = 66115-2016-F.T2-A7,
  RelativerPfad = Staatsexamen/66115/2016/03/Thema-2/Aufgabe-7.tex,
  ZitatSchluessel = examen:66115:2016:03,
  ZitatBeschreibung = {Seite 9},
  BearbeitungsStand = mit Lösung,
  Korrektheit = unbekannt,
  Ueberprueft = {unbekannt},
  Stichwoerter = {Bäume, Rot-Schwarz-Baum, AVL-Baum, Halde (Heap), B-Baum, R-Baum},
  EinzelpruefungsNr = 66115,
  Jahr = 2016,
  Monat = 03,
  ThemaNr = 2,
  AufgabeNr = 7,
}

Wofür eignen sich die folgenden Baum-Datenstrukturen im Vergleich zu den
anderen angeführten Baumstrukturen am besten, und warum. Sprechen Sie
auch die Komplexität der wesentlichen Operationen und die Art der
Speicherung an.\index{Bäume}
\footcite[Seite 9]{examen:66115:2016:03}

\begin{enumerate}

%%
% a)
%%

\item Rot-Schwarz-Baum\index{Rot-Schwarz-Baum}

\begin{bAntwort}
\begin{description}
\item[Einfügen (Zeitkomplexität)] \strut \\
$\mathcal{O}(\log n)$ (im Durchschnitt) \\
$\mathcal{O}(\log n)$ (im schlechtesten Fall)

\item[Löschen (Zeitkomplexität)] \strut \\
$\mathcal{O}(\log n)$ (im Durchschnitt) \\
$\mathcal{O}(\log n)$ (im schlechtesten Fall)

\item[Suchen (Zeitkomplexität)] \strut \\
$\mathcal{O}(\log n)$ (im Durchschnitt) \\
$\mathcal{O}(\log n)$ (im schlechtesten Fall)
\bFussnoteLink{tutorialspoint.com}{https://www.tutorialspoint.com/comparison-of-search-trees-in-data-structure}
\end{description}
\end{bAntwort}

%%
% b)
%%

\item AVL-Baum\index{AVL-Baum}

\begin{bAntwort}
\begin{description}
\item[Einfügen (Zeitkomplexität)] \strut \\
$\mathcal{O}(\log_2 n)$ (im Durchschnitt) \\
$\mathcal{O}(\log_2 n)$ (im schlechtesten Fall)

\item[Löschen (Zeitkomplexität)] \strut \\
$\mathcal{O}(\log_2 n)$ (im Durchschnitt) \\
$\mathcal{O}(\log_2 n)$ (im schlechtesten Fall)

\item[Suchen (Zeitkomplexität)] \strut \\
$\mathcal{O}(\log_2 n)$ (im Durchschnitt) \\
$\mathcal{O}(\log_2 n)$ (im schlechtesten Fall)
\bFussnoteLink{tutorialspoint.com}{https://www.tutorialspoint.com/comparison-of-search-trees-in-data-structure}
\end{description}
\end{bAntwort}

%%
% c)
%%

\item Binärer-Heap\index{Halde (Heap)}

\begin{bAntwort}
\begin{description}
\item[Verwendungszweck] zum effizienten Sortieren von Elementen.
\bFussnoteLink{deut. Wikipedia}{https://de.wikipedia.org/wiki/Binärer_Heap}

\item[Einfügen (Zeitkomplexität)] \strut \\
$\mathcal{O}(1)$ (im Durchschnitt) \\
$\mathcal{O}(\log n)$ (im schlechtesten Fall)

\item[Löschen (Zeitkomplexität)] \strut \\
$\mathcal{O}(\log n)$ (im Durchschnitt) \\
$\mathcal{O}(\log n)$ (im schlechtesten Fall)

\item[Suchen (Zeitkomplexität)] \strut \\
$\mathcal{O}(n)$ (im Durchschnitt) \\
$\mathcal{O}(n)$ (im schlechtesten Fall)
\bFussnoteLink{engl. Wikipedia}{https://en.wikipedia.org/wiki/Binary_heap}
\end{description}
\end{bAntwort}

%%
% d)
%%

\item B-Baum\index{B-Baum}

\begin{bAntwort}
\begin{description}
\item[Einfügen (Zeitkomplexität)] \strut \\
$\mathcal{O}(\log n)$ (im Durchschnitt) \\
$\mathcal{O}(\log n)$ (im schlechtesten Fall)

\item[Löschen (Zeitkomplexität)] \strut \\
$\mathcal{O}(\log n)$ (im Durchschnitt) \\
$\mathcal{O}(\log n)$ (im schlechtesten Fall)

\item[Suchen (Zeitkomplexität)] \strut \\
$\mathcal{O}(\log n)$ (im Durchschnitt) \\
$\mathcal{O}(\log n)$ (im schlechtesten Fall)
\bFussnoteLink{tutorialspoint.com}{https://www.tutorialspoint.com/comparison-of-search-trees-in-data-structure}
\end{description}
\end{bAntwort}

%%
% e)
%%

\item R-Baum\index{R-Baum}

\begin{bAntwort}
\begin{description}
\item[Verwendungszweck] Ein R-Baum erlaubt die schnelle Suche in
mehrdimensionalen ausgedehnten Objekten.
\bFussnoteLink{deut. Wikipedia}{https://de.wikipedia.org/wiki/R-Baum}

\item[Suchen (Zeitkomplexität)] \strut \\
$\mathcal{O}(\log_M n)$ (im Durchschnitt)
\bFussnoteLink{eng. Wikipedia}{https://en.wikipedia.org/wiki/R-tree}\\
$\mathcal{O}(n)$ (im schlechtesten Fall)
\bFussnoteLink{Simon Fraser University, Burnaby, Kanada}{https://www2.cs.sfu.ca/CourseCentral/454/jpei/slides/R-Tree.pdf}
\end{description}
\end{bAntwort}
\end{enumerate}

\end{document}
