\documentclass{bschlangaul-aufgabe}
\bLadePakete{mathe}
\begin{document}
\bAufgabenMetadaten{
  Titel = {Verständnis Komplexitätstheorie},
  Thematik = {Verständnis},
  Referenz = 66115-2016-F.T2-A3,
  RelativerPfad = Examen/66115/2016/03/Thema-2/Aufgabe-3.tex,
  ZitatSchluessel = theo:ab:4,
  ZitatBeschreibung = {StEx F2016 T2 A3, StEx H2017 T1 A3, Aufgabe 15},
  BearbeitungsStand = mit Lösung,
  Korrektheit = unbekannt,
  Ueberprueft = {unbekannt},
  Stichwoerter = {Komplexitätstheorie},
  EinzelpruefungsNr = 66115,
  Jahr = 2016,
  Monat = 03,
  ThemaNr = 2,
  AufgabeNr = 3,
}

% Check-Up
% i o theo ab 4
% i e 66115:2016:03 2 3
% i e 66115:2017:09 1 3

Beantworten Sie kurz, präzise und mit Begründung folgende Fragen: (Die
Begründungen müssen keine formellen mathematischen Beweise sein)
\footcite[StEx F2016 T2 A3, StEx H2017 T1 A3, Aufgabe 15]{theo:ab:4}
\index{Komplexitätstheorie}
\footcite{examen:66115:2016:03}

\begin{enumerate}

%%
% a)
%%

% 1:08

\item In der O-Notation insbesondere für die Zeitkomplexität von
Algorithmen lässt man i.\,A. konstante Faktoren oder kleinere Terme weg.
\ZB schreibt man anstelle $\mathcal{O}(3n^2 + 5)$ einfach nur
$\mathcal{O}(n^2)$. Warum macht man das so?

\begin{bAntwort}
Das Wachstum im Unendlich ist bestimmt durch den größten Exponenten.
Konstante falle bei einer asymptotischen. Analyse weg. nicht wesentlich
schneller
\end{bAntwort}

%%
% b)
%%

% 1h11min

\item Was ist die typische Vorgehensweise, wenn man für ein neues
Problem die NP-Vollständigkeit untersuchen will?

\begin{bAntwort}
Die alten Probleme werden reduziert. Das neue Problem ist größer als die
alten Probleme. Das Problem muss in NP liegen.

\begin{enumerate}
\item Problem $in$ NP durch Angabe eines nichtdeterministischen
Algorithmus in Polynomialzeit

\item Problem NP-schwer via Reduktion: $L_{\text{NP-vollständig}} \leq
L_{\text{neues Problem}}$
\end{enumerate}
\end{bAntwort}

%%
% c)
%%

% 1h14min

\item Was könnte man tun, um $P = NP$ zu beweisen?

\begin{bAntwort}
Es würde genügen, zu einem einzigen NP-Problem beweisen, dass es in P
liegt.

Zu einem Problem einen deterministen Turingmaschin finden, die es
in polynomineller Zeit löst.
\end{bAntwort}

%%
% d)
%%

% 1h17min

\item Sind NP-vollständige Problem mit Loop-Programmen lösbar? (Antwort
mit Begründung!)

\begin{bAntwort}
nicht lösbar mit Loop-Programmen. Begründung ähnlich wie bei der
Ackermann-Funktion. z. B. Passwort. Zu Passwort beliebig bräuchte man
beliebige for schleifen, was dem endlichen Anzahl an Loop-Schleifen
widerspricht.
\end{bAntwort}

%%
% e)
%%

% 1h26min

\item Wie zeigt man aus der NP-Härte des SAT-Problems die NP-Härte des
3SAT-Problems? (3SAT ist ein SAT-Problem wobei alle Klauseln maximal 3
Literale haben.)

\begin{bAntwort}
in den Lösungen enthalten
\end{bAntwort}

\end{enumerate}
\end{document}
