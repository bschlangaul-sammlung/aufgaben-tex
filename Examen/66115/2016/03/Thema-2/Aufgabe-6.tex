\documentclass{bschlangaul-aufgabe}
\bLadePakete{mathe,graph}
\begin{document}
\bAufgabenMetadaten{
  Titel = {Dijkstra Algorithmus},
  Thematik = {Karlsruhe nach Kassel},
  Referenz = 66115-2016-F.T2-A6,
  RelativerPfad = Staatsexamen/66115/2016/03/Thema-2/Aufgabe-6.tex,
  ZitatSchluessel = examen:66115:2016:03,
  ZitatBeschreibung = {Thema 2 Aufgabe 6 Seite 9},
  BearbeitungsStand = mit Lösung,
  Korrektheit = unbekannt,
  Ueberprueft = {unbekannt},
  Stichwoerter = {Algorithmus von Dijkstra},
  EinzelpruefungsNr = 66115,
  Jahr = 2016,
  Monat = 03,
  ThemaNr = 2,
  AufgabeNr = 6,
}

\begin{enumerate}

%%
% a)
%%

\item Berechnen Sie für folgenden Graphen den kürzesten Weg von
Karlsruhe nach Kassel und dokumentieren Sie den Berechnungsweg:
\index{Algorithmus von Dijkstra}
\footcite[Thema 2 Aufgabe 6 Seite 9]{examen:66115:2016:03}

\begin{bGraphenFormat}
A: 3 1
E: 2 3
F: 4 7
KA: 0 2
KS: 9 6
M: 6 1
MA: 0 6
N: 6 3
S: 7 5
W: 4 5
A -- KA: 250
A -- M: 84
E -- W: 186
F -- KS: 173
F -- MA: 85
F -- W: 217
M -- KS: 502
M -- N: 167
MA -- KA: 80
N -- S: 183
N -- W: 103
\end{bGraphenFormat}

\bPseudoUeberschrift{Verwendete Abkürzungen:}

\begin{description}
\item[A] Augsburg
\item[EF] Erfurt
\item[F] Frankfurt
\item[KA] Karlsruhe
\item[KS] Kassel
\item[M] München
\item[MA] Mannheim
\item[N] Nürnberg
\item[S] Stuttgart
\item[WÜ] Würzburg
\end{description}

Zahl = Zahl in Kilometern

\begin{center}
\begin{tikzpicture}[li graph]
\node (A) at (3,1) {A};
\node (E) at (2,3) {E};
\node (F) at (4,7) {F};
\node (KA) at (0,2) {KA};
\node (KS) at (9,6) {KS};
\node (M) at (6,1) {M};
\node (MA) at (0,6) {MA};
\node (N) at (6,3) {N};
\node (S) at (7,5) {S};
\node (W) at (4,5) {W};

\path (A) edge node {250} (KA);
\path (A) edge node {84} (M);
\path (E) edge node {186} (W);
\path (F) edge node {173} (KS);
\path (F) edge node {85} (MA);
\path (F) edge node {217} (W);
\path (M) edge node {502} (KS);
\path (M) edge node {167} (N);
\path (MA) edge node {80} (KA);
\path (N) edge node {183} (S);
\path (N) edge node {103} (W);
\end{tikzpicture}
\end{center}

\begin{bAntwort}
\begin{tabular}{llllllllllll}
\bf{Nr.}     & \bf{besucht} & \bf{A}       & \bf{E}       & \bf{F}       & \bf{KA}      & \bf{KS}      & \bf{M}       & \bf{MA}      & \bf{N}       & \bf{S}       & \bf{W}       \\
\hline
0            &              & $\infty$     & $\infty$     & $\infty$     & 0            & $\infty$     & $\infty$     & $\infty$     & $\infty$     & $\infty$     & $\infty$     \\
1            & KA           & \bf{250}     & $\infty$     & $\infty$     & \bf{0}       & $\infty$     & $\infty$     & 80           & $\infty$     & $\infty$     & $\infty$     \\
2            & MA           & |            & $\infty$     & 165          & |            & $\infty$     & $\infty$     & \bf{80}      & $\infty$     & $\infty$     & $\infty$     \\
3            & F            & |            & $\infty$     & \bf{165}     & |            & 338          & $\infty$     & |            & $\infty$     & $\infty$     & 382          \\
4            & A            & |            & $\infty$     & |            & |            & 338          & 334          & |            & $\infty$     & $\infty$     & 382          \\
5            & M            & |            & $\infty$     & |            & |            & 338          & \bf{334}     & |            & 501          & $\infty$     & 382          \\
6            & KS           & |            & $\infty$     & |            & |            & \bf{338}     & |            & |            & 501          & $\infty$     & 382          \\
7            & W            & |            & 568          & |            & |            & |            & |            & |            & 485          & $\infty$     & \bf{382}     \\
8            & N            & |            & 568          & |            & |            & |            & |            & |            & \bf{485}     & 668          & |            \\
9            & E            & |            & \bf{568}     & |            & |            & |            & |            & |            & |            & 668          & |            \\
10           & S            & |            & |            & |            & |            & |            & |            & |            & |            & \bf{668}     & |            \\
\end{tabular}

\noindent
\begin{tabular}{llll}
\bf{nach}                                                                           & \bf{Entfernung}                                                                     & \bf{Reihenfolge}                                                                    & \bf{Pfad}                                                                           \\
\hline
KA  $\rightarrow$  A                                                                & 250                                                                                 & 0                                                                                   & KA $\rightarrow$ A                                                                  \\
KA  $\rightarrow$  E                                                                & 568                                                                                 & 9                                                                                   & KA $\rightarrow$ MA $\rightarrow$ F $\rightarrow$ W $\rightarrow$ E                 \\
KA  $\rightarrow$  F                                                                & 165                                                                                 & 3                                                                                   & KA $\rightarrow$ MA $\rightarrow$ F                                                 \\
KA  $\rightarrow$  KA                                                               & 0                                                                                   & 1                                                                                   &                                                                                     \\
KA  $\rightarrow$  KS                                                               & 338                                                                                 & 6                                                                                   & KA $\rightarrow$ MA $\rightarrow$ F $\rightarrow$ KS                                \\
KA  $\rightarrow$  M                                                                & 334                                                                                 & 5                                                                                   & KA $\rightarrow$ A $\rightarrow$ M                                                  \\
KA  $\rightarrow$  MA                                                               & 80                                                                                  & 2                                                                                   & KA $\rightarrow$ MA                                                                 \\
KA  $\rightarrow$  N                                                                & 485                                                                                 & 8                                                                                   & KA $\rightarrow$ MA $\rightarrow$ F $\rightarrow$ W $\rightarrow$ N                 \\
KA  $\rightarrow$  S                                                                & 668                                                                                 & 10                                                                                  & KA $\rightarrow$ MA $\rightarrow$ F $\rightarrow$ W $\rightarrow$ N $\rightarrow$ S \\
KA  $\rightarrow$  W                                                                & 382                                                                                 & 7                                                                                   & KA $\rightarrow$ MA $\rightarrow$ F $\rightarrow$ W                                 \\
\end{tabular}

\end{bAntwort}

%%
% b)
%%

\item Könnte man den Dijkstra Algorithmus auch benutzen, um das
Travelling-Salesman Problem zu lösen?

\end{enumerate}

\end{document}
