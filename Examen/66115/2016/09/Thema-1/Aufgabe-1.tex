\documentclass{bschlangaul-aufgabe}
\bLadePakete{formale-sprachen,automaten}
\begin{document}
\bAufgabenMetadaten{
  Titel = {Aufgabe},
  Thematik = {Endlicher Automat, Zustände A-E, Alphabet 0-1},
  Referenz = 66115-2016-H.T1-A1,
  RelativerPfad = Examen/66115/2016/09/Thema-1/Aufgabe-1.tex,
  ZitatSchluessel = examen:66115:2016:09,
  BearbeitungsStand = nur Angabe,
  Korrektheit = unbekannt,
  Ueberprueft = {unbekannt},
  Stichwoerter = {Reguläre Sprache},
  EinzelpruefungsNr = 66115,
  Jahr = 2016,
  Monat = 09,
  ThemaNr = 1,
  AufgabeNr = 1,
}

Gegeben ist der deterministische endliche Automat
\bAutomat{zustaende={A,B,C,D,E},alphabet={0,1},start=A,ende=E}, wobei
\index{Reguläre Sprache}
\footcite{examen:66115:2016:09}

\begin{enumerate}
\item Minimieren Sie den Automaten mit dem bekannten
Minimierungsalgorithmus. Dokumentieren Sie die Schritte geeignet.

\item Geben Sie einen regulären Ausdruck für die erkannte Sprache an.

\item Geben Sie die Äquivalenzklassen der Myhill-Nerode-Äquivalenz der
Sprache durch reguläre Ausdrücke an.

\item Geben Sie ein Beispiel einer regulären Sprache an, für die kein
deterministischer endlicher Automat mit höchstens zwei Endzuständen
existiert.

\item Geben Sie ein Beispiel einer regulären Sprache an, für die kein
deterministischer endlicher Automat mit weniger als fünf Zuständen
existiert.
\end{enumerate}

\end{document}
