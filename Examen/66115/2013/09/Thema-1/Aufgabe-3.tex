\documentclass{bschlangaul-aufgabe}
\bLadePakete{mathe,automaten}
\begin{document}
\bAufgabenMetadaten{
  Titel = {Aufgabe},
  Thematik = {DOPP},
  Referenz = 66115-2013-H.T1-A3,
  RelativerPfad = Examen/66115/2013/09/Thema-1/Aufgabe-3.tex,
  ZitatSchluessel = examen:66115:2013:09,
  BearbeitungsStand = nur Angabe,
  Korrektheit = unbekannt,
  Ueberprueft = {unbekannt},
  Stichwoerter = {Entscheidbarkeit},
  EinzelpruefungsNr = 66115,
  Jahr = 2013,
  Monat = 09,
  ThemaNr = 1,
  AufgabeNr = 3,
}

Wir\index{Entscheidbarkeit}
\footcite{examen:66115:2013:09} betrachten das wie folgt definierte Problem DOPP:
\footcite[Aufgabe 8]{theo:ab:4}

\begin{description}
\item[GEGEBEN:]

Eine deterministische Turingmaschine $M$, eine Eingabe $x$ (für $M$),
ein Zustand $q$ (von $M$).

\item[GEFRAGT:]

Wird der Zustand $q$ bei der Berechnung von $M$ auf $x$ mindestens
zweimal besucht?
\end{description}

\begin{enumerate}

%%
% 1.
%%

\item Zeigen Sie durch Angabe einer Reduktion vom Halteproblem, dass
DOPP unentscheidbar.

%%
% 2.
%%
\item Begründen Sie, dass DOPP rekursiv aufzählbar (semi-entscheidbar)
ist.

\end{enumerate}

Die Reduktion $f \colon \text{HALTE} \leq_p \text{DOPP}$, bildet $c(M),
w$ auf $c(M'), xw, q$ ab, wobei $M'$ eine Turingmaschine mit folgendem
Verhalten ist:

\begin{itemize}
\item Sie verfügt über den neuen Startzustand $q$

\item Sie erweitert das Wort $w$ vorne um ein Zeichen $x \notin
\Gamma_M$ (dies ist nicht unbedingt notwendig, aber schöner, damit die
neue Maschine auch noch terminiert)

\item Für $q$ wird die Regel $(q, x) \rightarrow (z_0,
\bTuringLeerzeichen, R)$, wobei $z_0$ der Startzustand von $M$ ist

\item Alle Endzustände $z$ von $M$ erhalten eine neue Regel $(z,
\bTuringLeerzeichen) \rightarrow (q, \bTuringLeerzeichen, N)$
\end{itemize}

Hierbei wird davon ausgegangen, dass das Halteproblem eine
Turingmaschine mit Endzuständen als Eingabe hat und Aufgrund der
Akzeptanz eines Wortes in diesem Fall das Zeichen gelöscht wird.
\end{document}
