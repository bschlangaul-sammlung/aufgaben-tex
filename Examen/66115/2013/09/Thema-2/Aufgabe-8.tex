\documentclass{bschlangaul-aufgabe}
\bLadePakete{baum}
\begin{document}
\bAufgabenMetadaten{
  Titel = {Aufgabe 8},
  Thematik = {AVL-Baum 12,5,20,2,9,16,25,3,21},
  Referenz = 66115-2013-H.T2-A8,
  RelativerPfad = Examen/66115/2013/09/Thema-2/Aufgabe-8.tex,
  ZitatSchluessel = examen:66115:2013:09,
  BearbeitungsStand = mit Lösung,
  Korrektheit = korrekt und überprüft,
  Ueberprueft = {mit den Online-Tool VisuAlgo \url{https://visualgo.net/en/bst}},
  Stichwoerter = {AVL-Baum},
  EinzelpruefungsNr = 66115,
  Jahr = 2013,
  Monat = 09,
  ThemaNr = 2,
  AufgabeNr = 8,
}

Gegeben\index{AVL-Baum}\footcite{examen:66115:2013:09} sei der folgende
AVL-Baum $T$. Führen Sie auf $T$ folgende Operationen durch.

\begin{bProjektSprache}{Baum}
baum avl (
  setze: 12 5 20 2 9 16 25 3 21;
  drucke;
)
\end{bProjektSprache}

\begin{center}
\begin{tikzpicture}[b binaer baum]
\Tree
[.\node[label=0]{12};
  [.\node[label=-1]{5};
    [.\node[label=+1]{2};
      \edge[blank]; \node[blank]{};
      [.\node[label=0]{3}; ]
    ]
    [.\node[label=0]{9}; ]
  ]
  [.\node[label=+1]{20};
    [.\node[label=0]{16}; ]
    [.\node[label=-1]{25};
      [.\node[label=0]{21}; ]
      \edge[blank]; \node[blank]{};
    ]
  ]
]
\end{tikzpicture}
\end{center}

\begin{enumerate}

%%
% (a)
%%

\item Fügen Sie den Wert $22$ in $T$ ein. Balancieren Sie falls nötig und
geben Sie den entstandenen Baum (als Zeichnung) an.

\begin{bAntwort}
\begin{bBaum}{Nach dem Einfügen von „22“}
\begin{tikzpicture}[b binaer baum]
\Tree
[.\node[label=0]{12};
  [.\node[label=-1]{5};
    [.\node[label=+1]{2};
      \edge[blank]; \node[blank]{};
      [.\node[label=0]{3}; ]
    ]
    [.\node[label=0]{9}; ]
  ]
  [.\node[label=+2]{20};
    [.\node[label=0]{16}; ]
    [.\node[label=-2]{25};
      [.\node[label=+1]{21};
        \edge[blank]; \node[blank]{};
        [.\node[label=0]{22}; ]
      ]
      \edge[blank]; \node[blank]{};
    ]
  ]
]
\end{tikzpicture}
\end{bBaum}

\begin{bBaum}{Nach der Linksrotation}
\begin{tikzpicture}[b binaer baum]
\Tree
[.\node[label=0]{12};
  [.\node[label=-1]{5};
    [.\node[label=+1]{2};
      \edge[blank]; \node[blank]{};
      [.\node[label=0]{3}; ]
    ]
    [.\node[label=0]{9}; ]
  ]
  [.\node[label=+2]{20};
    [.\node[label=0]{16}; ]
    [.\node[label=-2]{25};
      [.\node[label=-1]{22};
        [.\node[label=0]{21}; ]
        \edge[blank]; \node[blank]{};
      ]
      \edge[blank]; \node[blank]{};
    ]
  ]
]
\end{tikzpicture}
\end{bBaum}

\begin{bBaum}{Nach der Rechtsrotation}
\begin{tikzpicture}[b binaer baum]
\Tree
[.\node[label=0]{12};
  [.\node[label=-1]{5};
    [.\node[label=+1]{2};
      \edge[blank]; \node[blank]{};
      [.\node[label=0]{3}; ]
    ]
    [.\node[label=0]{9}; ]
  ]
  [.\node[label=+1]{20};
    [.\node[label=0]{16}; ]
    [.\node[label=0]{22};
      [.\node[label=0]{21}; ]
      [.\node[label=0]{25}; ]
    ]
  ]
]
\end{tikzpicture}
\end{bBaum}
\end{bAntwort}

%%
% (b)
%%

\item Löschen Sie danach die $5$. Balancieren Sie $T$ falls nötig und
geben Sie den entstandenen Baum (als Zeichnung) an.
\end{enumerate}

\begin{bAntwort}
\begin{bBaum}{Nach dem Löschen von „5“}
\begin{tikzpicture}[b binaer baum]
\Tree
[.\node[label=0]{12};
  [.\node[label=-2]{9};
    [.\node[label=+1]{2};
      \edge[blank]; \node[blank]{};
      [.\node[label=0]{3}; ]
    ]
    \edge[blank]; \node[blank]{};
  ]
  [.\node[label=+1]{20};
    [.\node[label=0]{16}; ]
    [.\node[label=0]{22};
      [.\node[label=0]{21}; ]
      [.\node[label=0]{25}; ]
    ]
  ]
]
\end{tikzpicture}
\end{bBaum}

\begin{bBaum}{Nach der Linksrotation}
\begin{tikzpicture}[b binaer baum]
\Tree
[.\node[label=0]{12};
  [.\node[label=-2]{9};
    [.\node[label=-1]{3};
      [.\node[label=0]{2}; ]
      \edge[blank]; \node[blank]{};
    ]
    \edge[blank]; \node[blank]{};
  ]
  [.\node[label=+1]{20};
    [.\node[label=0]{16}; ]
    [.\node[label=0]{22};
      [.\node[label=0]{21}; ]
      [.\node[label=0]{25}; ]
    ]
  ]
]
\end{tikzpicture}
\end{bBaum}

\begin{bBaum}{Nach der Rechtsrotation}
\begin{tikzpicture}[b binaer baum]
\Tree
[.\node[label=+1]{12};
  [.\node[label=0]{3};
    [.\node[label=0]{2}; ]
    [.\node[label=0]{9}; ]
  ]
  [.\node[label=+1]{20};
    [.\node[label=0]{16}; ]
    [.\node[label=0]{22};
      [.\node[label=0]{21}; ]
      [.\node[label=0]{25}; ]
    ]
  ]
]
\end{tikzpicture}
\end{bBaum}
\end{bAntwort}

\end{document}
