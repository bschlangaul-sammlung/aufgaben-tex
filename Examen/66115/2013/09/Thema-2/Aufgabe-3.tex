\documentclass{bschlangaul-aufgabe}
\bLadePakete{automaten,minimierung,formale-sprachen}

\begin{document}
\bAufgabenMetadaten{
  Titel = {Aufgabe 3},
  Thematik = {Minimierung DFA},
  Referenz = 66115-2013-H.T2-A3,
  RelativerPfad = Staatsexamen/66115/2013/09/Thema-2/Aufgabe-3.tex,
  ZitatSchluessel = examen:66115:2013:09,
  BearbeitungsStand = mit Lösung,
  Korrektheit = unbekannt,
  Ueberprueft = {unbekannt},
  Stichwoerter = {Minimierungsalgorithmus},
  EinzelpruefungsNr = 66115,
  Jahr = 2013,
  Monat = 09,
  ThemaNr = 2,
  AufgabeNr = 3,
}

\let\f=\bFussnote
\let\l=\bLeereZelle
\def\Z#1#2{(#1, #2)}

Minimieren Sie den folgenden deterministischen Automaten mit den
Zuständen \bMenge{0, 1, 2, 3, 4, 5, 6}, dem Startzustand $0$ und den
Endzuständen \bMenge{3, 6}. Geben Sie \zB durch die Bezeichnung an,
welche Zustände zusammengefasst wurden.\index{Minimierungsalgorithmus}
\footcite{examen:66115:2013:09}

\begin{center}
\begin{tikzpicture}[li automat,node distance=1cm]
\node[state,initial] (0) {0};
\node[state,right=of 0] (1) {1};
\node[state,right=of 1] (2) {2};
\node[state,below right=of 2] (4) {4};
\node[state,accepting,above right=of 4] (3) {3};
\node[state,right=of 3] (5) {5};
\node[state,accepting,right=of 5] (6) {6};

\path (0) edge[above] node{a} (1);
\path (1) edge[above] node{a} (2);
\path (2) edge[above] node{a} (3);
\path (2) edge[above] node{b} (4);
\path (3) edge[above,bend left] node{b} (5);
\path (3) edge[above,loop above] node{a} (3);
\path (4) edge[above] node{a} (3);
\path (4) edge[above,loop below] node{b} (4);
\path (5) edge[above,bend left] node{a} (6);
\path (5) edge[above,bend left] node{b} (3);
\path (6) edge[above,bend left] node{b} (5);
\path (6) edge[below,bend left] node{a} (3);
\end{tikzpicture}
\end{center}

\begin{bAntwort}
\begin{center}
\begin{tabular}{|c||c|c|c|c|c|c|c|}
\hline
0 & \l  & \l  & \l  & \l  & \l  & \l  & \l  \\ \hline
1 & \f3 & \l  & \l  & \l  & \l  & \l  & \l  \\ \hline
2 & \f2 & \f2 & \l  & \l  & \l  & \l  & \l  \\ \hline
3 & \f1 & \f1 & \f1 & \l  & \l  & \l  & \l  \\ \hline
4 & \f2 & \f2 &     & \f1 & \l  & \l  & \l  \\ \hline
5 & \f2 & \f2 & \f2 & \f1 & \f2 & \l  & \l  \\ \hline
6 & \f1 & \f1 & \f1 &     & \f1 & \f1 & \l  \\ \hline\hline
  & 0   & 1   & 2   & 3   & 4   & 5   & 6   \\ \hline
\end{tabular}
\end{center}

\bFussnoten

\begin{liUebergangsTabelle}{a}{b}
\Z01 & \Z12 \f3 & \Z TT \\
\Z02 & \Z13 \f2 & \Z T4 \\
\Z04 & \Z13 \f2 & \Z T4 \\
\Z05 & \Z16 \f2 & \Z T3 \\
\Z12 & \Z23 \f2 & \Z T4 \\
\Z14 & \Z23 \f2 & \Z T4 \\
\Z15 & \Z26 \f2 & \Z T3 \\
\Z24 & \Z33 & \Z 44 \\
\Z25 & \Z36 & \Z 34 \f2 \\
\Z36 & \Z33 & \Z 55 \\
\Z45 & \Z36 & \Z 34 \f2 \\
\end{liUebergangsTabelle}

T = Trap-Zustand = Falle

\begin{center}
\begin{tikzpicture}[li automat,node distance=1cm]
\node[state,initial] (0) {0};
\node[state,right=of 0] (1) {1};
\node[state,right=of 1] (24) {24};
\node[state,accepting,right=of 24] (36) {36};
\node[state,right=of 36] (5) {5};

\path (0) edge[above] node{a} (1);
\path (1) edge[above] node{a} (24);
\path (24) edge[above,loop above] node{b} (24);
\path (24) edge[above] node{a} (36);
\path (36) edge[above,bend left] node{b} (5);
\path (36) edge[above,loop above] node{a} (36);
\path (5) edge[above,bend left] node{a,b} (36);
\end{tikzpicture}
\end{center}

\end{bAntwort}
\end{document}
