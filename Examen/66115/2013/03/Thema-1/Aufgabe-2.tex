\documentclass{bschlangaul-aufgabe}
\bLadePakete{formale-sprachen,cyk-algorithmus}
\begin{document}
\bAufgabenMetadaten{
  Titel = {Aufgabe 2},
  Thematik = {Kontextfreie Grammatiken},
  Referenz = 66115-2013-F.T1-A2,
  RelativerPfad = Staatsexamen/66115/2013/03/Thema-1/Aufgabe-2.tex,
  ZitatSchluessel = examen:66115:2013:03,
  BearbeitungsStand = mit Lösung,
  Korrektheit = unbekannt,
  Ueberprueft = {unbekannt},
  Stichwoerter = {Kontextfreie Sprache, CYK-Algorithmus},
  EinzelpruefungsNr = 66115,
  Jahr = 2013,
  Monat = 03,
  ThemaNr = 1,
  AufgabeNr = 2,
}

\let\l=\bKurzeTabellenLinie

Gegeben sei die Grammatik \bGrammatik{variablen={S, A, B, C},
alphabet={a,b}} und\index{Kontextfreie Sprache}
\footcite{examen:66115:2013:03}

\begin{bProduktionsRegeln}
S -> A B,
S -> C S,
A -> B C,
A -> B B,
A -> a,
B -> A C,
B -> b,
C -> A A,
C -> B A
\end{bProduktionsRegeln}
\bFlaci{Gr46a6j0a}

$L = L(G)$ ist die von G erzeugte Sprache.

\begin{enumerate}
%%
% a)
%%

\item Zeigen Sie, dass $G$ mehrdeutig ist.

\begin{bAntwort}
Das Wort $baab$ kann in zwei verschiedenen Ableitungen hergeleitet
werden:

\begin{enumerate}
\item \bAbleitung{S -> AB -> BCB -> bCB -> bAAB -> baAB -> baaB -> baab}

\item \bAbleitung{S -> CS -> BAS -> bAS -> baS -> baAB -> baaB -> baab}
\end{enumerate}
\end{bAntwort}

%%
% b)
%%

\item Entscheiden Sie mithilfe des Algorithmus von Cocke, Younger und
Kasami (CYK), ob das Wort $w = babaaa$ zur Sprache L gehört. Begründen
Sie Ihre Entscheidung.
\index{CYK-Algorithmus}

\begin{bAntwort}
\begin{tabular}{|c|c|c|c|c|c|}
b     & a     & b    & a    & a    & a \\\hline\hline

B     & A     & B    & A    & A    & A \l6
C     & S     & C    & C    & C \l5
-     & B     & A    & B \l4
A     & C     & A,C \l3
A,C   & B,C,A \l2
A,C,B \l1
\end{tabular}

\bWortNichtInSprache{babaaa}

Das Startsymbol $S$ ist nicht in der Zelle $V(1,5) = \bMenge{A, C, B}$
enthalten.
\end{bAntwort}

%%
% c)
%%

\item Geben Sie eine Ableitung für $w = babaaa$ an.

\begin{bAntwort}
\bAbleitung{A -> BB -> bB -> bAC -> baC -> baAA -> baBCA -> babCA ->
babAAA -> babaAA -> babaaA -> babaaa}
\end{bAntwort}

\end{enumerate}

\end{document}
