\documentclass{bschlangaul-aufgabe}
\bLadePakete{java}
\begin{document}
\bAufgabenMetadaten{
  Titel = {Aufgabe},
  Thematik = {Hashing mit Modulo 10},
  Referenz = 66115-2010-H.T2-A3,
  RelativerPfad = Examen/66115/2010/09/Thema-2/Aufgabe-3.tex,
  ZitatSchluessel = examen:66115:2010:09,
  ZitatBeschreibung = {Seite 4},
  BearbeitungsStand = nur Angabe,
  Korrektheit = unbekannt,
  Ueberprueft = {unbekannt},
  Stichwoerter = {Streutabellen (Hashing)},
  EinzelpruefungsNr = 66115,
  Jahr = 2010,
  Monat = 09,
  ThemaNr = 2,
  AufgabeNr = 3,
}

Gegegen sei ein Array der Größe 10, \zB \bJavaCode{int[] hashfeld = new
int [10]}. Die Hashfunktion sei der Wert modulo 10, $h(x) = x \% 10$.
Kollisionen werden mit linearer Verschiebung um 1 (modulo 10) gelöst.
\index{Streutabellen (Hashing)}
\footcite[Seite 4]{examen:66115:2010:09}

\bJavaCode{in(x)} bedeutet, dass die Zahl x eingefügt wird,
\bJavaCode{search(x)}, dass nach x gesucht wird mit den Antworten „ja“
bzw. „nein“ und \bJavaCode{out(x)}, dass x gelöscht wird, sofern x
gespeichert ist.

Es wird folgende Sequenz von Operationen auf ein anfangs leeres Array
ausgeführt:

\bJavaCode{in(19)},
\bJavaCode{in(29)},
\bJavaCode{in(39)},
\bJavaCode{in(10)},
\bJavaCode{out(29)},
\bJavaCode{out(39)},
\bJavaCode{search(29)},
\bJavaCode{in(11)},
\bJavaCode{in(17)},
\bJavaCode{out(10)},
\bJavaCode{in(2)},
\bJavaCode{in(22)}

Geben Sie den Inhalt von \bJavaCode{hashfeld} an

nach \bJavaCode{search(29)}\\
nach \bJavaCode{out(10)}\\
und nach \bJavaCode{in(22)}.
\end{document}
