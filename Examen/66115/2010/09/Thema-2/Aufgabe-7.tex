\documentclass{bschlangaul-aufgabe}
\bLadePakete{formale-sprachen}
\begin{document}
\bAufgabenMetadaten{
  Titel = {Aufgabe 8},
  Thematik = {p Zeichen a und aus q Zeichen b},
  Referenz = 66115-2010-H.T2-A7,
  RelativerPfad = Examen/66115/2010/09/Thema-2/Aufgabe-7.tex,
  ZitatSchluessel = examen:66115:2010:09,
  BearbeitungsStand = nur Angabe,
  Korrektheit = unbekannt,
  Ueberprueft = {unbekannt},
  Stichwoerter = {Turing-Maschine},
  EinzelpruefungsNr = 66115,
  Jahr = 2010,
  Monat = 09,
  ThemaNr = 2,
  AufgabeNr = 7,
}

Konstruieren Sie eine Turingmaschine $M$ mit $L(M) = L$, wobei $p, q
\geq 1$. Beschreiben Sie zusätzlich, wie $M$ arbeitet (Stil: $M$ liest
das Zeichen $a$ und speichert ....)\index{Turing-Maschine}
\footcite{examen:66115:2010:09}

\bAusdruck{w}{w \in \bMenge{a, b}*, w\text{ besteht aus }p\text{ Zeichen
}a\text{ und aus }q\text{ Zeichen }b }.

\end{document}
