\documentclass{bschlangaul-aufgabe}
\bLadePakete{formale-sprachen,cyk-algorithmus}
\begin{document}
\bAufgabenMetadaten{
  Titel = {Aufgabe 3},
  Thematik = {Kontextfreie Sprachen},
  Referenz = 66115-2020-F.T1-A3,
  RelativerPfad = Examen/66115/2020/03/Thema-1/Aufgabe-3.tex,
  ZitatSchluessel = examen:66115:2020:03,
  BearbeitungsStand = mit Lösung,
  Korrektheit = unbekannt,
  Ueberprueft = {unbekannt},
  Stichwoerter = {Kontextfreie Sprache, CYK-Algorithmus, Ableitung (Kontextfreie Sprache)},
  EinzelpruefungsNr = 66115,
  Jahr = 2020,
  Monat = 03,
  ThemaNr = 1,
  AufgabeNr = 3,
}

\let\m=\bMenge
\let\l=\bKurzeTabellenLinie

\begin{enumerate}

%%
% (a)
%%

\item Entwerfen Sie eine kontextfreie Grammatik für die folgende
kontextfreie Sprache über dem Alphabet \bAlphabet{a, b, c}:
\index{Kontextfreie Sprache}
\footcite{examen:66115:2020:03}

\begin{center}
\bAusdruck{a^{3n+2}wvc^n}{n \in \mathbb{N}_0, 2 \cdot |w|_b = |v|_a}
\end{center}

(Hierbei bezeichnet $|u|_x$, die Anzahl des Zeichens $x$ in dem Wort $u$.)

Erklären Sie den Zweck der einzelnen Nichtterminale (Variablen) und der
Grammatikregeln Ihrer Grammatik.

\begin{bAntwort}
\begin{bProduktionsRegeln}
S -> a a a S c | a a a A c,
A -> a a B,
B -> b B a a | b a a
\end{bProduktionsRegeln}
\bFlaci{Ghhs1xexw}
\end{bAntwort}

%%
% (b)
%%

\item Betrachten Sie die folgende kontextfreie Grammatik

\begin{displaymath}
G=(\m{A,B,C,D}, \m{a,b,c}, P, A)
\end{displaymath}

mit den Produktionen

\begin{bProduktionsRegeln}
A -> A B | C D | a,
B -> C C | c,
C -> D C | C B | b,
D -> D B | a,
\end{bProduktionsRegeln}
\bFlaci{Gf7556jn2}

Benutzen Sie den Algorithmus von Cocke-Younger-Kasami (CYK), um zu
zeigen, dass das Wort $abcab$ zu der von $G$ erzeugten Sprache $L(G)$
gehört.
\index{CYK-Algorithmus}

\begin{bAntwort}
\begin{tabular}{|c|c|c|c|c|}

a    & b    & c    & a    & b \\\hline\hline

A,D  & C    & B    & A,D  & C \l5
C    & C    & -    & C    \l4
C,C  & A    & -    \l3
A,A  & B    \l2
A,D,B,B \l1
\end{tabular}

\bWortInSprache{abcab}
\end{bAntwort}

%%
% (c)
%%

\item Finden Sie nun ein größtmögliches Teilwort von $abcab$, dass von
keinem der vier Nichtterminale von $G$ ableitbar ist.

%%
% (d)
%%

\item Geben Sie eine Ableitung des Wortes $abcab$ mit $G$ an.
\index{Ableitung (Kontextfreie Sprache)}

\begin{bAntwort}
\bAbleitung{
  A ->
  AB ->
  ACC ->
  ACBC ->
  ACBDC ->
  aCBDC ->
  abBDC ->
  abcDC ->
  abcaC ->
  abcab
}
\end{bAntwort}

%%
% (e)
%%

\item Beweisen Sie, dass die folgende formale Sprache über Z = {a,b} nicht kon-
textfrei ist: ,
L={a"b" |neN}.

\end{enumerate}
\end{document}
