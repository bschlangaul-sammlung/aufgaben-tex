\documentclass{bschlangaul-aufgabe}
\bLadePakete{formale-sprachen,spalten,tabelle}
\begin{document}
\bAufgabenMetadaten{
  Titel = {Aufgabe 1},
  Thematik = {Vornamen},
  Referenz = 66115-2020-H.T2-TA2-A1,
  RelativerPfad = Staatsexamen/66115/2020/09/Thema-2/Teilaufgabe-2/Aufgabe-1.tex,
  ZitatSchluessel = examen:66115:2020:09,
  BearbeitungsStand = mit Lösung,
  Korrektheit = unbekannt,
  Ueberprueft = {unbekannt},
  Stichwoerter = {Streutabellen (Hashing)},
  EinzelpruefungsNr = 66115,
  Jahr = 2020,
  Monat = 09,
  ThemaNr = 2,
  TeilaufgabeNr = 2,
  AufgabeNr = 1,
}

Eine Hashfunktion $h$ wird verwendet, um Vornamen auf die Buchstaben
\bMenge{A, \dots, Z} abzubilden. Dabei bildet $h$ auf den ersten
Buchstaben des Vornamens als Hashwert ab.
\index{Streutabellen (Hashing)}
\footcite{examen:66115:2020:09}

Sei $H$ die folgende Hashtabelle (Ausgangszustand):

\begin{multicols}{2}
\noindent
\begin{tabularx}{\linewidth}{|l||X|}
\hline
Schlüssel & Inhalt \\\hline\hline
A &\\\hline
B &\\\hline
C &\\\hline
D & Dirk \\\hline
E &\\\hline
F &\\\hline
G &\\\hline
H &\\\hline
I & Inge \\\hline
J &\\\hline
K & Kurt \\\hline
L &\\\hline
M &\\\hline
\end{tabularx}

\noindent
\begin{tabularx}{\linewidth}{|l||X|}
\hline
Schlüssel & Inhalt \\\hline\hline
N & \\\hline
O & \\\hline
P & \\\hline
Q & \\\hline
R & \\\hline
S & \\\hline
T & \\\hline
U & \\\hline
V & \\\hline
W & \\\hline
X & \\\hline
Y & \\\hline
Z & \\\hline
\end{tabularx}
\end{multicols}
\begin{enumerate}
%%
% a)
%%

\item Fügen Sie der Hashtabelle $H$ die Vornamen

\begin{center}
Martin, Michael, Norbert, Matthias, Alfons, Bert, Christel, Adalbert,
Edith, Emil
\end{center}

in dieser Reihenfolge hinzu, wobei Sie Kollisionen durch lineares
Sondieren (mit Inkrementieren zum nächsten Buchstaben) behandeln.

\begin{bAntwort}
\begin{multicols}{2}
\noindent
\begin{tabularx}{\linewidth}{|l||X|}
\hline
Schlüssel & Inhalt \\\hline\hline
A & Alfons \\\hline
B & Bert \\\hline
C & Christel \\\hline
D & Dirk \\\hline
E & Adalbert \\\hline
F & Edith \\\hline
G & Emil \\\hline
H & \\\hline
I & Inge \\\hline
J & \\\hline
K & Kurt \\\hline
L & \\\hline
M & Martin \\\hline
\end{tabularx}

\noindent
\begin{tabularx}{\linewidth}{|l||X|}
\hline
Schlüssel & Inhalt \\\hline\hline
N & Michael \\\hline
O & Norbert \\\hline
P & Matthias \\\hline
Q & \\\hline
R & \\\hline
S & \\\hline
T & \\\hline
U & \\\hline
V & \\\hline
W & \\\hline
X & \\\hline
Y & \\\hline
Z & \\\hline
\end{tabularx}
\end{multicols}
\bigskip
\end{bAntwort}

%%
% b)
%%

\item Fügen Sie der Hashtabelle $H$ die Vornamen

\begin{center}
Brigitte, Elmar, Thomas, Katrin, Diana, Nathan, Emanuel, Sebastian,
Torsten, Karolin
\end{center}

in dieser Reihenfolge hinzu, wobei Sie Kollisionen durch Verkettung der
Überläufer behandeln. (Hinweis: Verwenden Sie die Hashtabelle im
Ausgangszustand.)

\begin{bAntwort}
\begin{multicols}{2}
\noindent
\begin{tabularx}{\linewidth}{|l||X|}
\hline
Schlüssel & Inhalt \\\hline\hline
A &\\\hline
B & Brigitte \\\hline
C &\\\hline
D & Dirk, Diana \\\hline
E & Elmar, Emanuel \\\hline
F &\\\hline
G &\\\hline
H &\\\hline
I & Inge \\\hline
J &\\\hline
K & Kurt, Katrin, Karolin \\\hline
L &\\\hline
M &\\\hline
\end{tabularx}

\noindent
\begin{tabularx}{\linewidth}{|l||X|}
\hline
Schlüssel & Inhalt \\\hline\hline
N & Nathan \\\hline
O & \\\hline
P & \\\hline
Q & \\\hline
R & \\\hline
S & Sebastian \\\hline
T & Thomas, Torsten \\\hline
U & \\\hline
V & \\\hline
W & \\\hline
X & \\\hline
Y & \\\hline
Z & \\\hline
\end{tabularx}
\end{multicols}
\bigskip
\end{bAntwort}

\end{enumerate}
\end{document}
