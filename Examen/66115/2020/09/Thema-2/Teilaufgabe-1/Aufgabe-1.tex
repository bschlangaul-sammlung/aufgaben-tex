\documentclass{bschlangaul-aufgabe}
\bLadePakete{formale-sprachen,automaten,minimierung}
\begin{document}
\bAufgabenMetadaten{
  Titel = {Aufgabe 1},
  Thematik = {Minimierungsalgorithmus},
  Referenz = 66115-2020-H.T2-TA1-A1,
  RelativerPfad = Examen/66115/2020/09/Thema-2/Teilaufgabe-1/Aufgabe-1.tex,
  ZitatSchluessel = examen:66115:2020:09,
  BearbeitungsStand = mit Lösung,
  Korrektheit = unbekannt,
  Ueberprueft = {unbekannt},
  Stichwoerter = {Reguläre Sprache},
  EinzelpruefungsNr = 66115,
  Jahr = 2020,
  Monat = 09,
  ThemaNr = 2,
  TeilaufgabeNr = 1,
  AufgabeNr = 1,
}

\let\z=\bZustandsnameTiefgestellt
\let\f=\bFussnote
\let\l=\bLeereZelle
\let\Z=\bZustandsPaar

\begin{enumerate}

%%
% a)
%%

\item Geben Sie einen deterministischen endlichen Automaten (DEA) mit
minimaler Anzahl an Zuständen an, der dieselbe Sprache akzeptiert wie
folgender deterministischer endlicher Automat. Dokumentieren Sie Ihr
Vorgehen geeignet.\index{Reguläre Sprache}
\footcite{examen:66115:2020:09}

\begin{center}
\begin{tikzpicture}[li automat,scale=0.8,transform shape]
  \node[state,initial] (z0) at (2.14cm,-2.14cm) {$z_0$};
  \node[state] (z1) at (2.86cm,-6cm) {$z_1$};
  \node[state] (z2) at (6.71cm,-1.71cm) {$z_2$};
  \node[state] (z3) at (8cm,-4.29cm) {$z_3$};
  \node[state] (z4) at (10.71cm,-1cm) {$z_4$};
  \node[state] (z5) at (4.43cm,-3.57cm) {$z_5$};
  \node[state] (z6) at (11.86cm,-4cm) {$z_6$};
  \node[state] (z7) at (9.43cm,-6.43cm) {$z_7$};
  \node[state,accepting] (z8) at (14cm,-2.43cm) {$z_8$};

  \path (z0) edge[auto] node{$1$} (z1);
  \path (z0) edge[auto] node{$0$} (z2);
  \path (z1) edge[auto] node{$1$} (z7);
  \path (z1) edge[auto] node{$0$} (z5);
  \path (z2) edge[auto] node{$0$} (z3);
  \path (z2) edge[auto,bend left] node{$1$} (z7);
  \path (z3) edge[auto] node{$0$} (z5);
  \path (z3) edge[auto] node{$1$} (z1);
  \path (z4) edge[auto] node{$0$} (z3);
  \path (z4) edge[auto] node{$1$} (z7);
  \path (z5) edge[auto] node{$0$} (z0);
  \path (z5) edge[auto] node{$1$} (z7);
  \path (z6) edge[auto] node{$0$} (z8);
  \path (z6) edge[auto] node{$1$} (z4);
  \path (z7) edge[auto,bend right] node{$1$} (z8);
  \path (z7) edge[auto] node{$0$} (z6);
  \path (z8) edge[auto] node{$1$} (z4);
  \path (z8) edge[auto,loop above] node{$0$} (z8);
\end{tikzpicture}
\end{center}
\bFlaci{Aj5aei652}

\begin{bAntwort}

\bUeberschriftDreiecksTabelle

\bMinimierungErklaerung

\begin{center}
\begin{tabular}{|c||c|c|c|c|c|c|c|c|c|}
\hline
\z0 & \l  & \l  & \l  & \l  & \l  & \l  & \l  & \l  & \l  \\ \hline
\z1 & \f3 & \l  & \l  & \l  & \l  & \l  & \l  & \l  & \l  \\ \hline
\z2 & \f3 & \f4 & \l  & \l  & \l  & \l  & \l  & \l  & \l  \\ \hline
\z3 &     & \f3 & \f3 & \l  & \l  & \l  & \l  & \l  & \l  \\ \hline
\z4 & \f3 & \f4 &     & \f3 & \l  & \l  & \l  & \l  & \l  \\ \hline
\z5 & \f3 & \f4 &     & \f3 &     & \l  & \l  & \l  & \l  \\ \hline
\z6 & \f2 & \f2 & \f2 & \f2 & \f2 & \f2 & \l  & \l  & \l  \\ \hline
\z7 & \f2 & \f2 & \f2 & \f2 & \f2 & \f2 & \f2 & \l  & \l  \\ \hline
\z8 & \f1 & \f1 & \f1 & \f1 & \f1 & \f1 & \f1 & \f1 & \l  \\ \hline\hline
    & \z0 & \z1 & \z2 & \z3 & \z4 & \z5 & \z6 & \z7 & \z8 \\ \hline
\end{tabular}
\end{center}

\bFussnoten

% 0 2 1
% 1 5 7
% 2 3 7
% 3 5 1
% 4 3 7
% 5 0 7
% 6 8 4
% 7 6 8
% 8 8 4

\begin{liUebergangsTabelle}{0}{1}
\Z01 & \Z25 & \Z17 $^{x_3}$ \f3 \\
\Z02 & \Z23 & \Z17 \f3 \\
\Z03 & \Z25 & \Z11 \\
\Z04 & \Z23 & \Z17 \f3 \\
\Z05 & \Z20 & \Z17 \f3\\
\Z06 & \Z28 & \Z14 \f2 \\
\Z07 & \Z26 & \Z18 \f2 \\
\Z12 & \Z53 & \Z77 \f4 \\
\Z13 & \Z55 & \Z71 \f3 \\
\Z14 & \Z53 & \Z77 \f4 \\
\Z15 & \Z50 & \Z77 \f4 \\
\Z16 & \Z58 & \Z74 \f2 \\
\Z17 & \Z56 & \Z78 \f2 \\
\Z23 & \Z35 & \Z71 \f3 \\
\Z24 & \Z33 & \Z77 \\
\Z25 & \Z30 & \Z77 \\
\Z26 & \Z38 & \Z74 \f2 \\
\Z27 & \Z36 & \Z78 \f2 \\
\Z34 & \Z53 & \Z17 \f3 \\
\Z35 & \Z50 & \Z17 \f3 \\
\Z36 & \Z58 & \Z14 \f2 \\
\Z37 & \Z56 & \Z18 \f2 \\
\Z45 & \Z30 & \Z77 \\
\Z46 & \Z38 & \Z74 \f2 \\
\Z47 & \Z36 & \Z78 \f2 \\
\Z56 & \Z08 & \Z74 \f2 \\
\Z57 & \Z06 & \Z78 \f2 \\
\Z67 & \Z86 & \Z48 \f2\\
\end{liUebergangsTabelle}

\begin{center}
\begin{tikzpicture}[li automat]
  \node[state,initial] (z03) at (2.29cm,-2.14cm) {$z_{03}$};
  \node[state] (z6) at (9.43cm,-6.71cm) {$z_6$};
  \node[state] (z245) at (6.71cm,-3cm) {$z_{245}$};
  \node[state] (z7) at (5.71cm,-7.86cm) {$z_7$};
  \node[state,accepting] (z8) at (10.71cm,-2.71cm) {$z_8$};
  \node[state] (z1) at (2.14cm,-5.29cm) {$z_1$};

  \path (z03) edge[auto] node{$1$} (z1);
  \path (z03) edge[auto,bend left] node{$0$} (z245);
  \path (z6) edge[auto] node{$0$} (z8);
  \path (z6) edge[auto] node{$1$} (z245);
  \path (z245) edge[auto,bend left] node{$0$} (z03);
  \path (z245) edge[auto] node{$1$} (z7);
  \path (z7) edge[auto] node{$1$} (z8);
  \path (z7) edge[auto] node{$0$} (z6);
  \path (z8) edge[auto] node{$1$} (z245);
  \path (z8) edge[auto,loop above] node{$0$} (z8);
  \path (z1) edge[auto] node{$1$} (z7);
  \path (z1) edge[auto] node{$0$} (z245);
\end{tikzpicture}
\end{center}
\bFlaci{Aro484bz2}

\end{bAntwort}

%%
% b)
%%

\item Beweisen oder widerlegen Sie für folgende Sprachen über dem
Alphabet \bAlphabet{a ,b, c}, dass sie regulär sind.

\begin{enumerate}

%%
% (i)
%%

\item \bAusdruck[L_1]{a^i c u b^j v a c^k}
{u,v \in \bMenge{a,b}^* \text{ und } i, j, k \in \mathbb{N}_0}

\begin{bAntwort}
Die Sprache $L_1$ ist regulär. Nachweis durch regulären Ausdruck:

\begin{displaymath}
a^* c (a|b)^* b^* (a|b)^* a c^*
\end{displaymath}
\end{bAntwort}

%%
% (ii)
%%

\item \bAusdruck[L_2]{a^i c u b^j v a c^k}
{u, v \in \bMenge{a, b}^* \text{ und } i, j, k \in \mathbb{N}_0\text{ mit } k = i + j}

\begin{bAntwort}
Die Sprache $L_2$ ist nicht regulär. Widerlegung durch das Pumping-Lemma.

TODO
\end{bAntwort}

\end{enumerate}

%%
% c)
%%

\item Sei $L$ eine reguläre Sprache über dem Alphabet $\Sigma$. Für ein
festes Element $a \in \Sigma$ betrachten wir die Sprache
\bAusdruck[L_a]{a w}{w \in \Sigma^*, wa \in L}. Zeigen Sie, dass $L_a$
regulär ist.

\begin{bAntwort}
Die regulären Sprachen sind unter dem Komplement abgeschlossen.
\end{bAntwort}
\end{enumerate}
\end{document}
