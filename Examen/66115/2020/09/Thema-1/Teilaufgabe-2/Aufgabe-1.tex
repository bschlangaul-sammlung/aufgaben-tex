\documentclass{bschlangaul-aufgabe}
\bLadePakete{java}
\begin{document}
\bAufgabenMetadaten{
  Titel = {Aufgabe 1},
  Thematik = {Algorithmenanalyse},
  Referenz = 66115-2020-H.T1-TA2-A1,
  RelativerPfad = Examen/66115/2020/09/Thema-1/Teilaufgabe-2/Aufgabe-1.tex,
  ZitatSchluessel = examen:66115:2020:09,
  BearbeitungsStand = mit Lösung,
  Korrektheit = unbekannt,
  Ueberprueft = {unbekannt},
  Stichwoerter = {Algorithmen und Datenstrukturen},
  EinzelpruefungsNr = 66115,
  Jahr = 2020,
  Monat = 09,
  ThemaNr = 1,
  TeilaufgabeNr = 2,
  AufgabeNr = 1,
}

\newmintinline[p]{pascal}{}

Betrachten Sie die folgende Prozedur \p{countup}, die aus zwei
ganzzahligen Eingabewerten \p{n} und \p{m} einen ganzzahligen
Ausgabewert berechnet:
\index{Algorithmen und Datenstrukturen}
\footcite{examen:66115:2020:09}

\begin{minted}{pascal}
procedure countup(n, m : integer): integer
var x, y : integer;
begin
  x := n;
  y := 0;
  while (y < m) do
    x := x - 1;
    y := y + 1;
  end while
  return x;
end
\end{minted}
\begin{enumerate}

%%
% a)
%%

\item Führen Sie \p{countup(3,2)} aus. Geben Sie für jeden
Schleifendurchlauf jeweils den Wert der Variablen \p{n}, \p{m}, \p{x}
und \p{y} zu Beginn der while-Schleife und den Rückgabewert der Prozedur
an.

\begin{bAntwort}
\begin{tabular}{lllll}
n & m & x & y  & ausgeführter Code, der Änderung bewirkte \\\hline
3 & 2 & 3 & 0 \\
3 & 2 & 2 & 1 & \p{x := x - 1; y := y + 1;} \\
\end{tabular}

Rückgabewert: 1

\bPseudoUeberschrift{Java-Implementation der Prozedur}

\bJavaExamen{66115}{2020}{09}{counter/CountUp}
\end{bAntwort}

%%
% b)
%%

\item Gibt es Eingabewerte von \p{n} und \p{m}, für die die Prozedur
\p{countup} nicht terminiert? Begründen Sie Ihre Antwort.

\begin{bAntwort}
Nein. Mit jedem Schleifendurchlauf wird der Wert der Variablen \p{y} um
eins hochgezählt. Die Werte, die \p{y} annimmt, sind streng monoton
steigend. \p{y} nähert sich \p{m} an, bis \p{y} nicht mehr kleiner ist
als \p{m} und die Prozedur terminiert. An diesem Sachverhält ändern auch
sehr große Zahlen, die über die Variable \p{m} der Prozedur übergeben
werden, nichts.
\end{bAntwort}

%%
% c)
%%

\item Geben Sie die asymptotische worst-case Laufzeit der Prozedur
\p{countup} in der $\Theta$-Notation in Abhängigkeit von den
Eingabewerten \p{n} und/oder \p{m} an. Begründen Sie Ihre Antwort.

\begin{bAntwort}
Die Laufzeit der Prozedur ist immer $\Theta(m)$. Die Laufzeit hängt nur
von \p{m} ab. Es kann nicht zwischen best-, average and worst-case
unterschieden werden.
\end{bAntwort}
%%
% d)
%%

\item

Betrachten Sie nun die folgende Prozedur \p{countdown}, die aus zwei
ganzzahligen Eingabewerten \p{n} und \p{m} einen ganzzahligen
Ausgabewert berechnet:

\begin{minted}{pascal}
procedure countdown(n, m : integer) : integer
var x, y : integer;
begin
  x := n;
  y := 0;
  while (n > 0) do
    if (y < m) then
      x := x - 1;
      y := y + 1;
    else
      y := 0;
      n := n / 2; /* Ganzzahldivision */
    end if
  end while
  return x;
end
\end{minted}

Führen Sie \p{countdown(3, 2)} aus. Geben Sie für jeden
Schleifendurchlauf jeweils den Wert der Variablen \p{n}, \p{m}, \p{x}
und \p{y} zu Beginn der \p{while}-Schleife und den Rückgabewert der
Prozedur an.

\begin{bAntwort}
\begin{tabular}{lllll}
n & m & x & y & ausgeführter Code, der Änderung bewirkte \\\hline
3 & 2 & 3 & 0 & \\
3 & 2 & 2 & 1 & \p{x := x - 1; y := y + 1;} \\
3 & 2 & 1 & 2 & \p{x := x - 1; y := y + 1;} \\
1 & 2 & 1 & 0 & \p{y := 0; n := n / 2;} \\
1 & 2 & 0 & 1 & \p{x := x - 1; y := y + 1;} \\
1 & 2 & -1 & 2 & \p{x := x - 1; y := y + 1;} \\
\end{tabular}

Rückgabewert: -1

\bPseudoUeberschrift{Java-Implementation der Prozedur}

\bJavaExamen{66115}{2020}{herbst}{counter/CountDown}
\end{bAntwort}

%%
% e)
%%

\item Gibt es Eingabewerte von \p{n} und \p{m}, für die die Prozedur
\p{countdown} nicht terminiert?

Begründen Sie Ihre Antwort.

\begin{bAntwort}
Nein.

\begin{description}
\item[\p{n <= 0}]

terminiert sofort

\item[\p{m <= 0}]

Der Falsch-Block der Wenn-Dann-Bedingung erniedrigt \p{n} \p{n := n /
2;} bis \p{0} erreicht ist. Dann terminiert die Prozedur.

\item[\p{m > 0}]

Der erste Wahr-Block der Wenn-Dann-Bedingung erhöht \p{y} streng monoton
bis \p{y >= m}. 2. Falsch-Block der Wenn-Dann-Bedingung halbiert \p{n}
bis \p{0}. 1. und 2. solange bis $n = 0$
\end{description}
\end{bAntwort}

%%
% f)
%%

\item Geben Sie die asymptotische Laufzeit der Prozedur \p{countdown} in
der $\Theta$-Notation in Abhängigkeit von den Eingabewerten \p{n}
und/oder \p{m} an unter der Annahme, dass \p{m >= 0} und \p{n > 0}.
Begründen Sie Ihre Antwort.

\begin{bAntwort}
Anzahl der Wiederholungen der while-Schleife: $m + 1$:

\begin{itemize}
\item $m$ oft: bis $y < m$
\item $+1$ Halbierung von $n$ und $y$ auf $0$ setzen
\end{itemize}

wegen dem \p{n/2} ist die Laufzeit logarithmisch, ähnlich wie der worst
case bei der Binären Suche.

\begin{tabular}{lllll}
n & m & x & y & ausgeführter Code, der Änderung bewirkte \\\hline
16 & 3 & 16 & 0 \\
16 & 3 & 15 & 1 \\
16 & 3 & 14 & 2 \\
16 & 3 & 13 & 3 \\
8 & 3 & 13 & 0 & \p{y := 0; n := n / 2;} \\
8 & 3 & 12 & 1 \\
8 & 3 & 11 & 2 \\
8 & 3 & 10 & 3 \\
4 & 3 & 10 & 0 & \p{y := 0; n := n / 2;} \\
4 & 3 & 9 & 1 \\
4 & 3 & 8 & 2 \\
4 & 3 & 7 & 3 \\
2 & 3 & 7 & 0 & \p{y := 0; n := n / 2;} \\
2 & 3 & 6 & 1 \\
2 & 3 & 5 & 2 \\
2 & 3 & 4 & 3 \\
1 & 3 & 4 & 0 & \p{y := 0; n := n / 2;} \\
1 & 3 & 3 & 1 \\
1 & 3 & 2 & 2 \\
1 & 3 & 1 & 3 \\
\end{tabular}

$\Theta((m + 1) \log_2 n)$

Wegkürzen der Konstanten:

$\Rightarrow \Theta(m \log n)$
\end{bAntwort}

\end{enumerate}
\end{document}
