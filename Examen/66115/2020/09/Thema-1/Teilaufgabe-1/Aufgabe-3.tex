\documentclass{bschlangaul-aufgabe}
\bLadePakete{formale-sprachen,pumping-lemma}
\begin{document}
\bAufgabenMetadaten{
  Titel = {Aufgabe 3},
  Thematik = {Palindrom über Alphabet „abc“},
  Referenz = 66115-2020-H.T1-TA1-A3,
  RelativerPfad = Staatsexamen/66115/2020/09/Thema-1/Teilaufgabe-1/Aufgabe-3.tex,
  ZitatSchluessel = examen:66115:2020:09,
  BearbeitungsStand = mit Lösung,
  Korrektheit = unbekannt,
  Ueberprueft = {unbekannt},
  Stichwoerter = {Kontextfreie Sprache, Pumping-Lemma (Reguläre Sprache)},
  EinzelpruefungsNr = 66115,
  Jahr = 2020,
  Monat = 09,
  ThemaNr = 1,
  TeilaufgabeNr = 1,
  AufgabeNr = 3,
}

Seien \bAlphabet{a,b,c} und \bAusdruck{wc\hat{w}}{w \in \bMenge{a,b}*}.
Dabei ist $\hat{w}$ das zu $w$ gespiegelte Wort.
\index{Kontextfreie Sprache}
\footcite{examen:66115:2020:09}

\begin{enumerate}

%%
% a)
%%

\item Zeigen Sie, dass $L$ nicht regulär ist.
\index{Pumping-Lemma (Reguläre Sprache)}

\begin{bExkurs}[Pumping-Lemma für Reguläre Sprachen]
\bPumpingRegulaer
\end{bExkurs}

\begin{bAntwort}
$L$ ist regulär.
Dann gilt für $L$ das Pumping-Lemma.
Sei $j$ die Zahl aus
dem Pumping-Lemma.
Dann muss sich das Wort $a^j b c b a^j \in L$
aufpumpen lassen (da $|a^j b c b a^j| \geq j$).
$a^j b c b a^j = uvw$ ist eine passende Zerlegung laut Lemma.
Da $|uv| < j$, ist
$u = a^x$, $v = a^y$, $w = a^z b c b a^j$, wobei
$y > 0$ und
$x + y + z = j$.
Aber dann
$u v^0 w= a^{x+z} b c b a^j \notin L$, da $x + z < j$.
Widerspruch.
\bFussnoteUrl{https://userpages.uni-koblenz.de/~sofronie/gti-ss-2015/slides/endliche-automaten6.pdf}
\end{bAntwort}

%%
% b)
%%

\item Zeigen Sie, dass $L$ kontextfrei ist, indem Sie eine geeignete
Grammatik angeben und anschließend begründen, dass diese die Sprache $L$
erzeugt.

\begin{bAntwort}
\begin{bProduktionsRegeln}
S -> a S a | a C a | b S b | b C b,
C -> c
\end{bProduktionsRegeln}

\bAbleitung{S -> aSa -> abCba -> abcba}

\bAbleitung{S -> bSb -> bbSbb -> bbaSabb -> bbacabb}

\end{bAntwort}

\end{enumerate}
\end{document}
