\documentclass{bschlangaul-aufgabe}

\begin{document}
\bAufgabenMetadaten{
  Titel = {Aufgabe 4},
  Thematik = {Gödelisierung aller Registermaschinen (RAMs)},
  Referenz = 66115-2015-F.T2-A4,
  RelativerPfad = Staatsexamen/66115/2015/03/Thema-2/Aufgabe-4.tex,
  ZitatSchluessel = examen:66115:2015:03,
  BearbeitungsStand = mit Lösung,
  Korrektheit = unbekannt,
  Ueberprueft = {unbekannt},
  Stichwoerter = {Berechenbarkeit},
  EinzelpruefungsNr = 66115,
  Jahr = 2015,
  Monat = 03,
  ThemaNr = 2,
  AufgabeNr = 4,
}

Sei\index{Berechenbarkeit} \footcite{examen:66115:2015:03} M 0 , M 1 , .
. . eine Gödelisierung aller Registermaschinen (RAMs). Geben Sie für die
folgenden Mengen D 1 , D 2 , D 3 an, ob sie entscheidbar oder aufzählbar
sind. Begründen Sie Ihre Behauptungen, wobei Sie die Aufzählbarkeit und
Un- entscheidbarkeit des speziellen Halteproblems K 0 = {x ∈ N|M x haelt
bei Eingabe x} verwenden dürfen. D 1 = {x ∈ N|x < 9973und M x haelt bei
Eingabe x} D 2 = {x ∈ N|x ≥ 9973und M x haelt bei Eingabe x} D 3 = {x ∈
N|M x haelt nicht bei Eingabe x}
\footcite[Aufgabe 6]{theo:ab:4}

\begin{bAntwort}
D 1 ist eine endliche Menge und damit entscheidbar. Auch eine endliche
Teilmenge des Halteproblems. Anschaulich kann man sich dies so
verstellen: Man stellt dem Rechner eine Liste zur Verfügung, die alle
haltenden Maschinen M x mit x < 9973 enthält. Diese Liste kann zum
Beispiel vorab von einem Menschen erstellt worden sein, denn die Menge
der zu prüfenden Programme ist endlich.

D 2 x ≥ 9973 entscheidbar, L halt semi-entscheidbar → semi-entscheidbar
(Hier wäre auch eine Argumentation über die Cantorsche Paarungsfunktion
möglich). Es ist weiterhin nicht entscheidbar. Dazu betrachten wir dei
Reduktin des speziel- len Halteproblems H 0 : H 0 ≤ D 2 Für alle x <
9973 lassen wir M x durch eine Turingmaschine M y simulieren, die eine
höhere Nummer hat.

D 3 ist unentscheidbar, denn angenommen D 3 wäre semi-entscheidbar, dann
würde sofort folgen, dass L halt entscheidbar ist, da aus der
Semientscheidbarkeit von L halt und L halt die Entscheidbarkeit von L
halt folgen würde
\end{bAntwort}

\end{document}
