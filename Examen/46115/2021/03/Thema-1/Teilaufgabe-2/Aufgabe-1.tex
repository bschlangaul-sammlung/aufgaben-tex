\documentclass{bschlangaul-aufgabe}

\begin{document}
\bAufgabenMetadaten{
  Titel = {Aufgabe},
  Thematik = {Pseudo-Code Insertionsort, Bubblesort, Quicksort},
  Referenz = 46115-2021-F.T1-TA2-A1,
  RelativerPfad = Examen/46115/2021/03/Thema-1/Teilaufgabe-2/Aufgabe-1.tex,
  ZitatSchluessel = examen:46115:2021:03,
  BearbeitungsStand = mit Lösung,
  Korrektheit = unbekannt,
  Ueberprueft = {unbekannt},
  Stichwoerter = {Sortieralgorithmen},
  EinzelpruefungsNr = 46115,
  Jahr = 2021,
  Monat = 03,
  ThemaNr = 1,
  TeilaufgabeNr = 2,
  AufgabeNr = 1,
}

\begin{enumerate}

%%
% a)
%%

\item Geben Sie für folgende Sortierverfahren jeweils zwei Felder A und B
an, so dass das jeweilige Sortierverfahren angewendet auf A seine
Best-Case-Laufzeit und angewendet auf B seine Worst-Case-Laufzeit
erreicht. (Wir messen die Laufzeit durch die Anzahl der Vergleiche
zwischen Elementen der Eingabe.) Dabei soll das Feld A die Zahlen
1,2,...,7 genau einmal enthalten; das Feld B ebenso. Sie bestimmen also
nur die Reihenfolge der Zahlen.
\index{Sortieralgorithmen}
\footcite{examen:46115:2021:03}

Wenden Sie als Beleg für Ihre Aussagen das jeweilige Sortierverfahren
auf die Felder A und B an und geben Sie nach jedem größeren Schritt des
Algorithmus den Inhalt der Felder an.

Geben Sie außerdem für jedes Verfahren asymptotische Best- und
Worst-Case-Laufzeit für ein Feld der Länge n an.

Für drei der Sortierverfahren ist der Pseudocode angegeben. Beachten
Sie, dass die Feldindi- zes hier bei 1 beginnen. Die im Pseudocode
verwendete Unterroutine Swap(A, :, j) vertauscht im Feld A die Elemente
mit den Indizes i und j miteinander.

\begin{enumerate}
%%
% i)
%%

\item Insertionsort
%%
% ii)
%%

\item Bubblesort
%%
% iii)
%%

\item Quicksort

\end{enumerate}
Insertionsort(int[] A)
for 7 = 2 to A.length do
key = Alj]
i=j-1
while i>0 and Ali] > key do
Afi + 1] = Ali]
t=t-1
Ali + 1] = key

Bubblesort(int|] A)
n := length(A)
repeat
swapped = false
fori=lton—1do
if Ali — 1] > Alc] then
Swap(A,i — 1,7)

swapped := true

until not swapped

Quicksort(int[] A, @ = 1, r = A.length)
if 2<r then
m = Partition(A, 2, r)
| Quicksort(A, 2,m — 1)
Quicksort(A,m + 1,r)

int Partition (int|] A, int 2, intr)

pivot = Alr]

i=

for j =¢€tor—1do

if A[j] < pivot then

Swap(A, i, 5)
w=i+l

Swap(A,i,r)

return i

%%
% b)
%%

\item Geben Sie die asymptotische Best- und Worst-Case-Laufzeit von
Mergesort an.

\end{enumerate}
\end{document}
