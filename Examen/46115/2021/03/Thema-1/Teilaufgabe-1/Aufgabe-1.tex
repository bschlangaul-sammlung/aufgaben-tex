\documentclass{bschlangaul-aufgabe}

\begin{document}
\bAufgabenMetadaten{
  Titel = {Aufgabe 1},
  Thematik = {Kaugummi-Automat},
  Referenz = 46115-2021-F.T1-TA1-A1,
  RelativerPfad = Examen/46115/2021/03/Thema-1/Teilaufgabe-1/Aufgabe-1.tex,
  ZitatSchluessel = examen:46115:2021:03,
  BearbeitungsStand = nur Angabe,
  Korrektheit = unbekannt,
  Ueberprueft = {unbekannt},
  Stichwoerter = {Reguläre Sprache, Deterministisch endlicher Automat (DEA)},
  EinzelpruefungsNr = 46115,
  Jahr = 2021,
  Monat = 03,
  ThemaNr = 1,
  TeilaufgabeNr = 1,
  AufgabeNr = 1,
}

Ein moderner Kaugummi-Automat erhalte 20- und 50-Cent-Münzen. Eine
Packung Kaugummi kostet 120 Cent. Die interne Steuerung des
Kaugummi-Automaten verwendet einen deterministischen endlichen
Automaten, der die Eingabe als Folge von 20- und 50-Cent-Münzen (d. h.
als Wort über dem Alphabet {20,50}) erhält, und genau die Folgen
akzeptiert, die in der Summe 120 Cent ergeben.\index{Reguläre Sprache}
\footcite{examen:46115:2021:03}

\begin{enumerate}

%%
% (a)
%%

\item Geben Sie zwei Worte aus {20,50}* an, die der Automat akzeptiert.

%%
% (b)
%%

\item Zeichnen Sie einen deterministischen endlichen Automaten als
Zustandsgraph, der für die interne Steuerung des Kaugummi-Automaten
verwendet werden kann (d. h. der Automat akzeptiert genau alle Folgen
von 20- und 50-Cent-Münzen, deren Summe 120 ergibt).
\index{Deterministisch endlicher Automat (DEA)}

%%
% c)
%%

\item Geben Sie einen regulären Ausdruck an, der die akzeptierte Sprache
des Automaten erzeugt.
\end{enumerate}
\end{document}
