\documentclass{bschlangaul-aufgabe}
\bLadePakete{formale-sprachen}
\begin{document}
\bAufgabenMetadaten{
  Titel = {Aufgabe 2},
  Thematik = {Nonterminale SABCD, Terminale ab},
  Referenz = 46115-2021-F.T1-TA1-A2,
  RelativerPfad = Examen/46115/2021/03/Thema-1/Teilaufgabe-1/Aufgabe-2.tex,
  ZitatSchluessel = examen:46115:2021:03,
  BearbeitungsStand = mit Lösung,
  Korrektheit = unbekannt,
  Ueberprueft = {unbekannt},
  Stichwoerter = {Kontextfreie Sprache},
  EinzelpruefungsNr = 46115,
  Jahr = 2021,
  Monat = 03,
  ThemaNr = 1,
  TeilaufgabeNr = 1,
  AufgabeNr = 2,
}

\begin{enumerate}
%%
% a)
%%

\item Verwenden Sie den Algorithmus von Cocke, Younger und Kasami
(CYK-Algorithmus), um für die folgende kontextfreie Grammatik G = (V,D,
P, 5) mit Variablen V={s,A,B,C,D}, Terminalzeichen 3 = {a,b},
Produktionen\index{Kontextfreie Sprache}
\footcite{examen:46115:2021:03}

P={
S -> SB | AC |a,
A -> a,
B -> b,
C -> DD | AB,
D -> AB | DC | CD
}

und Startsymbol S zu prüfen, ob das Wort aabababb in der durch G
erzeugten Sprache liegt. Erläutern Sie dabei Ihr Vorgehen und den Ablauf
des CYK-Algorithmus.

%%
% b)
%%

\item Mit a’, wobei i € No = {0,1,2,...}, wird das Wort bezeichnet, das
aus der i-fachen Wie- derholung des Zeichens a besteht (d. h. a’ = € und
a? = aa‘~!, wobei e das leere Wort ist).

Sei die Sprache L definiert über dem Alphabet {a,b} als

L = {a"b™ | iE No, i > 1}.

Zeigen Sie, dass die Sprache Z nicht vom Typ 3 der Chomsky-Hierarchie
(d. h. nicht regulär) ist.
\end{enumerate}

\end{document}
