\documentclass{bschlangaul-aufgabe}
\bLadePakete{automaten}
\begin{document}
\bAufgabenMetadaten{
  Titel = {Aufgabe 2},
  Thematik = {NEA ab},
  Referenz = 46115-2021-F.T2-TA1-A2,
  RelativerPfad = Examen/46115/2021/03/Thema-2/Teilaufgabe-1/Aufgabe-2.tex,
  ZitatSchluessel = examen:46115:2021:03,
  BearbeitungsStand = mit Lösung,
  Korrektheit = unbekannt,
  Ueberprueft = {unbekannt},
  Stichwoerter = {Reguläre Sprache},
  EinzelpruefungsNr = 46115,
  Jahr = 2021,
  Monat = 03,
  ThemaNr = 2,
  TeilaufgabeNr = 1,
  AufgabeNr = 2,
}

Gegeben sei der folgende e-nichtdeterministische endliche Automat A über
dem Alphabet\index{Reguläre Sprache}
\footcite{examen:46115:2021:03}

\begin{center}
\begin{tikzpicture}[li automat]
  \node[state,initial] (q0) at (2.14cm,-1.57cm) {$q_0$};
  \node[state] (q1) at (6.29cm,-1.71cm) {$q_1$};
  \node[state] (q2) at (9.43cm,-1.71cm) {$q_2$};
  \node[state] (q3) at (2.14cm,-4.71cm) {$q_3$};
  \node[state,accepting] (q4) at (5.86cm,-4.71cm) {$q_4$};
  \node[state] (q5) at (9.71cm,-4.86cm) {$q_5$};

  \path (q0) edge[auto,bend left] node{$\varepsilon,a$} (q1);
  \path (q0) edge[auto] node{$a$} (q3);
  \path (q1) edge[auto,bend left] node{$b$} (q2);
  \path (q1) edge[auto,bend left] node{$a$} (q0);
  \path (q2) edge[auto] node{$a$} (q5);
  \path (q2) edge[auto,bend left] node{$\varepsilon$} (q0);
  \path (q2) edge[auto,bend left] node{$b$} (q1);
  \path (q3) edge[auto] node{$b$} (q4);
  \path (q5) edge[auto] node{$b$} (q4);
\end{tikzpicture}
\end{center}
\bFlaci{A54bbh2iz}

\begin{enumerate}

  %%
% a)
%%

\item Konstruieren Sie einen deterministischen endlichen Automaten für
A. Wenden Sie dafür die Potenzmengenkonstruktion an.

\begin{bAntwort}
\begin{center}
\begin{tikzpicture}[li automat]
  \node[state,initial] (q0) at (2.14cm,-2.14cm) {$q_0$};
  \node[state] (q013) at (5.71cm,-2cm) {$q_013$};
  \node[state,accepting] (q24) at (10.29cm,-5cm) {$q_24$};
  \node[state] (q0125) at (4.14cm,-4.71cm) {$q_0125$};
  \node[state] (q2) at (1.57cm,-6.29cm) {$q_2$};
  \node[state] (q12) at (6.29cm,-8.29cm) {$q_12$};

  \path (q0) edge[auto] node{$a$} (q013);
  \path (q0) edge[auto] node{$b$} (q2);
  \path (q013) edge[auto,loop above] node{$a$} (q013);
  \path (q013) edge[auto] node{$b$} (q24);
  \path (q24) edge[auto,bend left] node{$a$} (q0125);
  \path (q24) edge[auto] node{$b$} (q12);
  \path (q0125) edge[auto] node{$a$} (q013);
  \path (q0125) edge[auto,bend left] node{$b$} (q24);
  \path (q2) edge[auto] node{$a$} (q0125);
  \path (q2) edge[auto] node{$b$} (q12);
  \path (q12) edge[auto,loop above] node{$b$} (q12);
  \path (q12) edge[auto] node{$a$} (q0125);
\end{tikzpicture}
\end{center}
\bFlaci{Arnqcysfc}
\end{bAntwort}

%%
% b)
%%

\item Beschreiben Sie die von A akzeptierte Sprache L(A) mit eigenen
Worten und so einfach wie möglich.

\begin{bAntwort}
Endet auf ab, Präfix beliebig auch leer
\end{bAntwort}
\end{enumerate}

\end{document}
