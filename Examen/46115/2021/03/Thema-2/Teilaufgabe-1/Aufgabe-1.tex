\documentclass{bschlangaul-aufgabe}
\bLadePakete{automaten}
\begin{document}
\bAufgabenMetadaten{
  Titel = {Aufgabe 1},
  Thematik = {Alphabet abc},
  Referenz = 46115-2021-F.T2-TA1-A1,
  RelativerPfad = Examen/46115/2021/03/Thema-2/Teilaufgabe-1/Aufgabe-1.tex,
  ZitatSchluessel = 46115:2021:03,
  BearbeitungsStand = mit Lösung,
  Korrektheit = unbekannt,
  Ueberprueft = {unbekannt},
  Stichwoerter = {Reguläre Sprache},
  EinzelpruefungsNr = 46115,
  Jahr = 2021,
  Monat = 03,
  ThemaNr = 2,
  TeilaufgabeNr = 1,
  AufgabeNr = 1,
}
\index{Reguläre Sprache}
\footcite{46115:2021:03}

\begin{enumerate}
%%
% a)
%%

\item Betrachten Sie die formale Sprache $L \subseteq \{ a, b, c \}^*$:
aller Wörter, die entweder mit $b$ beginnen oder mit $a$ enden (aber
nicht beides gleichzeitig) und das Teilwort $cc$ enthalten. Entwerfen
Sie einen (vollständigen) deterministischen endlichen Automaten, der die
Sprache $L$ akzeptiert. (Hinweis: Es werden weniger als 10 Zustände
benötigt.)

\begin{bAntwort}

NEA:
\begin{center}
\begin{tikzpicture}[li automat]
  \node[state] (q0) at (3.29cm,-3.14cm) {$q_0$};
  \node[state] (q1) at (5.43cm,-3.14cm) {$q_1$};
  \node[state] (q2) at (7.71cm,-3.14cm) {$q_2$};
  \node[state,accepting] (q3) at (9.86cm,-3.14cm) {$q_3$};
  \node[state] (q4) at (3.29cm,-6cm) {$q_4$};
  \node[state] (q5) at (6cm,-5.86cm) {$q_5$};
  \node[state] (q6) at (8.57cm,-5.86cm) {$q_6$};
  \node[state,accepting] (q7) at (11cm,-6cm) {$q_7$};
  \node[state,initial] (q8) at (1.57cm,-5cm) {$q_8$};

  \path (q0) edge[auto] node{$b$} (q1);
  \path (q1) edge[auto] node{$c$} (q2);
  \path (q1) edge[auto,loop above] node{$a,b,c$} (q1);
  \path (q2) edge[auto] node{$c$} (q3);
  \path (q3) edge[auto,loop above] node{$a,b,c$} (q3);
  \path (q4) edge[auto] node{$c$} (q5);
  \path (q4) edge[auto,loop above] node{$a,b,c$} (q4);
  \path (q5) edge[auto] node{$c$} (q6);
  \path (q6) edge[auto] node{$a$} (q7);
  \path (q6) edge[auto,loop above] node{$a,b,c$} (q6);
  \path (q8) edge[auto] node{$\varepsilon$} (q4);
  \path (q8) edge[auto] node{$\varepsilon$} (q0);
\end{tikzpicture}
\end{center}
\bFlaci{Ar3pvv7ha}

konvertierter DEA:

\begin{center}
\begin{tikzpicture}[li automat]
  \node[state,initial] (q0) at (1.57cm,-3.86cm) {$q_0$};
  \node[state] (q3) at (3.29cm,-2.14cm) {$q_3$};
  \node[state] (q2) at (7.29cm,-2.14cm) {$q_2$};
  \node[state] (q7+q10) at (7.29cm,-6.86cm) {q7+q10};
  \node[state] (q4) at (4.29cm,-6.86cm) {$q_4$};
  \node[state,accepting] (q5+q8+q9) at (10.14cm,-2.14cm) {q5+q8+q9};
  \node[state] (q6) at (1.71cm,-6.86cm) {$q_6$};
  \node[state,accepting] (q1) at (10.29cm,-6.71cm) {$q_1$};

  \path (q0) edge[auto] node{$a$} (q6);
  \path (q0) edge[auto] node{$b$} (q3);
  \path (q0) edge[auto] node{$c$} (q4);
  \path (q3) edge[auto,bend left] node{$c$} (q2);
  \path (q3) edge[auto,loop above] node{$a,b$} (q3);
  \path (q2) edge[auto] node{$c$} (q5+q8+q9);
  \path (q2) edge[auto,bend left] node{$a,b$} (q3);
  \path (q7+q10) edge[auto,loop above] node{$b,c$} (q7+q10);
  \path (q7+q10) edge[auto,bend left] node{$a$} (q1);
  \path (q4) edge[auto] node{$c$} (q7+q10);
  \path (q4) edge[auto,bend left] node{$a,b$} (q6);
  \path (q5+q8+q9) edge[auto,loop above] node{$a,b,c$} (q5+q8+q9);
  \path (q6) edge[auto,bend left] node{$c$} (q4);
  \path (q6) edge[auto,loop left] node{$a,b$} (q6);
  \path (q1) edge[auto,bend left] node{$b,c$} (q7+q10);
  \path (q1) edge[auto,loop above] node{$a$} (q1);
\end{tikzpicture}
\end{center}
\bFlaci{Ai89m0txw}

\end{bAntwort}

%%
% b)
%%

\item Ist die folgende Aussage richtig? Begründen Sie Ihre Antwort.

„Jede Teilsprache einer regulären Sprache ist regulär, d. h. für ein
Alphabet  und formale Sprachen $L' \subseteq L \subseteq \Sigma^*$ ist
$L'$ regulär, falls $L$ regulär ist.“

\begin{bAntwort}
Ja. Reguläre Sprachen sind abgeschlossen unter dem Komplement und der
Vereinigung.
\end{bAntwort}

\end{enumerate}
\end{document}
