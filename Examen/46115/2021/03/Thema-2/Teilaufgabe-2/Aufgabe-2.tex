\documentclass{bschlangaul-aufgabe}
\bLadePakete{java,o-notation}
\begin{document}
\bAufgabenMetadaten{
  Titel = {Aufgabe 2},
  Thematik = {Java-Klasse Stack},
  Referenz = 46115-2021-F.T2-TA2-A2,
  RelativerPfad = Staatsexamen/46115/2021/03/Thema-2/Teilaufgabe-2/Aufgabe-2.tex,
  ZitatSchluessel = examen:46115:2021:03,
  BearbeitungsStand = mit Lösung,
  Korrektheit = unbekannt,
  Ueberprueft = {unbekannt},
  Stichwoerter = {Stapel (Stack), Implementierung in Java},
  EinzelpruefungsNr = 46115,
  Jahr = 2021,
  Monat = 03,
  ThemaNr = 2,
  TeilaufgabeNr = 2,
  AufgabeNr = 2,
}

\let\j=\bJavaCode
\let\O=\bONotationO

Gegeben sei die folgende Java-Implementierung eines Stacks.
\index{Stapel (Stack)}
\index{Implementierung in Java}
\footcite{examen:46115:2021:03}

\begin{minted}{java}
class Stack {
  private Item head;

  public Stack() {
    head = null;
  }

  public void push(int val) {
    if (head == null) {
      head = new Item(val, null);
    } else {
      head = new Item(val, head);
    }
  }

  public int pop() {
    // ...
  }

  public int size() {
    // ...
  }

  public int min() {
    // ...
  }

  class Item {
    private int val;
    private Item next;

    public Item(int val, Item next) {
      this.val = val;
      this.next = next;
    }
  }
}
\end{minted}

\begin{enumerate}

%%
% a)
%%

\item Implementieren Sie die Methode \j{pop} in einer objektorientierten
Programmiersprache Ihrer Wahl, die das erste Item des Stacks entfernt
und seinen Wert zurückgibt. Ist kein Wert im Stack enthalten, so soll
dies mit einer \j{IndexOutOfBoundsException} oder Ähnlichem gemeldet
werden.

Beschreiben Sie nun jeweils die notwendigen Änderungen an den bisherigen
Implementierungen, die für die Realisierung der folgenden Methoden
notwendig sind.

\begin{bAntwort}
\bJavaExamen[firstline=25,lastline=34]{46115}{2021}{03}{Stack}
\end{bAntwort}

%%
% b)
%%

\item \j{size} gibt in Laufzeit \O1 die Anzahl der enthaltenen Items
zurück.

\begin{bAntwort}
\bJavaExamen[firstline=13,lastline=38]{46115}{2021}{03}{Stack}
\end{bAntwort}

%%
% c)
%%

\item \j{min} gibt (zu jedem Zeitpunkt) in Laufzeit \O1 den Wert des
kleinsten Elements im Stack zurück.

\begin{bAntwort}
\bJavaExamen[firstline=13,lastline=23]{46115}{2021}{03}{Stack}
\bJavaExamen[firstline=40,lastline=43]{46115}{2021}{03}{Stack}
\end{bAntwort}

Sie dürfen jeweils alle anderen angegebenen Methoden der Klasse
verwenden, auch wenn Sie diese nicht implementiert haben. Sie können
anstelle von objektorientiertem Quellcode auch eine informelle
Beschreibung Ihrer Änderungen angeben.

\end{enumerate}

\begin{bAdditum}[Kompletter Java-Code]
\bJavaExamen{46115}{2021}{03}{Stack}
\bJavaTestDatei{examen/examen_46115/jahr_2021/fruehjahr/StackTest}
\end{bAdditum}
\end{document}
