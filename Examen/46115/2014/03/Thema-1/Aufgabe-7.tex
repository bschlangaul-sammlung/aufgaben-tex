\documentclass{bschlangaul-aufgabe}
\bLadePakete{baum}
\begin{document}
\bAufgabenMetadaten{
  Titel = {Aufgabe 7},
  Thematik = {Zahlenfolge in binärer Suchbaum, Min-Heap und AVL-Baum},
  Referenz = 46115-2014-F.T1-A7,
  RelativerPfad = Examen/46115/2014/03/Thema-1/Aufgabe-7.tex,
  ZitatSchluessel = examen:46115:2014:03,
  BearbeitungsStand = mit Lösung,
  Korrektheit = korrekt und überprüft,
  Ueberprueft = {überprüft mit den Online-Tools \url{https://visualgo.net/en/bst}, \url{https://www.cs.usfca.edu/~galles/visualization/Heap.html}},
  Stichwoerter = {Halde (Heap), Binärbaum, AVL-Baum, Min-Heap},
  EinzelpruefungsNr = 46115,
  Jahr = 2014,
  Monat = 03,
  ThemaNr = 1,
  AufgabeNr = 7,
}

Fügen Sie nacheinander die Zahlen 11, 1, 2, 13, 9, 10, 7, 5
\index{Halde (Heap)}\index{Binärbaum}\index{AVL-Baum}
\footcite{examen:46115:2014:03}

\begin{bProjektSprache}{Baum}
baum binär (
  setzte: 11 1 2 13 9 10 7 5;
  drucke;
)
\end{bProjektSprache}

\begin{enumerate}

%%
% (i)
%%

\item in einen leeren binären Suchbaum und zeichnen Sie den Suchbaum
jeweils nach dem Einfügen von „9“ und „5“

\begin{bAntwort}
\begin{bBaum}{Nach dem Einfügen von „9“}
\begin{tikzpicture}[b binaer baum]
\Tree
[.11
  [.1
    \edge[blank]; \node[blank]{};
    [.2
      \edge[blank]; \node[blank]{};
      [.9 ]
    ]
  ]
  [.13 ]
]
\end{tikzpicture}
\end{bBaum}

\begin{bBaum}{Nach dem Einfügen von „5“}
\begin{tikzpicture}[b binaer baum]
\Tree
[.11
  [.1
    \edge[blank]; \node[blank]{};
    [.2
      \edge[blank]; \node[blank]{};
      [.9
        [.7
          [.5 ]
          \edge[blank]; \node[blank]{};
        ]
        [.10 ]
      ]
    ]
  ]
  [.13 ]
]
\end{tikzpicture}
\end{bBaum}
\end{bAntwort}

%%
% (ii)
%%

\item in einen leeren Min-Heap ein, der bzgl. „$\leq$“ angeordnet ist und
geben Sie den Heap nach „9“ und nach „5“ an
\index{Min-Heap}

\begin{bAntwort}
\begin{bBaum}{Nach dem Einfügen von „9“}
\begin{tikzpicture}[b binaer baum]
\Tree
[.1
  [.9
    [.13 ]
    [.11 ]
  ]
  [.2 ]
]
\end{tikzpicture}
\end{bBaum}

\begin{bBaum}{Nach dem Einfügen von „5“}
\begin{tikzpicture}[b binaer baum]
\Tree
[.1
  [.5
    [.9
      [.13 ]
      \edge[blank]; \node[blank]{};
    ]
    [.11 ]
  ]
  [.2
    [.10 ]
    [.7 ]
  ]
]
\end{tikzpicture}
\end{bBaum}

\end{bAntwort}
%%
% (ii)
%%

\item in einen leeren AVL-Baum ein! Geben Sie den AVL Baum nach „2“ und
„5“ an und beschreiben Sie die ggf. notwendigen Rotationen beim Einfügen
dieser beiden Elemente!

\begin{bAntwort}
\begin{bBaum}{Nach dem Einfügen von „2“}
\begin{tikzpicture}[b binaer baum]
\Tree
[.\node[label=-2]{11};
  [.\node[label=+1]{1};
    \edge[blank]; \node[blank]{};
    [.\node[label=0]{2}; ]
  ]
  \edge[blank]; \node[blank]{};
]
\end{tikzpicture}
\end{bBaum}

\begin{bBaum}{Nach der Linksrotation}
\begin{tikzpicture}[b binaer baum]
\Tree
[.\node[label=-2]{11};
  [.\node[label=-1]{2};
    [.\node[label=0]{1}; ]
    \edge[blank]; \node[blank]{};
  ]
  \edge[blank]; \node[blank]{};
]
\end{tikzpicture}
\end{bBaum}

\begin{bBaum}{Nach der Rechtsrotation}
\begin{tikzpicture}[b binaer baum]
\Tree
[.\node[label=0]{2};
  [.\node[label=0]{1}; ]
  [.\node[label=0]{11}; ]
]
\end{tikzpicture}
\end{bBaum}

\begin{bBaum}{Nach dem Einfügen von „10“}
\begin{tikzpicture}[b binaer baum]
\Tree
[.\node[label=+2]{2};
  [.\node[label=0]{1}; ]
  [.\node[label=-1]{11};
    [.\node[label=+1]{9};
      \edge[blank]; \node[blank]{};
      [.\node[label=0]{10}; ]
    ]
    [.\node[label=0]{13}; ]
  ]
]
\end{tikzpicture}
\end{bBaum}

\begin{bBaum}{Nach der Rechtsrotation}
\begin{tikzpicture}[b binaer baum]
\Tree
[.\node[label=+2]{2};
  [.\node[label=0]{1}; ]
  [.\node[label=+2]{9};
    \edge[blank]; \node[blank]{};
    [.\node[label=0]{11};
      [.\node[label=0]{10}; ]
      [.\node[label=0]{13}; ]
    ]
  ]
]
\end{tikzpicture}
\end{bBaum}

\begin{bBaum}{Nach der Linksrotation}
\begin{tikzpicture}[b binaer baum]
\Tree
[.\node[label=0]{9};
  [.\node[label=-1]{2};
    [.\node[label=0]{1}; ]
    \edge[blank]; \node[blank]{};
  ]
  [.\node[label=0]{11};
    [.\node[label=0]{10}; ]
    [.\node[label=0]{13}; ]
  ]
]
\end{tikzpicture}
\end{bBaum}

\begin{bBaum}{Nach dem Einfügen von „5“}
\begin{tikzpicture}[b binaer baum]
\Tree
[.\node[label=-1]{9};
  [.\node[label=+1]{2};
    [.\node[label=0]{1}; ]
    [.\node[label=-1]{7};
      [.\node[label=0]{5}; ]
      \edge[blank]; \node[blank]{};
    ]
  ]
  [.\node[label=0]{11};
    [.\node[label=0]{10}; ]
    [.\node[label=0]{13}; ]
  ]
]
\end{tikzpicture}
\end{bBaum}
\end{bAntwort}
\end{enumerate}

\end{document}
