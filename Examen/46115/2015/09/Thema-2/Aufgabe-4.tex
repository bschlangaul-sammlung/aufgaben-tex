\documentclass{bschlangaul-aufgabe}
\bLadePakete{mathe,java,vollstaendige-induktion}
\begin{document}
\bAufgabenMetadaten{
  Titel = {Aufgabe 4},
  Thematik = {Methode function: Formale Verifikation - Induktionsbeweis},
  Referenz = 46115-2015-H.T2-A4,
  RelativerPfad = Staatsexamen/46115/2015/09/Thema-2/Aufgabe-4.tex,
  ZitatSchluessel = examen:46115:2015:09,
  BearbeitungsStand = mit Lösung,
  Korrektheit = korrekt,
  Ueberprueft = {unbekannt},
  Stichwoerter = {Vollständige Induktion},
  EinzelpruefungsNr = 46115,
  Jahr = 2015,
  Monat = 09,
  ThemaNr = 2,
  AufgabeNr = 4,
}

\let\m=\bInduktionMarkierung
\let\e=\bInduktionErklaerung

Gegeben\index{Vollständige Induktion}
\footcite{examen:46115:2015:09} sei die folgende Methode \bJavaCode{function}:

\bJavaExamen[firstline=4,lastline=9]{46115}{2015}{09}{Induktion}

\noindent
Beweisen Sie folgenden Zusammenhang mittels vollständiger Induktion:

\begin{displaymath}
\forall n \geq 1 \colon \text{function}(n) = f(n)\text{ mit }
f(n) := 1 - \frac{1}{n + 1}
\end{displaymath}

\noindent
Hinweis: Eventuelle Rechenungenauigkeiten, wie z. B. in Java, bei der
Behandlung von Fließkommazahlen (z. B. \bJavaCode{double}) sollen beim
Beweis nicht berücksichtigt werden - Sie dürfen also annehmen,
Fließkommazahlen würden mathematische Genauigkeit aufweisen.

\begin{bAntwort}
\bInduktionAnfang

$f(1) := 1 - \frac{1}{1 + 1} = 1 - \frac{1}{2} = \frac{1}{2}$

\bInduktionVoraussetzung

$f(n) := 1 - \frac{1}{n + 1}$

\bInduktionSchritt

\bPseudoUeberschrift{zu zeigen:}

$f(n + 1) := 1 - \frac{1}{(n + 1) + 1} = f(n)$

\bPseudoUeberschrift{Vorarbeiten (Java in Mathe umwandeln):}

$\text{function}(n) = \frac{1}{n \cdot (n + 1)} + f(n - 1)$

\begin{align*}
f(n + 1)
& = \frac{1}{\m{(n + 1)} \cdot
    (\m{(n + 1) }+ 1)} +
    f(\m{(n + 1)} - 1)
& \e{$n + 1$ eingesetzt}\\
%
& = \frac{1}{(n + 1) \cdot
    (\m{n + 2})} + f(\text{n})
& \e{vereinfacht}\\
%
& = \frac{1}{(n + 1) \cdot (n + 2)} +
    \m{1 - \frac{1}{n + 1}}
& \e{für $f(n)$ Formel eingesetzt}\\
%
& = 1 +
    \m{\frac{1}{(n + 1) \cdot (n + 2)}} -
    \frac{1}{n + 1}
& \e{1. Bruch an 2. Stelle geschrieben}\\
%
& = 1 +
    \frac{1}{(n + 1) \cdot (n + 2)} -
    \frac{1 \cdot \m{(n + 2)}}{(n + 1) \cdot \m{(n + 2)}}
& \e{2. Bruch mit $(n + 2)$ erweitert}\\
%
& = 1 +
    \frac{1 - (n + 2)}{(n + 1) \cdot (n + 2)}
& \e{die 2 Brüche subtrahiert}\\
%
& = 1 +
    \frac{1 - n \m{-} 2}{(n + 1) \cdot (n + 2)}
& \e{$-+2 = -2$}\\
%
& = 1 +
    \frac{\m{-1} - n}{(n + 1) \cdot (n + 2)}
& \e{$1-2=-1$}\\
%
& = 1 +
    \frac{\m{-1 \cdot (1 + n)}}{(n + 1) \cdot (n + 2)}
& \e{$(n + 1)$ ausgeklammert}\\
%
& = 1 +
    \left(\m{-1 \cdot} \frac{(1 + n)}{(n + 1) \cdot (n + 2)}\right)
& \e{minus vor den Bruch bringen}\\
%
& = 1 \m{-}
    \frac{(1 + n)}{(n + 1) \cdot (n + 2)}
& \e{plus minus ist minus}\\
%
& = 1 -
    \m{\frac{1}{n + 2}}
& \e{$(n + 1)$ gekürzt}\\
%
& = 1 -
    \frac{1}{\m{(n + 1)} + 1}
& \e{Umformen zur Verdeutlichung}\\
\end{align*}
\end{bAntwort}
\end{document}
