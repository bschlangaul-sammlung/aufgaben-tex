\documentclass{bschlangaul-aufgabe}
\bLadePakete{java}
\begin{document}
\bAufgabenMetadaten{
  Titel = {Aufgabe 6 (Stacks)},
  Thematik = {Mystery-Stacks},
  Referenz = 46115-2019-H.T1-A6,
  RelativerPfad = Staatsexamen/46115/2019/09/Thema-1/Aufgabe-6.tex,
  ZitatSchluessel = examen:46115:2019:09,
  BearbeitungsStand = mit Lösung,
  Korrektheit = unbekannt,
  Ueberprueft = {unbekannt},
  Stichwoerter = {Stapel (Stack)},
  EinzelpruefungsNr = 46115,
  Jahr = 2019,
  Monat = 09,
  ThemaNr = 1,
  AufgabeNr = 6,
}

\let\j=\bJavaCode

Gegeben sei die Implementierung eines Stacks ganzer Zahlen mit folgender
Schnittstelle:
\index{Stapel (Stack)}
\footcite{examen:46115:2019:09}

\bJavaExamen[firstline=3]{46115}{2019}{09}{mystery_stack/IntStack}

\noindent
Betrachten Sie nun die Realisierung der folgenden Datenstruktur
\j{Mystery}, die zwei Stacks benutzt.

\bJavaExamen[firstline=3,lastline=18]{46115}{2019}{09}{mystery_stack/Mystery}

\begin{enumerate}

%%
% (a)
%%

\item Skizzieren Sie nach jedem Methodenaufruf der im folgenden
angegebenen Befehlssequenz den Zustand der beiden Stacks eines Objekts
\j{m} der Klasse \j{Mystery}. Geben Sie zudem bei jedem Aufruf der
Methode \j{bar} an, welchen Wert diese zurückliefert.

\bJavaExamen[firstline=21,lastline=30]{46115}{2019}{herbst}{mystery_stack/Mystery}

\begin{bAntwort}
\def\e#1#2#3#4{
\bJavaCode{#1} &
\{ #2 \} &
\{ #3 \} &
#4\\
}

\begin{tabular}{l|l|l|l}
Code & Stack b & Stack b & Rückgabewert \\\hline\hline

\e{m.foo(3);}{3}{}{}
\e{m.foo(5);}{5, 3}{}{}
\e{m.foo(4);}{4, 5, 3}{}{}

\e{m.bar();}{}{5, 4}{3}

\e{m.foo(7);}{7}{5, 4}{}

\e{m.bar();}{7}{4}{5}

\e{m.foo(2);}{2, 7}{}{}

\e{m.bar();}{2, 7}{}{4}
\e{m.bar();}{}{2}{7}
\end{tabular}

\end{bAntwort}

%%
% (b)
%%

\item Sei $n$ die Anzahl der in einem Objekt der Klasse \j{Mystery}
gespeicherten Werte. Im folgenden wird gefragt, wieviele Aufrufe von
Operationen der Klasse \j{IntStack} einzelne Aufrufe von Methoden der
Klasse \j{Mystery} verursachen. Begründen Sie jeweils Ihre Antwort.

\begin{enumerate}

%%
% (i)
%%

\item Wie viele Aufrufe von Operationen der Klasse \j{IntStack}
verursacht die Methode \j{foo(x)} im besten Fall?

\begin{bAntwort}
Einen Aufruf, nämlich \bJavaCode{a.push(i)}
\end{bAntwort}

%%
% (ii)
%%

\item Wie viele Aufrufe von Operationen der Klasse \j{IntStack}
verursacht die Methode \j{foo(x)} im schlechtesten Fall?

\begin{bAntwort}
Einen Aufruf, nämlich \bJavaCode{a.push(i)}
\end{bAntwort}

%%
% (ii)
%%

\item Wie viele Aufrufe von Operationen der Klasse \j{IntStack}
verursacht die Methode \j{bar()} im besten Fall?

\begin{bAntwort}
Wenn der Stack \j{b} nicht leer ist, dann werden zwei Aufrufe
benötigt, nämlich \j{b.isEmpty()} und \j{b.pop()}
\end{bAntwort}

%%
% (iv)
%%

\item Wie viele Aufrufe von Operationen der Klasse \j{IntStack}
verursacht die Methode \j{bar()} im schlechtesten Fall?

\begin{bAntwort}
Wenn der Stack \j{b} leer ist, dann liegen all $n$ Objekte im Stack
\j{a}. Die Objekte im Stack \j{a} werden in der \j{while}-Schleife nach
\j{b} verschoben. Pro Objekt sind drei Aufrufe nötig, also $3 \cdot n$.
\j{b.isEmpty()} (erste Zeile in der Methode) und \j{b.pop()} (letzte
Zeile in der Methode) wird immer aufgerufen. Wenn alle Objekt von \j{a}
nach \j{b} verschoben wurden, wird zusätzlich noch einmal in der
Bedingung der \j{while}-Schleife \j{a.isEmpty()} aufgerufen. Im
schlechtesten Fall werden also $3 \cdot n + 3$ Operationen der Klasse
\j{IntStack} aufgerufen.
\end{bAntwort}
\end{enumerate}

%%
% (c)
%%

\item Welche allgemeinen Eigenschaften werden durch die Methoden \j{foo}
und \j{bar} realisiert? Unter welchem Namen ist diese Datenstruktur
allgemein bekannt?

\begin{bAntwort}
\begin{description}
\item[\j{foo()}] Legt das Objekt auf den Stack \j{a}. Das Objekt wird in
die Warteschlange eingereiht. Die Methode müsste eigentlich
\j{enqueue()} heißen.

\item[\j{bar()}] Verschiebt alle Objekte vom Stack \j{a} in umgekehrter
Reihenfolge in den Stack \j{b}, aber nur dann, wenn Stack \j{b} leer
ist. Entfernt dann den obersten Wert aus dem Stack \j{b} und gibt ihn
zurück. Das zuerst eingereihte Objekt wird aus der Warteschlange
entnommen. Die Methode müsste eigentlich \j{dequeue()} heißen.
\end{description}

Die Datenstruktur ist unter dem Namen Warteschlange oder Queue bekannt
\end{bAntwort}
\end{enumerate}

\end{document}
