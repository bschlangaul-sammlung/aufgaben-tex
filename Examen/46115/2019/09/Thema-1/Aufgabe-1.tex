\documentclass{bschlangaul-aufgabe}
\bLadePakete{formale-sprachen,automaten}
\begin{document}
\bAufgabenMetadaten{
  Titel = {Aufgabe 1},
  Thematik = {Reguläre Sprachen)},
  Referenz = 46115-2019-H.T1-A1,
  RelativerPfad = Staatsexamen/46115/2019/09/Thema-1/Aufgabe-1.tex,
  ZitatSchluessel = examen:46115:2019:09,
  BearbeitungsStand = mit Lösung,
  Korrektheit = unbekannt,
  Ueberprueft = {unbekannt},
  Stichwoerter = {Reguläre Sprache},
  EinzelpruefungsNr = 46115,
  Jahr = 2019,
  Monat = 09,
  ThemaNr = 1,
  AufgabeNr = 1,
}

\let\M=\bMenge

\begin{enumerate}

%%
% (a)
%%

\item Geben Sie einen regulären Ausdruck für die Sprache über dem
Alphabet \M{a,b} an, die genau alle Wörter enthält, die eine gerade
Anzahl $a$’s haben.\index{Reguläre Sprache}
\footcite{examen:46115:2019:09}

\begin{bAntwort}
$b^*(ab^*ab^*)^*$
\end{bAntwort}

%%
% (b)
%%

\item Sei $A$ der folgende DEA über dem Alphabet \M{a,b}:

\begin{center}
\begin{tikzpicture}[->,node distance=2cm]
\node[state,initial]
(1) {1};
\node[state,right=of 1]
(2) {2};
\node[state,right=of 2]
(3) {3};
\node[state,right=of 3,accepting]
(4) {4};
\node[state,below=of 1]
(5) {5};
\node[state,right=of 5]
(6) {6};
\node[state,right=of 6]
(7) {7};
\node[state,right=of 7]
(8) {8};

\path (1) edge[above] node{a} (2);
\path (2) edge[above] node{b} (3);
\path (3) edge[above] node{a} (4);
\path (4) edge[above,loop right] node{a,b} (4);
\path (8) edge[right] node{a} (4);
\path (8) edge[above] node{b} (7);
\path (7) edge[above] node{b} (6);
\path (6) edge[above] node{a} (5);
\path (1) edge[left] node{b} (5);
\path (5) edge[above,bend right] node{b} (7);
\path (2) edge[above,bend left] node{a} (5);
\path (5) edge[above,bend left] node{a} (2);
\path (6) edge[right] node{b} (2);
\path (3) edge[above] node{b} (6);
\path (7) edge[above] node{a} (4);
\end{tikzpicture}
\end{center}

Führen Sie den Minimierungsalgorithmus für $A$ durch und geben Sie den
minimalen äquivalenten DEA für $L(A)$ als Zeichnung an.

\end{enumerate}
\end{document}
