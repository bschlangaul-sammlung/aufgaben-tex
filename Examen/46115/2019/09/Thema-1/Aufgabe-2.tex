\documentclass{bschlangaul-aufgabe}
\bLadePakete{formale-sprachen,chomsky-normalform}
\begin{document}
\bAufgabenMetadaten{
  Titel = {Aufgabe 2},
  Thematik = {Sprachen L1 und L2},
  Referenz = 46115-2019-H.T1-A2,
  RelativerPfad = Examen/46115/2019/09/Thema-1/Aufgabe-2.tex,
  ZitatSchluessel = examen:46115:2019:09,
  BearbeitungsStand = mit Lösung,
  Korrektheit = unbekannt,
  Ueberprueft = {unbekannt},
  Stichwoerter = {Kontextfreie Sprache},
  EinzelpruefungsNr = 46115,
  Jahr = 2019,
  Monat = 09,
  ThemaNr = 1,
  AufgabeNr = 2,
}

\let\schrittE=\bChomskyUeberErklaerung

\begin{enumerate}

%%
% (a)
%%

\item Betrachten Sie die folgenden Sprachen:
\index{Kontextfreie Sprache}
\footcite{examen:46115:2019:09}

\bAusdruck[L_1]{a^n b^{2n} c^{2m} d^m}{n, m \in \mathcal{N}}

\bAusdruck[L_2]{a^n b^{2n} c^{2n} d^n}{n \in \mathcal{N}}

Zeigen Sie für Zı und La, ob sie kontextfrei sind oder nicht. Für den Beweis von Kontext-
Freiheit in dieser Frage reicht die Angabe eines Automaten oder einer Grammatik. (Beschrei-

ben Sie dann die Konstruktionsidee des Automaten oder der Grammatik.) Für den Beweis
von Nicht-Kontext-Freiheit verwenden Sie eine der üblichen Methoden.

%%
% (b)
%%

\item Eine kontextfreie Grammatik ist in Chomsky-Normalform, falls die folgenden
Bedingungen erfüllt sind:

\begin{itemize}
\item alle Regeln sind von der Form X — YZ oder X — o mit
Nichtterminalzeichen X,Y, Z und Terminalzeichen o,

\item alle Nichtterminalzeichen sind erreichbar vom Startsymbol und

\item alle Nichtterminalzeichen sind erzeugend, d. h. für jedes
Nichtterminalzeichen X gibt es ein Wort w über dem Terminalalphabet, so
dass X =>* w.
\end{itemize}

Bringen Sie die folgende Grammatik in Chomsky-Normalform.

\begin{bProduktionsRegeln}
S -> A A B | C D | a b c,
A -> A A A A | a,
B -> B B | S,
C -> C C C | C C,
D -> d,
\end{bProduktionsRegeln}
\bFlaci{Gf7f9tp7z}

Das Startsymbol der Grammatik ist S, das Terminalalphabet ist {a,b,c,d} und die Menge
der Nichtterminalzeichen ist {S,A,B,C,D}.

\begin{bAntwort}

\begin{enumerate}
\item \schrittE{1}

\bNichtsZuTun

\item \schrittE{2}

Eine rechte Seite in der C vorkommt, lässt sich wegen \bProduktionen{C
-> C C C | C C} nicht ableiten, weil es zu einer Endlosschleife kommt.
Wir entfernen die entsprechenden Regeln.

\begin{bProduktionsRegeln}
S -> A A B | a b c,
A -> A A A A | a,
B -> B B | A A B | a b c,
\end{bProduktionsRegeln}

\item \schrittE{3}

\begin{bProduktionsRegeln}
S -> A A B | T_a T_b T_c,
A -> A A A A | a,
B -> B B | A A B | T_a T_b T_c,
T_a  -> a,
T_b  -> b,
T_c  -> c,
\end{bProduktionsRegeln}

\item \schrittE{4}
\end{enumerate}

\begin{bProduktionsRegeln}
S   -> A S_1 | T_a S_2,
A   -> A A_1 | a,
B   -> B B | A S_1 | T_a S_2,
T_a  -> a,
T_b  -> b,
T_c  -> c,
S_1 -> A B,
S_2 -> T_b T_c,
A_1 -> A A_2,
A_2 -> A A,
\end{bProduktionsRegeln}
\end{bAntwort}

\end{enumerate}
\end{document}
