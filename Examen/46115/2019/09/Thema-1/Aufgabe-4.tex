\documentclass{bschlangaul-aufgabe}
\bLadePakete{java,mathe,sortieren}
\begin{document}
\bAufgabenMetadaten{
  Titel = {Aufgabe 4},
  Thematik = {(Sortierverfahren)},
  Referenz = 46115-2019-H.T1-A4,
  RelativerPfad = Staatsexamen/46115/2019/09/Thema-1/Aufgabe-4.tex,
  ZitatSchluessel = examen:46115:2019:09,
  BearbeitungsStand = mit Lösung,
  Korrektheit = unbekannt,
  Ueberprueft = {unbekannt},
  Stichwoerter = {Selectionsort},
  EinzelpruefungsNr = 46115,
  Jahr = 2019,
  Monat = 09,
  ThemaNr = 1,
  AufgabeNr = 4,
}

In der folgenden Aufgabe soll ein Feld $A$ von ganzen Zahlen
\emph{aufsteigend} sortiert werden. Das Feld habe $n$ Elemente $A[1]$
bis $A[n]$. Der folgende Algorithmus sei gegeben:
\index{Selectionsort}
\footcite{examen:46115:2019:09}

\begin{minted}{componentpascal}
var A : array[1..n] of integer;

procedure selection_sort
var i, j, smallest, tmp : integer;
begin
  for j := 1 to n-1 do begin
    smallest := j;
    for i := j + 1 to n do begin
      if A[i] < A[smallest] then
        smallest := i;
    end
    tmp = A[j];
    A[j] = A[smallest];
    A[smallest] = tmp;
  end
end
\end{minted}
\begin{enumerate}

%%
%
%%

\item Sortieren Sie das folgende Feld mittels des Algorithmus. Notieren
Sie alle Werte, die die Variable \emph{smallest} jeweils beim Durchlauf
der inneren Schleife annimmt. Geben Sie die Belegung des Feldes nach
jedem Durchlauf der äußeren Schleife in einer neuen Zeile an.

\begin{bAntwort}
\def\u#1{\bPseudoUeberschrift{nach #1. Durchlauf ( $j = #1$ )}}
\let\v=\bVertauschen

%%
%
%%

Ausgang

\v{>27 32 <3 6 17 44 42 29 8 14}

%%
%
%%

\u{1}

smallest: (1) 3

\v{3 >32 27 <6 17 44 42 29 8 14}

%%
%
%%

\u{2}

smallest: (2) 3 4

\v{3 6 >27 32 17 44 42 29 <8 14}

%%
%
%%

\u{3}

smallest: (3) 5 9

\v{3 6 8 >32 <17 44 42 29 27 <14}

%%
%
%%

\u{4}

smallest: (4) 5 10

\v{3 6 8 14 17 44 42 29 27 32}

%%
%
%%

\u{5}

smallest: (5) -

\v{3 6 8 14 17 >44 42 29 <27 32}

%%
%
%%

\u{6}

smallest: (6) 7 8 9

\v{3 6 8 14 17 27 >42 <29 44 32}

%%
%
%%

\u{7}

smallest: (7) 8

\v{3 6 8 14 17 27 29 >42 44 <32}

%%
%
%%

\u{8}

smallest: (8) 10

\v{3 6 8 14 17 27 29 32 44 42}

%%
%
%%

\u{9}

smallest: (9) 10

\v{3 6 8 14 17 27 29 32 >44 <42}

%%
%
%%

\bPseudoUeberschrift{fertig}

\v{3 6 8 14 17 27 29 32 42 44}

\end{bAntwort}

%%
%
%%

\item Der Wert der Variablen \emph{smallest} wird bei jedem Durchlauf
der äußeren Schleife mindestens ein Mal neu gesetzt. Wie muss das Feld
$A$ beschaffen sein, damit der Variablen \emph{smallest} ansonsten
niemals ein Wert zugewiesen wird? Begründen Sie Ihre Antwort.

\begin{bAntwort}
Wenn das Feld bereits aufsteigend sortiert ist, dann nimmt die Variable
\emph{smallest} in der innneren Schleife niemals einen neuen Wert an.
\end{bAntwort}

%%
%
%%

\item Welche Auswirkung auf die Sortierung oder auf die Zuweisungen an
die Variable \emph{smallest} hat es, wenn der Vergleich in Zeile 9 des
Algorithmus statt $A[i] < A[\text{smallest}]$ lautet $A[i] \leq
A[\text{smallest}]$? Begründen Sie Ihre Antwort.

\begin{bAntwort}
Der Algorithmus sortiert dann nicht mehr \emph{stabil}, \dh die
Eingabereihenfolge von Elementen mit \emph{gleichem Wert} wird beim
Sortieren nicht mehr \emph{bewahrt}.
\end{bAntwort}

%%
%
%%

\item Betrachten Sie den Algorithmus unter der Maßgabe, dass Zeile 9 wie
folgt geändert wurde:

\begin{minted}{componentpascal}
if A[i] > A[smallest] then
\end{minted}

Welches Ergebnis berechnet der Algorithmus nun?

\begin{bAntwort}
Der Algorithmus sortiert jetzt absteigend.
\end{bAntwort}

%%
%
%%

\item Betrachten Sie die folgende \emph{rekursive} Variante des
Algorithmus. Der erste Parameter ist wieder das zu sortierende Feld, der Parameter
$n$ ist die Größe des Feldes und der Parameter $index$ ist eine ganze
Zahl. Die Funktion \mintinline{componentpascal}{min_index(A, x, y)} berechnet
für $1 \leq x \leq y \leq n$ den Index des kleinsten Elements aus der
Menge \mintinline{componentpascal}{{A[x], A[x+1],...,A[y]}}

\begin{minted}{componentpascal}
procedure rek_selection_sort(A, n, index : integer)
var k, tmp : integer;
begin
if (Abbruchbedingung) then return;
  k = min_index(A, index, n);
  if k <> index then begin
  tmp := A[k];
  A[k] := A[index];
  A[index] := tmp;
  end
  (rekursiver Aufruf)
end
\end{minted}

Der initiale Aufruf des Algorithmus lautet:
\mintinline{componentpascal}{rek_selection_sort(A, n, 1)}

Vervollständigen Sie die fehlenden Angaben in der Beschreibung des
Algorithmus für

\begin{itemize}
\item die Abbruchbedingung in Zeile 4 und

\begin{bAntwort}
\texttt{n = index} bzw \texttt{n == index}

Begründung: Wenn der aktuelle Index so groß ist wie die Anzahl der
Elemente im Feld, dann muss / darf abgebrochen werden, denn dann ist das
Feld soriert.
\end{bAntwort}

\item den rekursiven Aufruf in Zeile 11.

\begin{bAntwort}
\mintinline{componentpascal}{rek_selection_sort(A, n, index + 1)}

Am Ende der Methode wurde an die Index-Position \emph{index} das
kleinste Element gesetzt, jetzt muss an die nächste Index-Position
(\emph{index + 1}) der kleinste Wert, der noch nicht sortieren Zahlen,
gesetzt werden.
\end{bAntwort}
\end{itemize}

Begründen Sie Ihre Antworten.
\end{enumerate}

\bJavaExamen[firstline=3]{46115}{2019}{09}{SelectionSort}
\end{document}
