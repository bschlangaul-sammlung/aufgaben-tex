\documentclass{bschlangaul-aufgabe}
\bLadePakete{graph}
\begin{document}
\bAufgabenMetadaten{
  Titel = {Aufgabe 8},
  Thematik = {Maximaler Spannbaum mit Jarník/Prim},
  Referenz = 46115-2019-H.T2-A8,
  RelativerPfad = Staatsexamen/46115/2019/09/Thema-2/Aufgabe-8.tex,
  ZitatSchluessel = examen:46115:2019:09,
  BearbeitungsStand = nur Angabe,
  Korrektheit = unbekannt,
  Ueberprueft = {unbekannt},
  Stichwoerter = {Algorithmus von Prim},
  EinzelpruefungsNr = 46115,
  Jahr = 2019,
  Monat = 09,
  ThemaNr = 2,
  AufgabeNr = 8,
}

\begin{enumerate}

%%
% (a)
%%

\item Durch folgende Adjazenzmatrix sei ein ungerichteter Graph $G$ mit
Kantenlängen gegeben.\index{Algorithmus von Prim}
\footcite{examen:46115:2019:09}

\begin{bGraphenFormat}
a: 0 0
b: 3 3
c: 2 0
d: 8 0
e: 0 5
f: 6 2
g: 6 5
h: 3 1
i: 5 4
a -- c: 7
a -- e: 9
a -- f: 2
a -- h: 4
b -- f: 3
b -- h: 5
b -- i: 2
c -- d: 4
c -- e: 4
d -- g: 2
d -- h: -3
e -- g: 0
f -- g: 8
f -- h: 10
f -- i: 2
i -- i: 20
\end{bGraphenFormat}

\begin{displaymath}
\begin{blockarray}{ccccccccc}
    &  a &  b &  c &  d &  e &  f &  g &  h &  i \\
\begin{block}{c(cccccccc)}
  a &  0 &  0 &  7 &  0 &  9 &  2 &  0 &  4 &  0 \\
  b &  0 &  0 &  0 &  0 &  0 &  3 &  0 &  5 &  2 \\
  c &  7 &  0 &  0 &  4 &  4 &  0 &  0 &  0 &  0 \\
  d &  0 &  0 &  4 &  0 &  0 &  0 &  2 & -3 &  0 \\
  e &  9 &  0 &  4 &  0 &  0 &  0 &  0 &  0 &  0 \\
  f &  2 &  3 &  0 &  0 &  0 &  0 &  8 & 10 &  2 \\
  g &  0 &  0 &  0 &  2 &  0 &  8 &  0 &  0 &  0 \\
  h &  4 &  5 &  0 & -3 &  0 & 10 &  0 &  0 &  0 \\
  i &  0 &  2 &  0 &  0 &  0 &  2 &  0 &  0 &  0 \\
\end{block}
\end{blockarray}
\end{displaymath}

\begin{center}
\begin{tikzpicture}[li graph]
\node (a) at (0,0) {a};
\node (b) at (3,3) {b};
\node (c) at (2,0) {c};
\node (d) at (8,0) {d};
\node (e) at (0,5) {e};
\node (f) at (6,2) {f};
\node (g) at (6,5) {g};
\node (h) at (3,1) {h};
\node (i) at (5,4) {i};

\path (a) edge node {7} (c);
\path (a) edge node {9} (e);
\path (a) edge node {2} (f);
\path (a) edge node {4} (h);
\path (b) edge node {3} (f);
\path (b) edge node {5} (h);
\path (b) edge node {2} (i);
\path (c) edge node {4} (d);
\path (c) edge node {4} (e);
\path (d) edge node {2} (g);
\path (d) edge node {-3} (h);
\path (e) edge node {0} (g);
\path (f) edge node {8} (g);
\path (f) edge node {10} (h);
\path (f) edge node {2} (i);
%\path (i) edge node {20} (i);
\end{tikzpicture}
\end{center}

Wenden Sie den Algorithmus von Jarník/Prim auf $G$ ausgehend von Knoten
$d$ an, um einen Spannbaum $T$ mit \emph{maximalem} Gewicht zu
berechnen. Erstellen Sie dazu eine Tabelle mit zwei Spalten und stellen
Sie jeden einzelnen Schritt des Verfahrens in einer eigenen Zeile dar.
Geben Sie in der ersten Spalte denjenigen Knoten $v$ an, der vom
Algorithmus als nächstes in $T$ aufgenommen wird (dieser sog.
„schwarze“ Knoten ist damit fertiggestellt). Führen Sie in der zweiten Spalte alle
anderen vom aktuellen Baum $T$ direkt erreichbaren Knoten $v$ (sog.
„graue Randknoten“) auf.

Geben Sie in der Tabelle Knoten stets als Tripel $(v, \delta, v.\pi)$
an, mit $v$ als Knotenname, $v.\pi$ als aktueller Vorgängerknoten
(anderer Knoten der Kante) und $\delta$ als Länge der Kante $\{v,
v.\pi\}$.

%%
% (b)
%%

\item Sei $G = (V, E,w)$ ein Graph mit Kantenlängen $w: E \rightarrow
\mathbb{N}$ und $T$ ein Spannbaum von $G$ mit maximalem Gewicht.
Beweisen oder widerlegen Sie die folgende Aussage:

Längste (einfache) Wege zwischen zwei Knoten $u,v \in V$ enthalten nur
Kanten aus $T$.
\end{enumerate}

\end{document}
