\documentclass{bschlangaul-aufgabe}
\bLadePakete{java}
\begin{document}
\bAufgabenMetadaten{
  Titel = {Aufgabe 7 (Heapify)},
  Thematik = {Heapify},
  Referenz = 46115-2019-H.T2-A7,
  RelativerPfad = Staatsexamen/46115/2019/09/Thema-2/Aufgabe-7.tex,
  ZitatSchluessel = examen:46115:2019:09,
  BearbeitungsStand = mit Lösung,
  Korrektheit = unbekannt,
  Ueberprueft = {unbekannt},
  Stichwoerter = {Halde (Heap)},
  EinzelpruefungsNr = 46115,
  Jahr = 2019,
  Monat = 09,
  ThemaNr = 2,
  AufgabeNr = 7,
}

Schreiben Sie in Pseudocode eine Methode \bJavaCode{heapify(int[] a)},
welche im übergebenen Array der Länge $n$ die Heapeigenschaft in
$\mathcal{O}(n)$ Schritten herstellt. D. h. als Ergebnis soll in $a$
gelten, dass $a[i] \leq a[2i + 1]$ und $a[i] \leq a[i + 2]$.
\index{Halde (Heap)}
\footcite{examen:46115:2019:09}

\begin{bAntwort}
\bJavaExamen[firstline=3]{46115}{2019}{09}{Heapify}
\end{bAntwort}

\end{document}
