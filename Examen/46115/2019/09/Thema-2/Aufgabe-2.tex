\documentclass{bschlangaul-aufgabe}
\bLadePakete{formale-sprachen,automaten}
\begin{document}
\bAufgabenMetadaten{
  Titel = {Aufgabe 2},
  Thematik = {Rechtslineare Grammatik},
  Referenz = 46115-2019-H.T2-A2,
  RelativerPfad = Staatsexamen/46115/2019/09/Thema-2/Aufgabe-2.tex,
  ZitatSchluessel = examen:46115:2019:09,
  BearbeitungsStand = mit Lösung,
  Korrektheit = unbekannt,
  Ueberprueft = {unbekannt},
  Stichwoerter = {Reguläre Sprache},
  EinzelpruefungsNr = 46115,
  Jahr = 2019,
  Monat = 09,
  ThemaNr = 2,
  AufgabeNr = 2,
}

\let\m=\bMenge

Gegeben\index{Reguläre Sprache}
\footcite{examen:46115:2019:09} ist die rechtslineare Grammatik
$G = (\m{a, b}, \m{S, A, B, C, D}, S, P)$ mit
\begin{bProduktionsRegeln}
S -> aA,
A -> bB,
A -> aD,
B -> aC,
B -> bB,
C -> bD,
C -> b,
D -> aA
\end{bProduktionsRegeln}. Sei $L$ die von $G$ erzeugte Sprache.
\bFlaci{Gpkv4ansc}

\begin{enumerate}

%%
% (a)
%%

\item Zeichnen Sie einen nichtdeterministischen endlichen Automaten, der
$L$ akzeptiert!

\begin{bAntwort}
\begin{center}
\begin{tikzpicture}[->,node distance=2cm]
\node[state,initial] (S) {S};
\node[state,above right=of S] (A) {A};
\node[state,below right=of S] (B) {B};
\node[state,right=of B] (C) {C};
\node[state,right=of A] (D) {D};
\node[state,accepting,above right=of C] (E) {E};

\path (A) edge[above,bend left] node{a} (D);
\path (A) edge[left] node{b} (B);
\path (B) edge[above,loop below] node{b} (B);
\path (B) edge[above] node{a} (C);
\path (C) edge[above] node{b} (E);
\path (C) edge[right] node{b} (D);
\path (D) edge[above,bend left] node{a} (A);
\path (S) edge[above] node{a} (A);

\end{tikzpicture}
\end{center}
\bFlaci{Ahh979eqz}
\end{bAntwort}

%%
% (b)
%%

\item Konstruieren Sie auf nachvollziehbare Weise einen regulären
Ausdruck $\alpha$ mit $L(\alpha)=L$!

\begin{bAntwort}
$(ab+ab|(a|b+)+ab+ab)$
(von Flaci automatisch konvertiert)
\end{bAntwort}

\end{enumerate}
\end{document}
