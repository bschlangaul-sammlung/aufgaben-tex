\documentclass{bschlangaul-aufgabe}
\bLadePakete{automaten,minimierung}

\begin{document}
\bAufgabenMetadaten{
  Titel = {Aufgabe 1},
  Thematik = {Reguläre Sprache},
  Referenz = 46115-2020-F.T1-A1,
  RelativerPfad = Staatsexamen/46115/2020/03/Thema-1/Aufgabe-1.tex,
  ZitatSchluessel = examen:46115:2020:03,
  BearbeitungsStand = mit Lösung,
  Korrektheit = unbekannt,
  Ueberprueft = {unbekannt},
  Stichwoerter = {Reguläre Sprache, Reguläre Ausdrücke},
  EinzelpruefungsNr = 46115,
  Jahr = 2020,
  Monat = 03,
  ThemaNr = 1,
  AufgabeNr = 1,
}

\let\f=\bFussnote
\let\l=\bLeereZelle
\def\p#1#2{(#1, #2)}
\def\P#1#2{\textbf{(#1, #2)}}
\let\k=\bAutomatenKante

\begin{enumerate}

%%
% (a)
%%

\item Betrachten Sie die formale Sprache $L \subseteq \{ 0, 1 \}^*$
aller Wörter, die $01$ oder $110$ als Teilwort enthalten.
\index{Reguläre Sprache}
\footcite{examen:46115:2020:03}

Geben Sie einen regulären Ausdruck für die Sprache $L$ an.
\index{Reguläre Ausdrücke}

\begin{bAntwort}
(0|1)*(01|110)(0|1)*
\end{bAntwort}

%%
% (b)
%%

\item Entwerfen Sie einen (vollständigen) deterministischen endlichen
Automaten, der die Sprache $L$ aus Teilaufgabe (a) akzeptiert. (Hinweis:
es werden nicht mehr als 6 Zustände benötigt.)

\begin{bAntwort}
\begin{center}
\begin{tikzpicture}[->,node distance=2cm]
\node[state,initial] (0) {$z_0$};
\node[state,below of=0] (1) {$z_1$};
\node[state,right of=0] (2) {$z_2$};
\node[state,right of=2] (3) {$z_3$};
\node[state,below of=3,accepting] (4) {$z_4$};

\path (0) edge[left] node{0} (1);
\path (0) edge[above] node{1} (2);
\path (1) edge[above,loop left] node{1} (1);
\path (1) edge[above] node{1} (4);
\path (2) edge[above] node{0} (1);
\path (2) edge[above] node{1} (3);
\path (3) edge[above,loop right] node{1} (3);
\path (3) edge[right] node{0} (4);
\path (4) edge[above,loop right] node{0,1} (4);
\end{tikzpicture}
\end{center}

\bFlaci{A54gek0vz}
\end{bAntwort}

%%
% (c)
%%

\item Minimieren Sie den folgenden deterministischen endlichen Automaten:

Machen Sie dabei Ihren Rechenweg deutlich!

\begin{center}
\begin{tikzpicture}[->,node distance=1.5cm]
\node[state] (a) {a};
\node[state,below=of a,initial] (b) {b};
\node[state,right=of a,accepting] (c) {c};
\node[state,right=of b,accepting] (d) {d};
\node[state,right=of c] (e) {e};
\node[state,right=of d] (f) {f};
\node[state,right=of e,accepting] (g) {g};
\node[state,right=of f,accepting] (h) {h};

\path (a) edge[above] node{0} (c);
\path (a) edge[right,pos=0.2] node{1} (d);
\path (b) edge[above] node{1} (d);
\path (b) edge[left,pos=0.2] node{0} (c);
\path (c) edge[above,pos=0.2] node{1} (f);
\path (c) edge[right,bend left] node{0} (d);
\path (d) edge[below,pos=0.2] node{1} (e);
\path (d) edge[left,bend left] node{0} (c);
\path (e) edge[above,pos=0.2] node{0} (h);
\path (e) edge[above] node{1} (c);
\path (f) edge[above,pos=0.2] node{0} (g);
\path (f) edge[above] node{1} (d);
\path (g) edge[above] node{0} (e);
\path (g) edge[right,bend left] node{1} (h);
\path (h) edge[above] node{0} (f);
\path (h) edge[left,bend left] node{1} (g);
\end{tikzpicture}
\end{center}
\bFlaci{Ajpw4j73w}

\begin{bAntwort}
\begin{center}
\begin{tabular}{|c||c|c|c|c|c|c|c|c|}
\hline
a & \l  & \l  & \l  & \l  & \l  & \l  & \l  & \l  \\ \hline
b &     & \l  & \l  & \l  & \l  & \l  & \l  & \l  \\ \hline
c & \f1 & \f1 & \l  & \l  & \l  & \l  & \l  & \l  \\ \hline
d & \f1 & \f1 &     & \l  & \l  & \l  & \l  & \l  \\ \hline
e & \f2 & \f2 & \f1 & \f1 & \l  & \l  & \l  & \l  \\ \hline
f & \f3 & \f3 & \f1 & \f1 &    & \l  & \l  & \l  \\ \hline
g & \f1 & \f1 & \f2 & \f2 & \f1 & \f1 & \l  & \l  \\ \hline
h & \f1 & \f1 & \f2 & \f2 & \f1 & \f1 &    & \l  \\ \hline\hline
% & a   & b   & c   & d   & e   & f   & g  & h   \\ \hline
  & a   & b   & \underline{c}   & \underline{d}   & e   & f   & \underline{g}   & \underline{h}   \\ \hline
\end{tabular}
\end{center}

\bFussnoten

Die Zustandpaare werden aufsteigend sortiert notiert.

\begin{liUebergangsTabelle}{0}{1}
\P ab & \p cc     & \p dd \\
\p ae & \p ch \f2 & \p cd \\
\p af & \p cg \f3 & \p dd \\
\p be & \p ch \f2 & \p cd \\
\p bf & \p cg \f3 & \p dg \\
\P cd & \P cd     & \P ef \\
\p cg & \p de \f2 & \p ef \\
\p ch & \p df \f2 & \p ff \\
\p dg & \p ce \f2 & \p ee \\
\p dh & \p cf \f2 & \p ef \\
\P ef & \P gh     & \P cd \\
\P gh & \P ef     & \P gh \\
\end{liUebergangsTabelle}

\begin{center}
\begin{tikzpicture}[->,node distance=1.5cm]
\node[state,initial] (ab) {ab};
\node[state,right=of ab] (cd) {cd};
\node[state,right=of cd] (ef) {ef};
\node[state,right=of ef,accepting] (gh) {gh};

\k {ab} {cd} {0,1} {above}
\k {cd} {cd} {0} {above,loop above}
\k {cd} {ef} {1} {above,bend left}
\k {ef} {cd} {1} {above,bend left}
\k {gh} {ef} {0} {above,bend left}
\k {ef} {gh} {0} {above,bend left}
\k {gh} {gh} {1} {above,loop above}
\end{tikzpicture}
\end{center}

\bFlaci{Arzvh5kyz}
\end{bAntwort}

%%
% (d)
%%

\item Ist die folgende Aussage richtig oder falsch? Begründen Sie Ihre
Antwort!

„Zu jeder regulären Sprache $L$ über dem Alphabet $\Sigma$ gibt es eine
Sprache $L' \subseteq \Sigma^*$, die $L$ enthält (\dh $L \subseteq
L’$) und nicht regulär ist.“
\end{enumerate}
\end{document}
