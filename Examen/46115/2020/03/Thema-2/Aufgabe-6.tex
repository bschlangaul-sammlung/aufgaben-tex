\documentclass{bschlangaul-aufgabe}
\bLadePakete{mathe,syntax,pseudo,grafik}

\begin{document}
\bAufgabenMetadaten{
  Titel = {Aufgabe 6},
  Thematik = {Nächstes rot-blaues Paar auf der x-Achse},
  Referenz = 46115-2020-F.T2-A6,
  RelativerPfad = Staatsexamen/46115/2020/03/Thema-2/Aufgabe-6.tex,
  IdentischeAufgabe = Staatsexamen/66115/2020/03/Thema-2/Aufgabe-8.tex,
  ZitatSchluessel = examen:66115:2020:03,
  BearbeitungsStand = mit Lösung,
  Korrektheit = unbekannt,
  Ueberprueft = {unbekannt},
  Stichwoerter = {Algorithmische Komplexität (O-Notation)},
  EinzelpruefungsNr = 46115,
  Jahr = 2020,
  Monat = 03,
  ThemaNr = 2,
  AufgabeNr = 6,
}
\def\blauesZeichen{\bullet}
\def\rotesZeichen{x}
\def\b#1{\node at (#1,0){\blauesZeichen};}
\def\r#1{\node at (#1,0){\rotesZeichen};}

Gegeben seien zwei nichtleere Mengen $R$ und $B$ von roten bzw. blauen
Punkten auf der x-Achse. Gesucht ist der minimale euklidische Abstand
$d(r, b)$ über alle Punktepaare $(r,b)$ mit $r \in R$ und $b \in B$.
Hier ist eine Beispielinstanz:\index{Algorithmische Komplexität (O-Notation)}
\footcite{examen:66115:2020:03}

\begin{center}
\begin{tikzpicture}
\draw (0,0) -- (8,0);
\node at (-1,0){x-Achse};
\r{0.1}
\r{0.5}
\b{2}
\node[label=r] at (3,0) (rot) {\rotesZeichen};
\node[label=b] at (3.7,0) {\blauesZeichen} edge[bend left,swap] (rot);
\r{8}
\b{5}
\b{5.2}
\b{5.4}
\b{6}

\node[label=blau] at (9,0.5) {\blauesZeichen};
\node[label=rot] at (9,-0.5) {\rotesZeichen};
\end{tikzpicture}
\end{center}

\noindent
Die Eingabe wird in einem Feld $A$ übergeben. Jeder Punkt $A[i]$ mit $1
\leq i \leq n$ hat eine x-Koordinate $A[i].x$ und eine Farbe $A[i].color
\in \{ \text{rot}, \text{blau} \}$. Das Feld $A$ ist nach x-Koordinate
sortiert, \dh es gilt $A[1].x < A[2].x < \dots < A[n].x$, wobei $n = |R|
+ |B|$.

\begin{enumerate}

%%
% (a)
%%

\item Geben Sie in Worten einen Algorithmus an, der den gesuchten
Abstand in $\mathcal{O}(n)$ Zeit berechnet.

\begin{bAntwort}
\bPseudoUeberschrift{Pseudo-Code}

\begin{algorithm}[H]
\newcommand{\meinkommentar}[1]{\texttt{\scriptsize #1}}
\SetCommentSty{meinkommentar}
$d_{min} := \max$ \tcp*{Setze $d_{min}$ zuerst auf einen maximalen Wert.}

\For(\tcp*{Iteriere über die Indizes des Punkte-Arrays $P$ bis zum vorletzten
Index $P[n-1]$}){$i$ in $0 \dots$ vorletzter Index}{

  \If(\tcp*{Berechne den Abstand nur, wenn die Punkte unterschiedliche Farben haben}){$P[n].color \neq P[n + 1].color$}{
    $d = P[n + 1].x - P[n].x$\\
    \If{$d < d_{min}$}{$d_{min} = d$}
  }
}
\caption{Minimaler Euklidischer Abstand}
\end{algorithm}

\bPseudoUeberschrift{Java}

\bJavaExamen[firstline=30,lastline=41]{66115}{2020}{03}{RedBluePairCollection}
\end{bAntwort}

%%
% (b)
%%

\item Begründen Sie kurz die Laufzeit Ihres Algorithmus.

\begin{bAntwort}
Da das Array der Länge $n$ nur einmal durchlaufen wird, ist die Laufzeit
$\mathcal{O}(n)$ sichergestellt.
\end{bAntwort}

%%
% (c)
%%

\item Begründen Sie die Korrektheit Ihres Algorithmus.

\begin{bAntwort}
In $d_{min}$ steht am Ende der gesuchte Wert (sofern nicht $d_{min} =
Integer.MAX\_VALUE$ geblieben ist)
\end{bAntwort}

%%
% (d)
%%

\item Wir betrachten nun den Spezialfall, dass alle blauen Punkte links
von allen roten Punkten liegen. Beschreiben Sie in Worten, wie man in
dieser Situation den gesuchten Abstand in $o(n)$ Zeit berechnen kann.
(Ihr Algorithmus darf also insbesondere nicht Laufzeit $\Theta(n)$
haben.)

\begin{bAntwort}
Zuerst müssen wir den letzten blauen Punkt finden. Das ist mit einer
binären Suche möglich. Wir beginnen mit dem ganzen Feld als Suchbereich
und betrachten den mittleren Punkt. Wenn er blau ist, wiederholen wir
die Suche in der zweiten Hälfte des Suchbereichs, sonst in der ersten,
bis wir einen blauen Punkt gefolgt von einem roten Punkt gefunden haben.

Der gesuchte minimale Abstand ist dann der Abstand zwischen dem
gefundenen blauen und dem nachfolgenden roten Punkt. Die Binärsuche hat
eine Worst-case-Laufzeit von $\mathcal{O}(\log n)$.
\end{bAntwort}
\end{enumerate}

\begin{bAdditum}[Komplette Java-Klasse]
\bJavaExamen{66115}{2020}{03}{RedBluePairCollection}
\end{bAdditum}

\end{document}
