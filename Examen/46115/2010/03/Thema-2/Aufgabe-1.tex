\documentclass{bschlangaul-aufgabe}
\bLadePakete{formale-sprachen,automaten}
\begin{document}
\bAufgabenMetadaten{
  Titel = {Aufgabe 1},
  Thematik = {Alphabet ab},
  Referenz = 46115-2010-F.T2-A1,
  RelativerPfad = Examen/46115/2010/03/Thema-2/Aufgabe-1.tex,
  ZitatSchluessel = examen:46115:2010:03,
  BearbeitungsStand = mit Lösung,
  Korrektheit = unbekannt,
  Ueberprueft = {unbekannt},
  Stichwoerter = {Reguläre Sprache},
  EinzelpruefungsNr = 46115,
  Jahr = 2010,
  Monat = 03,
  ThemaNr = 2,
  AufgabeNr = 1,
}

\def\z#1{$z_{#1}$}
\let\m=\bMenge

\section{Aufgabe 1
\index{Reguläre Sprache}
\footcite{examen:46115:2010:03}}
\begin{enumerate}

\item Gegeben ist die folgende Sprache L1 über dem Alphabet \bAlphabet{a,b}:

$L1 = \m{w \in \Sigma^* \, | \,
\text{die Anzahl der } a \text{ in } w \text{ ist gerade und } b \text{ kommt in } w \text{ genau einmal vor}}$.

\begin{enumerate}

%%
% a)
%%

\item Geben Sie einen deterministischen endlichen Automaten an, der die
Sprache $L1$ akzeptiert.

\begin{bAntwort}
\begin{center}
\begin{tikzpicture}[->,node distance=4cm]
\node[state,initial] (0) {\z0};
\node[state,right of=0] (1) {\z1};
\node[state,below of=0,accepting] (2) {\z2};
\node[state,right of=2] (3) {\z3};
\node[state,above right=2cm of 2] (t) {\z{t}};

\path (0) edge[bend left,above] node{a} (1);
\path (1) edge[bend left,above] node{a} (0);

\path (2) edge[bend left,above] node{a} (3);
\path (3) edge[bend left,above] node{a} (2);

\path (0) edge[right] node{b} (2);
\path (1) edge[right] node{b} (3);

\path (2) edge[right] node{b} (t);
\path (3) edge[right] node{b} (t);
\path (t) edge[loop above,right,pos=0.8] node{a,b} (t);
\end{tikzpicture}
\end{center}
\bFlaci{Ahfqe3eb7}
\end{bAntwort}

%%
% b)
%%

\item Geben Sie einen regulären Ausdruck an, der die Sprache $L1$
beschreibt.

\begin{bAntwort}
(aa)*(b|aba)(aa)*
\end{bAntwort}
\end{enumerate}

\item Die folgende Sprache $L2$ ist eine Erweiterung von $L1$:

$L1 = \m{w \in \Sigma^* \, | \, \text{die Anzahl der } a \text{ in } w \text{ ist gerade und die Anzahl der } b \text{ in } w \text{ ist ungerade}}.$

%%
% c)
%%
\begin{enumerate}
\item Geben Sie einen deterministischen endlichen Automaten an, der die
Sprache $L2$ akzeptiert.

\begin{bAntwort}
\bFussnoteUrl{https://www.informatik.uni-hamburg.de/TGI/lehre/vl/SS14/FGI1/Folien/fgi1_v2_handout.pdf}

gu = gerade Anzahl a’s, ungerade Anzahl b’s

ug = ungerade Anzahl a’s, gerade Anzahl b’s

\begin{center}
\begin{tikzpicture}[->,node distance=4cm]
\node[state,initial] (gg) {\z{gg}};
\node[state,right of=gg] (ug) {\z{ug}};
\node[state,below of=ug] (uu) {\z{uu}};
\node[state,below of=gg,accepting] (gu) {\z{gu}};

\path (gg) edge[bend left,above] node{0} (ug);
\path (ug) edge[bend left,above] node{0} (gg);

\path (gg) edge[bend left,left] node{1} (gu);
\path (gu) edge[bend left,left] node{1} (gg);

\path (gu) edge[bend left,above] node{0} (uu);
\path (uu) edge[bend left,above] node{0} (gu);

\path (ug) edge[bend left,left] node{1} (uu);
\path (uu) edge[bend left,left] node{1} (ug);
\end{tikzpicture}
\end{center}

\bFlaci{Af0vcjys9}
\end{bAntwort}

%%
% d)
%%

\item Geben Sie eine rechtslineare Grammatik an, die die Sprache $L2$
erzeugt.

\begin{bAntwort}
\begin{bProduktionsRegeln}
A -> a B | b D | b,
B -> b C | a A,
C -> a D | a | b B,
D -> b A | a C,
\end{bProduktionsRegeln}
\bFlaci{Ahfqe3eb7}
\end{bAntwort}

\end{enumerate}
\end{enumerate}
\end{document}
