\documentclass{bschlangaul-aufgabe}
\bLadePakete{automaten,formale-sprachen,potenzmengen-konstruktion}
\begin{document}
\bAufgabenMetadaten{
  Titel = {Aufgabe 1},
  Thematik = {Alphabet ab, vorvorletztes Zeichen a},
  Referenz = 46115-2016-H.T1-A1,
  RelativerPfad = Staatsexamen/46115/2016/09/Thema-1/Aufgabe-1.tex,
  ZitatSchluessel = examen:46115:2016:09,
  BearbeitungsStand = mit Lösung,
  Korrektheit = unbekannt,
  Ueberprueft = {unbekannt},
  Stichwoerter = {Potenzmengenalgorithmus},
  EinzelpruefungsNr = 46115,
  Jahr = 2016,
  Monat = 09,
  ThemaNr = 1,
  AufgabeNr = 1,
}

\let\p=\bPotenzmenge
\let\s=\bZustandsnameGross
\begin{enumerate}

%%
% a)
%%

\item Konstruieren Sie aus dem NEA mit der Potenzmengenkonstruktion
einen (deterministischen) EA, der dieselbe Sprache
akzeptiert.\index{Potenzmengenalgorithmus}
\footcite{examen:46115:2016:09}

\begin{center}
\begin{tikzpicture}[->,node distance=3cm]
\node[state,initial]
(0) {$z_0$};

\node[state,right of=0]
(1) {$z_1$};

\node[state,right of=1]
(2) {$z_2$};

\node[state,right of=2,accepting]
(3) {$z_3$};

\path (0) edge[above] node{a} (1);
\path (1) edge[above] node{a,b} (2);
\path (2) edge[above] node{a,b} (3);
\path (0) edge[above,loop] node{a,b} (0);
\end{tikzpicture}
\bFlaci{Aiqxdazuw}
\end{center}

\begin{bAntwort}
\def\z#1{
  \bZustandsMengenSammlung{#1}{{
    {0} {z0}
    {1} {z0, z1}
    {2} {z0, z1, z2}
    {3} {z0, z2}
    {4} {z0, z1, z2, z3}
    {5} {z0, z3}
    {6} {z0, z2, z3}
    {7} {z0, z1, z3}
  }}
}
\let\p=\bPotenzmenge

\begin{tabular}{l|l|l|l}
Name & Zustandsmenge & Eingabe a & Eingabe b \\\hline\hline
\s{0} &
\z{0} &
\z{1} &
\z{0} \\

\s{1} &
\z{1} &
\z{2} &
\z{3} \\

\s{2} &
\z{2} &
\z{4} &
\z{6} \\

\s{3} &
\z{3} &
\z{7} &
\z{5} \\

\s{4} &
\z{4} &
\z{4} &
\z{6} \\

\s{5} &
\z{5} &
\z{1} &
\z{0} \\

\s{6} &
\z{6} &
\z{7} &
\z{5} \\

\s{7} &
\z{7} &
\z{2} &
\z{3} \\
\end{tabular}

\begin{center}
\begin{tikzpicture}[->,node distance=2cm,y=-1cm]
\node[state]
(0) at (0,0) {\s{0}};

\node[state]
(1) at (2,0) {\s{1}};

\node[state]
(2) at (4,0) {\s{2}};

\node[state]
(3) at (3,4) {\s{3}};

\node[state,accepting]
(4) at (6,0) {\s{4}};

\node[state,accepting]
(5) at (0,2) {\s{5}};

\node[state,accepting]
(6) at (4,2) {\s{6}};

\node[state,accepting]
(7) at (6,4) {\s{7}};

\path (0) edge[above] node{a} (1);
\path (0) edge[above,loop] node{b} (0);

\path (1) edge[above] node{a} (2);
\path (1) edge[above left] node{b} (3);

\path (2) edge[above] node{a} (4);
\path (2) edge[left] node{b} (6);

\path (3) edge[above,bend left] node{a} (7);
\path (3) edge[above] node{b} (5);

\path (4) edge[above,loop] node{a} (4);
\path (4) edge[above] node{b} (6);

\path (5) edge[above] node{a} (1);
\path (5) edge[left] node{b} (0);

\path (6) edge[above] node{a} (7);
\path (6) edge[below] node{b} (5);

\path (7) edge[above right] node{a} (2);
\path (7) edge[above,bend left] node{b} (3);
\end{tikzpicture}
\bFlaci{Aiqnk06c9}
\end{center}
\end{bAntwort}

%%
% b)
%%

\item Beschreiben Sie möglichst einfach, welche Sprachen von den
folgenden regulären Ausdrücken beschrieben werden:

\begin{enumerate}
\item $(a | b)^*a$

\begin{bAntwort}
Sprache mit dem Alphabet \bAlphabet{a, b}: Alle Wörter der Sprache
enden auf $a$.
\end{bAntwort}

\item $(a|b)^*a(a|b)^*a(a|b)^*$

\begin{bAntwort}
Sprache mit dem Alphabet \bAlphabet{a, b}: Alle Wörter der Sprache
enthalten mindestens 2 $a$’s.
\end{bAntwort}

\item $(a|b)^*a(bb)^*a(a|b)^*$

\begin{bAntwort}
Sprache mit dem Alphabet \bAlphabet{a, b}: Alle Wörter der Sprache
enthalten mindestens 2 $a$’s, die von einer geradzahligen Anzahl von
$b$’s getrennt sind.
\end{bAntwort}
\end{enumerate}
\end{enumerate}

\end{document}
