\documentclass{bschlangaul-aufgabe}
\bLadePakete{komplexitaetstheorie}
\begin{document}
\bAufgabenMetadaten{
  Titel = {Aufgabe 2},
  Thematik = {SUBSET SUM, Raumausstattungsunternehmen},
  Referenz = 46115-2016-F.T2-A2,
  RelativerPfad = Examen/46115/2016/03/Thema-2/Aufgabe-2.tex,
  ZitatSchluessel = examen:46115:2016:03,
  BearbeitungsStand = mit Lösung,
  Korrektheit = unbekannt,
  Ueberprueft = {unbekannt},
  Stichwoerter = {Komplexitätstheorie, Polynomialzeitreduktion},
  EinzelpruefungsNr = 46115,
  Jahr = 2016,
  Monat = 03,
  ThemaNr = 2,
  AufgabeNr = 2,
}

\def\ssp{\bProblemName{}}

Ein Raumausstattungsunternehmen steht immer wieder vor dem Problem,
feststellen zu müssen, ob ein gegebener rechteckiger Fußboden mit
rechteckigen Teppichresten ohne Verschnitt ausgelegt werden kann. Alle
Längen sind hier ganzzahlige Meterbeträge. Haben sie beispielsweise zwei
Reste der Größen $3 \times 5$ und einen Rest der Größe $2 \times 5$, so
kann ein Fußboden der Größe $8 \times 5$ ausgelegt werden.
\index{Komplexitätstheorie}
\footcite{examen:46115:2016:03}

Das Unternehmen beauftragt eine Softwarefirma mit der Entwicklung eines
Programms, welches diese Frage für beliebige Größen von Fußboden und
Teppichresten entscheiden soll. Bei der Abnahme weist die
Softwarefirma darauf hin, dass das Programm im wesentlichen alle
Möglichkeiten durchprobiert und daher für große Eingaben schnell
ineffizient wird. Auf die Frage, ob man das nicht besser machen könne,
lautet die Antwort, dass das vorgelegte Problem NP-vollständig sei und
daher nach derzeitigem Kenntnisstand der theoretischen Informatik nicht
mehr zu erwarten sei.\footcite[Seite 15, Aufgabe 12]{theo:ab:4}

\begin{bExkurs}[\ssp]
\bProblemSubsetSum
\end{bExkurs}

\begin{enumerate}

%%
% (a)
%%

\item Fixieren Sie ein geeignetes Format für Instanzen des Problems und
geben Sie konkret an, wie die obige Beispielinstanz in diesem Format
aussieht.

\begin{bAntwort}
Problem $L$
\begin{description}
\item[1. Alternative]
$I = \{ x_1, y_1, \dots, x_n, y_n, c_x, c_y \}$

\item[2. Alternative]
$I = \{ w_1, \dots, w_n, c \}$
\end{description}
\end{bAntwort}

%%
% (b)
%%

\item Begründen Sie, dass das Problem in NP liegt.

\begin{bAntwort}
Es existiert ein nichtdeterministischer Algorithmus der das Problem in
Polynomialzeit entscheidet:

\begin{itemize}
\item nichtdeterministisch Untermenge raten ($\mathcal{O}(n)$)
\item Prüfe: ($\mathcal{O}(n$))
\end{itemize}

\begin{description}
\item[1. Alternative]

Elementsumme der Produkte ($x_i, y_i$) aus Untermenge $= c$

\item[2. Alternative]

Elementsumme der Untermenge $= c$
\end{description}
\end{bAntwort}

%%
% (c)
%%

\item Begründen Sie, dass das Problem NP-schwer ist durch Reduktion vom
NP-vollständigen Problem SUBSET-SUM.
\index{Polynomialzeitreduktion}

\begin{bAntwort}
\bPolynomiellReduzierbar{\ssp}{L}

\begin{description}

%%
%
%%

\item[1. Alternative]

Die Funktion $f$ ersetzt jedes $w_i$ durch $w_i$ , $1$ und $c$ durch
$c$, $1$ und startet TM für $L$

\begin{description}
\item[Berechenbarkeit:]

Hinzufügen von $1$ für jedes Element, offensichtlich in Polynomialzeit

\item[Korrektheit:]

$w \in \ssp \Leftrightarrow f (w) \in L$, offensichtlich, selbes Problem
mit lediglich anders notierter Eingabe
\end{description}

%%
%
%%

\item[2. Alternative]

Die Funktion $f$ startet TM für $L$

\begin{description}
\item[Berechenbarkeit:]

Identität, offensichtlich in Polynomialzeit

\item[Korrektheit:]

$w \in \ssp \Leftrightarrow f (w) \in L$ offensichtlich, selbes Problem

\end{description}

\end{description}
\end{bAntwort}

\end{enumerate}
\end{document}
