\documentclass{bschlangaul-aufgabe}
\bLadePakete{formale-sprachen,automaten}
\begin{document}
\bAufgabenMetadaten{
  Titel = {Aufgabe 4},
  Thematik = {Turingmaschinen},
  Referenz = 46115-2013-F.T1-A4,
  RelativerPfad = Examen/46115/2013/03/Thema-1/Aufgabe-4.tex,
  ZitatSchluessel = examen:46115:2013:03,
  BearbeitungsStand = mit Lösung,
  Korrektheit = unbekannt,
  Ueberprueft = {unbekannt},
  Stichwoerter = {Turing-Maschine},
  EinzelpruefungsNr = 46115,
  Jahr = 2013,
  Monat = 03,
  ThemaNr = 1,
  AufgabeNr = 4,
}

\section{Aufgabe 4
\index{Turing-Maschine}
\footcite{examen:46115:2013:03}}

Sei
\bAusdruck{u v}
{
  u \in \bMenge{a,b}^*,
  v \in \bMenge{c,d}^*,
  \#_a(u) = \#_c(v)
  \text{ und }
  \#_b(u) = \#_d(v))
}
wobei $\#_a(u)$ die Anzahl der in $u$ vorkommenden $a$’s ist.

\begin{enumerate}

%%
% a)
%%

\item Geben Sie eine Turingmaschine $M$ an, die $L$ erkennt.
Beschreiben Sie in Worten, wie Ihre Turingmaschine arbeitet.

\begin{bAntwort}
noch nicht fertig ... akzeptiert auch abcd!d
\begin{center}
\begin{tikzpicture}[li turingmaschine]
  \node[state,initial] (q0) at (2.29cm,-5.57cm) {$q_0$};
  \node[state] (q1) at (7.71cm,-3cm) {$q_1$};
  \node[state] (q2) at (7cm,-10cm) {$q_2$};
  \node[state] (q4) at (11.57cm,-5.86cm) {$q_4$};
  \node[state] (q3) at (7.29cm,-5.86cm) {$q_3$};
  \node[state,accepting] (q5) at (7.57cm,-4.29cm) {$q_5$};

  \bTuringKante[above]{q0}{q1}{
    a, A, R;
  }

  \bTuringKante[above]{q0}{q2}{
    b, B, R;
  }

  \bTuringKante[above,loop above]{q1}{q1}{
    a, a, R;
    b, b, R;
    d, d, R;
    C, C, R;
    D, D, R;
  }

  \bTuringKante[above]{q1}{q4}{
    c, C, R;
  }

  \bTuringKante[above]{q2}{q4}{
    d, D, R;
  }

  \bTuringKante[above,loop above]{q2}{q2}{
    a, a, R;
    c, c, R;
    b, b, R;
    C, C, R;
    D, D, R;
  }

  \bTuringKante[above,loop above]{q4}{q4}{
    c, c, R;
    d, d, R;
  }

  \bTuringKante[above]{q4}{q3}{
    LEER, LEER, L;
  }

  \bTuringKante[above,loop above]{q3}{q3}{
    d, d, L;
    c, c, L;
    C, C, L;
    D, D, L;
    A, A, L;
    B, B, L;
  }

  \bTuringKante[above]{q3}{q0}{
    a, a, N;
    b, b, N;
  }

  \bTuringKante[above]{q3}{q5}{
    LEER, LEER, N;
  }
\end{tikzpicture}
\end{center}
\bFlaci{Ajhi5w0ha}
\end{bAntwort}

%%
% b)
%%

\item Welche Laufzeit (Zeitkomplexität) hat Ihre Turingmaschine (in
O-Notation). Begründen Sie Ihre Angabe auf der Grundlage der
Beschreibung.

\end{enumerate}
\end{document}
