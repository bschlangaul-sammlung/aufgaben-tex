\documentclass{bschlangaul-aufgabe}
\bLadePakete{syntax}
\begin{document}
\bAufgabenMetadaten{
  Titel = {Aufgabe 6},
  Thematik = {Schreibtischlauf Haldensortierung},
  Referenz = 46115-2013-F.T2-A6,
  RelativerPfad = Staatsexamen/46115/2013/03/Thema-2/Aufgabe-6.tex,
  ZitatSchluessel = examen:46115:2013:03,
  ZitatBeschreibung = {Thema 2 Aufgabe 6},
  BearbeitungsStand = mit Lösung,
  Korrektheit = unbekannt,
  Ueberprueft = {unbekannt},
  Stichwoerter = {Mergesort, Heapsort},
  EinzelpruefungsNr = 46115,
  Jahr = 2013,
  Monat = 03,
  ThemaNr = 2,
  AufgabeNr = 6,
}

\section{Aufgabe 6
\footcite[Thema 2 Aufgabe 6]{examen:46115:2013:03}}

\begin{enumerate}

%%
% a)
%%

\item Vervollständigen Sie die folgende Sortierung mit MergeSort
(Sortieren durch Mischen) — beginnen Sie dabei Ihren
„rekursiven Abstieg“ immer im linken Teilfeld:
\index{Mergesort}

D | 40 5 89 95 85 84 || 14 25 20 52 7 71 |

Notation: Markieren Sie Zeilen mit D(ivide), in denen das Array zerlegt
wird, und mit M(erge), in denen Teilarrays zusammengeführt werden.
Beispiel:

D | 82 || 89 44 |

D 82 | 89 || 44 |

M 82 | 44 89 |

M | 44 82 89 |

\begin{bAntwort}
\begin{minted}[fontsize=\scriptsize]{md}
D | 40    5     89    95    85    84 || 14    25    20    52    7     71 |
D | 40    5     89 || 95    85    84 |
D | 40    5  || 89 |
D | 40 || 5  |
M | 5     40 |
M | 5     40    89 |
D                  |  95    85    84 |
D                  |  95    85 || 84 |
D                  |  95 || 85 |
M                  |  85    95 |
M                  |  84    85    95 |
M | 5     40    84    85    89    95 |
D                                    |  14    25    20 || 52    7     71 |
D                                    |  14    25 || 20 |
D                                    |  14 || 25 |
M                                    |  14    25 |
M                                    |  14    20   25  |
D                                                      |  52    7  || 71 |
D                                                      |  52 || 7  |
M                                                      |  7     52 |
M                                                      |  7     52    71 |
M                                    |  7     14   20     25    52    71 |
M                                    |  7     14   20     25    52    71 |
M | 5     7     14   20     25    40    52    71   84     85    89    95 |
\end{minted}
\end{bAntwort}

%%
% b)
%%

\item Sortieren Sie mittels HeapSort (Haldensortierung) die folgende
Liste weiter: Notation: Markieren Sie die Zeilen wie
folgt:\index{Heapsort}

\footcite[Sortieren II entnommen aus Algorithmen und
Datenstrukturen, Übungsblatt 4, Universität Würzburg, Aufgabe 4]{aud:pu:7}

\begin{description}
\item[I:] Initiale Heap-Eigenschaft hergestellt (größtes Element am
Anfang der Liste).

\item[R:] Erstes und letztes Element getauscht und letztes „gedanklich
entfernt“.

\item[S:] Erstes Element nach unten „versickert“ (Heap-Eigenschaft
wiederhergestellt).
\end{description}

\begin{bAntwort}
\begin{minted}[fontsize=\scriptsize]{md}
I | 99 63 91 4 36 81 76 |

R | 76 63 91 4 36 81 || 99 |
S | 91 63 81 4 36 76 || 99 |

R | 76 63 81 4 36 || 91 99 |
S | 81 76 63 4 36 || 91 99 |

R | 36 76 63 4 || 81 91 99 |
S | 76 36 63 4 || 81 91 99 |

R | 4 36 63 || 76 81 91 99 |
S | 63 4 36 || 76 81 91 99 |

R | 4 36 || 63 76 81 91 99 |
S | 36 4 || 63 76 81 91 99 |

R | 4 || 36 63 76 81 91 99 |
S | 4 || 36 63 76 81 91 99 |

R | 4 36 63 76 81 91 99 |
\end{minted}
\end{bAntwort}

\end{enumerate}

\end{document}
