\documentclass{bschlangaul-aufgabe}
\bLadePakete{baum}
\begin{document}
\bAufgabenMetadaten{
  Titel = {Aufgabe 6:},
  Thematik = {Halden - Heaps},
  Referenz = 46115-2017-H.T2-A6,
  RelativerPfad = Examen/46115/2017/09/Thema-2/Aufgabe-6.tex,
  ZitatSchluessel = examen:46115:2017:09,
  BearbeitungsStand = mit Lösung,
  Korrektheit = unbekannt,
  Ueberprueft = {unbekannt},
  Stichwoerter = {Halde (Heap)},
  EinzelpruefungsNr = 46115,
  Jahr = 2017,
  Monat = 09,
  ThemaNr = 2,
  AufgabeNr = 6,
}

Gegeben sei folgende Feld-Einbettung (Array-Darstellung) einer Min-Halde:
\index{Halde (Heap)}
\footcite{examen:46115:2017:09}

\begin{bProjektSprache}{Baum}
0 2 3 7 6 5 4 8 9 10 11 12
\end{bProjektSprache}

\begin{center}
\begin{tabular}{llllllllllll}
\bf{0}  & \bf{1}  & \bf{2}  & \bf{3}  & \bf{4}  & \bf{5}  & \bf{6}  & \bf{7}  & \bf{8}  & \bf{9}  & \bf{10} & \bf{11} \\
\hline
0       & 2       & 3       & 7       & 6       & 5       & 4       & 8       & 9       & 10      & 11      & 12      \\
\end{tabular}
\end{center}

\begin{enumerate}

%%
% a)
%%

\item Stellen Sie die Halde graphisch als (links-vollständigen) Baum
dar.

\begin{bAntwort}
\begin{center}
\begin{tikzpicture}[b binaer baum]
\Tree
[.0
  [.2
    [.7
      [.8 ]
      [.9 ]
    ]
    [.6
      [.10 ]
      [.11 ]
    ]
  ]
  [.3
    [.5
      [.12 ]
      \edge[blank]; \node[blank]{};
    ]
    [.4 ]
  ]
]
\end{tikzpicture}
\end{center}
\end{bAntwort}

%%
% b)
%%

\item Entfernen Sie das kleinste Element (die Wurzel 0) aus der obigen
initialen Halde, stellen Sie die Haldeneigenschaft wieder her und geben
Sie nur das Endergebnis in Felddarstellung an.

\begin{bAntwort}
\begin{tabular}{lllllllllll}
\bf{0}  & \bf{1}  & \bf{2}  & \bf{3}  & \bf{4}  & \bf{5}  & \bf{6}  & \bf{7}  & \bf{8}  & \bf{9}  & \bf{10} \\
\hline
2       & 6       & 3       & 7       & 10      & 5       & 4       & 8       & 9       & 12      & 11      \\
\end{tabular}
\end{bAntwort}

\begin{bAdditum}[zu b)]
\begin{bBaum}{Entfernen von „0“}
\begin{tikzpicture}[b binaer baum]
\Tree
[.12
  [.2
    [.7
      [.8 ]
      [.9 ]
    ]
    [.6
      [.10 ]
      [.11 ]
    ]
  ]
  [.3
    [.5 ]
    [.4 ]
  ]
]
\end{tikzpicture}
\end{bBaum}

\begin{bBaum}{Nach dem Vertauschen von „12“ und „2“}
\begin{tikzpicture}[b binaer baum]
\Tree
[.2
  [.12
    [.7
      [.8 ]
      [.9 ]
    ]
    [.6
      [.10 ]
      [.11 ]
    ]
  ]
  [.3
    [.5 ]
    [.4 ]
  ]
]
\end{tikzpicture}
\end{bBaum}

\begin{bBaum}{Nach dem Vertauschen von „12“ und „6“}
\begin{tikzpicture}[b binaer baum]
\Tree
[.2
  [.6
    [.7
      [.8 ]
      [.9 ]
    ]
    [.12
      [.10 ]
      [.11 ]
    ]
  ]
  [.3
    [.5 ]
    [.4 ]
  ]
]
\end{tikzpicture}
\end{bBaum}

\begin{bBaum}{Nach dem Vertauschen von „12“ und „10“}
\begin{tikzpicture}[b binaer baum]
\Tree
[.2
  [.6
    [.7
      [.8 ]
      [.9 ]
    ]
    [.10
      [.12 ]
      [.11 ]
    ]
  ]
  [.3
    [.5 ]
    [.4 ]
  ]
]
\end{tikzpicture}
\end{bBaum}
\end{bAdditum}

%%
% c)
%%

\item Fügen Sie nun den Wert 1 in die obige initiale Halde ein, stellen
Sie die Haldeneigenschaft wieder her und geben Sie nur das Endergebnis
in Felddarstellung an.

\begin{bAntwort}
\begin{tabular}{lllllllllllll}
\bf{0}  & \bf{1}  & \bf{2}  & \bf{3}  & \bf{4}  & \bf{5}  & \bf{6}  & \bf{7}  & \bf{8}  & \bf{9}  & \bf{10} & \bf{11} & \bf{12} \\
\hline
0       & 2       & 1       & 7       & 6       & 3       & 4       & 8       & 9       & 10      & 11      & 12      & 5       \\
\end{tabular}
\end{bAntwort}

\begin{bAdditum}[zu c)]

\begin{bBaum}{Nach dem Einfügen von „1“}
\begin{tikzpicture}[b binaer baum]
\Tree
[.0
  [.2
    [.7
      [.8 ]
      [.9 ]
    ]
    [.6
      [.10 ]
      [.11 ]
    ]
  ]
  [.3
    [.5
      [.12 ]
      [.1 ]
    ]
    [.4 ]
  ]
]
\end{tikzpicture}
\end{bBaum}

\begin{bBaum}{Nach dem Vertauschen von „1“ und „5“}
\begin{tikzpicture}[b binaer baum]
\Tree
[.0
  [.2
    [.7
      [.8 ]
      [.9 ]
    ]
    [.6
      [.10 ]
      [.11 ]
    ]
  ]
  [.3
    [.1
      [.12 ]
      [.5 ]
    ]
    [.4 ]
  ]
]
\end{tikzpicture}
\end{bBaum}

\begin{bBaum}{Nach dem Vertauschen von „1“ und „3“}
\begin{tikzpicture}[b binaer baum]
\Tree
[.0
  [.2
    [.7
      [.8 ]
      [.9 ]
    ]
    [.6
      [.10 ]
      [.11 ]
    ]
  ]
  [.1
    [.3
      [.12 ]
      [.5 ]
    ]
    [.4 ]
  ]
]
\end{tikzpicture}
\end{bBaum}
\end{bAdditum}
\end{enumerate}

\end{document}
